% Copyright 2024  李文威 (Wen-Wei Li).
% Permission is granted to copy, distribute and/or modify this
% document under the terms of the Creative Commons
% Attribution 4.0 International (CC BY 4.0)
% http://creativecommons.org/licenses/by/4.0/

% 目的: 字体相关设置, 呼叫相关宏包.
% 将由 AJbook.cls 引入
% 必须提供 \kaishu, \songti, \heiti, \thmheiti, \fangsong 几种字型切换命令, 在文档类中使用.
\ProvidesFile{font-setup-open.tex}[2022/12/19]

% 设置 xeCJK 字体及中文数字
%\setmainfont{TeX Gyre Pagella}	% 设置西文衬线字体
\setsansfont{TeX Gyre Heros}	% 设置西文无衬线字体

% 自用模式: 思源 Noto CJK SC 字体; 宜留意字体高低差异.
\setCJKmainfont[
	BoldFont=Noto Serif CJK SC Bold
]{Noto Serif CJK SC}

\setCJKsansfont[
	BoldFont=Noto Sans CJK SC Bold,
]{Noto Sans CJK SC}

\setCJKmonofont[
	BoldFont=Noto Sans CJK SC Bold,
]{Noto Sans CJK SC}


\setCJKfamilyfont{kai}[
	BoldFont=FandolKai-Regular.otf, ItalicFont=FandolKai-Regular.otf
]{FandolKai-Regular.otf}

\setCJKfamilyfont{song}[
	BoldFont=Noto Serif CJK SC Bold,
	ItalicFont=Noto Serif CJK SC
]{Noto Serif CJK SC}

\setCJKfamilyfont{fangsong}[
	BoldFont=FandolSong-Bold.otf,
	ItalicFont=FandolKai-Regular.otf
]{FandolFang-Regular.otf}

\setCJKfamilyfont{hei}[
	BoldFont=Noto Sans CJK SC Bold
]{Noto Sans CJK SC}

\setCJKfamilyfont{hei2}{Noto Sans CJK SC}

\setCJKfamilyfont{sectionfont}[
	BoldFont=* Black
]{Noto Sans CJK SC}

\setCJKfamilyfont{pffont}[
	BoldFont=* Medium
]{Noto Sans CJK SC}	% 证明用的字体
\setCJKfamilyfont{emfont}[
	BoldFont=Noto Sans CJK SC Medium
]{Noto Sans CJK SC}	% 强调用的字体

\defaultfontfeatures{Ligatures=TeX} 
% \XeTeXlinebreaklocale "zh"
% \XeTeXlinebreakskip = 0pt plus 1pt minus 0.1pt

% 以下设置字体相关命令, 用于定理等环境中.
\newcommand\kaishu{\CJKfamily{kai}} % 楷体
\newcommand\songti{\CJKfamily{song}} % 宋体
\newcommand\heiti{\CJKfamily{hei}}	% 黑体
\newcommand\thmheiti{\CJKfamily{hei2}}	% 用于定理名称的黑体
\newcommand\fangsong{\CJKfamily{fangsong}} % 仿宋
\renewcommand{\em}{\bfseries\mathversion{bold}\CJKfamily{emfont}} % 强调