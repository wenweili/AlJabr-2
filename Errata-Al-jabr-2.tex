%!TEX TS-program = xelatex
%!TEX encoding = UTF-8

% LaTeX source for the errata of the book ``代数学方法'' (Volume 2) in Chinese
% Copyright 2024  李文威 (Wen-Wei Li).
% Permission is granted to copy, distribute and/or modify this
% document under the terms of the Creative Commons
% Attribution 4.0 International (CC BY 4.0)
% http://creativecommons.org/licenses/by/4.0/

% 《代数学方法》卷一勘误表 / 李文威
% 使用自定义的文档类 AJerrata.cls. 自动载入 xeCJK.

\documentclass{AJerrata}

\usepackage{unicode-math}

\usepackage[unicode, colorlinks, psdextra, bookmarksnumbered,
	pdfpagelabels=true,
	pdfauthor={李文威 (Wen-Wei Li)},
	pdftitle={代数学方法卷二勘误},
	pdfkeywords={}
]{hyperref}

\setmainfont[
	BoldFont={texgyretermes-bold.otf},
	ItalicFont={texgyretermes-italic.otf},
	BoldItalicFont={texgyretermes-bolditalic.otf},
	PunctuationSpace=2
]{texgyretermes-regular.otf}

\setsansfont[
	BoldFont=FiraSans-Bold.otf,
	ItalicFont=FiraSans-Italic.otf
]{FiraSans-Regular.otf}

\setCJKmainfont[
	BoldFont=Noto Serif CJK SC Bold
]{Noto Serif CJK SC}

\setCJKsansfont[
	BoldFont=Noto Sans CJK SC Bold
]{Noto Sans CJK SC}

\setCJKfamilyfont{emfont}[
	BoldFont=FandolHei-Regular.otf
]{FandolHei-Regular.otf}	% 强调用的字体

\renewcommand{\em}{\bfseries\CJKfamily{emfont}} % 强调

\setmathfont[
	Extension = .otf,
	math-style= TeX,
]{texgyretermes-math}

\usepackage{mathrsfs}
\usepackage{stmaryrd} \SetSymbolFont{stmry}{bold}{U}{stmry}{m}{n}	% 避免警告 (stmryd 不含粗体故)
% \usepackage{array}
% \usepackage{tikz-cd}  % 使用 TikZ 绘图
\usetikzlibrary{positioning, patterns, calc, matrix, shapes.arrows, shapes.symbols}

\usepackage{myarrows}				% 使用自定义的可伸缩箭头
\usepackage{mycommand}				% 引入自定义的惯用的命令


\title{\bfseries 代数学方法(第二卷)勘误表 \\ 跨度: 2024 年 9 月正式出版迄今 }
\author{李文威}
\date{\today}

\begin{document}
	\maketitle

	\begin{Errata}
		\item[约定 2.6.3 第二行]
		\Orig 上确界 (或下确界)
		\Corr 下确界 (或上确界)
		\Thx{感谢黄行知指正}
		
		\item[推论 3.12.7 证明倒数第二行的显示公式]
		将末项的 $\mathrm{R}^1 F(Z)$ 换成 $\mathrm{R}^1 F(X)$
		\Thx{感谢黄行知指正}
		
		\item[约定 3.12.8]
		\Orig 高次左导出函子 (或右导出函子)
		\Corr 高次右导出函子 (或左导出函子)
		\Thx{感谢黄行知指正}
		
		\item[命题 3.13.13 证明]
		在``进入正题...''一段, 将最后的 $\psi^{-1}(c)$ 改为 $(\varprojlim \psi)^{-1}(c)$.
		
		\item[注记 3.14.8 之前的段落]
		\Orig $\cdots \to Q_1 \to Q_0 \to X \to 0$
		\Corr $\cdots \to Q_1 \to Q_0 \to Y \to 0$
		
		\item[\S 3.14 倒数第四段]
		\Orig 作为推论,... $\Hm^p(C) \otimes \Hm^q(D)$, 从它到 $\Hm^n(C \otimes D)$ ...
		\Corr 作为推论,... $\Hm_p(C) \otimes \Hm_q(D)$, 从它到 $\Hm_n(C \otimes D)$ ...
		\Thx{感谢黄行知指正}
		
		\item[定义 4.5.11 第三行]
		\Orig ... $X$ 同构 $Y$ 的...
		\Corr ... $X$ 通过 $Y$ 的...
		\Thx{感谢郑维喆指正}
		
		\item[定义 5.1.1]
		第一条的 $\mathrm{F}^{p+1}$ 改为 $\mathrm{F}^{p+1} X$, 定义之后第二段末尾的范畴 $\mathrm{F}_{\bullet}(\mathcal{A})$ 改为范畴 $\mathrm{Fil}_{\bullet}(\mathcal{A})$.
	\end{Errata}
\end{document}