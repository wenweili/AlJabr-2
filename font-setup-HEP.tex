% Copyright 2022  李文威 (Wen-Wei Li).
% Permission is granted to copy, distribute and/or modify this
% document under the terms of the Creative Commons
% Attribution 4.0 International (CC BY 4.0)
% http://creativecommons.org/licenses/by/4.0/

% 目的: 字体相关设置, 呼叫相关宏包.
% 将由 AJbook.cls 引入
% 必须提供 \kaishu, \songti, \heiti, \thmheiii, \fangsong 几种字型切换命令, 在文档类中使用.
\ProvidesFile{font-setup-HEP.tex}[2022/12/19]

% 设置 xeCJK 字体及中文数字
%\setmainfont{TeX Gyre Pagella}	% 设置西文衬线字体
\setsansfont{TeX Gyre Heros}	% 设置西文无衬线字体

% 出版模式: 方正字体
\setCJKmainfont[BoldFont=FZCuSong-B09S, ItalicFont=FZKai-Z03]{FZShuSong-Z01}
\setCJKsansfont[BoldFont=FZDaHei-B02S]{FZHei-B01}
\setCJKmonofont[BoldFont=FZDaHei-B02S]{FZHei-B01}
\setCJKfamilyfont{kai}{FZKai-Z03}
\setCJKfamilyfont{song}[BoldFont=FZCuSong-B09S, ItalicFont=FZKai-Z03]{FZShuSong-Z01}
\setCJKfamilyfont{fangsong}[BoldFont=FZCuSong-B09S, ItalicFont=FZKai-Z03]{FZFangSong-Z02}
\setCJKfamilyfont{hei}[BoldFont=FZDaHei-B02S]{FZHei-B01}
\setCJKfamilyfont{hei2}[BoldFont=FZDaHei-B02S]{FZHei-B01}
\setCJKfamilyfont{sectionfont}[BoldFont=FZDaHei-B02S]{FZHei-B01}
\setCJKfamilyfont{pffont}[BoldFont=FZDaHei-B02S]{FZHei-B01}
\setCJKfamilyfont{emfont}[BoldFont=FZHei-B01]{FZHei-B01}

\defaultfontfeatures{Ligatures=TeX} 
%\XeTeXlinebreaklocale "zh"
%\XeTeXlinebreakskip = 0pt plus 1pt minus 0.1pt

% 以下设置字体相关命令, 用于定理等环境中.
\newcommand\kaishu{\CJKfamily{kai}} % 楷体
\newcommand\songti{\CJKfamily{song}} % 宋体
\newcommand\heiti{\CJKfamily{hei}}	% 黑体
\newcommand\thmheiti{\CJKfamily{hei2}}	% 用于定理名称的黑体
\newcommand\fangsong{\CJKfamily{fangsong}} % 仿宋
\renewcommand{\thepart}{\Alph{part}} % 以大写字母表示部分
\renewcommand{\em}{\bfseries\mathversion{bold}\CJKfamily{emfont}} % 强调
