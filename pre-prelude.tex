% To be included, temporarily...
\chapter*{导言}

\section*{最终版草稿说明}
本版继续进行了纠错和细微的调整. 完整版将在出版手续完成后释出, 敬请期待. 依然欢迎读者批评指正, 必要的修改将会纳入勘误表.

\vspace{1em}
\begin{flushright}\begin{minipage}{0.3 \textwidth}
		\begin{tabular}{c}
			{\fangsong 李文威} \\
			2022 年 12 月 \\
			于石景山
		\end{tabular}
\end{minipage}\end{flushright}
\vspace{1em}

\section*{第四版草稿说明}
这一版草稿补全各章开头的简介, 也完成了针对正文和附录的一轮检查. 成果是清理了大量笔误, 数学错误和欠通顺的表述, 同时结构有所微调. 下一份流通的版本预期将是包含导言在内的成品, 现有内容也可能再作修正.

\vspace{1em}
\begin{flushright}\begin{minipage}{0.3 \textwidth}
		\begin{tabular}{c}
			{\fangsong 李文威} \\
			2022 年 11 月 \\
			于海淀山后
		\end{tabular}
\end{minipage}\end{flushright}
\vspace{1em}

\section*{第三版草稿说明}
相较于前一版草稿, 这版插入关于群的同调与上同调的第六章, 新增附录, 同时将前一版的一些正文内容移入附录. 既有内容也有所清理和改善/改正. 这一版本在附录部分加入了 Freyd--Mitchell 嵌入定理的一个迂回证明, 应用 $\IndC$ 化的技巧, 尽管正文部分几乎不用到嵌入定理.

全书框架大致已定型, 惟导言和各章开头仍待撰写. 如第二版草稿说明所述, 这是巨大的缺漏.

此外, 全书还需要一次完整检查, 目前版本无疑潜藏不少错误, 更有许多尚待理顺的内容.

\vspace{1em}
\begin{flushright}\begin{minipage}{0.3 \textwidth}
		\begin{tabular}{c}
			{\fangsong 李文威} \\
			2022 年 8 月 \\
			于海淀山后
		\end{tabular}
\end{minipage}\end{flushright}
\vspace{1em}

\section*{第二版草稿说明}
目前版本共有八章. 除了新添三章, 前五章也作了微小的改正. 按照写作计划, 未来很可能再插入或附加其他内容, 已有内容也不排除再作调整, 特别是一些枝蔓可能需要修剪.

此外, 必须强调的是数学著作绝不只是定义和论证的堆积, 所以在导言和各章开头的导读尚未真正完成的情况下, 书稿的任何一章都是不完整的. 呈现这样的半成品只能说是权宜之计.

书稿必然还有大量错误和不妥之处, 望广大读者继续批评斧正.

\vspace{1em}
\begin{flushright}\begin{minipage}{0.3 \textwidth}
		\begin{tabular}{c}
			{\fangsong 李文威} \\
			2022 年 3 月 \\
			于高碑店
		\end{tabular}
\end{minipage}\end{flushright}
\vspace{1em}

\section*{第一版草稿说明}

现前这份书稿, 是计划中的《代数学方法》卷二的一部分, 预估约占一半的内容. 除了缺少后续章节, 导言和每章开头的阅读提示或者极简略, 或者付之阙如. 习题部分也还相当贫乏. 这是一份不折不扣的半成品. 可以预期的是将有大量的笔误和数学错误. 现有的章节也可能在未来改动.

之所以公布这么一份草稿, 主要是我相信它对读者们是有益的, 读者群体也应当是广泛的, 其次则是抵达下一个进度节点尚需时日, 最后, 我盼望各位的反馈能让成品更快更好地面世.

很显然, 《代数学方法》卷一覆盖本书所需的全部代数知识. 由于卷一和此处使用的都是标准术语, 通行的同类教材也应该能提供这些背景. 建议读者先阅读开头的凡例部分, 确认符号和惯例.

本卷的编写精神大致上和卷一类似, 但由于处理的内容不同, 具体风格也有所改变. 这些差异只能在未来的导言里仔细解释. 如同卷一, 本书并不鼓励初学者循序阅读, 除非对抽象方法已经有充分高的接受度. 具体的阅读须知将见于导言和每章开头, 遗憾的是这些内容只有待全书定型后方能撰写.

编撰过程中广泛参考了各种相关著作, 一部分已在正文引用, 剩余部分计划待全书完成后在导言部分列出, 并非有意遗漏, 敬请谅解.

当前内容大致相当于传统上的同调代数, 这是本真意义的``线性代数''的一个真子集. 从现代的观点看, 前五章达到了一个勉强够用的体系, 但即使作为经典同调代数的基础训练, 仍然是意犹未尽. 剩余内容或要寄望于将来的第六章. 至于群的上同调这类标准应用, 则要留待更后面的章节处理.

关于使用这份文件的许可协议, 请详阅封面页的链接. 本人欢迎一切建议, 批评和指正.

\vspace{1em}
\begin{flushright}\begin{minipage}{0.3 \textwidth}
		\begin{tabular}{c}
			{\fangsong 李文威} \\
			2021 年 3 月 \\
			于北京大学图书馆
		\end{tabular}
\end{minipage}\end{flushright}