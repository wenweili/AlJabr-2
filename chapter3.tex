% LaTeX source for book ``代数学方法'' in Chinese
% Copyright 2024  李文威 (Wen-Wei Li).
% Permission is granted to copy, distribute and/or modify this
% document under the terms of the Creative Commons
% Attribution 4.0 International (CC BY 4.0)
% http://creativecommons.org/licenses/by/4.0/

% To be included
\chapter{复形}\label{sec:cplx}
复形的概念在 \S\ref{sec:cohomology} 已有初步介绍. 对于 Abel 范畴 $\mathcal{A}$ 上的复形 $X = (X^n, d^n)_n$, 上同调 $\Hm^n(X) := \Ker(d^n)/\Image(d^{n+1})$ 从 $X$ 萃取重要的不变量.逆 上同调全为零的复形即正合列, 又称零调复形.

本章 \S\S\ref{sec:Hom-cplx}---\ref{sec:double-cplx} 有许多内容并不涉及上同调, 而适用于一般的加性范畴 $\mathcal{A}$. 记 $\mathcal{A}$ 上的复形所成范畴为 $\cate{C}(\mathcal{A})$. 若无另外申明, 本章考虑的复形都是注记 \ref{rem:cochain-vs-chain} 所称的上链复形, 提及 $\Bbbk$ 时均指任意交换环.

对于加性范畴 $\mathcal{A}$ 上给定的复形 $X$ 和 $Y$, 同伦是 $\Hom_{\cate{C}(\mathcal{A})}(X, Y)$ 上的一个等价关系, 它也可以从 \S\ref{sec:Hom-cplx} 探讨的 $\Hom$ 复形 $\Hom^\bullet(X, Y)$ 来理解. 当 $\mathcal{A}$ 是 Abel 范畴时, 同伦不影响态射在上同调层次的作用. 其次, \S\ref{sec:mapping-cone} 介绍的映射锥 $\Cone(f)$ 将在导出范畴的研究中扮演要角, 它与同伦密切相关, 两者都有直接的拓扑诠释.

双复形 $X = (X^{p, q}, \dhori^{p, q}, \dvert^{p, q})_{p, q}$ 可以设想为带有两个独立方向的复形, 也可以等价地理解为``复形的复形''; 双复形 $X$ 能通过直和 (或直积) 摊平为复形 $\tot_{\oplus} X$ (或 $\tot_{\Pi} X$), 称为 $X$ 的全复形, 其确切定义涉及一些正负号, 这些细节是 \S\ref{sec:double-cplx} 的主题.

同调代数的基本技巧之一是从复形的短正合列 $0 \to X' \xrightarrow{f} X \xrightarrow{g} X'' \to 0$ 导出上同调的长正合列
\[ \cdots \to \Hm^n(X') \xrightarrow{\Hm^n(f)} \Hm^n(X) \xrightarrow{\Hm^n(g)} \Hm^n(X'') \xrightarrow{\text{连接态射}} \Hm^{n+1}(X') \to \cdots, \]
长正合列也可以从映射锥的观点来理解, 但需要一些不尽平凡的论证. 这些都是 \S\S\ref{sec:Abel-cplx}---\ref{sec:cone-vs-long-exact-sequence} 的内容.

为了帮助读者熟悉复形的种种基本操作, \S\ref{sec:HH} 将依托 $(R, R)$-双模 $M$ 的 Hochschild 同调 $\HHm_n(M)$ 和上同调 $\HHm^n(M)$ 进行一次整合演练. Hochschild 同调和上同调在数学中的用途甚广, 本章仅以教学为目的作粗浅介绍; 它们在此后各章还会作为例子反复现身.

基于 \S\ref{sec:truncation-functors} 介绍的截断函子, 我们将在 \S\ref{sec:double-cplx-coh} 说明如何联系双复形的纵向或横向上同调以及全复形的上同调, 这部分要求双复形具备某些有界性条件. 一些教材选择以谱序列来处理这些性质, 本书则效法 \cite{KS06}, 迂回地予以直接证明.

对于 Abel 范畴 $\mathcal{A}$ 的对象 $X$, 所谓 $X$ 的解消分成左右两种版本, 具体写作正合列
\[ 0 \to X \to I^0 \to I^2 \to \cdots \quad \text{或} \quad \cdots \to P_1 \to P_0 \to X \to 0. \]
这也相当于通过拟同构 $X \to I := (I^n)_n$ 或 $P := (P_n)_n \to X$ 将 $X$ 代换为一个具有更好性质的复形 $I$ 或链复形 $P$; 所谓拟同构, 是指在上同调或同调层次诱导出同构的态射. 如要求所有 $I^n$ (或 $P_n$) 为内射 (或投射) 对象, 便称之为 $X$ 的内射 (或投射) 解消. 我们将在 \S\ref{sec:resolutions} 深入地探讨解消. 解消还具有同伦意义下的函子性和唯一性: 以内射解消为例, 若在 $\mathcal{A}$ 中有实线部分图表
\[\begin{tikzcd}
	0 \arrow[r] & X' \arrow[r] & (I')^0 \arrow[r] & (I')^1 \arrow[r] & (I')^2 \arrow[r] & \cdots \\
	0 \arrow[r] & X \arrow[u, "f"] \arrow[r] & I^0 \arrow[dashed, u, "{\beta^0}"] \arrow[r] & I^1 \arrow[r] \arrow[dashed, u, "{\beta^1}"] & I^2 \arrow[r] \arrow[dashed, u, "{\beta^2}"] & \cdots 
\end{tikzcd}\]
使得两行皆正合, 每个 $(I')^n$ 皆为内射对象, 则存在虚线箭头 $\beta^0, \beta^1, \ldots$ 使全图交换, 而且复形之间的态射 $\beta: I \to I'$ 精确到同伦还是唯一的. 这是定理 \ref{prop:resolution-homotopy} 和引理 \ref{prop:ext-alpha-beta} 的综合运用, 背后有更广的表述.

事实上, \S\ref{sec:resolutions} 不仅考虑 $\mathcal{A}$ 的对象 $X$, 还扩及复形的解消, 相关结论包括方便的 Cartan--Eilenberg 解消 (定理 \ref{prop:CE-resolution}); 复形版本的一些论证不免变得冗长, 尽管思路终归是清晰的.

通过以内射 (或投射) 解消代替 $\mathcal{A}$ 的对象, 让左正合 (或右正合) 函子 $F: \mathcal{A} \to \mathcal{B}$ 作用于解消的每一项, 再取 $\Hm^n$ (或 $\Hm_n = \Hm^{-n}$), 便引向了右导出函子 $\mathrm{R}^n F$ (或左导出函子 $\mathrm{L}_n F$) 的经典定义, 而 $\mathcal{A}$ 中的短正合列诱导导出函子的长正合列; 详见 \S\ref{sec:derived-primer}. 基于复形的解消, 导出函子还能在下有界或上有界复形上求值, 经典文献中称此为超导出函子.

从右 (或左) 导出函子及其长正合列自然地提炼出上同调 (或同调) $\delta$-函子的概念, 而导出函子在其中是``泛''的. 命题 \ref{prop:erasable-univ-delta} 通过定义 \ref{def:erasable} 的可拭 (或余可拭) 条件来刻画泛 $\delta$-函子, 这是 Grothendieck 的贡献. 与此相关的成套技术构成了经典同调代数的核心内容.

另一个重要构造是双函子 $F: \mathcal{A}_1 \times \mathcal{A}_2 \to \mathcal{B}$ 的右 (或左) 导出函子, 前提是 $F$ 对两个变元皆左 (或右) 正合; 日常范例包括 $\Ext$ 与 $\Tor$ 函子, 见 \S\ref{sec:Ext-Tor}, 前者必然涉及相反范畴上的复形. 定理 \ref{prop:balanced-primer} 将考量称为平衡双函子的一类双函子, 说明对两个变元的求导殊途同归, 其证明基于双复形的上同调的基本性质.

在 $\mathcal{A}$ 有正合可数积的前提下, \S\ref{sec:lim1} 讨论的 $\lim^1$ 是右导出函子的另一类实例; 它是 $\varprojlim$ 的第一个右导出函子, 而更高阶的右导出函子全为 $0$ (定理 \ref{prop:lim1}). 为了获取确保 $\lim^1 = 0$ 的实用手段, 我们将在定义 \ref{def:ML} 引进 Mittag-Leffler 条件.

最后, 我们在 \S\ref{sec:K-injectives} 探讨何谓 K-内射复形和 K-投射复形, 然后定义 K-内射和 K-投射解消. 它们分别是内射解消和投射解消对无界复形的自然推广; 由于 K-内射和 K-投射是关于复形整体性质的定义, 它们比逐项定义的内射和投射解消显得更简洁自然. 借由这些概念, 导出函子的定义便能够扩及无界复形. 问题在于这些解消的存在性: 定理 \ref{prop:K-injective-resolution} 和 \ref{prop:K-projective-resolution} 包含部分结果, 论证相对复杂, 需要关于 $\lim^1$ 的一些性质.

\begin{wenxintishi}	
	处理 $\Bbbk$-模复形 ($\Bbbk$ 是交换环) 的经验有助于更快地理解本章内容, 然而并非绝对必要. 在\CHref{sec:Abel-cat}的基础上, 处理比模范畴更宽泛的 Abel 范畴也没有根本的困难.
	
	读者若希望尽速理解导出函子的经典定义, 又缺乏拓扑方面的动机, 建议先略过 \S\ref{sec:mapping-cone}, \S\ref{sec:opposite-cplx} 和 \S\ref{sec:cone-vs-long-exact-sequence} 中关于映射锥和映射柱的内容. 此外, \S\ref{sec:opposite-cplx} 旨在联系 $\cate{C}(\mathcal{A}^{\opp})$ 和 $\cate{C}(\mathcal{A})^{\opp}$, 该处涉及的一些正负号需要细心处理, 但它们对整体思路影响甚微; 建议初次阅读时只看陈述, 跳过证明. 

	由于后续内容均不依赖 Hochschild 同调和上同调理论, 请读者衡量时间和兴趣来决定略过与否; 这些内容并不困难, 而且在后续章节还会作为例子或注记被反复提及.

	同样地, \S\ref{sec:lim1} 的内容在除 \S\ref{sec:K-injectives} 之外的后续部分仅为例子出现. \S\ref{sec:K-injectives} 的内容仅在 \S\ref{sec:unbounded-derived} 和 \S\ref{sec:otimesL} 引用; 倘若读者对无界导出函子兴趣不大, 则可先看定义部分以及例 \ref{eg:bdd-K-resolution}, 省略其余.
\end{wenxintishi}

\section{加性范畴上的复形}\label{sec:additive-cplx}
定义 \ref{def:complex} 已初步介绍了复形的概念和常用记法. 选定加性范畴 $\mathcal{A}$, 并考虑其上的无穷长, 亦即无端点项的复形 $X = (X^n, d^n)_{n \in \Z}$; 态射 $d^n$ 经常标为 $d_X^n$. 基于历史的缘由, 这些态射 $d_X^n$ 也被称为``微分''.
\index{weifen@微分 (differential)}

\begin{definition}\label{def:cplx-cat}
	\index{fuxing}
	\index[sym1]{CA@$\cate{C}(\mathcal{A})$}
	从复形 $X$ 到 $Y$ 的态射定义为资料 $(f^n)_{n \in \Z}$, 也简记为 $f$, 其中 $f^n \in \Hom_{\mathcal{A}}(X^n, Y^n)$ 须满足
	\[ d_Y^n f^n = f^{n+1} d_X^n, \quad n \in \Z. \]

	全体复形及其间的态射构成范畴 $\cate{C}(\mathcal{A})$: 复形 $X$ 到自身的恒等态射由 $(\identity_X)^n := \identity_{X^n}$ 确定, 态射合成按 $(gf)^n = g^n f^n$ 定义. 加法 $f + g := (f^n + g^n)_{n \in \Z}$ 使 $\Hom_{\cate{C}(\mathcal{A})}(X, Y)$ 成为交换群.	
\end{definition}

若 $\mathcal{A}$ 在定义 \ref{def:k-linear-cat} 下是 $\Bbbk$-线性的, 其中 $\Bbbk$ 是选定的交换环, 则 $\Hom_{\cate{C}(\mathcal{A})}(X, Y)$ 也同样对逐项的纯量乘法构成 $\Bbbk$-模. 在本节的结果中, ``加性''一词全都可以代换为更广泛的``$\Bbbk$-线性'', 但为了节约笔墨, 今后仅陈述加性版本.

复形定义中的关键性质是 $d^{n+1} d^n = 0$, 这点也可以用微分分次对象的语言改述\footnote{相关讨论将在 \S\ref{sec:spectral-sequence} 接续.}.

\begin{definition}[分次对象]\label{def:graded-obj}
	\index{fenciduixiang@分次对象 (graded object)}
	\index{fenciduixiang!$\Z^m$-分次, 双分次}
	取 $\mathcal{A}$ 为任意范畴, 回忆积范畴 $\mathcal{A}^{\Z}$ 的构造:
	\begin{compactitem}
		\item 其对象形如 $X = (X^n)_{n \in \Z}$, 其中 $X^n \in \Obj(\mathcal{A})$;
		\item 其态射形如 $f = (f^n: X^n \to Y^n)_{n \in \Z}$, 无其他条件, 而合成是逐项 (或曰逐次) 操作的, 亦即 $(gf)^n = g^n f^n$;
		\item 定义 $\mathcal{A}^{\Z}$ 的自同构 $T$ 为下述``平移''函子: $(TX)^n = X^{n+1}$, $(Tf)^n = f^{n+1}$.
	\end{compactitem}
	对象 $X = (X^n)_n$ 也称为 $\mathcal{A}$ 上的 $\Z$-分次对象, 简称\emph{分次对象}. 
	
	推而广之, $\mathcal{A}^{\Z^m}$ 的对象称为 $\Z^m$-分次对象, 写作 $(X^{n_1, \ldots, n_m})_{n_1, \ldots, n_m}$, 其态射则写作 $(f^{n_1, \ldots, n_m})_{n_1, \ldots, n_m}$. 此范畴带有一族相交换的平移函子 $T_1, \ldots, T_m$. 特例 $m=2$ 又称\emph{双分次对象}.
\end{definition}

若 $\mathcal{A}$ 是加性范畴, 则 $\mathcal{A}^{\Z^m}$ 亦然: 态射相加, 对象作直和等一切操作都是逐次施行的.

\begin{definition}[微分分次对象]
	\index{weifenfenciduixiang@微分分次对象 (differential graded object)}
	设 $\mathcal{A}$ 为加性范畴, 其上的微分分次对象定义为资料 $(X, d)$, 其中 $X \in \Obj(\mathcal{A}^{\Z})$ 而 $d: X \to TX$ 满足 $(Td)d = 0$. 态射 $(X,d) \to (Y,d)$ 定义为满足 $(Tf) d = df$ 的态射 $f \in \Hom_{\mathcal{A}^{\Z}}(X, Y)$.
\end{definition}

将微分分次对象 $(X, d)$ 中的 $d$ 具体表成态射族 $d_X^n: X^n \to X^{n+1}$, 则 $d$ 的条件是 $d_X^{n+1} d_X^n = 0$, 而态射的条件是 $f^{n+1} d^n_X = d^n_Y f^n$, 复归定义 \ref{def:cplx-cat}. 因此 $\mathcal{A}$ 上的微分分次对象范畴同构于复形范畴. 本节将不加说明地切换两种视角.

研究复形的第一步是了解 $\cate{C}(\mathcal{A})$ 中的各种极限. 积范畴 $\mathcal{A}^{\Z}$ 的版本相对容易, 其极限可以对每个 $n \in \Z$ 取, 亦即逐项或曰逐次地在 $\mathcal{A}$ 中构造. 以下说明如何将此提升到 $\cate{C}(\mathcal{A})$ 层次. 记 $U: \cate{C}(\mathcal{A}) \to \mathcal{A}^{\Z}$ 为忘却函子, 映微分分次对象 $(X,d)$ 为分次对象 $X$.

\begin{lemma}[极限的逐项构造]\label{prop:limits-cplx}
	任给函子 $\alpha: I \to \cate{C}(\mathcal{A})$, 记 $\overline{\alpha} := U\alpha$. 假设 $\varinjlim \overline{\alpha}$ 存在, 则它可以唯一地升级为 $\cate{C}(\mathcal{A})$ 的对象, 使之给出 $\varinjlim \alpha$; 对 $\varprojlim$ 亦有相应结果.

	按定义 \ref{def:create-limit} 的语言, 这相当于说忘却函子 $U$ 生 $\varinjlim$ 和 $\varprojlim$.
\end{lemma}
\begin{proof}
	设 $\alpha: I \to \cate{C}(\mathcal{A})$ 映 $i \in \Obj(I)$ 为微分分次对象 $\alpha(i) = (\overline{\alpha}(i), d_i)$. 若 $\varinjlim \overline{\alpha}$ 存在, 则诸 $d_i$ 确定 $d: \varinjlim \overline{\alpha} \to (T\varinjlim \overline{\alpha} \simeq \varinjlim T\overline{\alpha})$ (自同构 $T$ 自动保 $\varinjlim$), 其刻画是交换图表
	\[\begin{tikzcd}
		\varinjlim \overline{\alpha} \arrow[r, "d"] & \varinjlim T\overline{\alpha} \arrow[r, "\sim"] & T\varinjlim \overline{\alpha} \\
		\overline{\alpha}(i) \arrow[u] \arrow[r, "{d_i}"'] & T\overline{\alpha}(i) \arrow[u] \arrow[ru] &
	\end{tikzcd} \qquad (i \in \Obj(I)) . \]

	按此由 $(Td_i) d_i = 0$ 推得 $(Td)d = 0$, 给出 $(\varinjlim \overline{\alpha}, d) \in \Obj(\cate{C}(\mathcal{A}))$. 图表说明这是使 $(\overline{\alpha}(i), d_i) \to (\varinjlim \overline{\alpha}, d)$ 为态射的唯一选法.
	
	以下验证它在 $\cate{C}(\mathcal{A})$ 中作为 $\varinjlim \alpha$ 的泛性质. 在 $\cate{C}(\mathcal{A})$ 中给定一族相容的态射 $f_i: \alpha(i) \to L$, 忘却给出 $f_i: \overline{\alpha}(i) \to U(L)$, 它们唯一地确定 $f: \varinjlim \overline{\alpha} \to U(L)$. 关键在于证 $f$ 为 $\cate{C}(\mathcal{A})$ 的态射. 虽是意料之中, 我们还是郑重其事地作图
	\[\begin{tikzcd}
		\varinjlim \overline{\alpha} \arrow[dd, "f"'] \arrow[rr, "d"] & & \varinjlim T\overline{\alpha} \arrow[r, "\sim"] & T\varinjlim \overline{\alpha} \arrow[dd, "{Tf}"] \\
		& \overline{\alpha}(i) \arrow[lu] \arrow[ld, "{f_i}"] \arrow[r, "d_i"] & T\overline{\alpha}(i) \arrow[ru] \arrow[rd, "{Tf_i}"] \arrow[u] & \\
		U(L) \arrow[rrr, "d_L"'] & & & TU(L)
	\end{tikzcd} \qquad (i \in \Obj(I)). \]
	每个构件都可以按定义或构造验证交换性. 于是 $d_L f$ 和 $(Tf)d$ 合成每个 $\overline{\alpha}(i) \to \varinjlim \overline{\alpha}$ 皆交换, 由此推得 $d_L f = (Tf) d$. 明所欲证.
\end{proof}

\begin{proposition}\label{prop:additive-cat-cplx}
	范畴 $\cate{C}(\mathcal{A})$ 是加性范畴; 如果 $\mathcal{A}$ 完备 (或余完备), 则 $\cate{C}(\mathcal{A})$ 亦然.
\end{proposition}
\begin{proof}
	以引理 \ref{prop:limits-cplx} 将所需的极限逐次地化到 $\mathcal{A}$ 上.
\end{proof}

\begin{definition}[平移函子]\label{def:translation-functor}
	\index{pingyihanzi@平移函子 (translation functor)}
	\index[sym1]{[n]@$[n]$}
	设 $X$ 为 $\cate{C}(\mathcal{A})$ 的对象, $n \in \Z$. 定义复形 $X[n]$ 如下:\footnote{在 $d_{X[n]}$ 的定义中插入 $(-1)^n$ 是有好处的, 它也和不加正负号的版本 $X[n]^\circ := \left(X^{n+k}, d_X^{k+n}\right)_k$ 相互同构: 考虑 $n=1$ 情形, 按 $\cdots, +, -, +, \cdots$ 定义 $X[1] \rightiso X[1]^\circ$ 便是.}
	\[ \left( X[n]\right)^k := X^{k+n}, \quad d_{X[n]}^k := (-1)^n d_X^{k+n}. \]
	这给出加性函子 $[n]: \cate{C}(\mathcal{A}) \to \cate{C}(\mathcal{A})$: 若 $f: X \to Y$ 为复形之间的态射, 则 $f[n]: X[n] \to Y[n]$ 由 $f[n]^k := f^{n+k}$ 给出.
\end{definition}

显然 $X[n]$ 仍是复形, 故函子是良定义的. 以下性质属显然.
\begin{itemize}
	\item $[0] = \identity_{\cate{C}(\mathcal{A})}$.
	\item $[n][m] = [n+m]$. 因此 $[n]$ 是从 $\cate{C}(\mathcal{A})$ 的自同构, 以 $[-n]$ 为逆.
	\item $(d_X^n)_{n \in \Z}$ 给出复形之间的态射 $d_X: X \to X[1]$ (提示: $d_X^{n+1} d_X^n = 0$).
\end{itemize}

\begin{convention}\label{con:concentrated-cplx}
	对于 $\mathcal{A}$ 的任何对象 $S$, 可将其视同复形 $(S^n, d^n_S)_{n \in \Z}$, 其中 $S^0 := S$ 而其余项为 $0$, 并且 $d_S^n$ 全取为零态射. 这给出全忠实加性函子 $\mathcal{A} \hookrightarrow \cate{C}(\mathcal{A})$. 推而广之, 对任意 $n \in \Z$, 复形 $S[-n]$ 是置 $S$ 于第 $n$ 次项, 其余各项为 $0$ 之复形.
\end{convention}

以下结果是自明的.

\begin{proposition}\label{prop:CA-functoriality}
	\index[sym1]{CF@$\cate{C}F$}
	给定加性函子 $F: \mathcal{A} \to \mathcal{A}'$, 则 $(X^n, d_X^n)_n \mapsto (FX^n, Fd_X^n)_n$ 给出函子 $\cate{C}F: \cate{C}(\mathcal{A}) \to \cate{C}(\mathcal{A}')$. 它和平移函子相交换: $[m] \circ \cate{C}F = \cate{C}F \circ [m]$ 对所有 $m \in \Z$ 皆成立.
\end{proposition}

\begin{remark}\label{rem:chain-cplx-translation}
	\index{pingyihanzi}
	\index[sym1]{[n]}
	本节一切结果都适用于注记 \ref{rem:cochain-vs-chain} 提及的链复形 $(X_\bullet, d_\bullet)$. 链复形范畴上的平移函子按
	\[ X[n]_k := X_{k+n}, \quad d^{X[n]}_k = (-1)^n d^X_{n+k} \]
	来定义. 若按 $X_n = X^{-n}$ 和 $d_n = d^{-n}$ 进行过渡, 则链复形的平移 $[1]$ 对应于复形的平移 $[-1]$.
\end{remark}

\section{\texorpdfstring{$\Hom$}{Hom} 复形与同伦}\label{sec:Hom-cplx}
先前已经定义过两个复形之间的 $\Hom$. 本节说明如何将其升级为复形, 并且将态射的合成升级为复形上的某种乘法. 仍取 $\mathcal{A}$ 为加性范畴. 当 $\Bbbk$ 是任意交换环而 $\mathcal{A}$ 是 $\Bbbk$-线性范畴时, 本节所有陈述都有不言自明的 $\Bbbk$-线性版本.

回忆到取 $\Hom$ 集给出函子 $\Hom: \mathcal{A}^{\opp} \times \mathcal{A} \to \cate{Ab}$. 不带上标的 $\Hom$ 指涉 $\mathcal{A}$ 或 $\cate{C}(\mathcal{A})$ 中的 $\Hom$ 集, 必要时将以下标区别范畴.

\begin{definition}[$\Hom$ 复形]\label{def:Hom-cplx}
	\index[sym1]{Hombullet@$\Hom^{\bullet}$}
	设 $X, Y$ 为 $\cate{C}(\mathcal{A})$ 的对象, 对每个 $n \in \Z$ 定义
	\[ \Hom^n\left(X, Y \right) := \prod_{k \in \Z} \Hom_{\mathcal{A}}\left( X^k, Y^{k+n} \right); \]
	态射的逐项合成给出
	\begin{equation}\label{eqn:Hom-cplx-multiplication}\begin{aligned}
		\Hom^n \left(Y, Z \right) \times \Hom^m\left( X, Y \right) & \longrightarrow \Hom^{n+m}\left(X , Z \right) \\
		(g, f) & \longmapsto gf := \left( g^{k+m} f^k \right)_{k \in \Z}.
	\end{aligned}\end{equation}
	注意到 $d_X = (d_X^k)_k \in \Hom^1\left(X, X\right)$, 对 $d_Y$ 亦同, 按此定义
	\begin{align*}
		d^n = d_{\Hom^\bullet(X, Y)}^n: \Hom^n\left(X, Y \right) & \longrightarrow \Hom^{n+1}\left(X, Y \right) \\
		f & \longmapsto d_Y f - (-1)^n f d_X.
	\end{align*}
\end{definition}

运算 \eqref{eqn:Hom-cplx-multiplication} 显然满足结合律和对加法的分配律. 给定 $\cate{C}(\mathcal{A})$ 中的态射 $u: \underline{X} \to X$ 和 $v: Y \to \underline{Y}$, 分别视同 $\Hom^0(\underline{X}, X)$ 和 $\Hom^0(Y, \underline{Y})$ 的元素, 则对每个 $n \in \Z$ 都有同态
\begin{equation}\label{eqn:Hom-cplx-functoriality}\begin{aligned}
		\Hom^n(u, v): \Hom^n\left( X, Y\right) & \longrightarrow \Hom^n\left(\underline{X}, \underline{Y}\right) \\
		f & \longmapsto vfu.
\end{aligned}\end{equation}

\begin{definition-proposition}
	\index{Hom fuxing@$\Hom$ 复形 ($\Hom$ complex)}
	以上资料 $\left( \Hom^n(X, Y), d^n \right)_{n \in \Z}$ 给出 $\cate{C}(\cate{Ab})$ 的对象, 称为 \emph{$\Hom$ 复形}. 当 $X$ 和 $Y$ 变动, 此构造连同 \eqref{eqn:Hom-cplx-functoriality} 给出加性函子
	\[ \Hom^\bullet: \cate{C}(\mathcal{A})^{\opp} \times \cate{C}(\mathcal{A}) \to \cate{C}(\cate{Ab}). \]
\end{definition-proposition}
\begin{proof}
	先验证 $d^{n+1} d^n = 0$. 因为 $d_Y^2 = 0 = d_X^2$, 直接计算给出
	\begin{align*}
		d^{n+1} \left( d^n f \right) & = d_Y \left( d_Y f - (-1)^n f d_X \right) - (-1)^{n+1} \left( d_Y f - (-1)^n f d_X \right) d_X \\
		& = (-1)^{n+1} d_Y f d_X - (-1)^{n+1} d_Y f d_X = 0.
	\end{align*}
	
	其次是函子性. 将 \eqref{eqn:Hom-cplx-functoriality} 定义的 $\Hom^\bullet(u, v)$ 编入图表
	\[\begin{tikzcd}
		\Hom^n\left( X^\bullet, Y^\bullet \right) \arrow[d, "{d^n}"'] \arrow[r, "{\Hom^{n+1}(u, v)}" inner sep=0.7em] & \Hom^n\left( \underline{X}^\bullet, \underline{Y}^\bullet \right) \arrow[d, "{\underline{d}^n}"] \\
		\Hom^{n+1}\left( X^\bullet, Y^\bullet \right) \arrow[r, "{\Hom^{n+1}(u, v)}"' inner sep=0.7em] & \Hom^{n+1}\left( \underline{X}^\bullet, \underline{Y}^\bullet \right) ;
	\end{tikzcd}\]
	由于 $d_X u = u d_{\underline{X}}$, $v d_Y = d_{\underline{Y}} v$, 易见图表交换.
\end{proof}

此外, 取 $d_{\Hom^\bullet(X, Y)}$ 和平移相容. 证明比陈述容易得多.

\begin{lemma}\label{prop:Hom-cplx-d-translate}
	设 $n, m \in \Z$ 而 $f \in \Hom^n(X, Y)$. 将 $f$ 等同于 $\Hom^{n-m}(X, Y[m])$ 的元素 $\underline{f} = (f^k)_k$, 或 $\Hom^{n-m}(X[-m], Y)$ 的元素 $\overline{f} = (f^{k-m})_k$, 则
	\begin{align*}
		d^{n-m}_{\Hom^\bullet(X, Y[m])} \underline{f} & \xlongequal{\text{等同于}} (-1)^m d^n_{\Hom^\bullet(X, Y)} f, \\ d^{n-m}_{\Hom^\bullet(X[-m], Y)} \overline{f} & \xlongequal{\text{等同于}} d^n_{\Hom^\bullet(X, Y)} f.
	\end{align*}
	这导致 $\Hom^\bullet(X, Y[m]) = \Hom^\bullet(X, Y)[m] \simeq \Hom^\bullet(X[-m], Y)$.
\end{lemma}
\begin{proof}
	以第一式为例,
	\[ (d^{n-m}_{\Hom^\bullet(X, Y[m])} \underline{f})^k = d_{Y[m]}^{k+n-m} \underline{f}^k - (-1)^{n-m} \underline{f}^{k+1} d_X^k : X^k \to Y^{k+n+1}, \]
	这也等于 $(-1)^m \left( d_Y^{k+n} f^k - (-1)^n f^{k+1} d_X^k\right)$. 关于 $\overline{f}$ 的论证类似: 虽然 $\overline{f}$ 的情形缺少符号 $(-1)^m$, 依然有 $\Hom$ 复形的同构, 详见定义 \ref{def:translation-functor} 的脚注.
\end{proof}

定义于 \eqref{eqn:Hom-cplx-multiplication} 的乘法运算还满足 Leibniz 律.

\begin{lemma}\label{prop:Hom-cplx-Leibniz}
	设 $X, Y, Z$ 为复形, $(g, f) \in \Hom^n \left(Y, Z\right) \times \Hom^m\left( X, Y\right)$. 相对于 \eqref{eqn:Hom-cplx-multiplication} 之乘法, 以下等式在 $\Hom^{n+m+1}(X, Z)$ 中成立
	\[ d^{n+m}(gf) = (d^n g) f + (-1)^n g (d^m f). \]
\end{lemma}
\begin{proof}
	直接计算, 敬请读者练习.
\end{proof}

\begin{lemma}\label{prop:Hom-as-cocycle}
	设 $n \in \Z$ 而 $f \in \Hom^n\left( X, Y \right)$, 则 $d^n f = 0$ 等价于 $f \in \Hom\left(X, Y[n] \right)$.
\end{lemma}
\begin{proof}
	通过引理 \ref{prop:Hom-cplx-d-translate} 将 $f$ 视同 $\Hom^0(X, Y[n])$ 的元素, 从而将问题化约到 $n=0$ 情形. 易见 $d^0 f = 0 \iff d_Y f = f d_X$.
\end{proof}

\begin{definition}\label{def:homotopy}
	\index{tonglun@同伦 (homotopy)}
	\index{linglun@零伦 (null-homotopic)}
	设 $n \in \Z$. 将 $\Hom(X, Y[n])$ 视同 $\Hom^n(X, Y)$ 的子集.
	\begin{compactitem}
		\item 若 $f, g \in \Hom(X, Y[n])$ 而 $g - f = d^{n-1} h$, 则称 $h$ 是从 $f$ 到 $g$ 的\emph{同伦}.
		\item 对于 $f, g$ 如上, 若存在从 $f$ 到 $g$ 的同伦 $g$, 则称 $f$ 和 $g$ 同伦. 与零态射同伦的 $f$ 称为\emph{零伦}的.
	\end{compactitem}

	作为 $n=0$ 时的特例, $f \in \Hom(X, Y)$ 零伦当且仅当存在一族态射 $h^m: X^m \to Y^{m-1}$ 使得
	\[ \forall m \in \Z, \quad f^m = d_Y^{m-1} h^m + h^{m+1} d_X^m . \]
\end{definition}

同伦是等价关系. 引理 \ref{prop:Hom-as-cocycle} 在 $\cate{Ab}$ (或 $\Bbbk\dcate{Mod}$) 中给出关键等式
\[ \Hm^n\left(\Hom^\bullet(X, Y), d^\bullet \right) = \Hom\left(X, Y[n]\right) \big/ \{\text{零伦态射} \}. \]

下一则结果说明零伦态射不但对加法封闭, 还构成理想.

\begin{lemma}\label{prop:null-homotopic-composition}
	设 $X \xrightarrow{f} Y[m] $和 $Y \xrightarrow{g} Z[n]$ 为复形之间的态射. 若 $f$ 或 $g$ 零伦, 则 $g[m] \circ f: X \to Z[n+m]$ 零伦.
\end{lemma}
\begin{proof}
	以下省略 $d$ 的上标. 设 $f = d h$, 其中 $h \in \Hom^{-1}(X, Y[m])$. 将合成放在 $\Hom^\bullet$ 中理解, 不妨将 $g[m] \circ f$ 简写成 $gf$. 引理 \ref{prop:Hom-cplx-Leibniz} 蕴涵 $d(gh) = (dg)h + (-1)^n g(d h) = (-1)^n gf$, 故 $gf$ 零伦. 至于 $g = dh$ 情形, 论证无异.
\end{proof}

省略符号 $d_{\Hom^\bullet}$, 引理 \ref{prop:Hom-cplx-Leibniz} 和 \ref{prop:null-homotopic-composition} 说明 $\Hom^\bullet$ 上的乘法 \eqref{eqn:Hom-cplx-multiplication} 诱导
\begin{equation}\label{eqn:homotopy-cat-cmposition}
	\Hm^n\left(\Hom^\bullet(Y, Z) \right) \times \Hm^m\left(\Hom^\bullet(X, Y) \right) \to \Hm^{n+m}\left( \Hom^\bullet(X, Z) \right) ,
\end{equation}
二元运算 \eqref{eqn:homotopy-cat-cmposition} 从 \eqref{eqn:Hom-cplx-multiplication} 继承结合律和对加法的分配律.

\begin{definition}\label{def:cplx-homotopy-cat}
	\index[sym1]{KA@$\cate{K}(\mathcal{A})$}
	从 $\cate{C}(\mathcal{A})$ 定义范畴 $\cate{K}(\mathcal{A})$, 使得
	\begin{align*}
		\Obj\left( \cate{K}(\mathcal{A}) \right) & := \Obj\left( \cate{C}(\mathcal{A}) \right), \\
		\Hom_{\cate{K}(\mathcal{A})}\left( X, Y \right) & := \Hm^0\left( \Hom^\bullet\left( X, Y \right), d^\bullet_{\Hom} \right),
	\end{align*}
	态射的合成由 \eqref{eqn:homotopy-cat-cmposition} 确定, 取 $n=m=0$.
\end{definition}

观察到 $\cate{K}(\mathcal{A})$ 是加性范畴: 态射集的加法运算显然, 零对象来自 $\cate{C}(\mathcal{A})$ 的零对象, 而积 (或余积) 的存在性来自以下更一般的性质: 存在自然同构
\begin{align*}
	\Hom_{\cate{K}(\mathcal{A})}\left( X, \prod_{i \in I} Y_i \right) & \simeq \prod_{i \in I} \Hom_{\cate{K}(\mathcal{A})}(X, Y_i) \\
	\text{或}\quad \Hom_{\cate{K}(\mathcal{A})}\left( \coprod_{i \in I} X_i, Y \right) & \simeq \prod_{i \in I} \Hom_{\cate{K}(\mathcal{A})}(X_i, Y),
\end{align*}
其中 $I$ 是任意集, 前提是 $\prod_{i \in I} Y_i$ (或 $\coprod_i X_i$) 存在于 $\cate{C}(\mathcal{A})$ 层次. 由于在 $\cate{C}(\cate{Ab})$ 中取直积和取 $\Hm^0$ 可交换, 一切归结为 $\Hom$ 复形的显然同构
\begin{align*}
	\Hom^\bullet\left(X, \prod_i Y_i\right) & \simeq \prod_i \Hom^\bullet(X, Y_i) \\
	\text{或}\quad \Hom^\bullet\left(\coprod_i X_i, Y\right) & \simeq \prod_i \Hom^\bullet(X_i, Y).
\end{align*}

平移函子 $[n]$ 诱导加性范畴 $\cate{K}(\mathcal{A})$ 的自同构, 仍记为 $[n]$. 加性函子 $\cate{C}(\mathcal{A}) \to \cate{K}(\mathcal{A})$ 在对象集上定为恒等映射, 在态射集上定为商映射. 下述泛性质是明白的.

\begin{proposition}\label{prop:homotopy-category-universal-property}
	若 $\mathcal{B}$ 为 $\cate{Ab}$-范畴, 而加性函子 $F: \cate{C}(\mathcal{A}) \to \mathcal{B}$ 映零伦态射映为零态射, 则 $F$ 唯一地通过 $\cate{C}(\mathcal{A}) \to \cate{K}(\mathcal{A})$ 分解.
\end{proposition}

下一章的习题将说明 $\cate{K}(\mathcal{A})$ 通常不是 Abel 范畴.

\begin{remark}\index{Hom fuxing}\index[sym1]{Homlbullet@$\Hom_\bullet$}
	\label{rem:Hom-chain-cplx}
	对于链复形 (注记 \ref{rem:cochain-vs-chain}) 也有相应的 $\Hom_\bullet$:
	\begin{gather*}
		\Hom_n(X, Y) := \prod_{k \in \Z} \Hom\left(X_k, Y_{k+n} \right), \\
		d_n f = d_Y f - (-1)^n f d_X, \quad f \in \Hom_n(X, Y).
	\end{gather*}
	事实上, 按注记 \ref{rem:cochain-vs-chain} 介绍的手法取 $X^n = X_{-n}$, $d^n = d_{-n}$ 等等, 则 $\left( \Hom_n(X, Y), d_n \right)_n$ 正是对应到 $\left( \Hom^n(X, Y), d^n \right)_n$ 的链复形:
	\begin{align*}
		\Hom_n\left(X, Y \right) & = \prod_{k \in \Z} \Hom\left( X_k, Y_{n+k} \right) \xlongequal{h = -k} \prod_{h \in \Z} \Hom\left(X^h, Y^{h-n} \right) \\
		& = \Hom^{-n} \left(X, Y\right),
	\end{align*}
	而 $d_n$ 和 $d^{-n}$ 的关联类此. 当然, 对于链复形也有 $\cate{K}(\mathcal{A})$ 的相应版本.
\end{remark}

给定加性范畴之间的加性函子 $F: \mathcal{A} \to \mathcal{B}$, 对于任意 $X, Y \in \Obj(\cate{C}(\mathcal{A}))$, 逐项取 $F$ 给出 $\cate{C}(\cate{Ab})$ 的态射
\begin{align*}
	\Hom^\bullet\left(X, Y\right) & \to \Hom^\bullet\left(\cate{C}F(X), \cate{C}F(Y) \right) \\
	(f^n)_{n \in \Z} & \mapsto (F(f^n))_{n \in \Z} ,
\end{align*}
两边同取 $\Hm^0$, 得到 $\Hom_{\cate{K}(\mathcal{A})}(X, Y) \to \Hom_{\cate{K}(\mathcal{B})}(\cate{C}F(X), \cate{C}F(Y))$. 如是得到函子 $\cate{K}F: \cate{K}(\mathcal{A}) \to \cate{K}(\mathcal{B})$ 使得下图交换
\begin{equation}\label{eqn:KF}\begin{tikzcd}
	\cate{C}(\mathcal{A}) \arrow[r, "{\cate{C}F}"] \arrow[d] & \cate{C}(\mathcal{B}) \arrow[d] \\
	\cate{K}(\mathcal{A}) \arrow[r, "{\cate{K}F}"'] & \cate{K}(\mathcal{B}),
\end{tikzcd}\end{equation}
其第一行如命题 \ref{prop:CA-functoriality}. 这些构造和伴随对是兼容的.
\index[sym1]{KF@$\cate{K}F$}

\begin{proposition}\label{prop:KF-adjoint}
	考虑伴随对 $(F, G, \varphi)$, 其中
	$\begin{tikzcd}
		F: \mathcal{A} \arrow[shift left, r] & \mathcal{B} \arrow[shift left, l] : G
	\end{tikzcd}$ 是加性范畴之间的加性函子, 则
	\[\begin{tikzcd}
		\cate{K}F: \cate{K}(\mathcal{A}) \arrow[shift left, r] & \cate{K}(\mathcal{B}) \arrow[shift left, l] : \cate{K}G
	\end{tikzcd}\]
	也自然地成为伴随对.
\end{proposition}
\begin{proof}
	伴随对的资料 $\varphi$ 是一族双射 $\varphi_{A, B}: \Hom_{\mathcal{B}}(FA, B) \simeq \Hom_{\mathcal{A}}(A, GB)$, 对 $A$ 和 $B$ 具函子性. 命题 \ref{prop:automatic-k-morphism} 确保它们是线性的, 由此诱导复形的典范同构
	\[ \Hom^\bullet\left( \cate{C}F(X), Y \right) \simeq \Hom^\bullet\left( X, \cate{C}G(Y) \right), \]
	映 $(f^n)_{n \in \Z}$ 为 $(\varphi_{A, B}(f^n))_{n \in \Z}$. 鉴于 \eqref{eqn:KF}, 两边同取 $\Hm^0$ 便使 $\cate{K}F$ 成为 $\cate{K}G$ 的左伴随.
\end{proof}

\section{映射锥}\label{sec:mapping-cone}
本节仍取定加性范畴 $\mathcal{A}$ 和相应的复形范畴 $\cate{C}(\mathcal{A})$.

\begin{definition}\label{def:Cone}
	\index{yingshezhui@映射锥 (mapping cone)}
	\index[sym1]{Conef@$\Cone(f)$}
	给定 $\cate{C}(\mathcal{A})$ 中的态射 $f: X \to Y$, 其\emph{映射锥} $\Cone(f)$ 定为以下复形
	\begin{align*}
		\Cone(f)^n & := X^{n+1} \oplus Y^n, \\
		d^n_{\Cone(f)} & := \;\text{矩阵表法} \;
		\begin{tikzpicture}[baseline]
			\matrix (M) [matrix of math nodes, ampersand replacement=\&, left delimiter=(, right delimiter=)] {
					-d_X^{n+1} \& 0 \\
					f^{n+1} \& d_Y^n \\
				};
				\node[above=1.2em] at (M-1-1) {\scriptsize $X^{n+1}$};
				\node[above=1.2em] at (M-1-2) {\scriptsize $Y^n$};
				\node[left=2.7em] at (M-1-1) {\scriptsize $X^{n+2}$};
				\node[left=2.7em] at (M-2-1) {\scriptsize $Y^{n+1}$};
		\end{tikzpicture} \\
		& : X^{n+1} \oplus Y^n \to X^{n+2} \oplus Y^{n+1} ,
	\end{align*}
	其中 $n \in \Z$. 由于 $d_{X[1]}^n = -d_X^{n+1}$, 另有简练记法
	\[ \Cone(f) := \left( X[1] \oplus Y , \; \begin{pmatrix} d_{X[1]} & 0 \\ f[1] & d_Y \end{pmatrix} \right) . \]
\end{definition}

易证 $\Cone(f)$ 仍是复形, 这是以下矩阵计算的结论:
\[ \begin{pmatrix} -d_X^{n+1} & 0 \\ f^{n+1} & d_Y^n \end{pmatrix} \begin{pmatrix} -d_X^n & 0 \\ f^n & d_Y^{n-1} \end{pmatrix} =
\begin{pmatrix} d_X^{n+1} d_X^n & 0 \\ - f^{n+1} d_X^n + d_Y^n f^n & d_Y^n d_Y^{n-1} \end{pmatrix}. \]

\begin{example}
	设 $f: X \to Y$ 是 $\mathcal{A}$ 中的态射. 将 $X, Y$ 视同 $\cate{C}(\mathcal{A})$ 的对象 (集中于 $0$ 次项), 则 $\Cone(f)$ 无非是复形 $[X \xrightarrow{f} Y]$ (次数为 $-1, 0$, 其他项全为 $0$).
\end{example}

以下两则函子性按定义是自明的.
\begin{proposition}\label{prop:cone-functorial}
	给定 $\cate{C}(\mathcal{A})$ 中的交换图表
	\[\begin{tikzcd}
		X \arrow[d, "f"'] \arrow[r, "\varphi"] & X' \arrow[d, "{f'}"] \\
		Y \arrow[r, "\psi"'] & Y'
	\end{tikzcd}\]
	则 $\bigl(\begin{smallmatrix} \varphi[1] & 0 \\ 0 & \psi \end{smallmatrix}\bigr)$ 给出态射 $\Cone(f) \to \Cone(f')$.
\end{proposition}

\begin{proposition}\label{prop:Cone-A}
	设 $F: \mathcal{A} \to \mathcal{A}'$ 为加性函子, $f: X \to Y$ 为 $\cate{C}(\mathcal{A})$ 中的态射. 相应的函子 $\cate{C}F: \cate{C}(\mathcal{A}) \to \cate{C}(\mathcal{A}')$ 在 $\cate{C}(\mathcal{A}')$ 中满足典范同构
	\[ (\cate{C}F)(\Cone(f)) \simeq \Cone(\cate{C}F(f)). \]
\end{proposition}

映射锥可以置入以下的``三角'', 这是今后一切理论的基础.
\begin{definition}\label{def:cone-triangle}
	\index[sym1]{alphafbetaf@$\alpha(f), \beta(f)$}
	考虑 $\cate{C}(\mathcal{A})$ 中的态射 $f: X \to Y$. 对之可在 $\cate{C}(\mathcal{A})$ 中构造典范态射
	\[ Y \xrightarrow{\alpha(f)} \Cone(f) \xrightarrow{\beta(f)} X[1] , \]
	具体以矩阵表达如下:
	\begin{align*}
		\alpha(f)^n & := \text{嵌入}\; \begin{pmatrix} 0 \\ \identity_{Y^n} \end{pmatrix}: Y^n \to X[1]^n \oplus Y^n , \\
		\beta(f)^n & := \text{投影}\; \begin{pmatrix} \identity_{X[1]^n} & 0 \end{pmatrix}: X[1]^n \oplus Y^n \to X[1]^n .
	\end{align*}
	易见它们的确给出 $\cate{C}(\mathcal{A})$ 中的态射.
\end{definition}

以下收集关于映射锥的几条同伦性质. 首先我们说明映射锥可以设想为态射的``同伦余核''或``同伦核''. 这是同伦论的基本思想在复形层次的体现.

\begin{proposition}\label{prop:homotopy-kernel-cokernel}
	选定 $\cate{C}(\mathcal{A})$ 中的态射 $f: X \to Y$. 存在典范双射
	\[\begin{tikzcd}[row sep=small]
		\Hom\left( \Cone(f), T \right) \arrow[leftrightarrow, r, "1:1"] & \left\{ (u, h): u \in \Hom(Y,T), \; h: \text{从 $uf$ 到 $0$ 的同伦} \right\}, \\
		\Hom\left( T, \Cone(f)[-1] \right) \arrow[leftrightarrow, r, "1:1"] & \left\{ (v, k): v \in \Hom(T,X), \; k: \text{从 $fv$ 到 $0$ 的同伦} \right\},
	\end{tikzcd}\]
	其中 $T$ 为 $\cate{C}(\mathcal{A})$ 的任意对象, $\Hom := \Hom_{\cate{C}(\mathcal{A})}$. 更具体地说,
	\begin{compactitem}
		\item $\tilde{u}: \Cone(f) \to T$ 的像 $(u, h)$ 满足 $u = \tilde{u} \circ \alpha(f)$,
		\item $\tilde{v}: T \to \Cone(f)[-1]$ 的像 $(v, k)$ 满足 $v = \beta(f)[-1] \circ \tilde{v}$.
	\end{compactitem}
\end{proposition}
\begin{proof}
	首先处理 $\Hom\left( \Cone(f), T \right)$ 的情形. 态射 $\tilde{u}: \Cone(f) \to T$ 相当于一族 $\mathcal{A}$ 中的态射 $h^n: X^{n+1} \to T^n$ 和 $u^n: Y^n \to T^n$, 其中 $n \in \Z$, 所需条件写成矩阵等式
	\[\begin{pmatrix}
		h^{n+1} & u^{n+1}
	\end{pmatrix} \begin{pmatrix}
		-d_X^{n+1} & 0 \\
		f^{n+1} & d_Y^n
	\end{pmatrix} = d_T^n \begin{pmatrix}
		h^n & u^n
	\end{pmatrix}. \]
	具体展开, 可见它相当于说 $u = (u^n)_n: Y \to T$ 是 $\cate{C}(\mathcal{A})$ 中的态射, 而 $h := (h^{n-1})_n$ 满足 $d_{\Hom^\bullet(X, T)}^{-1} h = uf$. 此即所求的双射; 根据 $\alpha(f)$ 的定义, $u = \tilde{u} \circ \alpha(f)$ 是自明的.
	
	其次, 考虑态射 $\tilde{v}: T \to \Cone(f)[-1]$. 这相当于 $\mathcal{A}$ 的态射族 $v^n: T^n \to X^n$ 和 $\underline{k}^n: T^n \to Y^{n-1}$, 所需条件写作
	\[\begin{pmatrix}
		d_X^n & 0 \\
		-f^n & -d_Y^{n-1}
	\end{pmatrix} \begin{pmatrix}
		v^n \\ \underline{k}^n
	\end{pmatrix} = \begin{pmatrix}
		v^{n+1} \\ \underline{k}^{n+1}
	\end{pmatrix} d_T^n
	\quad (n \in \Z). \]
	它相当于说 $v = (v^n)_n: T \to X$ 是 $\cate{C}(\mathcal{A})$ 中的态射, 而 $k := (-\underline{k}^n)_n$ 满足 $d_{\Hom^\bullet(T, Y)}^{-1} k = fv$. 根据 $\beta(f)$ 的定义, $v = \beta(f)[-1] \circ \tilde{v}$ 亦属显然.
\end{proof}

\begin{remark}[同伦余核, 同伦核]\label{rem:homotopy-kernel-cokernel}
	\index{tonglunyuhe@同伦余核, 同伦核 (homotopy cokernel, homotopy kernel)}
	由命题 \ref{prop:homotopy-kernel-cokernel} 推得: 态射 $u: Y \to T$ (或 $v: T \to X$) 满足 $uf$ (或 $fv$) 零伦当且仅当它通过 $\alpha(f): Y \to \Cone(f)$ (或 $\beta(f)[-1]: \Cone(f)[-1] \to X$) 分解. 这自然让人联想到余核 (或核) 的泛性质, 差别在于:
	\begin{compactitem}
		\item 此处以 $uf$ (或 $fv$) 零伦 来替代精确等式 $= 0$;
		\item 命题给出的分解 $\tilde{u}$ (或 $\tilde{v}$) 不仅坐实了 $uf$ (或 $fv$) 零伦这一事实, 还包含了它如何同伦于 $0$, 这是更高一阶的资料.
	\end{compactitem}
	这些性质无法简单地在 $\cate{C}(\mathcal{A})$ 或 $\cate{K}(\mathcal{A})$ 中按照初等范畴论的概念来处理; 实际上 $\cate{K}(\mathcal{A})$ 中鲜少有核或余核. 基于此, $Y \to \Cone(f)$ 又称 $f$ 的\emph{同伦余核}, 而 $\beta(f)[-1]: \Cone(f)[-1] \to X$ 又称 $f$ 的\emph{同伦核}. 耐人寻味的是映射锥 $\Cone(f)$ 及其平移在此意义下兼具``余核''以及``核''两种角色.
\end{remark}

称 $\cate{C}(\mathcal{A})$ 中的态射 $f$ 典范地同伦于 $g$, 如果存在典范的 $h$ 使得 $g -f = d^{-1} h$; 若 $f$ 典范地同伦于 $0$, 则称它典范地零伦.

\begin{proposition}\label{prop:cone-homotopy}
	选定加性范畴 $\mathcal{A}$.
	\begin{enumerate}[(i)]
		\item 若 $f: X \to Y$ 是 $\cate{C}(\mathcal{A})$ 中的同构, 则 $\identity_{\Cone(f)}$ 典范地零伦.
		\item 对于 $\cate{C}(\mathcal{A})$ 中的任意态射 $f: X \to Y$, 定义 \ref{def:cone-triangle} 的态射满足于
		\[ \beta(f) \circ \alpha(f) = 0, \]
		而 $\alpha(f) \circ f$ 和 $f[1] \circ \beta(f)$ 皆典范地零伦.
	\end{enumerate}
\end{proposition}
\begin{proof}
	首先考虑 (i). 基于映射锥的函子性 (命题 \ref{prop:cone-functorial}), 不妨设 $f = \identity_X$. 定义
	\[ s^n := \begin{pmatrix} 0 & \identity_{X^n} \\ 0 & 0 \end{pmatrix} : X^{n+1} \oplus X^n \to X^n \oplus X^{n-1}, \quad n \in \Z. \]
	直接计算给出
	\begin{multline*}
		d^{n-1}_{\Cone(\identity_X)} s^n + s^{n+1} d^n_{\Cone(\identity_X)} = \\
		\begin{pmatrix} -d_X^n & 0 \\ \identity_{X^n} & d_X^{n-1} \end{pmatrix} \begin{pmatrix} 0 & \identity_{X^n} \\ 0 & 0 \end{pmatrix}  + \begin{pmatrix} 0 & \identity_{X^{n+1}} \\ 0 & 0 \end{pmatrix} \begin{pmatrix} -d_X^{n+1} & 0 \\ \identity_{X^{n+1}} & d_X^n \end{pmatrix}
		= \identity_{X^{n+1} \oplus X^n},
	\end{multline*}
	故 $s = (s^n)_{n \in \Z}$ 使 $\identity_{\Cone(\identity_X)}$ 零伦.
	
	接着考虑 (ii). 易见 $\beta(f) \circ \alpha(f) = 0$. 剩下的同伦来自命题 \ref{prop:homotopy-kernel-cokernel}: 取 $T = \Cone(f)$, 则 $\tilde{u} := \identity_{\Cone(f)} \in \Hom(\Cone(f), T)$ 使 $\alpha(f) \circ f$ 零伦; 取 $T = \Cone(f)[-1]$, 则 $\tilde{v} := \identity_{\Cone(f)[-1]} \in \Hom(T, \Cone(f)[-1])$ 使 $f \circ \beta(f)[-1]$ 零伦, 也使 $f[1] \circ \beta(f)$ 零伦.
\end{proof}

定义 \ref{def:cone-triangle} 自 $f$ 引出两个新态射 $\alpha(f)$ 和 $\beta(f)$. 精确到同伦, 它们的映射锥并不产生新对象; 且从 $\Cone(\alpha(f))$ 入手来说明这点. 首先观察到
\[ \Cone(\alpha(f))^n = Y^{n+1} \oplus \Cone(f)^n = Y^{n+1} \oplus X^{n+1} \oplus Y^n. \]
按矩阵写法,
\begin{align*}
	\alpha(\alpha(f)) & = \begin{pmatrix}
		0 & 0 \\
		\identity_{X[1]} & 0 \\
		0 & \identity_Y
	\end{pmatrix} : \Cone(f) \to \Cone(\alpha(f)) , \\
	\beta(\alpha(f)) & = \begin{pmatrix}
		\identity_{Y[1]} & 0 & 0
	\end{pmatrix} : \Cone(\alpha(f)) \to Y[1].
\end{align*}

\begin{lemma}\label{prop:cone-alpha}
	对于 $\cate{C}(\mathcal{A})$ 中的态射 $f: X \to Y$, 用矩阵写法可定义一对态射
	\[ \begin{pmatrix} -f[1] \\ \identity_{X[1]} \\ 0 \end{pmatrix} : \begin{tikzcd} X[1] \arrow[r, shift left, "\phi"] & \Cone(\alpha(f)) \arrow[l, shift left, "\psi"] \end{tikzcd} : \begin{pmatrix} 0 & \identity_{X[1]} & 0 \end{pmatrix}. \]
	它们满足 $\psi \circ \phi = \identity_{X[1]}$, 使下图在 $\cate{K}(\mathcal{A})$ 中交换
	\[\begin{tikzcd}
		\Cone(f) \arrow[d, "{\identity_{\Cone(f)}}"'] \arrow[r, "\beta(f)"] & X[1] \arrow[d, "\phi"] \arrow[r, "{-f[1]}"] & Y[1] \arrow[d, "{\identity_{Y[1]}}"] \\
		\Cone(f) \arrow[r, "{\alpha(\alpha(f))}"'] & \Cone(\alpha(f)) \arrow[r, "\beta(\alpha(f))"'] & Y[1]
	\end{tikzcd}\]
	而且 $\phi, \psi$ 在 $\cate{K}(\mathcal{A})$ 中互逆.
\end{lemma}
\begin{proof}
	根据之前对 $\Cone(\alpha(f))$ 的描述, $d_{\Cone(\alpha(f))}^n$ 按矩阵写法表作
	\[ \begin{pmatrix} -d_Y^{n+1} & 0 & 0 \\ 0 & -d_X^{n+1} & 0 \\ \identity_{Y^{n+1}} & f^{n+1} & d_Y^n \end{pmatrix} : Y^{n+1} \oplus X^{n+1} \oplus Y^n \to Y^{n+2} \oplus X^{n+2} \oplus Y^{n+1}. \]
	问题化为在 $\cate{C}(\mathcal{A})$ 中验证以下断言:
	\begin{itemize}
		\item $\phi := (\phi^n)_{n \in \Z}$ 和 $\psi := (\psi^n)_{n \in \Z}$ 都是复形之间的态射;
		\item $\psi \circ \phi = \identity_{X[1]}$;
		\item $\psi \circ \alpha(\alpha(f)) = \beta(f)$;
		\item $\beta(\alpha(f)) \circ \phi = -f[1]$;
		\item 存在 $s = (s^n)_{n \in \Z} \in \Hom^{-1}\left(\Cone(\alpha(f)), \Cone(\alpha(f))\right)$ 使得 $\identity_{\Cone(\alpha(f))} - \phi \circ \psi = d^{-1}_{\Hom^\bullet} (s)$.
	\end{itemize}
	前四条都是初等的. 以矩阵写法取
	\[ s^n := \begin{pmatrix} 0 & 0 & \identity_{Y^n} \\ 0 & 0 & 0 \\ 0 & 0 & 0 \end{pmatrix}: \Cone(\alpha(f))^n \to \Cone(\alpha(f))^{n-1} \]
	则可验证最后一条断言.
\end{proof}

至于 $\beta(f)$ 的情形, 我们稍事修改, 考虑 $-\beta(f)[-1]: \Cone(f)[-1] \to X$ 的映射锥.
\begin{definition}\label{def:Cyl}
	\index{yingshezhu@映射柱 (mapping cylinder)}
	\index[sym1]{Cylf@$\Cyl(f)$}
	对于 $\cate{C}(\mathcal{A})$ 中的态射 $f: X \to Y$, 其\emph{映射柱}定为
	\[ \Cyl(f) := \Cone\left( \Cone(f)[-1] \xrightarrow{-\beta(f)[-1]} X \right). \]
	它带有典范态射 $X \to \Cyl(f) \to \Cone(f)$.
\end{definition}

映射柱有和映射锥相同形式的函子性 (命题 \ref{prop:cone-functorial}). 鉴于注记 \ref{rem:homotopy-kernel-cokernel}, $\Cyl(f)$ 可与加性范畴中的余像相比拟 --- 它近乎``核的余核'' (命题 \ref{prop:Im-Coim-additive}), 因此也可以视作 $f$ 的同伦余像.

具体地说, $\Cyl(f)^n = \Cone(f)^n \oplus X^n = X^{n+1} \oplus Y^n \oplus X^n$, 态射 $X \to \mathrm{Cyl}(f)$ (或 $\mathrm{Cyl}(f) \to \Cone(f)$) 是向第三个直和项的嵌入 (或向前两个直和项的投影), 而
\[ d_{\Cyl(f)}^n =
	\begin{pmatrix} d_{\Cone(f)}^n & 0 \\ -\beta(f)^n & d_X^n \end{pmatrix}
	= \begin{pmatrix}
		-d_X^{n+1} & 0 & 0 \\
		f^{n+1} & d_Y^n & 0 \\
		-\identity_{X^{n+1}} & 0 & d_X^n
\end{pmatrix} .\]

\begin{lemma}\label{prop:cone-beta}
	对于 $\cate{C}(\mathcal{A})$ 中的态射 $f: X \to Y$, 可定义一对态射
	\[ \begin{pmatrix} 0 \\ \identity_Y \\ 0 \end{pmatrix} : \begin{tikzcd} Y \arrow[r, shift left, "\phi"] & \Cyl(f) \arrow[l, shift left, "\psi"] \end{tikzcd} : \begin{pmatrix} 0 & \identity_Y & f \end{pmatrix}. \]
	它们满足 $\psi \circ \phi = \identity_Y$, 使下图在 $\cate{K}(\mathcal{A})$ 中交换
	\[\begin{tikzcd}
		X \arrow[d, "{\identity_X}"'] \arrow[r, "f"] & Y \arrow[d, "\phi"] \arrow[r, "{\alpha(f)}"] & \Cone(f) \arrow[d, "{\identity_{\Cone(f)}}"] \\
		X \arrow[r] & \Cyl(f) \arrow[r] & \Cone(f)
	\end{tikzcd}\]
	而且 $\phi, \psi$ 在 $\cate{K}(\mathcal{A})$ 中互逆. 进一步, $f$ 分解为 $X \to \Cyl(f) \xrightarrow{\psi} Y$ 的合成.
\end{lemma}
\begin{proof}
	请读者直接按定义验证 $\phi$ 和 $\psi$ 确实给出复形的态射. 只要确立这点, 则 $\psi \circ \phi = \identity_Y$ 一望可知. 接着证 $\phi$ 和 $\psi$ 在 $\cate{K}(\mathcal{A})$ 中互逆: 类似于引理 \ref{prop:cone-alpha}, 取
	\begin{align*}
		s^n & := \begin{pmatrix} 0 & 0 & -\identity_{X^n} \\ 0 & 0 & 0 \\ 0 & 0 & 0 \end{pmatrix}: X^{n+1} \oplus Y^n \oplus X^n \to X^n \oplus Y^{n-1} \oplus X^{n-1}, \\
		s & := (s^n)_{n \in \Z},
	\end{align*}
	并且验证 $\identity_{\Cyl(f)} - \phi \circ \psi = d^{-1}_{\Hom^\bullet} (s)$ 即可.
	
	图表右侧方块在 $\cate{C}(\mathcal{A})$ 中已经交换. 至于左侧方块, 简单观察到 $X \to \mathrm{Cyl}(f) \xrightarrow{\psi} Y$ 合成为 $f$ 便是.
\end{proof}

引理 \ref{prop:cone-alpha} 连同引理 \ref{prop:cone-beta} 表明: 精确到 $\cate{K}(\mathcal{A})$ 中的同构, 任何态射 $f: X \to Y$ 皆能被一个简单得多的投影 $\Cone(\alpha(f))[-1] \to Y$ (或嵌入 $X \to \Cyl(f)$) 来替换, 而且此构造对 $f$ 具函子性. 这在同调代数或同伦论中是一个重要思想.

映射柱在 $f = \identity_X$ 的特例另有妙用, 它可以用来诠释同伦. 对于熟悉同调论的读者, 一切自有拓扑诠释, 但此处只论其代数版本.

\begin{proposition}\label{prop:cylinder-homotopy}
	\index[sym1]{CylX@$\Cyl_X$}
	设 $X$ 为 $\cate{C}(\mathcal{A})$ 的对象, 命 $\mathrm{Cyl}_X := \mathrm{Cyl}(\identity_X)$. 注意到 $\mathrm{Cyl}_X^n = X^{n+1} \oplus X^n \oplus X^n$.
	\begin{enumerate}[(i)]
		\item 在 $\cate{C}(\mathcal{A})$ 中有态射
		$\begin{tikzcd}
			X \arrow[shift left, r, "i_0"] \arrow[shift right, r, "i_1"'] & \Cyl_X \arrow[r, "j"] & X
		\end{tikzcd}$,
		以矩阵记法对每个 $n \in \Z$ 定义为
		\begin{equation*}
			i_0^n := \begin{pmatrix} 0 \\ 0 \\ \identity_{X^n} \end{pmatrix} , \quad
			i_1^n := \begin{pmatrix} 0 \\ \identity_{X^n} \\ 0 \end{pmatrix} , \quad j^n := \begin{pmatrix} 0 & \identity_{X^n} & \identity_{X^n} \end{pmatrix},
		\end{equation*}
		它们满足 $j i_0 = \identity_X = j i_1$, 而且 $j$ 在 $\cate{K}(\mathcal{A})$ 中的像是同构.
		\item 对 $\cate{C}(\mathcal{A})$ 的任意对象 $Y$, 我们有双射
		\begin{align*}
			\left\{\begin{array}{r|l}
				(f, g, h) & f, g \in \Hom_{\cate{C}(\mathcal{A})}(X, Y) \\
				& h \in \Hom^{-1}(X, Y) \\
				& g - f = d^{-1}_{\Hom^\bullet(X, Y)} h
			\end{array}\right\} & \xrightarrow{1:1} \Hom_{\cate{C}(\mathcal{A})}\left( \mathrm{Cyl}_X, Y \right) \\
			(f, g, h) & \longmapsto \tilde{h} = (\tilde{h}^n)_{n \in \Z}, \; \tilde{h}^n := \begin{pmatrix} h^n & g^n & f^n \end{pmatrix}.
		\end{align*}
		特别地, $f = \tilde{h} i_0$ 而 $g = \tilde{h} i_1$.
	\end{enumerate}
\end{proposition}
\begin{proof}
	对于 (i), 注意到 $i_0$ 无非是定义 \ref{def:Cyl} 中的典范态射 $X \to \mathrm{Cyl}(\identity_X)$, 另一方面 $i_1$ 则是引理 \ref{prop:cone-beta} 中的态射 $\phi: X \to \mathrm{Cyl}(\identity_X)$; 引理 \ref{prop:cone-beta} 的交换图表蕴涵两者在 $\cate{K}(\mathcal{A})$ 中给出同一个同构.
	
	至于 $j$, 容易验证 $j^{n+1} d_{\Cyl_X}^n = \bigl(0 \; d_X^n \; d_X^n \bigr) = d_X^n j^n$. 因此 $j$ 确实是态射, 而 $j i_0 = \identity_X = j i_1$ 是自明的. 既然 $i_0$ 和 $i_1$ 在 $\cate{K}(\mathcal{A})$ 中是同构, $j$ 亦然.

	对于 (ii), 将 $\Hom^0(\mathrm{Cyl}_X, Y)$ 的任意元素 $\tilde{h}$ 表作 $\left((h^n, g^n, f^n)\right)_{n \in \Z}$, 其中
	\[ h^n: X^{n+1} \to Y^n, \quad f^n: X^n \to Y^n, \quad g^n: X^n \to Y^n \]
	都是 $\mathcal{A}$ 的态射. 省略上标并以矩阵记法来计算
	\begin{align*}
		\tilde{h} \; d_{\mathrm{Cyl}_X} & = \begin{pmatrix} h & g & f \end{pmatrix} \begin{pmatrix} -d_X & 0 & 0 \\ \identity_X & d_X & 0 \\ -\identity_X & 0 & d_X \end{pmatrix} = \begin{pmatrix} -hd_X + g - f & g d_X & f d_X \end{pmatrix} , \\
		d_Y \; \tilde{h} & = \begin{pmatrix} d_Y h & d_Y g & d_Y f \end{pmatrix}.
	\end{align*}
	因此 $\tilde{h}$ 是复形的态射当且仅当 $f, g \in \Hom_{\cate{C}(\mathcal{A})}(X, Y)$ 而 $g-f = h d_X + d_Y h$.
\end{proof}

\begin{remark}\label{rem:cone-chain}
	\index{yingshezhui}\index{yingshezhu}
	映射锥和映射柱当然有同调版本 (见注记 \ref{rem:cochain-vs-chain}): 给定链复形的态射 $f: X \to Y$, 取
	\[\begin{array}{rlrl}
		\Cone(f)_n & :=(X[-1] \oplus Y)_n, &
		\Cyl(f)_n & := (X[-1] \oplus Y \oplus X)_n, \\
		d_{\Cone(f)} & := \begin{pmatrix} d_{X[-1]} & 0 \\ f[-1] & d_Y \end{pmatrix}, &
		d_{\Cyl(f)} & := \begin{pmatrix} d_{X[-1]} & 0 & 0 \\ f[-1] & d_Y & 0 \\ -\identity_{X[-1]} & 0 & d_X \end{pmatrix}.
	\end{array}\]

	本节所有陈述都能移植到链复形的情形, 特别地, 存在典范态射
	\[ X \xrightarrow{f} Y \xrightarrow{\alpha(f)} \Cone(f) \xrightarrow{\beta(f)} X[-1], \quad X \to \Cyl(f) \to \Cone(f). \]
\end{remark}

\section{相反范畴上的复形}\label{sec:opposite-cplx}
选定加性范畴 $\mathcal{A}$. 本节旨在沟通 $\mathcal{A}$ 和 $\mathcal{A}^{\opp}$ 上的复形. 由于涉及 $\mathcal{A}^{\opp}$ 的函子经常出现, 譬如 $\Hom$ 函子, 这一工序尽管简单却是必要的, 而其中涉及的一些正负号也需要适度的留意. 建议初学的读者略过本节, 或者先大致地浏览.

以下内容分成三个面向: 复形, 同伦, 映射锥. 尔后将探讨的导出范畴版本 (命题 \ref{prop:derived-cat-op}) 是这些结果的直接应用. 回忆到若 $f: X \to Y$ 是 $\mathcal{A}$ 中的态射, 则 $f^{\opp}$ 代表 $\mathcal{A}^{\opp}$ 中对应的态射 $Y \to X$.

\begin{definition-proposition}\label{def:sigma}
	\index[sym1]{sigma@$\sigma$}
	加性范畴的同构 $\sigma: \cate{C}(\mathcal{A}^{\opp}) \to \cate{C}(\mathcal{A})^{\opp}$ 定义如下: 对于 $\cate{C}(\mathcal{A}^{\opp})$ 的对象 $X$, 在 $\mathcal{A}$ 中定义
	\[ (\sigma X)^n := X^{-n}, \quad d_{\sigma X}^n = (-1)^{n+1} \left[ d_X^{-n-1, \opp}: (\sigma X)^n \to (\sigma X)^{n+1} \right], \quad n \in \Z. \]
	对于 $\cate{C}(\mathcal{A}^{\opp})$ 的态射 $f = \left( f^n: X^n \to Y^n \right)_n$, 其像 $\sigma f$ 取作 $\cate{C}(\mathcal{A})$ 的态射
	\[ (\sigma f)^n := \left(f^{-n} \right)^{\opp} : (\sigma Y)^n \to (\sigma X)^n, \]
	亦即 $\cate{C}(\mathcal{A})^{\opp}$ 的态射 $\sigma X \to \sigma Y$. 对所有 $m \in \Z$, 存在自然同构
	\[ s_m: \sigma \circ [m] \rightiso [-m] \circ \sigma, \]
	左式的 $[-m]$ 视同从 $\cate{C}(\mathcal{A})^{\opp}$ 到自身的函子 (严格写法应是 $[-m]^{\opp}$).
\end{definition-proposition}
\begin{proof}
	将 $s_m$ 逐步化到 $m=1$ 情形. 定义 $s_1 = (s_{1, X})_X$ 如下. 对每个 $n$, 取
	\begin{equation}\label{eqn:sigma-s}\begin{tikzcd}[row sep=tiny]
		s_{1,X}^n : \sigma(X{[1]})^n = X^{1-n} \arrow[r, "\sim"', "{(-1)^{n-1}}" inner sep=0.6em] & X^{1-n} = (\sigma X){[-1]}^n .
	\end{tikzcd}\end{equation}
	一切归结为 $s_{1, X}^{n+1} d^n_{\sigma(X[1])} = -d_{X[1]}^{-n-1, \opp} = d_X^{-n, \opp} = (-1)^n d_{\sigma X}^{n-1} = d^n_{(\sigma X)[-1]} s_{1,X}^n$.
\end{proof}

由于 $n(n+1) \equiv 0 \pmod{2}$, 函子 $\sigma$ 操作两次返回自身, 故它确实是范畴之间的同构.

\begin{remark}\label{rem:sigma-vs-cohomology}
	当 $\mathcal{A}$ 是 Abel 范畴时, $d_{\sigma X}^\bullet$ 带的正负号并不改变定义 \ref{def:cohomology} 介绍的上同调; 因此 $\Hm^{-n}(\sigma X) \in \Obj(\mathcal{A})$ 对应到 $\Hm^n(X) \in \Obj(\mathcal{A}^{\opp})$.
\end{remark}

接着探讨 $\sigma$ 和同伦的关系.

\begin{proposition}\label{prop:sigma-homotopy}
	以上定义的 $\sigma$ 诱导加性范畴的等价 $\cate{K}(\mathcal{A}^{\opp}) \rightiso \cate{K}(\mathcal{A})^{\opp}$.
\end{proposition}
\begin{proof}
	选定 $\cate{C}(\mathcal{A}^{\opp})$ 的对象 $X, Y$. 对于 $f = (f^k)_k \in \Hom^{-1}_{\cate{C}(\mathcal{A}^{\opp})}(X, Y)$, 用以下词典在 $\mathcal{A}$ 中定义对应的 $\sigma f \in \Hom^{-1}_{\cate{C}(\mathcal{A})}(\sigma Y, \sigma X)$:
	\[\begin{array}{|c|c|c|c|} \hline
		\text{范畴} & \multicolumn{3}{c|}{\text{态射}} \\ \hline
		\mathcal{A}^{\opp} & X^k \xrightarrow{(-1)^{k+1} f^k} Y^{k-1} & d_Y^{k-1} f^k + f^{k+1} d_X^k & (d^{-1}f)^k \\
		\mathcal{A} & (\sigma Y)^{-k+1} \xrightarrow{(\sigma f)^{-k+1}} (\sigma X)^{-k} & (\sigma f)^{-k+1} d_{\sigma Y}^{-k} + d_{\sigma X}^{-k-1} (\sigma f)^{-k} & d^{-1} (\sigma f)^{-k} \\ \hline
	\end{array}\]	
	这便足以说明 $\Hom_{\cate{K}(\mathcal{A}^{\opp})}(X, Y) \rightiso \Hom_{\cate{K}(\mathcal{A})}(\sigma Y, \sigma X) = \Hom_{\cate{K}(\mathcal{A})^{\opp}}(\sigma X, \sigma Y)$.
\end{proof}

最后, 我们通过之前构造的 $\sigma$ 和同构族 $(s_m)_m$ 来比较 $\cate{C}(\mathcal{A})$ 和 $\cate{C}(\mathcal{A}^{\opp})$ 的映射锥. 这部分的细节是比较琐碎的.

\begin{proposition}\label{prop:sigma-triangulated}
	\index[sym1]{sigma}
	对于 $\cate{C}(\mathcal{A}^{\opp})$ 中的任意态射 $f: X \to Y$, 在 $\cate{C}(\mathcal{A})$ 中\footnote{此处的 $\alpha(\cdot)$ 和 $\beta(\cdot)$ 都是相对于 $\cate{C}(\mathcal{A})$ 来定义的. 若在 $\cate{C}(\mathcal{A})^{\opp}$ 中考量, 则图表的箭头须倒转.}有交换图表
	\[\begin{tikzcd}[column sep=large]
		\sigma\left( X{[1]} \right) \arrow[r, "{\sigma(\beta(f))}"] \arrow[d, "{s_{1,X}}"'] & \sigma\left(\Cone(f)\right) \arrow[r, "{\sigma(\alpha(f))}"] \arrow[d, "\theta"] & \sigma Y \arrow[r, "\sigma(f)"] \arrow[d, "\identity"] & \sigma X \arrow[d, "{\identity}"] \\
		(\sigma X){[-1]} \arrow[r, "{\alpha(\sigma(f))[-1]}"' inner sep=0.6em] & \Cone(\sigma(f)){[-1]} \arrow[r, "{\beta(\sigma(f))[-1]}"' inner sep=0.6em] & \sigma Y \arrow[r, "{\sigma(f)}"' inner sep=0.6em] & \sigma X
	\end{tikzcd}\]
	其中 $\theta$ 是一个典范同构, 而 $s_{1,X}$ 是定义--命题 \ref{def:sigma} 给出的同构.
\end{proposition}
\begin{proof}
	倘若将 $\Cone(f)$ 换作 $X[1] \oplus Y$, 断言则是容易的, 所求同构取 $(s_{1, X}, \identity_{\sigma Y})$ 即可. 这里的麻烦在于 $d_{\Cone(f)}$ 的矩阵有非对角项 $f[1]$. 尽管如此, 我们还是循相同方法, 以 \eqref{eqn:sigma-s} 来对每个 $n \in \Z$ 定义同构
	\begin{multline*}
		\theta^n: \sigma\left(\Cone(f)\right)^n = \sigma\left(X[1]\right)^n \oplus (\sigma Y)^n \\
		\xrightarrow[\sim]{(s_{1,X}, \identity)^n = ((-1)^{n-1} \identity, \identity) } (\sigma X)[-1]^n \oplus (\sigma Y)^n \xlongequal{\text{换位}} \left( (\sigma Y) \oplus (\sigma X)[-1] \right)^n ,
	\end{multline*}	
	而 $\mathcal{A}$ 的态射 $d_{\sigma(\Cone(f))}^n$ 按此对应到
	\[ \begin{pmatrix}
		d_{\sigma Y}^n & 0 \\
		* & d_{(\sigma X)[-1]}^n
	\end{pmatrix} : \left( (\sigma Y) \oplus (\sigma X)[-1] \right)^n \to \left( (\sigma Y) \oplus (\sigma X)[-1] \right)^{n+1} . \]
	如证明开头所述, 对角项不成问题, 重点在于确定矩阵的左下角元素. 基于定义--命题 \ref{def:sigma}, 它等于 $(-1)^{n+1}$ 乘以 $\mathcal{A}$ 中的合成态射
	\[\begin{tikzcd}[row sep=tiny]
		(\sigma Y)^n \arrow[r] & (\sigma(X[1]))^{n+1} \arrow[r, "\sim"] & (\sigma X)[-1]^{n+1} \\
		Y^{-n} \arrow[equal, u] \arrow[r, "{(f^{-n})^{\opp}}"'] & X^{-n} \arrow[equal, u] \arrow[r, "{s_{1, X}^{n+1} = (-1)^n}"'] & X^{-n} \arrow[equal, u]
	\end{tikzcd}\]
	的产物, 亦即 $-\sigma(f)^n$. 由此可见
	\[ \theta := (\theta^n)_{n \in \Z}: \sigma(\Cone(f)) \rightiso \Cone(\sigma(f))[-1] \]
	确实是 $\cate{C}(\mathcal{A})$ 中的同构. 一旦确立这点, 交换性的验证便没有本质困难.
\end{proof}

毋庸赘言, 本节的结果也适用于链复形, 并且可以推广到 $\mathcal{A}$ 为 $\Bbbk$-线性的情形.

\section{双复形}\label{sec:double-cplx}
本节依然取 $\mathcal{A}$ 为加性范畴. 约略地说, 双复形可以设想为复形的二维版本, 带有纵, 横两个方向的微分态射.

\begin{definition}[双复形]\label{def:double-cplx}
	\index{shuangfuxing@双复形 (double complex)}
	\index[sym1]{dhoridvert@$\dhori$, $\dvert$}
	加性范畴 $\mathcal{A}$ 上的双复形意谓 $\mathcal{A}$ 中的一族对象 $\left( X^{p, q} \right)_{(p, q) \in \Z^2}$, 连同态射 $\dhori^{p, q}: X^{p, q} \to X^{p+1, q}$ 和 $\dvert^{p, q}: X^{p, q} \to X^{p, q+1}$, 满足于
	\[ \dhori^{p+1,q} \dhori^{p,q} = 0, \quad \dvert^{p, q+1} \dvert^{p,q} = 0, \quad \dhori^{p, q+1} \dvert^{p,q} = \dvert^{p+1, q} \dhori^{p,q}. \]
	上述资料照例简记为 $(X^{\bullet, \bullet}, \dhori, \dvert)$, $X^{\bullet, \bullet}$ 或 $X$.
\end{definition}

关于 $\dhori, \dvert$ 的条件可简写为
\[ \dhori^2 = 0, \quad \dvert^2 = 0, \quad \dhori \dvert = \dvert \dhori. \]

\begin{definition}\label{def:bicplx}
	给定加性范畴 $\mathcal{A}$, 从双复形 $X$ 到 $Y$ 的态射意谓一族态射
	\[ f = \left( f^{p,q}: X^{p, q} \to Y^{p, q} \right)_{(p, q) \in \Z^2}, \]
	使得对所有 $p, q$ 都有
	\[ \dhori_Y^{p, q} f^{p, q} = f^{p+1, q} \dhori_X^{p, q}, \quad \dvert_Y^{p, q} f^{p, q} = f^{p, q+1} \dvert_X^{p, q}. \]
	上述关系可以简写为 $\dhori_Y f = f \dhori_X$ 和 $\dvert_Y f = f \dvert_X$.
\end{definition}

一如命题 \ref{prop:additive-cat-cplx} 的情形, 这些定义使 $\mathcal{A}$ 上的所有双复形构成加性范畴 $\cate{C}^2(\mathcal{A})$.
\index[sym1]{C2A@$\cate{C}^2(\mathcal{A})$}

双复形 $X$ 形象地表作
\[\begin{tikzcd}
	& \vdots & \vdots & \\
	\cdots \arrow[r] & X^{p, q+1} \arrow[r] \arrow[u] & X^{p+1, q+1} \arrow[r] \arrow[u] & \cdots \\
	\cdots \arrow[r] & X^{p, q} \arrow[r, "{\dhori^{p,q}}"'] \arrow[u, "{\dvert^{p,q}}"] & X^{p+1, q} \arrow[r] \arrow[u] & \cdots \\
	& \vdots \arrow[u] & \vdots \arrow[u] &
\end{tikzcd}\]
每行 $\left( X^{\bullet, q}, d^{\bullet, q} \right)$ 和每列 $\left( X^{p, \bullet}, d^{p, \bullet} \right)$ 都是复形. 因此双复形可以按列或按行收纳.

\begin{definition}\label{def:bicplx-F1}
	\index[sym1]{F1F2@$F_{\mathrm{I}}, F_{\mathrm{II}}$}
	加性函子 $F_{\mathrm{I}}: \cate{C}^2(\mathcal{A}) \to \cate{C}(\cate{C}(\mathcal{A}))$ 定义如下: 对于双复形 $X$, 取 $(F_{\mathrm{I}} X)^p = X^{p, \bullet}$ 而 $d_{F_{\mathrm{I}} X}^p = \dhori^{p, \bullet}: (F_{\mathrm{I}} X)^p \to (F_{\mathrm{I}} X)^{p+1}$. 类似地, 按 $(F_{\mathrm{II}} X)^q = X^{\bullet, q}$ 和 $d_{F_{\mathrm{II}} X}^q = \dvert^{\bullet, q}$ 定义加性函子 $\cate{C}^2(\mathcal{A}) \to \cate{C}(\cate{C}(\mathcal{A}))$. 此处 $p, q \in \Z$.
\end{definition}

定义 \ref{def:bicplx} 蕴涵 $F_{\mathrm{I}}$, $F_{\mathrm{II}}$ 都是加性范畴之间的同构. 示意如下:
\begin{equation}\label{eqn:bicplx-diagram}\begin{tikzcd}[
		/tikz/execute at end picture={
			\node[rectangle, draw, fit=(NW) (SE), inner xsep=0.5em, inner ysep=0em] {};
		}]
		\vdots & |[alias=NW]| & {} & \\
		(F_{\mathrm{II}} X)^q \arrow[u, "{d^q_{F_{\mathrm{II}} X}}"] \arrow[phantom, r, ":=" description, sloped] & \cdots \arrow[r] & X^{p,q} \arrow[r, "{\dhori^{p,q}}"'] \arrow[u, "{\dvert^{p,q}}"] & \cdots \\
		\vdots \arrow[u] & & {} \arrow[u] & |[alias=SE]| \\
		& \cdots \arrow[r] & (F_{\mathrm{I}} X)^p \arrow[r, "{d^p_{F_{\mathrm{I}} X}}"] \arrow[phantom, u, ":=" description, sloped] & \cdots
\end{tikzcd}\end{equation}

\begin{definition}[全复形]\label{def:total-cplx}
	\index{quanfuxing@全复形 (total complex)}
	\index[sym1]{tot@$\tot_{\oplus}$, $\tot_{\Pi}$, $\tot$}
	设 $\mathcal{A}$ 具有可数余积, 以直和符号 $\bigoplus$ 标记, 而 $X$ 是 $\cate{C}^2(\mathcal{A})$ 的对象. 定义 $\mathcal{A}$ 上的复形 $\tot_{\oplus}(X)$ 如下:
	\begin{compactitem}
		\item $\tot_{\oplus}(X)^n := \bigoplus_{p+q=n} X^{p, q}$,
		\item $d^n: \tot_{\oplus}(X)^n \to \tot_{\oplus}(X)^{n+1}$ 拉回到 $X^{p,q}$ 等于 $\dhori^{p,q} + (-1)^p \dvert^{p,q}$.
	\end{compactitem}
\end{definition}

省略上标, 按定义写出 $d^2: X^{p, q} \to X^{p+2, q} \oplus X^{p+1, q+1} \oplus X^{p, q+2}$ 可得
\[ d^2 = \left( \dhori^2, \; (-1)^p \dhori \dvert + (-1)^{p+1} \dvert \dhori , \; \dvert^2 \right) = (0, 0, 0). \]

在函子 $\tot_{\oplus}$ 的构造中以积 $\prod$ 代余积, 可以类似地得到 $\tot_{\Pi} X \in \Obj(\cate{C}(\mathcal{A}))$, 前提是所论的积存在. 此处 $d^n: \tot_{\Pi}(X)^n \to \tot_{\Pi}(X)^{n+1}$ 的定义和 $\tot_{\oplus}$ 情形神似: 我们要求 $d^{n-1}$ 投影到 $X^{p,q}$ 等于
\[ \tot_{\Pi}^{n-1}(X) \xrightarrow{\text{投影}} X^{p-1, q} \oplus X^{p,q-1} \xrightarrow{(\dhori^{p-1, q}, (-1)^p \dvert^{p, q-1})} X^{p, q} \]
的合成; 同理可证 $d^2 = 0$. 不致混淆时, $\tot_{\oplus} X$ 和 $\tot_{\Pi} X$ 都被称为 $X$ 的\emph{全复形}.

以下性质应当是自明的.

\begin{proposition}
	在相应的条件下, $\tot_{\oplus}$ 和 $\tot_{\Pi}$ 都是从 $\cate{C}^2(\mathcal{A})$ 到 $\cate{C}(\mathcal{A})$ 的加性函子.
\end{proposition}

在全复形的定义中, 上标 $p$ 和 $q$ 乍看并不对称. 为了消除这个错觉, 以下来定义 $\cate{C}^2(\mathcal{A})$ 的加性自同构 $\mathrm{swap}$, 使得 $\mathrm{swap}(X)^{p,q} = X^{q, p}$, 并且互换 $\dhori$ 和 $\dvert$. 即将与之搭配的还有符号 $(-1)^{pq}$, 它和 \cite[定义--定理 7.4.4]{Li1} 的 Koszul 符号律本质上是同一套机制.

\index[sym1]{swap@$\mathrm{swap}$}
\begin{proposition}\label{prop:double-cplx-swap}
	对每个 $(p,q) \in \Z^2$ 和 $\cate{C}^2(\mathcal{A})$ 的对象 $X$, 定义一族同构
	\[ r^{p, q}_X := (-1)^{pq} \identity_{X^{p,q}} : X^{p, q} \to \mathrm{swap}(X)^{q,p} . \]
	\begin{compactitem}
		\item 设 $\mathcal{A}$ 有可数余积, 则 $(r_X^{p,q})_{(p,q) \in \Z^2}$ 诱导同构 $r_X: \tot_{\oplus}(X) \rightiso \tot_{\oplus}(\mathrm{swap}(X))$.
		\item 设 $\mathcal{A}$ 有可数积, 则 $(r_X^{p,q})_{(p,q) \in \Z^2}$ 诱导同构 $r_X: \tot_{\Pi}(X) \rightiso \tot_{\Pi}(\mathrm{swap}(X))$.
	\end{compactitem}
\end{proposition}
\begin{proof}
	对所有 $(p,q) \in \Z^2$, 图表
	\[\begin{tikzcd}[row sep=large]
		X^{p,q} \arrow[d, "(-1)^{pq} \identity"'] \arrow[r, "{ (\dhori, (-1)^p \dvert) }" inner sep=0.8em] & X^{p+1, q} \oplus X^{p, q+1} \arrow[d, "{((-1)^{(p+1)q} \identity, (-1)^{p(q+1)} \identity) }"] \\
		X^{p,q} \arrow[r, "{((-1)^q \dhori, \dvert)}"' inner sep=0.8em] & X^{p+1, q} \oplus X^{p, q+1}
	\end{tikzcd}\]
	交换, 故 $r_X$ 确为复形之间的态射.
\end{proof}

在关于 $\tot_{\oplus} X$ (或 $\tot_{\Pi} X$) 的定义中, 可数余积 (或积) 的存在条件可以适当弱化: 如果对所有 $n$, 至多仅有有限个 $(p, q)$ 满足 $p+q=n$ 而 $X^{p,q} \neq 0$, 则全复形定义中的余积 (或积) 化为有限直和. 这时可将两种全复形统一记为 $\tot(X)$. 定义 \ref{def:Cf-Supp} 之后将有进一步的讨论.

\begin{remark}\label{rem:k-fold-cplx}
	准此要领, 对任何 $k \in \Z_{\geq 1}$ 皆可定义 $k$ 重复形为资料
	\[ \left( X^{p_1, \ldots, p_k} \in \Obj(\mathcal{A}) \right)_{p_1, \ldots, p_k \in \Z}, \quad {}^1 d, \ldots, {}^k d, \]
	其中 ${}^i d^{p_1, \ldots, p_k}: X^{p_1, \ldots, p_k} \to X^{p_1, \ldots, p_i + 1, \ldots, p_k}$, 而
	\[ {}^i d^2 = 0, \quad {}^i d {}^j d = {}^j d {}^i d \quad \text{(省略上标)}. \]
	全体 $k$ 重复形构成加性范畴 $\cate{C}^k(\mathcal{A})$. 按定义 \ref{def:bicplx-F1} 的方式还可以定义一族加性函子 $F_i: \cate{C}^k(\mathcal{A}) \to \cate{C}(\cate{C}^{k-1}(\mathcal{A}))$ 使得每个 $F_i$ 都是范畴等价 ($k \geq 2$ 而 $i = 1, \ldots, k$).
	
	在此情形下, 假设 $\mathcal{A}$ 有可数余积或积, 则可依样画葫芦地定义全复形函子
	\[ \tot_{\oplus} \; \text{或} \; \tot_{\Pi}: \cate{C}^k(\mathcal{A}) \to \cate{C}(\mathcal{A}). \]
	细节与双复形的情形无异, 毋须赘述.
\end{remark}

定义从 $\cate{C}^2(\mathcal{A})$ 到自身的同构 $[m]_{\mathrm{I}} := F_{\mathrm{I}}^{-1} \circ [m] \circ F_{\mathrm{I}}$ 和 $[m]_{\mathrm{II}} := F_{\mathrm{II}}^{-1} \circ [m] \circ F_{\mathrm{II}}$, 分别对应到双复形的横向和纵向平移. 对所有 $a, b \in \Z$ 都有 $[a]_{\mathrm{I}} [b]_{\mathrm{II}} = [b]_{\mathrm{II}} [a]_{\mathrm{I}}$. 现在来研究这些函子对全复形的影响, 以及它们和自同构 $\mathrm{swap}$ 的关系.

% Reference: KS06, p.290
\begin{proposition}\label{prop:tot-shift}
	对所有 $X \in \Obj(\cate{C}^2(\mathcal{A}))$ 和 $(p, q) \in \Z^2$, 记标准嵌入 $X^{p, q} \hookrightarrow \mathrm{tot}_{\oplus}(X)^{p+q}$ 为 $\iota_{p, q}$. 对所有 $n \in \Z$, 定义
	\begin{itemize}
		\item $\theta_X^n: \tot_{\oplus} \left( X[1]_{\mathrm{I}} \right)^n \to \tot_{\oplus} \left( X \right)[1]^n$, 使得它拉回到 $X[1]_{\mathrm{I}}^{p, q} = X^{p+1, q}$ 上等于 $\iota_{p+1,q}$;
		\item $(\theta'_X)^n: \tot_{\oplus} \left( X[1]_{\mathrm{II}} \right)^n \to \tot_{\oplus} \left( X \right)[1]^n$, 使得它拉回到 $X[1]_{\mathrm{II}}^{p, q} = X^{p, q+1}$ 上等于 $(-1)^p \iota_{p,q+1}$.
	\end{itemize}
	则它们给出 $\cate{C}(\mathcal{A})$ 中的典范同构
	\begin{align*}
		\theta_X & = (\theta_X^n)_n: \tot_{\oplus} \left( X[1]_{\mathrm{I}} \right) \rightiso \tot_{\oplus} \left(X\right)[1], \\
		\theta'_X & = ((\theta'_X)^n)_n: \tot_{\oplus} \left( X[1]_{\mathrm{II}} \right) \rightiso \tot_{\oplus} \left(X\right)[1],
	\end{align*}
	连同反交换图表 (亦即: 两路合成差一个负号)
	\[\begin{tikzcd}
		\tot_{\oplus} \left( X[1]_{\mathrm{I}}[1]_{\mathrm{II}}\right) \arrow[r, "{\theta_{X[1]_{\mathrm{II}}}}" inner sep=0.6em] \arrow[d, "{\theta'_{X[1]_{\mathrm{I}}}}"'] & \tot_{\oplus}\left(X[1]_{\mathrm{II}}\right)[1] \arrow[d, "{\theta'_X [1]}"] \\
		\tot_{\oplus} \left( X[1]_{\mathrm{I}}\right)[1] \arrow[r, "{\theta_X [1]}"' inner sep=0.6em] & \tot_{\oplus}(X)[2]
	\end{tikzcd}\]
	和交换图表
	\[\begin{tikzcd}[column sep=large]
		\tot_{\oplus} \left( X[1]_{\mathrm{I}}[1]_{\mathrm{II}} \right) \arrow[r, "\text{先 $\theta$ 后 $\theta'$}"] \arrow[d, "{r_{X[1]_{\mathrm{I}}[1]_{\mathrm{II}}}}"'] & \tot_{\oplus}(X)[2] \arrow[d, "{r_X[2]}"] \\
		\tot_{\oplus}\left( \mathrm{swap}(X)[1]_{\mathrm{I}}[1]_{\mathrm{II}} \right) \arrow[r, "\text{先 $\theta'$ 后 $\theta$}"'] & \tot_{\oplus}\left(\mathrm{swap}(X)\right)[2],
	\end{tikzcd}\]
	其中 $r$ 的定义如命题 \ref{prop:double-cplx-swap}. 若以 $\tot_{\Pi}$ 代 $\tot_{\oplus}$, 仍然有同样的 $\theta_X$, $\theta'_X$ 连同相应的反交换图表.
\end{proposition}
\begin{proof}
	直接按全复形的定义 \ref{def:total-cplx} 和平移函子的定义 \ref{def:translation-functor} 来验证 $\theta_X$ 和 $\theta'_X$ 是复形的态射, 细节繁而不难. 反交换图表则归结为如下观察: 图表
	\[\begin{tikzcd}
		X^{p+1, q+1} \arrow[d, "{(-1)^p \identity}"'] \arrow[r, "\identity"] & X^{p+1, q+1} \arrow[d, "{(-1)^{p+1} \identity}"] \\
		X^{p+1, q+1} \arrow[r, "\identity"'] & X^{p+1, q+1}
	\end{tikzcd}\]
	反交换.
	
	现在考虑关于交换图表的断言. 选定 $(p, q) \in \Z^2$, 将两路合成拉回到 $(X[1]_{\mathrm{I}}[1]_{\mathrm{II}})^{p, q}$ 作比较. 先前已说明先 $\theta$ 后 $\theta'$ (对 $X$) 和先 $\theta'$ 后 $\theta$ (对 $\mathrm{swap}(X)$) 分别给出
	\[ (-1)^{p+1}, \; (-1)^q \; \in \Aut(X^{p+1, q+1}) \quad \text{(省略 $\identity$)}. \]
	另一方面, $r_{X[1]_{\mathrm{I}}[1]_{\mathrm{II}}}$ 和 $r_X[2]$ 拉回 $X^{p+1, q+1}$ 分别是 $(-1)^{pq}$ 和 $(-1)^{(p+1)(q+1)}$. 然而 $(p+1)(q+1) - pq \equiv (p+1) - q \pmod{2}$. 明所欲证.
\end{proof}

作为上述反交换图表的简单推广, 请读者证明对于所有 $a, b \in \Z$, 按两种方式定义的
\[ \tot_{\oplus}(X[a]_{\mathrm{I}}[b]_{\mathrm{II}}) \rightrightarrows \tot_{\oplus}(X)[a+b] \]
(先 $\theta$ 后 $\theta'$ 和先 $\theta'$ 后 $\theta$) 相差 $(-1)^{ab}$; 另一方面, 上述交换图表对 $X[a]_{\mathrm{I}} [b]_{\mathrm{II}}$ 仍然成立. 以 $\tot_{\Pi}$ 代 $\tot_{\oplus}$ 亦同.

双复形的主要应用场景之一是关于双函子的研究. 何谓双函子?

\begin{convention}\label{con:bifunctor}
	\index{shuanghanzi@双函子 (bifunctor)}
	形如 $F: \mathcal{A}_1 \times \mathcal{A}_2 \to \mathcal{B}$ 的函子称为\emph{双函子}. 一旦选定对象 $X_i \in \Obj(\mathcal{A}_i)$, 便得到单变元函子 $F(X_1, \cdot): \mathcal{A}_2 \to \mathcal{B}$ 和 $F(\cdot, X_2): \mathcal{A}_1 \to \mathcal{B}$. 由此可以谈论 $F$ 对各个变元的加性, 正合性等诸多概念.
\end{convention}

\begin{definition-proposition}\label{def:bifunctor-cplx}
	\index[sym1]{C2F@$\cate{C}^2 F, \cate{C}_{\oplus} F, \cate{C}_{\Pi} F$}
	设 $\mathcal{A}_1$, $\mathcal{A}_2$, $\mathcal{B}$ 为加性范畴, 而且双函子 $F: \mathcal{A}_1 \times \mathcal{A}_2 \to \mathcal{B}$ 对每个变元都是加性的, 此时有相应的函子
	\[\begin{tikzcd}[row sep=tiny, column sep=small]
		\cate{C}^2 F: \cate{C}(\mathcal{A}_1) \times \cate{C}(\mathcal{A}_2) \arrow[r] & \cate{C}^2(\mathcal{B}) \\
		(X, Y) \arrow[mapsto, r] & (F(X, Y))^{p, q} := F(X^p, Y^q), \\
		& \dhori^{p, q} = F\left( d_X^p, \identity \right), \; \dvert^{p, q} = F\left( \identity, d_Y^q\right) .
	\end{tikzcd}\]

	因此, 在 $\mathcal{B}$ 有可数余积或可数积的前提下, 可分别定义
	\begin{gather*}
		\cate{C}_{\oplus} F := \tot_{\oplus} \circ \cate{C}^2 F: \cate{C}(\mathcal{A}_1) \times \cate{C}(\mathcal{A}_2) \to \cate{C}(\mathcal{B}), \\
		\cate{C}_{\Pi} F := \tot_{\Pi} \circ \cate{C}^2 F: \cate{C}(\mathcal{A}_1) \times \cate{C}(\mathcal{A}_2) \to \cate{C}(\mathcal{B}).
	\end{gather*}
	如果所论的 $\tot_{\oplus}$ 和 $\tot_{\Pi}$ 仅涉及有限直和, 则 $\cate{C}_{\oplus} = \cate{C}_{\Pi}$ 相等.
\end{definition-proposition}
\begin{proof}
	双复形的条件 $\dhori \dvert = \dvert \dhori$ 是双函子定义的直接应用, 其余皆属显然.
\end{proof}

双复形范畴中也有同伦的概念. 设 $f, g: X \to Y$ 是 $\cate{C}^2(\mathcal{A})$ 中的一对态射, 则从 $f$ 到 $g$ 的同伦是指满足以下条件的两族态射 \index{tonglun!双复形版本}
\begin{equation*}\begin{gathered}\begin{tikzcd}
	Y^{p-1, q} & X^{p,q} \arrow[l, "{h^{p,q}}"'] \arrow[d, "{k^{p, q}}"] \\
	& Y^{p, q-1}
\end{tikzcd} \qquad (p,q) \in \Z^2 , \\
	\dvert^{p-1, q} h^{p, q} = h^{p, q+1} \dvert^{p, q}, \qquad \dhori^{p, q-1} k^{p,q} = k^{p+1, q} \dhori^{p,q}, \\
	g^{p,q} - f^{p, q} = \dhori^{p-1, q} h^{p,q} + h^{p+1, q} \dhori^{p,q} + \dvert^{p, q-1} k^{p,q} + k^{p, q+1} \dvert^{p,q}.
\end{gathered}\end{equation*}
这是合理的: 从 $\dvert h = h \dvert$, $\dhori k = k \dhori$ 和双复形的定义, 容易检查 $\dhori h + h \dhori + \dvert k + k \dvert$ 总是给出 $\cate{C}^2(\mathcal{A})$ 的态射.

双复形的同伦反映在全复形上. 详言之, 一旦有 $(h^{p,q}, k^{p,q})_{p,q \in \Z}$, 便可以定义 $\hat{h} \in \Hom^{-1}\left( \tot_{\oplus}(X), \tot_{\oplus}(Y) \right)$, 使 $\hat{h}$ 拉回到直和项 $X^{p,q}$ 上等于
\[ \left( h^{p,q}, (-1)^p k^{p,q} \right): X^{p,q} \to Y^{p-1, q} \oplus Y^{p, q-1}, \]
这将使 $\tot_{\oplus}(g) - \tot_{\oplus}(f) = d^{-1} \hat{h}$; 以相同手法处理 $\tot_{\Pi}$.

综上, 从双复形的同伦关系可以定义 $\cate{K}^2(\mathcal{A})$ 连同函子 $\cate{C}^2(\mathcal{A}) \to \cate{K}^2(\mathcal{A})$, 而在可数余积 (或积) 存在的前提下, 函子 $\tot_{\oplus}$ (或 $\tot_{\Pi}$) 下降为 $\cate{K}^2(\mathcal{A}) \to \cate{K}(\mathcal{A})$. \index[sym1]{K2A@$\cate{K}^2(\mathcal{A})$}

回到加性范畴之间的双函子 $F: \mathcal{A}_1 \times \mathcal{A}_2 \to \mathcal{B}$, 要求它对每个变元都是加性的. 以下是 \eqref{eqn:KF} 的双函子版本.

\begin{proposition}\label{prop:bifunctor-cplx-homotopy}
	\index[sym1]{K2F@$\cate{K}^2 F, \cate{K}_{\oplus} F, \cate{K}_{\Pi} F$}
	取双函子 $F$ 如上, 则 $\cate{C}^2 F$ 分解为 $\cate{K}^2 F: \cate{K}(\mathcal{A}_1) \times \cate{K}(\mathcal{A}_2) \to \cate{K}^2 (\mathcal{B})$.
	
	同理, $\cate{C}_{\oplus} F$ 或 $\cate{C}_{\Pi} F$ 分解为 $\cate{K}(\mathcal{A}_1) \times \cate{K}(\mathcal{A}_2) \to \cate{K}(\mathcal{B})$, 分别记为 $\cate{K}_{\oplus} F$ 或 $\cate{K}_{\Pi} F$, 前提是函子 $\cate{C}_{\oplus} F$ 或 $\cate{C}_{\Pi} F$ 有定义.
\end{proposition}
\begin{proof}
	以 $\cate{C}_{\oplus} F$ 的情形为例, 问题在于对 $\cate{C}(\mathcal{A}_1)$ 的零伦态射 $f_1 = d^{-1} g: X_1 \to Y_1$ 和 $\cate{C}(\mathcal{A}_2)$ 的任意态射 $f_2: X_2 \to Y_2$ 说明对应的 $(\cate{C}^2 F)(f_1, f_2): \cate{C}^2 F(X_1, X_2) \to \cate{C}^2 F(Y_1, Y_2)$ 零伦. 显然的取法是 $h^{p,q} := F(g^p, f_2^q)$ 和 $k^{p,q} := 0$; 因为 $f_2$ 是态射, $h$ 的确与 $\dvert = F(\identity, d)$ 交换. 对第二个变元也类似地处理.
\end{proof}

命题 \ref{prop:tot-shift} 给出典范同构
\begin{gather*}
	\cate{C}_{\oplus} F(X[1], Y) \simeq \cate{C}_{\oplus}F(X, Y)[1] \simeq \cate{C}_{\oplus}(X, Y[1]), \\
	\cate{K}_{\oplus} F(X[1], Y) \simeq \cate{K}_{\oplus}F(X, Y)[1] \simeq \cate{K}_{\oplus}(X, Y[1]).
\end{gather*}
以 $\Pi$ 代 $\oplus$ 亦同.

\begin{example}[$\Hom$ 双复形]\label{eg:Hom-bicplx}
	\index{Hom shuangfuxing@$\Hom$ 双复形 ($\Hom$ double complex)}
	\index[sym1]{Hombb@$\Hom^{\bullet, \bullet}$}
	考虑双函子 $\Hom(\cdot, \cdot): \mathcal{A}^{\opp} \times \mathcal{A} \to \cate{Ab}$, 它映对象 $(S, T)$ 为 $\Hom_{\mathcal{A}}(S, T)$. 以下说明 $\Hom$ 复形 $\Hom^\bullet(X, Y)$ 典范地同构于合成函子
	\[ \cate{C}(\mathcal{A})^{\opp} \times \cate{C}(\mathcal{A}) \xrightarrow{(\sigma^{-1}, \identity)} \cate{C}(\mathcal{A}^{\opp}) \times \cate{C}(\mathcal{A}) \xrightarrow{\cate{C}_{\Pi} \Hom(\cdot, \cdot)} \cate{C}(\cate{Ab}) \]
	在对象 $(X, Y)$ 处的取值, 其中的 $\sigma$ 如定义--命题 \ref{def:sigma}.

	为此, 我们首先考虑从 $\mathcal{A} \times \mathcal{A}^{\opp}$ 到 $\cate{Ab}$ 的双函子 $F(T, S) = \Hom_{\mathcal{A}}(S, T)$, 以及
	\[ \cate{C}(\mathcal{A}) \times \cate{C}(\mathcal{A})^{\opp} \xrightarrow{(\identity, \sigma^{-1})} \cate{C}(\mathcal{A}) \times \cate{C}(\mathcal{A}^{\opp}) \xrightarrow{\cate{C}^2 F} \cate{C}^2(\cate{Ab}). \]
	设 $X, Y \in \Obj(\cate{C}(\mathcal{A}))$. 称对象 $(Y, X)$ 在上述合成函子之下的像 $\Hom^{\bullet, \bullet}(X, Y)$ 为 \emph{$\Hom$ 双复形}. 鉴于显然的交换图表
	\[\begin{tikzcd}[column sep=large]
		\cate{C}(\mathcal{A})^{\opp} \times \cate{C}(\mathcal{A}) \arrow[d, "\simeq"', "\text{对调}"] \arrow[r, "{(\sigma^{-1}, \identity)}" inner sep=0.6em] & \cate{C}(\mathcal{A}^{\opp}) \times \cate{C}(\mathcal{A}) \arrow[r, "{\cate{C}^2 \Hom(\cdot, \cdot)}" inner sep=0.6em] \arrow[d, "\simeq"', "\text{对调}"] & \cate{C}^2(\cate{Ab}) \arrow[d, "\mathrm{swap}"] \\
		\cate{C}(\mathcal{A}) \times \cate{C}(\mathcal{A})^{\opp} \arrow[r, "{(\identity, \sigma^{-1})}"' inner sep=0.6em] & \cate{C}(\mathcal{A}) \times \cate{C}(\mathcal{A}^{\opp}) \arrow[r, "{\cate{C}^2 F}"' inner sep=0.6em] & \cate{C}^2(\cate{Ab})
	\end{tikzcd}\]
	和命题 \ref{prop:double-cplx-swap}, 原问题归结为证 $\tot_{\Pi} \Hom^{\bullet, \bullet}(X, Y)$ 典范地同构于 $\Hom^\bullet(X, Y)$.
	
	细观定义可见
	\begin{equation*}\begin{aligned}
		\Hom^{p, q}(X, Y) & = \Hom_{\mathcal{A}}\left( X^{-q}, Y^p \right), \\
		\dhori^{p, q} & = (d_Y^p)_* : \Hom_{\mathcal{A}}\left( X^{-q}, Y^p \right) \to \Hom_{\mathcal{A}}\left( X^{-q}, Y^{p+1} \right), \\
		\dvert^{p, q} & = (-1)^{q+1} (d_X^{-q-1})^* : \Hom_{\mathcal{A}}\left( X^{-q}, Y^p \right) \to \Hom_{\mathcal{A}}\left( X^{-q-1}, Y^p \right).
	\end{aligned}\end{equation*}
	现在来确定 $\tot_{\Pi} \Hom^{\bullet, \bullet}(X, Y)$. 首先,
	\[ \left(\tot_{\Pi} \Hom^{\bullet, \bullet}(X, Y)\right)^n = \prod_{p+q=n} \Hom_{\mathcal{A}}\left( X^{-q}, Y^p \right) = \prod_{k \in \Z} \Hom\left( X^k, Y^{k+n} \right) \]
	(代入 $k = -q$), 此即 $\Hom^n(X, Y)$. 下一步描述 $d_{\tot_{\Pi} \Hom^{\bullet, \bullet}(X, Y)}^n$: 它在 $\Hom^{n+1}(X, Y)$ 中的第 $(p, q)$ 个坐标 ($p + q = n+1$) 来自
	\[\begin{tikzcd}[column sep=large]
		\Hom^{p-1, q}(X, Y) \arrow[r, "{\dhori^{p-1, q}}"] & \Hom^{p, q}(X, Y) \\
		& \Hom^{p, q-1}(X, Y) \arrow[u, "{(-1)^p \dvert^{p, q-1}}"'] .
	\end{tikzcd}\]
	水平箭头是 $(d_Y^{p-1})_*$, 而垂直箭头是 $(-1)^{p+q} (d_X^{-q})^* = -(-1)^n (d_X^{-q})^*$. 和 $\Hom$ 复形的定义 \ref{def:Hom-cplx} 比较, 立见 $\tot_{\Pi} \Hom^{\bullet, \bullet}(X, Y) = \Hom^\bullet(X, Y)$. 明所欲证.

	最后, 若 $\mathcal{A}$ 是 $\Bbbk$-线性的, 则 $\Hom^{\bullet, \bullet}(X, Y)$ 升级为映至 $\cate{C}^2(\Bbbk\dcate{Mod})$ 的函子.
\end{example}

\section{Abel 范畴上的复形}\label{sec:Abel-cplx}
本节设 $\mathcal{A}$ 为 Abel 范畴. 一如既往, 本节关于加性的陈述都能推广到 $\mathcal{A}$ 为 $\Bbbk$-线性的情形.

对 Abel 范畴上的复形 $X$ 可以探讨上同调 (定义 \ref{def:cohomology}). 回忆到 $[n]$ 不仅平移复形的上标, 还将 $d_X^\bullet$ 乘上符号 $(-1)^n$, 但后者并不改变各个 $d_X$ 的核与像. 由此得到
\[ \Hm^k\left( X[n] \right) = \Hm^{n+k} \left(X \right), \quad k, n \in \Z . \]

命题 \ref{prop:additive-cat-cplx} 已确保 $\cate{C}(\mathcal{A})$ 为加性范畴. 以下说明它还是 Abel 范畴.

\begin{proposition}\label{prop:abelian-cat-cplx}
	范畴 $\cate{C}(\mathcal{A})$ 是 Abel 范畴. 确切地说, 对所有 $X, Y \in \Obj(\cate{C}(\mathcal{A}))$, 我们有 $X \oplus Y = (X^n \oplus Y^n, (d_X^n, d_Y^n))_{n \in \Z}$, 而对于任何态射 $f: X \to Y$, 可取
	\begin{gather*}
		\Ker(f)^n = \left( \Ker(f^n) \right)_{n \in \Z}, \quad \Coker(f)^n = \left( \Coker(f^n) \right)_{n \in \Z}, \\
		\Image(f)^n = \left( \Image(f^n) \right)_{n \in \Z}, \quad \Coim(f)^n = \left( \Coim(f^n) \right)_{n \in \Z}.
	\end{gather*}

	设 $\cate{C}(\mathcal{A})$ 的态射 $f$ 和 $g$ 可合成且 $gf=0$, 则 $H := \Hm\left[ X \xrightarrow{f} Y \xrightarrow{g} Z\right]$ 是以下复形:
	\begin{itemize}
		\item 第 $n$ 次项为 $H^n := \Hm\left[ X^n \xrightarrow{f^n} Y^n \xrightarrow{g^n} Z^n \right]$,
		\item 态射 $d_H^n: H^n \to H^{n+1}$ 由 $d_X^n$, $d_Y^n$ 连同 $\Hm[\cdots]$ 的函子性 (命题 \ref{prop:homology-functoriality}) 确定.
	\end{itemize}
	作为推论, $X \xrightarrow{f} Y \xrightarrow{g} Z$ 正合当且仅当 $X^n \xrightarrow{f^n} Y^n \xrightarrow{g^n} Z^n$ 对每个 $n \in \Z$ 皆正合.
\end{proposition}
\begin{proof}
	命题 \ref{prop:additive-cat-cplx} 业已说明如何以从复形到分次对象的忘却函子 $U: \cate{C}(\mathcal{A}) \to \mathcal{A}^{\Z}$ 将所需的 $\varinjlim$ 和 $\varprojlim$ 逐次地化到 $\mathcal{A}$ 上; 根本在于 $U$ 生 $\varinjlim$ 和 $\varprojlim$ (引理 \ref{prop:limits-cplx}). 这就给出关于 $X \oplus Y$ 和 $\Ker(f)$, $\Coker(f)$ 等等的逐次构造.

	特别地, 根据刻画 \eqref{eqn:strict-morphism}, 典范态射 $\Coim(f) \to \Image(f)$ 由 $\mathcal{A}$ 中的 $\Coim(f^n) \to \Image(f^n)$ 给出. 逐次同构即复形同构. 综上, $\cate{C}(\mathcal{A})$ 的态射皆严格. 故 $\mathcal{A}$ 是 Abel 范畴. 对于 $\Hm[X \to Y \to Z]$ 的描述也是类似处理.
\end{proof}

\begin{remark}
	同理可证双复形范畴 $\cate{C}^2(\mathcal{A})$, 乃至 $k$ 重复形范畴 $\cate{C}^k(\mathcal{A})$ 仍是 Abel 范畴, 其中 $k \in \Z_{\geq 1}$; 见注记 \ref{rem:k-fold-cplx}.
\end{remark}

接着考察 $\cate{C}(\mathcal{A})$ 中的态射 $f: X \to Y$, 命题 \ref{prop:homology-functoriality} 对每个 $n \in \Z$ 唯一地确定 $\Hm^n(f): \Hm^n(X) \to \Hm^n(Y)$, 使得下图交换:
\begin{equation}\label{eqn:induced-morphism-cohomology}\begin{tikzcd}
	\Ker(d_X^n) \arrow[twoheadrightarrow, r] \arrow[d, "\text{由 $f$ 诱导}"'] & \Hm^n(X) \arrow[hookrightarrow, r] \arrow[d, "{\Hm^n(f)}"] & \Coker(d_X^{n-1}) \arrow[d, "\text{由 $f$ 诱导}"] \\
	\Ker(d_Y^n) \arrow[twoheadrightarrow, r] & \Hm^n(Y) \arrow[hookrightarrow, r] & \Coker(d_Y^{n-1}) .
\end{tikzcd}\end{equation}

对给定的态射 $X \xrightarrow{f} Y \xrightarrow{g} Z$, 此刻画即刻给出 $\Hm^n(gf) = \Hm^n(g) \Hm^n(f)$; 此外 $\Hm^n(\identity_X) = \identity_{\Hm^n(X)}$.

\begin{proposition}[上同调作为函子]\label{prop:H-functor}
	\index[sym1]{Hn}
	对所有 $n \in \Z$, 上述定义给出加性函子 $\Hm^n: \cate{C}(\mathcal{A}) \to \mathcal{A}$.
\end{proposition}
\begin{proof}
	性质 $\Hm^n(gf) = \Hm^n(g) \Hm^n(f)$ 和 $\Hm^n(\identity) = \identity$ 直接来自 \eqref{eqn:induced-morphism-cohomology} 的刻画; 同理可得 $\Hm^n(f_1 + f_2) = \Hm^n(f_1) + \Hm^n(f_2)$ 以及 $\Hm^n(tf) = t\Hm^n(f)$, 若 $\mathcal{A}$ 是 $\Bbbk$-线性的而 $t \in \Bbbk$.
\end{proof}

复形之间的短正合列自动诱导上同调的长正合列, 这是长正合列在同调论中最初等的形式.

\begin{proposition}[短正合列诱导长正合列]\label{prop:long-exact-sequence-ses}
	\index{changzhenghelie@长正合列 (long exact sequence)}
	设 $0 \to X \xrightarrow{f} Y \xrightarrow{g} Z \to 0$ 是 $\cate{C}(\mathcal{A})$ 中的短正合列, 则基于定理 \ref{prop:snake-lemma} 可导出 $\mathcal{A}$ 中的典范正合列
	
	\begin{equation*}\begin{tikzcd}[row sep=large]
	\cdots \arrow[r] & \Hm^{n-1}(Y) \arrow[r, "{\Hm^{n-1}(g)}" inner sep=0.7em] \arrow[phantom, d, ""{coordinate, name=A}] & \Hm^{n-1}(Z) \arrow[dll, rounded corners, "{\delta^{n-1}}" description, to path={
		-- ([xshift=2ex]\tikztostart.east)
		|- (A) [near end]\tikztonodes
		-| ([xshift=-2ex]\tikztotarget.west)
		-- (\tikztotarget)}] \\
	\Hm^n(X) \arrow[r, "{\Hm^n(f)}"] & \Hm^n(Y) \arrow[r, "{\Hm^n(g)}"] \arrow[phantom, d, ""{coordinate, name=B}] & \Hm^n(Z) \arrow[dll, rounded corners, "{\delta^n}" description, to path={
		-- ([xshift=2ex]\tikztostart.east)
		|- (B) [near end]\tikztonodes
		-| ([xshift=-2ex]\tikztotarget.west)
		-- (\tikztotarget)}] \\
	\Hm^{n+1}(X) \arrow[r, "{\Hm^{n+1}(f)}"' inner sep=0.7em] & \Hm^{n+1}(Y) \arrow[r] & \cdots
	\end{tikzcd}\end{equation*}
	
	它具备如下的函子性: 若有 $\cate{C}(\mathcal{A})$ 中的行正合交换图表
	\[\begin{tikzcd}[row sep=small]
		0 \arrow[r] & X \arrow[r] \arrow[d] & Y \arrow[r] \arrow[d] & Z \arrow[d] \arrow[r] & 0 \\
		0 \arrow[r] & \underline{X} \arrow[r] & \underline{Y} \arrow[r] & \underline{Z} \arrow[r] & 0
	\end{tikzcd}\]
	则它们给出交换图表
	\[\begin{tikzcd}[column sep=small, row sep=small]
		\cdots \arrow[r] & \Hm^n(X) \arrow[d] \arrow[r] & \Hm^n(Y) \arrow[r] \arrow[d] & \Hm^n(Z) \arrow[r] \arrow[d] & \Hm^{n+1}(X) \arrow[r] \arrow[d] & \cdots \\
		\cdots \arrow[r] & \Hm^n(\underline{X}) \arrow[r] & \Hm^n(\underline{Y}) \arrow[r] & \Hm^n(\underline{Z}) \arrow[r] & \Hm^{n+1}(\underline{X}) \arrow[r] & \cdots .
	\end{tikzcd}\]
\end{proposition}
\begin{proof}
	对每个 $n \in \Z$ 皆有正合列 $\Coker(d_X^n) \to \Coker(d_Y^n) \to \Coker(d_Z^n) \to 0$: 对
	\[\begin{tikzcd}
		0 \arrow[r] & X^{n-1} \arrow[r] \arrow[d] & Y^{n-1} \arrow[r] \arrow[d] & Z^{n-1} \arrow[r] \arrow[d] & 0 \\
		0 \arrow[r] & X^n \arrow[r] & Y^n \arrow[r] & Z^n \arrow[r] & 0
	\end{tikzcd}\]
	取定理 \ref{prop:snake-lemma} 之正合列的 $\Coker$ 部分便是. 类似道理, 也有正合列 $0 \to \Ker(d_X^n) \to \Ker(d_Y^n) \to \Ker(d_Z^n)$. 它们可以置入行正合的交换图表
	\begin{equation}\label{eqn:long-exact-sequence-ses-aux}\begin{tikzcd}
		& \Coker(d_X^{n-2}) \arrow[r] \arrow[d] & \Coker(d_Y^{n-2}) \arrow[r] \arrow[d] & \Coker(d_Z^{n-2}) \arrow[d] \arrow[r] & 0 \\
		0 \arrow[r] & \Ker(d_X^n) \arrow[r] & \Ker(d_Y^n) \arrow[r] & \Ker(d_Z^n) &
	\end{tikzcd}\end{equation}
	垂直箭头分别由 $d_X^{n-1}$, $d_Y^{n-1}$, $d_Z^{n-1}$ 诱导, 具有形如 $\Coker(d^{n-2}) \stackrel{d^{n-1}}{\twoheadrightarrow} \Image(d^{n-1}) \hookrightarrow \Ker(d^n)$ 的满--单分解 (省略下标).
	
	根据引理 \ref{prop:mono-epi-ker-coker} (iii), 可从 $\Coker(d^{n-2}) \twoheadrightarrow \Image(d^{n-1})$ 确定 \eqref{eqn:long-exact-sequence-ses-aux} 中垂直箭头的核, 再由 \eqref{eqn:homology-3} 依序得到 $\Hm^{n-1}(X)$, $\Hm^{n-1}(Y)$, $\Hm^{n-1}(Z)$; 同理, 从 $\Image(d^{n-1}) \hookrightarrow \Ker(d^n)$ 确定垂直箭头的余核, 则依序得到 $\Hm^n(X)$, $\Hm^n(Y)$ 和 $\Hm^n(Z)$. 应用定理 \ref{prop:snake-lemma} 以得出连接态射 $\delta^{n-1}: \Hm^{n-1}(Z) \to \Hm^n(X)$ 和长正合列中涉及 $\Hm^n$ 和 $\Hm^{n-1}$ 的项. 应用注记 \ref{rem:connecting-canonical} 可得函子性.
\end{proof}

\begin{definition}[拟同构]\label{def:quasi-isomorphism}
	\index{nitonggou@拟同构 (quasi-isomorphism)}
	设 $f: X \to Y$ 为 $\cate{C}(\mathcal{A})$ 中的态射. 若 $\Hm^n(f)$ 对所有 $n \in \Z$ 皆为同构, 则称 $f$ 为拟同构.
\end{definition}

\begin{proposition}\label{prop:null-homotopic-cohomology}
	设 $f: X \to Y$ 为 $\cate{C}(\mathcal{A})$ 的态射, $n \in \Z$. 若 $f$ 零伦, 则 $\Hm^n(f) = 0$.
\end{proposition}
\begin{proof}
	取 $h \in \Hom^{-1}(X, Y)$ 使得 $f = d_Y h + h d_X$. 合成态射 $\Ker(d_X^n) \hookrightarrow X^n \xrightarrow{h^{n+1} d_X^n} Y^n$ 为 $0$, 另一方面 $d_Y^{n-1} h^n$ 通过 $\Image(d_Y^{n-1})$ 分解. 两者搭配给出 $\Hm^n(f) = 0$.
\end{proof}

\begin{corollary}
	对所有 $n \in \Z$, 上同调函子 $\Hm^n: \cate{C}(\mathcal{A}) \to \mathcal{A}$ 唯一地通过 $\cate{K}(\mathcal{A})$ 分解. 故拟同构的概念可以扩及 $\cate{K}(\mathcal{A})$ 的态射. 若 $f: X \to Y$ 在 $\cate{K}(\mathcal{A})$ 中为同构, 则 $f$ 是拟同构.
\end{corollary}
\begin{proof}
	结合命题 \ref{prop:homotopy-category-universal-property} 和 \ref{prop:null-homotopic-cohomology}.
\end{proof}


\section{映射锥和长正合列}\label{sec:cone-vs-long-exact-sequence}
长正合列是同调代数的主要工具, 本节旨在比较从短正合列构造长正合列的三种方法. 精确到一些正负号, 它们殊途同归. 以下取 $\mathcal{A}$ 为 Abel 范畴.

\begin{proposition}\label{prop:cone-triangle-ses}
	设 $f: X \to Y$ 为 $\cate{C}(\mathcal{A})$ 中的态射. 令 $\alpha(f)$, $\beta(f)$ 如定义 \ref{def:cone-triangle}, 则
	\[ 0 \to Y \xrightarrow{\alpha(f)} \Cone(f) \xrightarrow{\beta(f)} X[1] \to 0 \]
	是 $\cate{C}(\mathcal{A})$ 中的正合列.
\end{proposition}
\begin{proof}
	回忆 $\alpha(f)$ 和 $\beta(f)$ 的定义. 既然正合性可以逐次检验 (命题 \ref{prop:abelian-cat-cplx}), 断言遂归结为
	\[ 0 \to Y^n \xrightarrow{(0, \identity)} X^{n+1} \oplus Y^n \xrightarrow{\text{投影}} X^{n+1} \to 0, \]
	对每个 $n \in \Z$ 的正合性, 此即命题 \ref{prop:biproduct-ses} 的内容.
\end{proof}

给定 $\cate{C}(\mathcal{A})$ 中的态射 $f: X \to Y$, 命题 \ref{prop:cone-triangle-ses} 的短正合列连同命题 \ref{prop:long-exact-sequence-ses} 给出 $\mathcal{A}$ 中的长正合列
\begin{equation}\label{eqn:cone-long-exact-sequence}
	\cdots \to \Hm^n(Y) \xrightarrow{\Hm^n(\alpha(f))} \Hm^n(\Cone(f)) \xrightarrow{\Hm^n(\beta(f))} \underbracket{\Hm^n(X[1])}_{= \Hm^{n+1}(X)} \xrightarrow{\xi^n} \Hm^{n+1}(Y) \to \cdots
\end{equation}
标为 $\xi^n$ 者是诱导自短正合列的连接态射; 推论 \ref{prop:cone-connecting} 将证明 $\xi^n$ 无非是 $\Hm^{n+1}(f): \Hm^{n+1}(X) \to \Hm^{n+1}(Y)$, 所以映射锥的长正合列 \eqref{eqn:cone-long-exact-sequence} 其每段都是明确的, 都来自 $f$, $\alpha(f)$ 和 $\beta(f)$.

下述结果则从另一个方向表明: 从 $\cate{C}(\mathcal{A})$ 中任意短正合列出发, 也可以自然地和映射锥建立联系. 方式分两种.

\begin{lemma}\label{prop:cone-qis}
	给定 $\cate{C}(\mathcal{A})$ 中的短正合列 $0 \to X \xrightarrow{f} Y \xrightarrow{g} Z \to 0$, 则
	\[ \Phi := (0, g): \Cone(f) \to Z, \quad \Phi' := (f[1], 0): X[1] \to \Cone(g) \]
	皆是复形之间的拟同构, 它们具有以下性质:
	\begin{enumerate}[(i)]
		\item $\Phi \circ \alpha(f) = g$ 而 $\beta(g) \circ \Phi' = f[1]$,
		\item $\alpha(g)\Phi + \Phi'\beta(f): \Cone(f) \to \Cone(g)$ 典范地零伦.
	\end{enumerate}
\end{lemma}
\begin{proof}
	端详 $\cate{C}(\mathcal{A})$ 的交换图表
	\[\begin{tikzcd}[column sep=small]
		0 \arrow[r] & X \arrow[r, "\identity_X"] \arrow[d, "{\identity_X}"'] & X \arrow[r] \arrow[d, "f"] & 0 \arrow[r] \arrow[d] & 0 \\
		0 \arrow[r] & X \arrow[r, "f"'] & Y \arrow[r, "g"'] & Z \arrow[r] & 0
	\end{tikzcd} \quad \begin{tikzcd}[column sep=small]
		0 \arrow[r] & X \arrow[d] \arrow[r, "f"] & Y \arrow[d, "g"'] \arrow[r, "g"] & Z \arrow[d, "\identity_Z"] \arrow[r] & 0 \\
		0 \arrow[r] & 0 \arrow[r] & Z \arrow[r, "\identity_Z"'] & Z \arrow[r] & 0
	\end{tikzcd}\]
	每一行皆正合; 映射锥的函子性 (命题 \ref{prop:cone-functorial}) 遂给出 $\cate{C}(\mathcal{A})$ 中的态射
	\begin{equation}\label{eqn:cone-qis-aux}\begin{tikzcd}[row sep=tiny]
		0 \arrow[r] & \Cone\left( \identity_X \right) \arrow[r] & \Cone\left( X \xrightarrow{f} Y \right) \arrow[r] & \Cone\left( 0 \to Z \right) \arrow[r] & 0 \\
		0 \arrow[r] & \Cone(X \to 0) \arrow[r] & \Cone\left( Y \xrightarrow{g} Z \right) \arrow[r] & \Cone\left( \identity_Z \right) \arrow[r] & 0
	\end{tikzcd}\end{equation}
	逐次考察, 并利用前述交换图表及映射锥的定义, 可见 \eqref{eqn:cone-qis-aux} 每行皆正合. 此外, 容易看出 $\Cone(f) \to \Cone(0 \to Z) \simeq Z$ (或 $X[1] \simeq \Cone(X \to 0) \to \Cone(g)$) 即是 $\Phi$ (或 $\Phi'$). 一旦知道它们是复形的态射, (i) 的等式便是自明的.
	
	对 \eqref{eqn:cone-qis-aux} 应用命题 \ref{prop:long-exact-sequence-ses}, 便对每个 $n \in \Z$ 得到正合列
	\begin{gather*}
		\Hm^n\left( \Cone(\identity_X) \right) \to \Hm^n \left( \Cone(f) \right) \xrightarrow{\Hm^n(\Phi)} \Hm^n(Z) \to \Hm^{n+1} \left( \Cone(\identity_X) \right), \\
		\Hm^{n-1}\left( \Cone(\identity_Z) \right) \to \Hm^n(X[1]) \xrightarrow{\Hm^n(\Phi')} \Hm^n\left( \Cone(g) \right) \to \Hm^n\left( \Cone(\identity_Z) \right).
	\end{gather*}
	引理 \ref{prop:cone-homotopy} (i) 和命题 \ref{prop:null-homotopic-cohomology} 表明左右端点项皆为 $0$, 故 $\Phi$ 和 $\Phi'$ 是拟同构.
	
	最后验证 (ii). 按照 $\Cone(f)^n = X^{n+1} \oplus Y^n$ 和 $\Cone(g)^n = Y^{n+1} \oplus Z^n$ 将态射写成矩阵
	\begin{gather*}\begin{tikzcd}[row sep=tiny, column sep=large, ampersand replacement=\&]
		\& Z \arrow[rd, "\alpha(g)" description, "{\bigl(\begin{smallmatrix} 0 \\ \identity \end{smallmatrix} \bigr)}" inner sep=0.6em] \& \\
		\Cone(f) \arrow[ru, "\Phi" description, "{\bigl( 0 \; g \bigr)}" inner sep=0.6em] \arrow[rd, "\beta(f)" description, "{\big( \identity \; 0 \bigr)}"' inner sep=0.6em] \& \& \Cone(g) \\
		\& X[1] \arrow[ru, "{\Phi'}" description, "{\bigl( \begin{smallmatrix} f[1] \\ 0 \end{smallmatrix} \bigr)}"' inner sep=0.6em] \&
	\end{tikzcd}, \quad
		\alpha(g) \Phi + \Phi' \beta(f) = \begin{pmatrix} f[1] & 0 \\ 0 & g  \end{pmatrix}.
	\end{gather*}
	另外取 $s = (s^n)_n \in \Hom^{-1}\left( \Cone(f), \Cone(g) \right)$ 为
	\[ s^n := \begin{pmatrix}
		0 & \identity_{Y^n} \\
		0 & 0
	\end{pmatrix}: X^{n+1} \oplus Y^n \to Y^n \oplus Z^{n-1} , \]
	这也是唯一合理的取法. 直接的矩阵计算表明 $d_{\Cone(g)}^{n-1} s^n + s^{n+1} d_{\Cone(f)}^n$ 等于
	\[\begin{pmatrix}
		-d_Y^n & 0 \\
		g^n & d_Z^{n-1}
	\end{pmatrix} \begin{pmatrix}
		0 & \identity_{Y^n} \\
		0 & 0
	\end{pmatrix} + \begin{pmatrix}
		0 & \identity_{Y^{n+1}} \\
		0 & 0
	\end{pmatrix} \begin{pmatrix}
		-d_X^{n+1} & 0 \\
		f^{n+1} & d_Y^n
	\end{pmatrix} = \begin{pmatrix}
		f^{n+1} & 0 \\
		0 & g^n
	\end{pmatrix}.\]
	故 $\alpha(g)\Phi + \Phi'\beta(f)$ 确实零伦.
\end{proof}

综上, 从 $\cate{C}(\mathcal{A})$ 中的短正合列 $0 \to X \xrightarrow{f} Y \xrightarrow{g} Z \to 0$ 出发, 有三种方式在 $\mathcal{A}$ 中导出长正合列
\[\begin{tikzcd}[row sep=large]
	\cdots \arrow[r] & \Hm^{n-1}(Y) \arrow[r] \arrow[phantom, d, ""{coordinate, name=A}] & \Hm^{n-1}(Z) \arrow[dll, rounded corners, to path={
		-- ([xshift=2ex]\tikztostart.east)
		|- (A) [near end]\tikztonodes
		-| ([xshift=-2ex]\tikztotarget.west)
		-- (\tikztotarget)}] \\
	\Hm^n(X) \arrow[r] & \Hm^n(Y) \arrow[r] \arrow[phantom, d, ""{coordinate, name=B}] & \Hm^n(Z) \arrow[dll, rounded corners, to path={
		-- ([xshift=2ex]\tikztostart.east)
		|- (B) [near end]\tikztonodes
		-| ([xshift=-2ex]\tikztotarget.west)
		-- (\tikztotarget)}] \\
	\Hm^{n+1}(X) \arrow[r] & \Hm^{n+1}(Y) \arrow[r] & \cdots .
\end{tikzcd}\]
\begin{enumerate}[(A)]
	\item 直接应用命题 \ref{prop:long-exact-sequence-ses} 得到长正合列
	\[\begin{tikzcd}
		\cdots \arrow[r] & \Hm^n(Y) \arrow[r, "{\Hm^n(g)}" inner sep=0.6em] & \Hm^n(Z) \arrow[r, "{\delta^n}" inner sep=0.6em] & \Hm^{n+1}(X) \arrow[r, "{\Hm^{n+1}(f)}" inner sep=0.6em] & \Hm^{n+1}(Y) \arrow[r] & \cdots
	\end{tikzcd}\]
	\item 应用引理 \ref{prop:cone-qis} 的拟同构 $\Phi: \Cone(f) \to Z$, 配合 \eqref{eqn:cone-long-exact-sequence} 得出交换图表
	\[\begin{tikzcd}[column sep=scriptsize]
		\cdots \arrow[r] & \Hm^n(Y) \arrow[r, "{\Hm^n(\alpha(f))}" inner sep=0.6em] \arrow[equal, d] & \Hm^n\left( \Cone(f) \right) \arrow[r, "{\Hm^n(\beta(f))}" inner sep=0.6em] \arrow[d, "{\Hm^n(\Phi)}"] & \Hm^n(X[1]) \arrow[r] \arrow[equal, d] & \Hm^{n+1}(Y) \arrow[r] \arrow[equal, d] & \cdots \\
		\cdots \arrow[r] & \Hm^n(Y) \arrow[r, "{\Hm^n(g)}"' inner sep=0.6em] & \Hm^n(Z) \arrow[r, "{\eta^n}"' inner sep=0.6em] & \Hm^{n+1}(X) \arrow[r, "{\xi^n}"' inner sep=0.6em] & \Hm^{n+1}(Y) \arrow[r] & \cdots
	\end{tikzcd}\]
	其中
	\begin{compactitem}
		\item $\eta^n := \Hm^n(\beta(f)) \Hm^n(\Phi)^{-1}$,
		\item $\xi^n$ 是 \eqref{eqn:cone-long-exact-sequence} 中的连接态射 $\Hm^{n+1}(X) = \Hm^n(X[1]) \to \Hm^{n+1}(Y)$.
	\end{compactitem}
	已知第一行正合, 故第二行亦然.

	\item 同上, 但改用引理 \ref{prop:cone-qis} 的拟同构 $\Phi': X[1] \to \Cone(g)$, 得到交换图表
	\[\begin{tikzcd}[column sep=scriptsize]
		\cdots \arrow[r] & \Hm^n(Z) \arrow[r, "{\Hm^n(\alpha(g))}" inner sep=0.6em] & \Hm^n\left( \Cone(g) \right) \arrow[r, "{\Hm^n(\beta(g))}" inner sep=0.6em] & \Hm^n(Y[1]) \arrow[r] & \Hm^{n+1}(Z) \arrow[r] & \cdots \\
		\cdots \arrow[r] & \Hm^n(Z) \arrow[r, "{(\eta')^n}"' inner sep=0.6em] \arrow[equal, u] & \Hm^n(X[1]) \arrow[r, "{\Hm^n(f[1])}"' inner sep=0.6em] \arrow[u, "{\Hm^n(\Phi')}"', "\simeq"] & \Hm^{n+1}(Y) \arrow[r, "{(\xi')^n}"' inner sep=0.6em] \arrow[equal, u] & \Hm^{n+1}(Z) \arrow[r] \arrow[equal, u] & \cdots
	\end{tikzcd}\]
	其中
	\begin{compactitem}
		\item $(\eta')^n := \Hm^n(\Phi')^{-1} \Hm^n(\alpha(g))$,
		\item $(\xi')^n$ 来自 \eqref{eqn:cone-long-exact-sequence} (以 $g$ 代 $f$) 中的连接态射 $\Hm^{n+1}(Y) = \Hm^n(Y[1]) \to \Hm^{n+1}(Z)$.
	\end{compactitem}
	已知第一行正合, 故第二行亦然.
\end{enumerate}

这三种长正合列有何异同? 我们先来陈述结论.

\begin{itemize}
	\item 构造 (A) 和 (B) 仅差一些负号: 命题 \ref{prop:connecting-eta-delta} 将说明 $\eta^n = -\delta^n$, 而推论 \ref{prop:cone-connecting} 则蕴涵 $\xi^n = \Hm^{n+1}(f)$.
	\item 构造 (A) 和 (C) 的产物相同: 推论 \ref{prop:connecting-eta-delta-2} 将基于 (B) 的结果来说明 $(\eta')^n = \delta^n$, 而推论 \ref{prop:cone-connecting} 蕴涵 $(\xi')^n = \Hm^{n+1}(g)$.
	\item 因此 (B) 和 (C) 仅在连接态射 $\Hm^n(Z) \to \Hm^{n+1}(X)$ 处差一个负号, 这点可由三角范畴的旋转公理 (TR3) 得到解释, 详见之后的命题 \ref{prop:ses-vs-triangle}.
\end{itemize}
接着着手来确立这些关系.

\begin{proposition}\label{prop:connecting-eta-delta}
	在上述场景中, $\eta^n = -\delta^n$ 对所有 $n \in \Z$ 皆成立.
\end{proposition}
\begin{proof}
	以下论证取自 \cite[Proposition 12.3.6]{KS06}. 由于论证比较曲折, 敬邀读者先尝试 $\mathcal{A} = R\dcate{Mod}$ 的具体情形, 其中 $R$ 是环; 图追踪 \cite[\S 6.8]{Li1} 的办法对之给出直截了当的证明.

	对于一般的 Abel 范畴 $\mathcal{A}$, 首先选定 $n \in \Z$, 取纤维积
	\[ S := \Coker(d_Y^{n-1}) \dtimes{\Coker(d_Z^{n-1})} \Hm^n(Z). \]
	连接同态 $\delta^n$ 如何构造? 它是施 \eqref{eqn:snake-conn} 于 \eqref{eqn:long-exact-sequence-ses-aux} 的图表
	\[\begin{tikzcd}
		& \Coker(d_X^{n-1}) \arrow[r] \arrow[d] & \Coker(d_Y^{n-1}) \arrow[r] \arrow[d] & \Coker(d_Z^{n-1}) \arrow[d] \arrow[r] & 0 \\
		0 \arrow[r] & \Ker(d_X^{n+1}) \arrow[r] & \Ker(d_Y^{n+1}) \arrow[r] & \Ker(d_Z^{n+1}) &
	\end{tikzcd}\]
	的产物, 在此调整为
	\begin{equation}\label{eqn:connecting-eta-delta-0}\begin{tikzcd}
		& S \arrow[twoheadrightarrow, r, "w"] \arrow[d, "v"] \arrow[ldd, bend right, "u"'] \arrow[rd, phantom, "\Box" description] & \Hm^n(Z) \arrow[hookrightarrow, d] \\
		& \Coker(d_Y^{n-1}) \arrow[twoheadrightarrow, r] \arrow[d, "a"] & \Coker(d_Z^{n-1}) \arrow[d] \\
		\Ker(d_X^{n+1}) \arrow[hookrightarrow, r, "b"'] \arrow[twoheadrightarrow, d] & \Ker(d_Y^{n+1}) \arrow[r, "c"'] & \Ker(d_Z^{n+1}) \\
		\Hm^{n+1}(X) & &
	\end{tikzcd}\end{equation}
	的形式; 图中行列皆正合, 箭头 $a$ 由 $d_Y^n$ 诱导, $b$ 由 $f^{n+1}$ 诱导, 而箭头 $u$ 的存在性和唯一性是缘于右半部交换蕴涵 $cav = 0$. 细心回顾 \eqref{eqn:snake-conn} 的构造可见图表
	\begin{equation}\label{eqn:connecting-eta-delta-0.5}\begin{tikzcd}[row sep=tiny]
		& \Hm^n(Z) \arrow[rd, "\delta^n"] & \\
		S \arrow[twoheadrightarrow, ru, "w"] \arrow[rd, "u"'] & & \Hm^{n+1}(X) \\
		& \Ker\left( d_X^{n+1} \right) \arrow[twoheadrightarrow, ru] &
	\end{tikzcd} \quad \text{交换}; \end{equation}
	因为 $w$ 已知满, 此图表也唯一地确定了 $\delta^n$.
	
	为了和映射锥作比较, 请按定义验证下图交换
	\[\begin{tikzcd}
		\Ker(d_X^{n+1}) \oplus \Coker(d_Y^{n-1}) \arrow[d, "{(b,a)}"'] \arrow[hookrightarrow, r] & X^{n+1} \oplus \Coker\left( d_Y^{n-1} \right) \arrow[twoheadrightarrow, r] & \Coker\left( d_{\Cone(f)}^{n-1} \right) \arrow[d, "{d_{\Cone(f)}^n}"] \\
		\Ker(d_Y^{n+1}) \arrow[r, "\sim"] & 0 \oplus \Ker\left( d_Y^{n+1} \right) \arrow[hookrightarrow, r] & \Ker\left( d_{\Cone(f)}^{n+1} \right)
	\end{tikzcd}\]
	其中横向箭头的定义理应是自明的. 既然 \eqref{eqn:connecting-eta-delta-0} 的扇形部分交换, 上图蕴涵
	\[ S \xrightarrow{(-u, v)} \Ker\left( d_X^{n+1} \right) \oplus \Coker\left( d_Y^{n-1} \right) \xrightarrow[\text{第一行}]{\text{上图}} \Coker\left( d_{\Cone(f)}^{n-1} \right) \xrightarrow{d_{\Cone(f)}^n} \Ker\left( d_{\Cone(f)}^{n+1} \right) \]
	的合成为 $0$, 故 $S \xrightarrow{(-u, v)} \Ker\left( d_X^{n+1} \right) \oplus \Coker\left( d_Y^{n-1} \right) \to \Coker\left( d_{\Cone(f)}^{n-1} \right)$ 唯一地分解为 $S \to \Hm^n(\Cone(f)) \hookrightarrow \Coker\left( d_{\Cone(f)}^{n-1} \right)$. 兹断言下图交换:
	\begin{equation}\label{eqn:connecting-eta-delta-1}\begin{tikzcd}
		S \arrow[ddd, bend right=60, "w"'] \arrow[dd] \arrow[r, "{(-u, v)}"] & \Ker\left( d_X^{n+1} \right) \oplus \Coker\left( d_Y^{n-1} \right) \arrow[dd, "\text{自明}"] \arrow[r, "\text{投影}"] & \Ker\left( d_X^{n+1} \right) \arrow[twoheadrightarrow, d] \\
		& & \Hm^{n+1}(X) \arrow[hookrightarrow, d] \\
		\Hm^n\left( \Cone(f) \right) \arrow[d, "\simeq"', "{\Hm^n(\Phi)}"] \arrow[hookrightarrow, r] & \Coker\left( d_{\Cone(f)}^{n-1} \right) \arrow[r, "\text{由 $\beta(f)$ 诱导}"' inner sep=0.5em] \arrow[d, "\text{由 $\Phi$ 诱导}"' inner sep=0.2em] & \Coker\left( d_X^n \right) \\
		\Hm^n(Z) \arrow[hookrightarrow, r] & \Coker\left( d_Z^{n-1} \right) . &
	\end{tikzcd}\end{equation}
	诚然, 验证各个方块的交换性都是例行公事; 基于 \eqref{eqn:connecting-eta-delta-0} 的右上方块和 $\Phi = (0, g)$, 可见图表 \eqref{eqn:connecting-eta-delta-1} 的
	\[\begin{tikzcd}
		S \arrow[r, "{(-u, v)}"]  \arrow[bend right=60, d, "w"'] & \Ker\left( d_X^{n+1} \right) \oplus \Coker\left( d_Y^{n-1} \right) \arrow[d] \\
		\Hm^n(Z) \arrow[hookrightarrow, r] & \Coker\left( d_Z^{n-1} \right)
	\end{tikzcd}\]
	部分也交换, 而又由于 $\Hm^n(Z) \to \Coker(d_Z^{n-1})$ 为单, 与之合成立见 \eqref{eqn:connecting-eta-delta-1} 余下的左侧弓形部分亦交换.

	观察到 $\Hm^n(\Cone(f)) \hookrightarrow \Coker \left( d_{\Cone(f)}^{n-1} \right) \to \Coker(d_X^n)$ 和 $\Hm^n(\Cone(f)) \xrightarrow{\Hm^n(\beta(f))} \Hm^{n+1}(X) \hookrightarrow \Coker(d_X^n)$ 的合成相同, 皆来自向 $X^{n+1}$ 的投影. 故 \eqref{eqn:connecting-eta-delta-1} 进一步给出交换图表
	\[\begin{tikzcd}
		S \arrow[d, "w"'] \arrow[rr, "-u"] & & \Ker\left( d_X^{n+1} \right) \arrow[d] \\
		\Hm^n(Z) \arrow[r, "\sim", "{\Hm^n(\Phi)^{-1}}"' inner sep=0.5em] & \Hm^n\left( \Cone(f) \right) \arrow[r, "{\Hm^n(\beta(f))}"' inner sep=0.5em] & \Hm^{n+1}(X) .
	\end{tikzcd}\]
	鉴于 \eqref{eqn:connecting-eta-delta-0.5}, 这蕴涵 $S \xrightarrow{w} \Hm^n(Z) \xrightarrow{\delta^n} \Hm^{n+1}(X)$ 和 $S \xrightarrow{w} \Hm^n(Z) \xrightarrow{\eta^n} \Hm^{n+1}(X)$ 的合成差一个负号. 又因为 $w$ 满, 故 $\eta^n = -\delta^n$. 明所欲证.
\end{proof}

\begin{corollary}\label{prop:connecting-eta-delta-2}
	在上述场景中, $(\eta')^n = \delta^n$ 对所有 $n \in \Z$ 皆成立.
\end{corollary}
\begin{proof}
	沿用引理 \ref{prop:cone-qis} 的记号. 基于命题 \ref{prop:connecting-eta-delta}, 问题归结为证 $(\eta')^n + \eta^n = 0$, 但后者是 $\alpha(g)\Phi + \Phi'\beta(f): \Cone(f) \to \Cone(g)$ 零伦 (引理 \ref{prop:cone-qis} (ii)) 的直接结论.
\end{proof}

\begin{corollary}\label{prop:cone-connecting}
	给定 $\cate{C}(\mathcal{A})$ 中的任意态射 $f: X \to Y$, 长正合列 \eqref{eqn:cone-long-exact-sequence} 中的连接态射 $\xi^n: \Hm^{n+1}(X) \simeq \Hm^n(X[1]) \to \Hm^{n+1}(Y)$ 等于 $\Hm^{n+1}(f)$.
\end{corollary}
\begin{proof}
	对短正合列 $0 \to Y \to \Cone(f) \to X[1] \to 0$ 应用引理 \ref{prop:cone-qis}, 可得拟同构 $(0, \beta(f))$ 及行正合交换图表
	\[\begin{tikzcd}
		0 \arrow[r] & Y \arrow[r, "\alpha(f)"] & \Cone(f) \arrow[r, "{\alpha(\alpha(f))}"] & \Cone(\alpha(f)) \arrow[d, "{\Phi = (0, \beta(f))}"] \arrow[r] & 0 \\
		0 \arrow[r] & Y \arrow[r, "\alpha(f)"'] \arrow[equal, u] & \Cone(f) \arrow[r, "\beta(f)"'] \arrow[equal, u] & X[1] \arrow[r] & 0
	\end{tikzcd}\]
	引理 \ref{prop:cone-alpha} 论及的态射 $\psi: \Cone(\alpha(f)) \to X[1]$ 正是此处的 $(0, \beta(f))$: 它的第 $n$ 次项是从 $Y^{n+1} \oplus \Cone(f)^n = Y^{n+1} \oplus X^{n+1} \oplus Y^n$ 到 $X^{n+1}$ 的投影. 该引理的交换图表遂给出
	\begin{equation}\label{eqn:cone-connecting-aux}
		\text{在 $\cate{K}(\mathcal{A})$ 中} \quad f[1] \circ (0, \beta(f)) + \beta(\alpha(f)) = 0 .
	\end{equation}

	现在回忆到 $\xi^n: \Hm^n(X[1]) \to \Hm^{n+1}(Y)$ 是以下短正合列的连接态射:
	\[ 0 \to Y \xrightarrow{\alpha(f)} \Cone(f) \xrightarrow{\beta(f)} X[1] \to 0, \]
	对此短正合列应用命题 \ref{prop:connecting-eta-delta}, 不难推得合成态射
	\[ \Hm^n(\Cone(\alpha(f))) \xrightarrow[\sim]{\Hm^n((0, \beta(f)))} \Hm^n(X[1]) \xrightarrow{\xi^n} \Hm^{n+1}(Y) \]
	等于 $-\Hm^n(\beta(\alpha(f)))$. 然而 \eqref{eqn:cone-connecting-aux} 蕴涵 $\Hm^n(\beta(\alpha(f))) = -\Hm^n(f[1]) \circ \Hm^n((0, \beta(f)))$. 于是 $\xi^n = \Hm^n(f[1]) = \Hm^{n+1}(f)$.
\end{proof}

\begin{corollary}\label{prop:quism-cone}
	承上, $f$ 是拟同构当且仅当 $\Hm^n(\Cone(f)) = 0$ 对所有 $n \in \Z$ 成立.
\end{corollary}
\begin{proof}
	将推论 \ref{prop:cone-connecting} 代入 \eqref{eqn:cone-long-exact-sequence} 的长正合列, 按图索骥.
\end{proof}

\section{练习: Hochschild 同调与上同调}\label{sec:HH}
本节回归具体的数学. 取 $\Bbbk$ 为交换环, 将 $\Bbbk$-模之间的张量积 $\dotimes{\Bbbk}$ 简写作 $\otimes$, 将 $\Bbbk$-模 $M$ 的 $n$ 重张量积记为 $M^{\otimes n}$, 并约定 $M^{\otimes 0} := \Bbbk$. 对于 $\Bbbk$-代数 $R$, 按惯例, $(R, R)$-双模带有的 $\Bbbk$ 的左乘和右乘默认相等, 因此 $(R, R)$-双模也等同于左 $R \otimes R^{\opp}$-模.

\begin{definition}[杠复形]\label{def:bar-Hochschild}
	\index{gangfuxing@杠复形 (bar complex)}
	\index[sym1]{BR@$\mathsf{B}R$}
	选定 $\Bbbk$-代数 $R$. 对每个 $n \in \Z_{\geq 0}$ 定义 $(R,R)$-双模
	\begin{gather*}
		\mathsf{B}_n R := R \otimes R^{\otimes n} \otimes R = R^{\otimes (n+2)}, \\
		r(r_0 \otimes \cdots \otimes r_{n+1})r' = rr_0 \otimes \cdots \otimes r_{n+1} r' ,
	\end{gather*}
	其中 $r, r', r_0, \ldots, r_{n+1} \in R$. 对所有 $n \geq 1$ 定义双模同态
	\begin{equation*}\begin{split}
		b_n: \mathsf{B}_n R & \to \mathsf{B}_{n-1} R, \\
		b_n \left( r_0 \otimes \cdots \otimes r_{n+1} \right) & := \sum_{k=0}^n (-1)^k \cdots \otimes r_k r_{k+1} \otimes \cdots .
	\end{split}\end{equation*}
	 我们称 $\mathsf{B} R := \left( \mathsf{B}_n R, b_n \right)_{n \geq 0}$ 为 $R$ 的杠复形; 这是注记 \ref{rem:cochain-vs-chain} 所谓的链复形.
\end{definition}

经典的记法是将 $r_0 \otimes \cdots \otimes r_{n+1} \in \mathsf{B}_n R$ 记作 $(r_0 | \cdots | r_{n+1})$. 这是``杠''的由来. 同态 $b$ 的作用相当于以所有可能方式抽掉任一道杠, 符号交错加总. 由此容易看出 $b^2 = 0$: 诚然, $b^2 (r_0 | \cdots | r_{n+1})$ 是形如
\begin{gather*}
	(\cdots |r_{h-1} r_h| \cdots |r_{k-1} r_k| \cdots ) \quad \text{或} \quad ( \cdots | r_{h-1} r_h r_{h+1} | \cdots )
\end{gather*}
的元素的线性组合; 要从 $(\cdots |r_{h-1} | \cdots | r_k | \cdots)$ 抽杠得到这样的项, 恰有两种方式, 其符号相消. 因此 $\mathsf{B} R$ 确实是链复形.

现将杠复形作下图所示的增广, 记为 $\mathsf{B}' R$, 其中的态射 $b_0$ 也称为增广同态.
\[\begin{tikzcd}[row sep=small]
	\cdots \arrow[r, "b_2"] & \mathsf{B}_1 R \arrow[r, "b_1"] & \mathsf{B}_0 R \arrow[r, "b_0"] & (\mathsf{B}_{-1} R := R) \arrow[r] & 0 \\
	& & (r_0 | r_1) \arrow[mapsto, r] \arrow[phantom, u, "\in" description, sloped] & r_0 r_1 \arrow[phantom, u, "\in" description, sloped] &
\end{tikzcd}\]\index[sym1]{B'R@$\mathsf{B}'R$}

\begin{lemma}\label{prop:HH-bar-exactness}
	增广后的链复形 $\mathsf{B}' R$ 正合; 更精确地说, 若将 $\mathsf{B}' R$ 视为 $\Bbbk$-模构成的链复形, 则 $\identity_{\mathsf{B}'R}$ 零伦.
\end{lemma}
\begin{proof}
	今将构造一族 $\Bbbk$-模同态 $h_n: \mathsf{B}'_n R \to \mathsf{B}'_{n+1} R$, 其中 $n \geq -1$, 使得 $b_{n+1} h_n + h_{n-1} b_n = \identity_{\mathsf{B}'_n R}$ (当然地约定 $h_{-2} = 0$, $b_{-1} = 0$); 这将使 $\identity_{\mathsf{B}' R}$ 零伦. 具体取
	\[ h_n(r_0 | \cdots | r_{n+1}) = (1_R | r_0 | \cdots | r_{n+1}), \]
	然后按线性延拓到 $\mathsf{B}'_n R$ 上. 对任意 $n \geq 0$ 和 $r_0, \ldots, r_{n+1} \in R$, 我们有
	\begin{align*}
		b_{n+1} h_n \left(r_0 | \cdots | r_{n+1} \right) & = (r_0 | \cdots | r_{n+1}) + \sum_{k=0}^n (-1)^{k+1} (1_R | \cdots | r_k r_{k+1} | \cdots), \\
		h_{n-1} b_n \left(r_0 | \cdots | r_{n+1} \right) & = \sum_{k=0}^n (-1)^k (1_R | \cdots | r_k r_{k+1} | \cdots ).
	\end{align*}
	由此立见
	\[ (b_{n+1} h_n + h_{n-1} b_n)(r_0 | \cdots | r_{n+1}) = (r_0 | \cdots | r_{n+1}), \]
	而因为 $b_0 h_{-1}(r) = b_0(1_R|r) = r$, 此式对 $n = -1$ 也平凡地成立.
\end{proof}

复形 $\mathsf{B}R$ 及 $\mathsf{B}'R$ 的定义乍看出乎奇思妙想. 往后的例 \ref{eg:HH-comonad} 将从余单子的高度来观照这些链复形, 予以自然的解释.

\begin{convention}\label{con:Re-R-R}
	\index[sym1]{R-e@$R^e$}
	命 $R^e := R \otimes R^{\opp}$. 对于任意 $(R, R)$-双模 $M$, 包括 $M=R$ 的特例, 今后
	\begin{compactitem}
		\item 按 $(r \otimes r') m = rmr'$ 将 $M$ 作成左 $R^e$-模,
		\item 按 $m (r \otimes r') := r'mr$ 将 $M$ 作成右 $R^e$-模.
	\end{compactitem}
\end{convention}

对所有 $M$, 依此可定义由 $\Bbbk$-模构成的链复形 $M \dotimes{R^e} \mathsf{B} R$ 和复形 $\Hom_{R^e}\left( \mathsf{B}R, M \right)$.

\begin{definition}[G.\ Hochschild]\label{def:HH}
	\index{Hochschild@Hochschild 同调, 上同调}
	\index[sym1]{HH@$\HHm_n$, $\HHm^n$}
	对所有 $n$, 定义 $\Bbbk$-线性函子
	$\HHm_n, \HHm^n: (R,R)\dcate{Mod} \rightrightarrows \Bbbk\dcate{Mod}$ 如下:
	\[\begin{array}{cl}
		\text{Hochschild 同调} & \HHm_n(M) := \Hm_n\left( M \dotimes{R^e} \mathsf{B} R \right), \\
		\text{Hochschild 上同调} & \HHm^n(M) := \Hm^n \left( \Hom_{R^e}(\mathsf{B}R, M)  \right).
	\end{array}\]
	对于特例 $M=R$, 上式所定义的对象分别称为 $R$ 的 Hochschild 同调 $\HHm_n(R)$ 与上同调 $\HHm^n(R)$.
\end{definition}

为了简化 $\HHm_n(M)$ 和 $\HHm^n(M)$ 的描述, 兹引入两种 Hochschild 复形
\begin{equation}\label{eqn:HH-cplx}\begin{aligned}
	\index[sym1]{CRM@$C_\bullet(R, M)$, $C^\bullet(R, M)$}
	C_\bullet(R,M) & := \left[ \cdots \to M \otimes R^{\otimes n} \xrightarrow{d_n} \cdots \xrightarrow{d_2} M \otimes R \xrightarrow{d_1} M \right], \\
	C^\bullet(R,M) & := \left[ M \xrightarrow{d^0} \Hom_{\Bbbk}(R, M) \xrightarrow{d^1} \cdots \to \Hom_{\Bbbk}(R^{\otimes n}, M) \xrightarrow{d^n} \cdots \right],
\end{aligned}\end{equation} \index{Hochschild fuxing@Hochschild 复形 (Hochschild complex)}
次数分别为 $\ldots, 2, 1, 0$ 和 $0, 1, 2, \ldots$, 其他项补 $0$. 同态 $d_n$ 和 $d^n$ 按以下方式定义.
\begin{enumerate}
	\item 依旧以杠标记 $M \otimes R^{\otimes n}$ 的元素 $m \otimes r_1 \otimes \cdots \otimes r_n$ 为 $(m | r_1 | \cdots | r_n)$. 命
	\begin{multline}\label{eqn:HH-cplx-d}
		d_n (m|r_1| \cdots |r_n) = \\
		\underbracket{(mr_1 | r_2 | \cdots |r_n) + \sum_{k=1}^{n-1} (-1)^k (m|\cdots |r_k r_{k+1} | \cdots)}_{=: d'_n(m|r_1| \cdots| r_n)} + (-1)^n (r_n m | r_1 | \cdots | r_{n-1}).
	\end{multline}
	\item 将 $\Hom_{\Bbbk}(R^{\otimes n}, M)$ 的元素视同 $n$ 重 $\Bbbk$-线性映射 $R^n \to M$ (见 \cite[\S 7.5]{Li1}). 命
	\begin{multline*}
		(d^n f)(r_1, \ldots, r_{n+1}) = \\
		r_1 f(r_2, \ldots, r_{n+1}) + \sum_{k=1}^n (-1)^k f(\ldots, r_k r_{k+1}, \ldots) + (-1)^{n+1} f(r_1, \ldots, r_n) r_{n+1}.
	\end{multline*}
\end{enumerate}

对所有 $n \in \Z_{\geq 0}$, 我们有同构
\[\begin{tikzcd}[row sep=tiny]
	M \dotimes{R^e} \mathsf{B}_n R \arrow[leftrightarrow, r, "\sim"] & M \otimes R^{\otimes n} \\
	m \otimes (r_0 \otimes \cdots \otimes r_{n+1}) \arrow[mapsto, r] & (r_{n+1} m r_0 | r_1 | \cdots | r_n | 1_R) \\
	m \otimes (1_R \otimes r_1 \otimes \cdots \otimes r_n \otimes 1_R )& (m | r_1 | \cdots | r_n) \arrow[mapsto, l]
\end{tikzcd}\]
和
\[\begin{tikzcd}[row sep=tiny, column sep=small]
	\Hom_{R^e}(\mathsf{B}_nR, M) \arrow[leftrightarrow, r, "\sim"] & \Hom_{\Bbbk}(R^{\otimes n}, M) \\
	\varphi \arrow[mapsto, r] & {\left[ (r_1, \ldots, r_n) \mapsto \varphi(1_R, r_1, \ldots, r_n, 1_R) \right]} \\
	{\left[ r_0 \otimes \cdots \otimes r_{n+1} \mapsto r_0 f(r_1, \ldots, r_n) r_{n+1} \right]} & f \arrow[mapsto, l] .
\end{tikzcd}\]
简短的计算表明 $\identity_M \otimes b_n$ 通过同构对应于 $d_n$, 而 $b_n^*$ 则对应于 $d^n$. 于是上述同构是复形的同构, 给出
\[ \HHm^n(R, M) \simeq \Hm^n(C^\bullet(R, M)), \quad \HHm_n(R, M) \simeq \Hm_n(C_\bullet(R, M)) . \]
建议初学的读者动手进行详细验证.

\begin{remark}
	设 $R$ 交换, 则任意 $R$-模 $M$ 按 $rmr' := rr' m$ 作成 $(R,R)$-双模. 这时 $C_\bullet(M,R)$ (或 $C^\bullet(M,R)$) 成为 $R$-模的复形, 方式是命
	\[ r \cdot (m|r_1| \cdots |r_n) = (rm|r_1| \cdots | r_n) \quad \text{或} \quad (r \cdot f)(r_1, \ldots, r_n) = r f(r_1, \ldots, r_n), \]
	因此 $\HHm_n(M)$ 和 $\HHm^n(M)$ 都升级为 $R$-模. 这点自然也可以从杠复形的层面来论证.
\end{remark}

\begin{example}
	取 $M = R = \Bbbk$, 按上述构造立见 $d_1, d_2, \ldots$ 依序是 $0$, $\identity$, $0$, $\identity$ 等, 于是 $\HHm_0(\Bbbk) = \Bbbk$ 而 $\HHm_{\geq 1}(\Bbbk) = 0$. 类似的论证给出 $\HHm^0(\Bbbk) = \Bbbk$ 和 $\HHm^{\geq 1}(\Bbbk) = 0$.
\end{example}

对于较高的次数 $n$, 按原始定义计算 $\HHm^n(M)$ 和 $\HHm_n(M)$ 一般是困难的. 之后的例 \ref{eg:HH-Tor} 和 \ref{eg:HH-Ext} 将分别说明
\[ \HHm_n(M) \simeq \Tor^{R^e}_n(M, R), \quad \HHm^n(M) \simeq \Ext_{R^e}^n(R, M), \]
前提是 $R$ 作为 $\Bbbk$-模分别是平坦和投射的; 届时就能以更简单的复形来计算 $\HHm_n$ 和 $\HHm^n$ 的一些特例. 对于一般的 $R$, 我们将在\CHref{sec:simplicial}的习题部分介绍如何将 $\HHm_n$ 和 $\HHm^n$ 诠释为相对 $\Tor$ 和相对 $\Ext$.

\begin{example}[零次情形: 中心和余中心]
	\index{zhongxin}
	\index{yuzhongxin@余中心 (cocenter)}
	首先来考察 $\HHm^0(M)$. 据定义可见 $d^0: M = C^0(R, M) \to C^1(R, M) = \Hom_{\Bbbk}(R, M)$ 映 $m$ 为 $[r \mapsto rm - mr]$. 因此
	\[ \HHm^0(M) = \left\{ m \in M: \forall r \in R, \; rm=mr \right\} \]
	右式可以合理地称为 $M$ 的\emph{中心}; 当 $M = R$ 时, 它无非是环论意义的中心.
	
	其次考虑 $\HHm_0(M)$. 记 $[M, R]$ 为形如 $mr-rm$ 的元素在 $M$ 中生成的 $\Bbbk$-子模, 其中 $m \in M$ 而 $r \in R$. 由于 $d_1(m|r) = mr - rm$, 我们有
	\[ \Image\left[ d_1: M \otimes R \to M \right] = [M, R], \quad \HHm_0(M) = M/[M, R]. \]
	特别地, $M=R$ 的特例给出所有 $[r', r] := r'r - rr'$ 在 $R$ 中生成的子模 $[R, R]$, 对应的 $\Bbbk$-模 $R/[R,R]$ 称为 $R$ 的\emph{余中心}. 一切满足 $\varphi(rr') = \varphi(r'r)$, 亦即性质近乎``迹'' 的 $\Bbbk$-模同态 $\varphi: R \to N$ 都唯一地通过 $R/[R, R] = \HHm_0(R)$ 分解.
\end{example}

\begin{example}[一次情形: 求导]\label{eg:HH1}
	\index[sym1]{Der@$\mathrm{Der}_{\Bbbk}(R, M)$}
	\index[sym1]{InnDer@$\mathrm{Inn}_{\Bbbk}(R, M)$}
	现在来探讨 $\HHm^1(M)$. 记 $[r, m] := rm - mr$, 则
	\begin{align*}
		\Ker(d^1) & = \left\{ D \in \Hom_{\Bbbk}(R, M) : \forall r_1, r_2 \in R, \; r_1 D(r_2) - D(r_1 r_2) + D(r_1) r_2 = 0 \right\}, \\
		\Image(d^0) & = \left\{ [\cdot, m] \in \Hom_{\Bbbk}(R, M) : m \in M \right\}.
	\end{align*}

	关于 $\Ker(d^1)$ 的条件可以改写成 Leibniz 律 $D(r_1 r_2) = r_1 D(r_2) + D(r_1) r_2$. 具此性质的 $\Bbbk$-模同态 $D$ 应当设想为取值在 $M$ 上的求导运算, 它们构成的 $\Bbbk$-模记为 $\Der_{\Bbbk}(R, M)$; 其中形如 $[\cdot, m]$ 的元素构成子模 $\mathrm{Inn}_{\Bbbk}(R, M)$, 于是商模
	\[ \HHm^1(M) = \Der_{\Bbbk}(R, M) / \mathrm{Inn}_{\Bbbk}(R, M) \]
	分类了 $R$ 上所有取值在 $M$ 的求导运算, 精确到 $\mathrm{Inn}_{\Bbbk}(R, M)$.
	
	现在假设 $R$ 交换以诠释 $\HHm_1(M)$. 我们需要一些准备: 定义以符号 $\widetilde{\dd r}$ 为基的自由 $R$-模 $\bigoplus_{r \in R} R \widetilde{\dd r}$, 再定义由以下元素生成的子模 $N$:
	\[ \widetilde{\dd(r+r')} - \widetilde{\dd r} - \widetilde{\dd r'}, \quad \widetilde{\dd tr} - t \widetilde{\dd r}, \quad \widetilde{\dd rr'} - r \widetilde{\dd r'} - r' \widetilde{\dd r}, \]
	其中 $r, r' \in R$ 而 $t \in \Bbbk$. 由此定义 $R$-模
	\begin{align*}
		\Omega_{R|\Bbbk} & := \bigoplus_{r \in R} R \widetilde{\dd r} \bigg/ N \\
		& = \sum_{r \in R} R \dd r, \quad \dd r := \widetilde{\dd r} \;\text{的像},
	\end{align*} \index{Kahler weifenxingshi@Kähler 微分形式 (Kähler differentials)} \index[sym1]{OmegaRk@$\Omega_{R\vert\Bbbk}$}
	称之为 $\Bbbk$-代数 $R$ 的 Kähler 微分形式模; 它由以下泛性质刻画:
	\[\begin{tikzcd}[row sep=tiny]
		\Hom_R(\Omega_{R|\Bbbk}, M) \arrow[leftrightarrow, r, "\sim"] & \Der_{\Bbbk}(R, M) \\
		\varphi \arrow[mapsto, r] & {\left[ r \mapsto \varphi(\dd r) \right]} \\
		{\left[ \dd r \mapsto D(r) \right]} & D \arrow[mapsto, l].
	\end{tikzcd} \qquad M: R\text{-模}, \]
	请读者直接验证. 此同构表明 $r \mapsto \dd r$ 给出的 $R \to \Omega_{R|\Bbbk}$ (对应于 $\varphi = \identity$) 是``泛求导''. 相关内容理应是交换环论的主题, 点到为止.

	回到 $\HHm_1(M)$. 继续要求 $R$ 交换. 设 $M$ 为 $R$-模, 按 $rmr' := (rr')m$ 作成 $(R, R)$-双模. 由此立见 $d_1: M \otimes R \to M$ 为 $0$, 而 $d_2: M \otimes R^{\otimes 2} \to M \otimes R$ 的像由形如 $(rm|r') - (m|rr') + (r'm|r)$ 的元素生成. 综上可得双向的 $\Bbbk$-模同态
	\[\begin{tikzcd}[row sep=tiny]
		(M \otimes R) / \Image(d_2) \arrow[shift right, r] & M \dotimes{R} \Omega_{R|\Bbbk} \arrow[shift right, l] \\
		(m|r) + \Image(d_2) \arrow[mapsto, r] & m \otimes \dd r \\
		(r'm|r) + \Image(d_2) & m \otimes r' \dd r \arrow[mapsto, l] .
	\end{tikzcd}\]
	既有 $\Image(d_2)$ 和 $\Omega_{R|\Bbbk}$ 的描述, 例行的验证表明同态良定义而且互逆; 事实上, 它们还是 $R$-模的同构. 这就在 $R$ 交换的情形给出 $\HHm_1(M) \simeq M \dotimes{R} \Omega_{R|\Bbbk}$. 特别地, $\HHm_1(R) \simeq \Omega_{R|\Bbbk}$.
\end{example}

Hochschild 同调和上同调有丰富的内涵, 它与求导和微分形式的联系并非偶然. 习题将给出更多针对 Hochschild 上同调的诠释.

回到 Hochschild 链复形. 考虑到 $R^e$ 在 $\mathsf{B}_n R$ 和在 $M$ 上的作用, 直观的思路应是将 $C_n(R, M)$ 的元素 $(m|r_1| \cdots|r_n)$ 排列成环形
\begin{center}\begin{tikzpicture}[scale=1.2]
		\node[rotate=30] at (120:1) {$r_n$};
		\node[rotate = 0]  at (90:1) {$m$};
		\node[rotate=-15] at (60:1) {$r_1$};
		\node[rotate=-45]  at (30:1) {$r_2$};
		\draw (135:0.8) -- (135:1.2);
		\draw (105:0.8) -- (105:1.2);
		\draw (75:0.8) -- (75:1.2);
		\draw(45:0.8) -- (45:1.2);
		\draw[loosely dotted, thick] (10:1) arc [start angle=10, end angle=-210, radius=1];
\end{tikzpicture}\end{center}
于是 $d_n(m|r_1| \cdots |r_n)$ 即以 $n+1$ 种方式抽杠, 交错加总的产物. 对于特例 $M=R$, 图像有明显的旋转对称性. 从代数的视角, 我们对每个 $n \in \Z_{\geq 0}$ 定义 $\Bbbk$-模同态
\[\begin{tikzcd}[row sep=tiny]
	t_n: R^{\otimes (n+1)} \arrow[r] & R^{\otimes (n+1)} \\
	(r_0 | \cdots | r_n) \arrow[mapsto, r] & (-1)^n \cdot (r_n | r_0 | \cdots | r_{n-1}) ;
\end{tikzcd}\]
这导致 $t_n^{n+1} = \identity$. 它体现对称性.

旋转对称性通向\emph{循环同调}理论. 从历史的角度, 研究循环同调至少有两个动机. 一是为了寻求 de Rham 理论在非交换情形的类比. 二是着眼于 K-理论的研究与应用, 包括指标定理的种种推广. 本节仅将 Hochschild 同调, 上同调与循环同调作为轻便的教具, 目的在熟悉复形操作, 浅尝辄止. 有心深入的读者可参阅专著, 如 \cite{Lo98, Wi19} 等.

以下提及的复形 (或双复形) 均默认为链复形 (或链双复形).

对所有 $n \geq 0$, 命 $N_n := \identity + t_n + \cdots + (t_n)^n \in \End_{\Bbbk}(R^{\otimes (n+1)})$. 定义\emph{循环双复形} $\mathrm{CC}(R) = (\mathrm{CC}(R)_{p,q})_{(p,q) \in \Z^2}$ 为双复形 \index{xunhuanshuangfuxing@循环双复形 (cyclic double complex)} \index[sym1]{CCR@$\mathrm{CC}(R)$}
\[\begin{tikzcd}[
	/tikz/execute at end picture={
		\node[rectangle, draw, fit=(NW) (SE), inner xsep=1.5em, inner ysep=1.5em] {};
	}]
	\vdots & |[alias=NW]| & \vdots \arrow[d, "{d_{q+1}}"] & \vdots \arrow[d, "{d'_{q+1}}"] & \\
	q & \cdots & R^{\otimes (q+1)} \arrow[d, "{d_q}"] \arrow[l, "{N_q}"] & R^{\otimes (q+1)} \arrow[d, "{d'_q}"] \arrow[l, "{\identity - t_q}"] & \cdots \arrow[l] \\
	q-1 & \cdots & R^{\otimes q} \arrow[d] \arrow[l, "{N_{q-1}}"] & R^{\otimes q} \arrow[d] \arrow[l, "{\identity - t_{q-1}}"] & \cdots \arrow[l] \\
	\vdots & & \vdots & \vdots & |[alias=SE]| \\
	& \cdots & p: \text{偶数} & p+1: \text{奇数} & \cdots
\end{tikzcd}\]
各项都是 $\Bbbk$-模, $q < 0$ 的项全设为 $0$; 态射 $d_q$ 和 $d'_q$ 的定义见诸 \eqref{eqn:HH-cplx-d}. 观察到
\begin{compactitem}
	\item 偶数列是 \eqref{eqn:HH-cplx} 的 Hochschild 链复形 $C_\bullet(R, R)$.
	\item 易证奇数列是增广杠复形 $\mathsf{B}' R$, 但平移下标使它始于 $0$ 次项. 于是引理 \ref{prop:HH-bar-exactness} 蕴涵奇数列皆正合, 事实上它们在 $\cate{K}(\Bbbk\dcate{Mod})$ 中为 $0$.
\end{compactitem}

为了说明 $\mathrm{CC}(R)$ 确实是双复形, 需要以下观察. 论证是初等而且有趣的, 而且没有本质上的困难, 故留作本章习题.

\begin{lemma}\label{prop:CC-cplx}
	设 $R$ 为 $\Bbbk$-代数. 对所有 $q \in \Z_{\geq 1}$, 有 $\Hom_{\Bbbk}\left( R^{\otimes (q+1)}, R^{\otimes q}\right)$ 中的等式
	\[ d_q (\identity - t_q) = (\identity - t_{q-1}) d'_q, \quad d'_q N_q = N_{q-1} d_q. \]
\end{lemma}

对所有双复形 $C = (C_{i,j})_{(i, j) \in \Z^2}$ 和 $m \in \Z$, 定义其横向\footnote{本书的惯例是以下标 $\mathrm{I}$ 代表横向操作, 以 $\mathrm{II}$ 代表纵向操作.}平移 $C_{\mathrm{I}}[m]$ 为双复形 $(C_{m+i, j})_{i,j}$. 对所有 $p \in \Z$, 定义简单粗暴的横向截断函子 $\sigma_{\mathrm{I}, \leq p} C$ (或 $\sigma_{\mathrm{I}, \geq p} C$), 它将满足 $i > p$ (或 $i < p$) 的 $C_{i,j}$ 代换为 $0$, 其余不变. 因此我们有双复形的短正合列 (注意顺序!)
\begin{equation}\label{eqn:homology-I-brute-truncation}
	0 \to \sigma_{\mathrm{I}, \leq p} C \to C \to \sigma_{\mathrm{I}, \geq p+1} C \to 0.
\end{equation}
此外, 对所有 $a \leq b$ 定义函子 $\sigma_{\mathrm{I}, [a, b]} := \sigma_{\mathrm{I}, \leq b} \sigma_{\mathrm{I}, \geq a} = \sigma_{\mathrm{I}, \geq a} \sigma_{\mathrm{I}, \leq b}$.

将这些函子应用于 $\mathrm{CC}(R)$, 再取全复形 $\tot_{\Pi}$ (定义 \ref{def:total-cplx}), 便抵达以下定义.

\begin{definition}[B.\ Feigin, B.\ Tsygan; A.\ Connes]\label{def:HC}
	\index{xunhuantongdiao@循环同调 (cyclic homology)}
	\index{zhouqixunhuantongdiao@周期循环同调 (periodic cyclic homology)}
	\index[sym1]{HPHC@$\mathrm{HP}_n$, $\mathrm{HC}_n$, $\mathrm{HP}^n$, $\mathrm{HC}^n$}
	对 $\Bbbk$-代数 $R$ 和任意 $n \in \Z$, 定义
	\begin{align*}
		\mathrm{HP}_n(R) & := \Hm_n\left( \tot_{\Pi}(\mathrm{CC}(R)) \right) & \text{(周期循环同调)}, \\
		\mathrm{HC}_n(R) & := \Hm_n\left( \tot\left(\sigma_{\mathrm{I}, \geq 0} \mathrm{CC}(R)\right) \right) & \text{(循环同调)}.
%		\mathrm{HC}^-_n(R) & := \Hm_n\left( \tot_{\Pi}\left(\sigma_{\mathrm{I}, \leq 1} \mathrm{CC}(R)\right) \right) & \text{(负循环同调)}
	\end{align*}
	它们都是函子 $\Bbbk\dcate{Alg} \to \Bbbk\dcate{Mod}$.
\end{definition}

由于 $\sigma_{\mathrm{I}, \geq 0} \mathrm{CC}(R)$ 落在第一象限, 其全复形仅涉及有限直和 $\bigoplus_{\substack{p+q=n \\ p,q \geq 0}} \mathrm{CC}_{p,q}(R)$, 相应的 $\tot$ 不必加下标.

按定义立见 $\mathrm{HC}_{< 0}(R) = 0$, 而 $\mathrm{HC}_0(R)$ 等于 $R$ 对 $\Image[d_1: R^{\otimes 2} \to R]$ 和 $\Image[\identity - t_0] = 0$ 取商的产物, 亦即 $R/[R, R]$. 习题将给出更多特例的计算.

循环复形具备周期性 $\mathrm{CC}(R) = \mathrm{CC}(R)_{\mathrm{I}}[-2]$, 导致同构 $\mathrm{HP}_n(R) \rightiso \mathrm{HP}_{n-2}(R)$. 对于循环同调, 对应的场景是``左移两格''的满态射
\begin{align*}
	\sigma_{\mathrm{I}, \geq 0} \mathrm{CC}(R) & \to \left(\sigma_{\mathrm{I}, \geq 0} \mathrm{CC}(R) \right)_{\mathrm{I}}[-2] \\
	\mathrm{CC}(R)_{p,q} & \to \begin{cases} \mathrm{CC}(R)_{p-2, q} \; \text{(恒等)}, & p \geq 2 \\ 0, & 0 \leq p < 2. \end{cases}
\end{align*}
它的核由 $\mathrm{CC}(R)$ 的第 $0, 1$ 列, 亦即子双复形 $\sigma_{\mathrm{I}, [0,1]} \mathrm{CC}(R)$ 给出. 这就定出循环同调的 Connes 周期算子 $S: \mathrm{HC}_n(R) \to \mathrm{HC}_{n-2}(R)$. \index{Connes zhouqisuanzi@Connes 周期算子 (Connes periodicity operator)}

\begin{theorem}[A.\ Connes]\label{prop:Connes-exact}
	设 $R$ 为 $\Bbbk$-代数. 我们有典范的长正合列
	\[ \cdots \xrightarrow{S} \mathrm{HC}_{n-1}(R) \xrightarrow{B} \HHm_n(R) \xrightarrow{I} \mathrm{HC}_n(R) \xrightarrow{S} \mathrm{HC}_{n-2}(R) \to \cdots , \]
	其中 $S$ 是 Connes 周期算子, $I$ 由嵌入 $C_\bullet(R, R) \xrightarrow{\text{第 $0$ 列}} \sigma_{\mathrm{I}, \geq 0} \mathrm{CC}(R)$ 给出, 而 $B$ 是适当的连接同态.
\end{theorem}
\begin{proof}
	第一步是运用 \eqref{eqn:homology-I-brute-truncation} 得到双复形的短正合列
	\[\begin{tikzcd}
		0 \arrow[r] & \text{第 $0$ 列} \arrow[r] & \sigma_{\mathrm{I}, [0,1]} \mathrm{CC}(R) \arrow[r] & \text{第 $1$ 列} \arrow[r] & 0 .
	\end{tikzcd}\]
	取全复形只涉及有限直和; 逐次考察, 可见其产物仍是短正合列
	\[\begin{tikzcd}
		0 \arrow[r] & \text{第 $0$ 列} \arrow[r] & \tot\left( \sigma_{\mathrm{I}, [0,1]} \mathrm{CC}(R)\right) \arrow[r] & \text{第 $1$ 列} \arrow[r] & 0. \\
		& C_\bullet(R, R) \arrow[u, "\sim" sloped] & & \mathsf{B}'R \arrow[u, "\sim" sloped] &
	\end{tikzcd}\]
	已知 $\mathsf{B}' R$ 正合, 应用命题 \ref{prop:long-exact-sequence-ses} 的长正合列遂知 $C_\bullet(R, R) \to  \tot\left( \sigma_{\mathrm{I}, [0,1]} \mathrm{CC}(R)\right)$ 是复形的拟同构.
	
	接着考虑定义 $S$ 时提及的短正合列
	\[\begin{tikzcd}
		0 \arrow[r] & \sigma_{\mathrm{I}, [0,1]} \mathrm{CC}(R) \arrow[r] & \sigma_{\mathrm{I}, \geq 0} \mathrm{CC}(R) \arrow[r] & \left( \sigma_{\mathrm{I}, \geq 0} \mathrm{CC}(R)\right)_{\mathrm{I}}[-2] \arrow[r] & 0 .
	\end{tikzcd}\]
	取全复形后仍是短正合列, 而根据上一步, 三个全复形的 $n$ 次上同调分别等同于 $\HHm_n(R)$, $\mathrm{HC}_n(R)$ 和 $\mathrm{HC}_{n-2}(R)$. 明所欲证.
\end{proof}

\begin{remark}
	\index{xunhuanshuangfuxing}
	\index{zhouqixunhuanshangtongdiao@周期循环上同调 (periodic cyclic cohomology)}
	\index{xunhuanshangtongdiao@循环上同调 (cyclic cohomology)}
	\index[sym1]{HPHC}
	以 $\mathrm{CC}'(R)^{p,q} := \Hom_{\Bbbk}\left(\mathrm{CC}(R)_{p,q}, \Bbbk\right)$ 定义上链复形意义下的循环双复形 $\mathrm{CC}'(R)$; 按定义 \ref{def:HC} 如法炮制, 得到
	\begin{align*}
		\mathrm{HP}^n(R) & := \Hm^n\left( \tot_{\oplus}(\mathrm{CC}'(R)) \right) & \text{(周期循环上同调)}, \\
		\mathrm{HC}^n(R) & := \Hm^n\left( \tot\left(\sigma_{\mathrm{I}}^{\geq 0} \mathrm{CC}'(R)\right) \right) & \text{(循环上同调)}
	\end{align*}
	等等, 其中 $\sigma_{\mathrm{I}}^{\geq 0}$ 仍代表横向的暴力截断函子. 这时定理 \ref{prop:Connes-exact} 的长正合列仍有对应版本
	\[ \cdots \xrightarrow{S} \mathrm{HC}^{n+1}(R) \xrightarrow{I} \HHm^{n+1}(R) \xrightarrow{B} \mathrm{HC}^n(R) \xrightarrow{S} \mathrm{HC}^{n+2}(R) \to \cdots \]
	对应的 Connes 周期算子 $S$ 来自于``右移两格''的单态射, 亦即
	\[ \sigma_{\mathrm{I}}^{\geq 0} \mathrm{CC}'(R) \to \left(\sigma_{\mathrm{I}}^{\geq 0} \mathrm{CC}'(R) \right)_{\mathrm{I}}[2]. \]
\end{remark}

随着之后掌握的工具渐多, 我们还会反复回归 Hochschild 同调, 上同调以及循环同调, 上同调的讨论.

\section{截断函子}\label{sec:truncation-functors}
本节伊始, 选定加性范畴 $\mathcal{A}$.

\begin{definition}\label{def:cplx-cat-variant}
	\index{fuxing!上有界, 下有界, 有界 (bounded above, bounded below, bounded)}
	\index[sym1]{CbA@$\cate{C}^+(\mathcal{A})$, $\cate{C}^-(\mathcal{A})$, $\cate{C}^{\bdd}(\mathcal{A})$}
	\index[sym1]{CstA@$\cate{C}^{[s,t]}(\mathcal{A})$, $\cate{C}^{\geq s}(\mathcal{A})$, $\cate{C}^{\leq t}(\mathcal{A})$}
	对复形 $X \in \Obj(\cate{C}(\mathcal{A}))$ 采用以下术语, 并标注相应的全子范畴:
	\begin{center}\begin{tabular}{|c|c|c|c|}\hline
		术语 & 有界 & 下有界 & 上有界 \\
		条件 & $|n| \gg 0 \implies X^n = 0 $ & $n \ll 0 \implies X^n = 0$ & $n \gg 0 \implies X^n = 0$ \\
		全子范畴 & $\cate{C}^{\bdd}(\mathcal{A})$ & $\cate{C}^+(\mathcal{A})$ & $\cate{C}^-(\mathcal{A})$ \\ \hline
	\end{tabular}\end{center}
	
	推而广之, 对于 $-\infty \leq s \leq t \leq +\infty$, 我们记 $\cate{C}^{[s,t]}(\mathcal{A})$ 为满足 $n \notin [s, t] \implies X^n = 0$ 的复形构成的全子范畴, 并且记
	\[ \cate{C}^{\geq s}(\mathcal{A}) := \cate{C}^{[s, +\infty]}(\mathcal{A}), \quad \cate{C}^{\leq t}(\mathcal{A}) := \cate{C}^{[-\infty, t]}(\mathcal{A}). \]
\end{definition}

这些全子范畴都是加性范畴, 而 $\cate{C}^+(\mathcal{A})$, $\cate{C}^-(\mathcal{A})$ 和 $\cate{C}^{\bdd}(\mathcal{A}) = \cate{C}^+(\mathcal{A}) \cap \cate{C}^-(\mathcal{A})$ 还是子 Abel 范畴. 平移函子 $[n]$ 保持 $\cate{C}^+(\mathcal{A})$, $\cate{C}^-(\mathcal{A})$ 和 $\cate{C}^{\bdd}(\mathcal{A})$ 不变, 但映 $\cate{C}^{[a,b]}(\mathcal{A})$ 为 $\cate{C}^{[a-n, b-n]}(\mathcal{A})$.

此外, $\mathcal{A}$ 自然地等同于 $\cate{C}^{\geq 0}(\mathcal{A}) \cap \cate{C}^{\leq 0}(\mathcal{A})$.

对于任意 $\star \in \{+, -, \bdd\}$, 态射同伦的概念 (定义 \ref{def:homotopy}) 可以限制到 $\cate{C}^{\star}(\mathcal{A})$ 上. 定义 \ref{def:cplx-homotopy-cat} 因而有如下推广.

\begin{definition}\label{def:cplx-homotopy-cat-variant}
	\index[sym1]{KbA@$\cate{K}^+(\mathcal{A})$, $\cate{K}^-(\mathcal{A})$, $\cate{K}^{\bdd}(\mathcal{A})$}
	对于 $\star \in \{+, -, \bdd\}$, 我们有 $\cate{K}(\mathcal{A})$ 的加性全子范畴 $\cate{K}^\star(\mathcal{A})$, 它对平移函子保持封闭.
\end{definition}

今起要求 $\mathcal{A}$ 为 Abel 范畴. 我们将探讨如何将复形 $X$ 的 $< n$ (或 $> n$) 次部分截断. 朴素的思路是将其余各项全换为 $0$. 此法简则简矣, 却打乱了上同调, 是故我们引入更精细的版本.

\begin{definition}[截断函子]\label{def:truncation-cplx}
	\index{jieduanhanzi@截断函子 (truncation functor)}
	\index[sym1]{taun@$\tau^{\leq n}$, $\tau^{\geq n}$}
	\index[sym1]{tautilden@$\tilde{\tau}^{\leq n}$, $\tilde{\tau}^{\geq n}$}
	设 $\mathcal{A}$ 为 Abel 范畴, $n \in \Z$. 给定复形 $X$, 命
	\[\begin{tikzcd}
		[column sep=small, every cell/.append style = {font = \small}]
		\tau^{\leq n} X \arrow[phantom, r, "" {name=A, yshift=2ex}] \arrow[hookrightarrow, d] & \cdots \arrow[r] & X^{n-2} \arrow[r] & X^{n-1} \arrow[r] & \Ker(d^n) \arrow[r] \arrow[hookrightarrow, d] & 0 \arrow[r] & 0 \arrow[r] & \cdots \\
		\tilde{\tau}^{\leq n} X \arrow[hookrightarrow, d] & \cdots \arrow[r] & X^{n-2} \arrow[r] & X^{n-1} \arrow[r] & X^n \arrow[r] & \Coim(d^n) \arrow[r] \arrow[hookrightarrow, d] & 0 \arrow[r] & \cdots \\
		X \arrow[twoheadrightarrow, d] & \cdots \arrow[r] & X^{n-2} \arrow[r] & X^{n-1} \arrow[r] \arrow[twoheadrightarrow, d] & X^n \arrow[r] & X^{n+1} \arrow[r] & X^{n+2} \arrow[r] & \cdots \\
		\tilde{\tau}^{\geq n} X \arrow[twoheadrightarrow, d] & \cdots \arrow[r] & 0 \arrow[r] & \Image(d^{n-1}) \arrow[r] & X^n \arrow[r] \arrow[twoheadrightarrow, d] & X^{n+1} \arrow[r] & X^{n+2} \arrow[r] & \cdots \\
		\tau^{\geq n} X \arrow[phantom, r, "" {name=B, yshift=-2ex}] & \cdots \arrow[r] & 0 \arrow[r] & 0 \arrow[r] & \Coker(d^{n-1}) \arrow[r] & X^{n+1} \arrow[r] & X^{n+2} \arrow[r] & \cdots
		\arrow[dash, thick, from=A, to=B]
	\end{tikzcd}\]
	其中省略的项是自明的, 垂直方向仅标注除 $\identity$ 和 $0$ 之外的态射. 这给出左侧各复形之间的态射; 注意到 $\Coim \simeq \Image$.
\end{definition}

易见 $\tau^{\leq n}$, $\tilde{\tau}^{\leq n}$, $\tau^{\geq n}$, $\tilde{\tau}^{\geq n}$ 都是加性函子. 我们有  $\tau^{\leq n} = [-n] \circ \tau^{\leq 0} \circ [n]$, 对于 $\tilde{\tau}^{\leq n}$, $\tau^{\geq n}$, $\tilde{\tau}^{\geq n}$ 亦同.

此外, 若 $m \leq n$, 则有自明的满态射 $\tau^{\geq m} X \twoheadrightarrow \tau^{\geq n} X$ 和 $\tilde{\tau}^{\geq m} X \twoheadrightarrow \tilde{\tau}^{\geq n} X$, 以及自明的单态射 $\tau^{\leq m} X \hookrightarrow \tau^{\leq n} X$ 和 $\tilde{\tau}^{\leq m} X \hookrightarrow \tilde{\tau}^{\leq n} X$.

\begin{lemma}\label{prop:truncation-ses}
	上述诸态射对所有 $k \in \Z$ 诱导 $\mathcal{A}$ 中的同构
	\begin{align*}
		\Hm^k\left( \tau^{\leq n} X\right) & \rightiso \Hm^k\left( \tilde{\tau}^{\leq n} X\right) \simeq \begin{cases} \Hm^k(X), & k \leq n \\ 0, & k > n \end{cases}, \\
		\Hm^k\left( \tilde{\tau}^{\geq n} X\right) & \rightiso \Hm^k\left( \tau^{\geq n} X\right) \simeq \begin{cases} \Hm^k(X), & k \geq n \\ 0, & k < n \end{cases},
	\end{align*}
	以及 $\cate{C}(\mathcal{A})$ 中的短正合列
	\begin{equation*}\begin{gathered}
		0 \to \tilde{\tau}^{\leq n-1} X \to \tau^{\leq n} X \to \Hm^n(X)[-n] \to 0, \\
		0 \to \Hm^n(X)[-n] \to \tau^{\geq n} X \to \tilde{\tau}^{\geq n+1} X \to 0, \\
		0 \to \tau^{\leq n} X \to X \to \tilde{\tau}^{\geq n+1} X \to 0, \\
		0 \to \tilde{\tau}^{\leq n-1} X \to X \to \tau^{\geq n} X \to 0, \\
		0 \to \tau^{\leq n} X \to \tilde{\tau}^{\leq n} X \to \Cone\left(\identity_{\Image(d_X^n) [-n-1]} \right) \to 0,
	\end{gathered}\end{equation*}
	此处 $\Hm^n(X)[-n]$ 按约定 \ref{con:concentrated-cplx} 理解. 所有态射对 $X$ 皆具函子性.
\end{lemma}
\begin{proof}
	态射 $\tilde{\tau}^{\leq n-1} X \to \tau^{\leq n} X$ 在 $n$ 次项是 $\Image(d^{n-1}) \hookrightarrow \Ker(d^n)$, 在其他项为 $\identity$; 态射 $\tau^{\leq n} X \to \Hm^n(X)[-n]$ 在 $n$ 次项是
	\[ \Ker(d^n) \twoheadrightarrow \Ker(d^n)/\Image(d^{n-1}) = \Hm^n(X). \]
	
	态射 $\tau^{\geq n} X \to \tilde{\tau}^{\geq n+1} X$ 在 $n$ 次项是由 $d^n$ 诱导的 $\Coker(d^{n-1}) \twoheadrightarrow \Image(d^n)$, 在其他项为 $\identity$; 态射 $\Hm^n(X)[-n] \to \tau^{\geq n} X$ 在 $n$ 次项是
	\[ \Ker(d^n)/\Image(d^{n-1}) \hookrightarrow X^n/\Image(d^{n-1}) = \Coker(d^{n-1}). \]
	
	剩下的验证全是例行公事, 交给读者练习.
\end{proof}

因此 $\tau^{\leq n}$, $\tilde{\tau}^{\leq n}$ (或 $\tau^{\geq n}$, $\tilde{\tau}^{\geq n}$) 的效果确实是将次数 $> n$ (或 $< n$) 的上同调截断.

\begin{remark}\label{rem:truncation-sigma}
	对于复形 $X \in \Obj(\cate{C}(\mathcal{A}))$, 套用定义--命题 \ref{def:sigma} 的函子 $\sigma$ 可得 $\sigma X \in \Obj(\cate{C}(\mathcal{A}^{\opp})) = \Obj(\cate{C}(\mathcal{A}^{\opp})^{\opp})$. 精确到一些在此无害的正负号, 这相当于在复形中倒转箭头再翻转标号. 鉴于 $\Ker$ 和 $\Coker$ 的对偶性, $\tau^{\geq n}(X)$ 在此操作下对应到 $\tau^{\leq -n}(\sigma X) \in \Obj(\cate{C}(\mathcal{A}^{\opp}))$, 如此等等. 同理, 因为 $\Image$ 和 $\Coim$ 对偶, $\tilde{\tau}^{\geq n}(X) \in \Obj(\cate{C}(\mathcal{A}))$ 对应于 $\tilde{\tau}^{\leq -n}(\sigma X) \in \Obj(\cate{C}(\mathcal{A}^{\opp}))$.
\end{remark}

相较于 $\tilde{\tau}^{\leq n}$ 和 $\tilde{\tau}^{\geq n}$, 函子 $\tau^{\leq n}$ 和 $\tau^{\geq n}$ 有一些更方便的性质. 首先是伴随关系.

\begin{proposition}\label{prop:truncation-adjunction}
	对所有 $n \in \Z$, 我们有伴随关系
	\begin{gather*}
		\Hom_{\cate{C}(\mathcal{A})}(X, Y) \simeq \Hom_{\cate{C}^{\leq n}(\mathcal{A})}\left(X, \tau^{\leq n} Y \right), \quad X \in \Obj(\cate{C}^{\leq n}(\mathcal{A})) \\ \Hom_{\cate{C}(\mathcal{A})}(X, Y) \simeq \Hom_{\cate{C}^{\geq n}(\mathcal{A})}\left(\tau^{\geq n} X, Y \right), \quad Y \in \Obj(\cate{C}^{\geq n}(\mathcal{A})).
	\end{gather*}
\end{proposition}
\begin{proof}
	就截断的定义看是明白的. 请读者写下对应的单位和余单位态射: 它们或者是定义 \ref{def:truncation-cplx} 写下的态射, 或者是 $\identity$.
\end{proof}

其次, $\tau^{\leq n}$ 和 $\tau^{\geq n}$ 下降到 $\cate{K}(\mathcal{A})$ 的层面.

\begin{definition-proposition}\label{def:K-truncation-functors}
	\index[sym1]{taun}
	对所有 $n \in \Z$, 函子 $\tau^{\leq n}$ (或 $\tau^{\geq n}$) 自然地诱导从 $\cate{K}(\mathcal{A})$ 到其自身的函子, 仍记为 $\tau^{\leq n}$ (或 $\tau^{\geq n}$).
\end{definition-proposition}
\begin{proof}
	设 $h \in \Hom^{-1}(X, Y)$. 对 $\tau^{\leq n}$ 的情形, 取 $\overline{h}^n: \Ker(d_X^n) \to Y^{n-1}$ 为 $h^n$ 和 $\Ker(d_X^n) \to X^n$ 的合成, 依照下图定义 $\overline{h} \in \Hom^{-1}\left( \tau^{\leq n} X, \tau^{\leq n} Y \right)$
	\[\begin{tikzcd}
		\cdots \arrow[r] & X^{n-2} \arrow[r] \arrow[ld, "h^{n-2}" description] & X^{n-1} \arrow[r] \arrow[ld, "{h^{n-1}}" description] & \Ker(d_X^n) \arrow[r] \arrow[ld, "{\overline{h}^n}" description] & 0 \arrow[r] \arrow[ld] & 0 \arrow[r] \arrow[ld] & \cdots \\
		\cdots \arrow[r] & Y^{n-2} \arrow[r] & Y^{n-1} \arrow[r] & \Ker(d_Y^n) \arrow[r] & 0 \arrow[r] & 0 \arrow[r] & \cdots ;
	\end{tikzcd}\]
	不难看出 $\tau^{\leq n}\left( d^{-1} h \right) = d^{-1} \overline{h}$.
	
	至于 $\tau^{\geq n}$ 的情形, 改取 $\underline{h}^{n+1}: X^{n+1} \to \Coker(d_Y^{n-1})$ 为 $h^{n+1}$ 和 $Y^n \to \Coker(d_Y^{n-1})$ 的合成, 以此定义 $\underline{h} \in \Hom^{-1}\left( \tau^{\geq n} X, \tau^{\geq n} Y \right)$; 其余思路类似. 两种情况当然是对偶的.
\end{proof}

以下简单而关键的性质直接导自定义 \ref{def:truncation-cplx}.

\begin{proposition}\label{prop:truncate-twice}
	对所有整数 $a < b$, $n$ 和 $X \in \Obj(\mathcal{A})$ 都有
	\begin{gather*}
		\tau^{\leq a} \tau^{\geq b}(X) = 0 = \tau^{\geq b} \tau^{\leq a}(X), \\
		\tau^{\leq n }\tau^{\geq n}(X) \simeq \Hm^n(X)[-n] \simeq \tau^{\geq n} \tau^{\leq n}(X) \quad \text{(典范同构).}
	\end{gather*}
\end{proposition}

\section{双复形的上同调}\label{sec:double-cplx-coh}
本节旨在探讨双复形 $X$ 沿水平或垂直方向的上同调, 以及它们和全复形的上同调之间的关系. 相关结果将用于研究双函子的导出函子, 论证取法 \cite[\S 12.5]{KS06}.

令 $\mathcal{A}$ 为 Abel 范畴. 一如既往地, 当 $\mathcal{A}$ 为 $\Bbbk$-线性 Abel 范畴时, 所有结果都有相应的推广.

回忆到定义 \ref{def:bicplx-F1} (或更显豁的图表 \eqref{eqn:bicplx-diagram}) 引入了一对可逆加性函子
\[\begin{tikzcd}
	\cate{C}^2(\mathcal{A}) \arrow[r, shift left, "{F_{\mathrm{I}}}"] \arrow[r, shift right, "{F_{\mathrm{II}}}"'] & \cate{C}(\cate{C}(\mathcal{A})) ,
\end{tikzcd}\]
以此对任意双复形 $X$ 定义 (以下 $p, q, n \in \Z$):
\begin{align*}
	\Hm_{\mathrm{I}}^p(X) & := \Hm^p \circ F_{\mathrm{I}}
	= \text{复形}\; \left[ \begin{tikzcd}[row sep=small]
		\vdots \\
		\Hm^p(X^{\bullet, q+1}, \dhori) \arrow[u, "{\dvert}" inner sep=0.6em] \\
		\Hm^p(X^{\bullet, q}, \dhori) \arrow[u, "{\dvert}" inner sep=0.6em] \\
		\vdots \arrow[u, "{\dvert}" inner sep=0.6em]
	\end{tikzcd}\right] \quad \text{(参见命题 \ref{prop:abelian-cat-cplx})}, \\
	\Hm_{\mathrm{II}}^q(X) & := \Hm^q \circ F_{\mathrm{II}} \\
	& = \text{复形}\; \left[ \cdots \xrightarrow{\dhori} \Hm^q(X^{p, \bullet}, \dvert) \xrightarrow{\dhori} \Hm^q(X^{p+1, \bullet}, \dvert) \xrightarrow{\dhori} \cdots \right] , \\
	\tau_{\mathrm{I}}^{\leq n} & := (F_{\mathrm{I}})^{-1} \circ \tau^{\leq n} \circ F_{\mathrm{I}}, \\
	\tau_{\mathrm{II}}^{\leq n} & := (F_{\mathrm{II}})^{-1} \circ \tau^{\leq n} \circ F_{\mathrm{II}},
\end{align*} \index[sym1]{HmIpX@$\Hm_{\mathrm{I}}^p(X)$, $\Hm_{\mathrm{II}}^q(X)$} \index[sym1]{tauI@$\tau_{\mathrm{I}}^{\leq n}$, $\tau_{\mathrm{II}}^{\leq n}, \ldots$}
以及 $\tilde{\tau}_{\mathrm{I}}^{\leq n}$, $\tau_{\mathrm{I}}^{\geq n}$, $\tilde{\tau}_{\mathrm{I}}^{\geq n}$ 和 $\tilde{\tau}_{\mathrm{II}}^{\leq n}$, $\tau_{\mathrm{II}}^{\geq n}$, $\tilde{\tau}_{\mathrm{II}}^{\geq n}$. 于是仍有函子之间的态射
\begin{gather*}
	\tau_{\mathrm{I}}^{\leq n} \to \tilde{\tau}_{\mathrm{I}}^{\leq n} \to \identity_{\cate{C}^2(\mathcal{A})} \to \tilde{\tau}_{\mathrm{I}}^{\geq n} \to \tau_{\mathrm{I}}^{\geq n}, \\
	\tau_{\mathrm{II}}^{\leq n} \to \tilde{\tau}_{\mathrm{II}}^{\leq n} \to \identity_{\cate{C}^2(\mathcal{A})} \to \tilde{\tau}_{\mathrm{II}}^{\geq n} \to \tau_{\mathrm{II}}^{\geq n}.
\end{gather*}

\begin{definition}\label{def:Cf-Supp}
	\index[sym1]{SuppX@$\Supp(X)$}
	\index[sym1]{C2f@$\cate{C}^2_f(\mathcal{A})$}
	对双复形 $X$ 记 $\Supp(X) := \left\{ (p,q) \in \Z^2 : X^{p,q} \neq 0 \right\}$. 命 $\cate{C}^2_f(\mathcal{A})$ 为由 $\cate{C}^2(\mathcal{A})$ 的如下对象 $X$ 构成的全子范畴: 我们要求对所有 $n \in \Z$, 集合 $\{ (p, q) \in \Supp(X): p + q = n \}$ 有限.
\end{definition}

\begin{example}
	如果 $\Supp(X) \subset \Z_{\geq 0}^2$ (第一象限), 或 $\Supp(X) \subset \Z_{\leq 0}^2$ (第三象限), 又或者 $\Supp(X)$ 是有限多个列或行之并, 则 $X$ 是 $\cate{C}^2_f(\mathcal{A})$ 的对象.
\end{example}

全复形 $\tot_{\oplus}(X)$ 和 $\tot_{\Pi}(X)$ 对 $\cate{C}^2_f(\mathcal{A})$ 的所有对象 $X$ 都有定义, 无须对 $\mathcal{A}$ 另加条件. 而且此时 $\tot_{\oplus}(X) = \tot_{\Pi}(X)$; 由此得到加性函子 $\tot: \cate{C}^2_f(\mathcal{A}) \to \cate{C}(\mathcal{A})$.

现在作一则简单然而必要的观察: 设 $\cate{C}^2_f(\mathcal{A})$ 的对象 $X$ 以及 $n \in \Z$ 给定, 则存在有限子集 $S \subset \Z^2$, 使得 $\Hm^n\left( \tot(X) \right)$ 由资料 $\left( X^{p,q}, \dhori^{p, q} , \dvert^{p,q} \right)_{(p, q) \in S}$ 完全确定.

\begin{lemma}\label{prop:tot-exact}
	全复形函子 $\tot: \cate{C}^2_f(\mathcal{A}) \to \cate{C}(\mathcal{A})$ 是正合函子 (定义 \ref{def:exact-functor}).
\end{lemma}
\begin{proof}
	设 $X \xrightarrow{f} Y \xrightarrow{g} Z$ 为 $\cate{C}^2_f(\mathcal{A})$ 中的正合列. 回忆相关定义, 例如命题 \ref{prop:abelian-cat-cplx}, 可见问题在于对所有 $n \in \Z$ 证
	\[ \bigoplus_{p+q=n} X^{p,q} \xrightarrow{\bigoplus_{p+q=n} f^{p,q}} \bigoplus_{p+q=n} Y^{p,q} \xrightarrow{\bigoplus_{p+q=n} g^{p,q}} \bigoplus_{p+q=n} Z^{p,q} \]
	是 $\mathcal{A}$ 中的正合列; 因为直和有限, 一切归结为 $X^{p,q} \xrightarrow{f^{p,q}} Y^{p,q} \xrightarrow{g^{p,q}} Z^{p,q}$ 的正合性.
\end{proof}

次一引理尽管看似复杂, 却纯粹是定义的操演, 它适用于任意加性范畴 $\mathcal{A}$.

\begin{lemma}\label{prop:double-cplx-tot-aux0}
	设 $h \in \Z$, 而 $Y$ 为 $\cate{C}(\mathcal{A})$ 的对象. 由此构造
	\begin{compactitem}
		\item $\cate{C}(\mathcal{A})$ 的对象 $Y[-h]$;
		\item $\cate{C}(\cate{C}(\mathcal{A}))$ 的对象 $Y_{\mathrm{I}}[-h]$: 其 $h$ 次项为 $Y$, 其余为 $0$.
	\end{compactitem}
	此时有 $\cate{C}(\mathcal{A})$ 中的典范同构
	\begin{gather*}
		(\tot \circ F_{\mathrm{I}}^{-1}) \Cone\left( \identity_{Y_{\mathrm{I}}[-h]} \right) \simeq \Cone\left( \identity_{Y[-h]} \right), \\
		(\tot \circ F_{\mathrm{I}}^{-1}) \left( Y_{\mathrm{I}}[-h] \right) \simeq Y[-h].
	\end{gather*}
\end{lemma}
\begin{proof}
	按定义, $\Cone\left( \identity_{Y_{\mathrm{I}}[-h]} \right)$ 是 $\cate{C}^{[h-1, h]}(\cate{C}(\mathcal{A}))$ 的对象 $Y \xrightarrow{\identity_Y} Y$ (集中于次数 $h-1$ 和 $h$). 因此 $Z := F_{\mathrm{I}}^{-1} \Cone\left( \identity_{Y_{\mathrm{I}}[-h]} \right)$ 等于下图所示的双复形:
	\[\begin{tikzcd}
		n \in \Z & \vdots & \vdots \\
		(q = n-h+1) & Y^{n-h+1} \arrow[u, "d_Y"] \arrow[r, "\identity"] & Y^{n-h+1} \arrow[u, "d_Y"] \\
		(q = n-h) & Y^{n-h} \arrow[r, "\identity"] \arrow[u, "d_Y"] & Y^{n-h} \arrow[u, "d_Y"] \\
		& (p = h-1) \arrow[phantom, u, "\cdots" description, sloped] & (p = h) \arrow[phantom, u, "\cdots" description, sloped]
	\end{tikzcd}\]
	而 $p \notin \{h-1, h\}$ 对应的列全为 $0$. 按全复形的定义 \ref{def:total-cplx} 立见
	\begin{align*}
		\tot(Z)^n & = Y^{n-h+1} \oplus Y^{n-h} = Y[-h]^{n+1} \oplus Y[-h]^n \\
		& = \Cone\left( \identity_{Y[-h]} \right)^n , \\
		d_{\tot(Z)}^n & = \begin{pmatrix}
			(-1)^{h-1} d_Y^{n-h+1} & 0 \\
			\identity_{Y^{n-h+1}} & (-1)^{h} d_Y^{n-h}
		\end{pmatrix}
		= \begin{pmatrix}
			- d_{Y[-h]}^{n+1} & 0 \\
			\identity_{Y[-h]^{n+1}} & d_{Y[-h]}^n
		\end{pmatrix} \\
		& = d_{\Cone\left( \identity_{Y[-h]} \right)}^n .
	\end{align*}
	第二个同构可以类似地检验, 但更加容易, 故留给读者.
\end{proof}

\begin{lemma}\label{prop:double-cplx-tot-aux}
	设 $X$ 为 $\cate{C}^2_f(\mathcal{A})$ 的对象, $q \in \Z$.
	\begin{enumerate}[(i)]
		\item 自然态射 $\tot\left(\tau_{\mathrm{I}}^{\leq q} X\right) \to \tot\left(\tilde{\tau}_{\mathrm{I}}^{\leq q} X\right)$ 是拟同构;
		\item 存在典范短正合列
		\[ 0 \to \tot\left( \tilde{\tau}_{\mathrm{I}}^{\leq q-1} (X) \right) \to \tot\left( \tau_{\mathrm{I}}^{\leq q}(X) \right) \to \Hm_{\mathrm{I}}^q(X)[-q] \to 0. \]
	\end{enumerate}
\end{lemma}
\begin{proof}
	在 $\cate{C}(\cate{C}(\mathcal{A}))$ 中应用引理 \ref{prop:truncation-ses} 得到短正合列
	\begin{gather*}
		0 \to \tau^{\leq q} F_{\mathrm{I}} (X) \to \tilde{\tau}^{\leq q} F_{\mathrm{I}} (X) \to \Cone\left( \identity_{\Image\left( d^q_{F_{\mathrm{I}} X} \right)_{\mathrm{I}} [-q-1]} \right) \to 0, \\
		0 \to \tilde{\tau}^{\leq q-1} F_{\mathrm{I}}(X) \to \tau^{\leq q} F_{\mathrm{I}}(X) \to \Hm^q_{\mathrm{I}}(X)_{\mathrm{I}}[-q] \to 0.
	\end{gather*}
	符号 $(\cdots)_{\mathrm{I}}[\cdots]$ 的意义如引理 \ref{prop:double-cplx-tot-aux0}. 现在对这些短正合列取 $\tot \circ F_{\mathrm{I}}^{-1}$. 引理 \ref{prop:tot-exact} 表明 $\tot$ 正合, 此外 $F_{\mathrm{I}}^{-1}$ 当然也正合; 代入引理 \ref{prop:double-cplx-tot-aux0} 遂有 $\cate{C}(\mathcal{A})$ 中的短正合列
	\begin{gather*}
		0 \to \tot\left(\tau_{\mathrm{I}}^{\leq q} X\right) \to \tot\left( \tilde{\tau}_{\mathrm{I}}^{\leq q} X \right) \to \Cone\left( \identity_{\Image( d_{F_{\mathrm{I}} X}^q )[-q-1]} \right) \to 0, \\
		0 \to \tot\left( \tilde{\tau}_{\mathrm{I}}^{\leq q-1} X \right) \to \tot\left( \tau_{\mathrm{I}}^{\leq q}X \right) \to \Hm_{\mathrm{I}}^q(X)[-q] \to 0.
	\end{gather*}
	第二式即 (ii). 至于 (i) 则是第一式配合命题 \ref{prop:long-exact-sequence-ses} 和 \ref{prop:cone-homotopy} (i) 的产物.
\end{proof}

对于 $\cate{C}^2(\mathcal{A})$ 的任意对象 $X$, 记 $\Hm_{\mathrm{I}}(X)$ 和 $\Hm_{\mathrm{II}}(X)$ 为如下双复形 ($p,q \in \Z$): \index[sym1]{HmIX@$\Hm_{\mathrm{I}}(X)$, $\Hm_{\mathrm{II}}(X)$}
\begin{equation}\label{eqn:HIHII}\begin{aligned}
	\left( \Hm_{\mathrm{I}}(X)^{p, \bullet}, \dvert_{\Hm_{\mathrm{I}}(X)}^{p, \bullet} \right) & := \Hm_{\mathrm{I}}^p(X), &
	\dhori_{\Hm_{\mathrm{I}}(X)}^{\bullet, \bullet} := 0, \\
	\left( \Hm_{\mathrm{II}}(X)^{\bullet, q}, \dhori_{\Hm_{\mathrm{II}}(X)}^{\bullet, q} \right) & := \Hm_{\mathrm{II}}^q(X), &
	\dvert_{\Hm_{\mathrm{II}}(X)}^{\bullet, \bullet} := 0,
\end{aligned}\end{equation}
换言之, $\Hm_{\mathrm{I}}(X)$ (或 $\Hm_{\mathrm{II}}(X)$) 是沿着 $\dhori$ (或 $\dvert$) 方向对 $X$ 取上同调的产物. 它们都是 $\cate{C}^2(\mathcal{A})$ 到自身的函子. 进而可定义 $\Hm_{\mathrm{II}} \Hm_{\mathrm{I}}(X)$ 和 $\Hm_{\mathrm{I}} \Hm_{\mathrm{II}} (X)$, 两者都满足 $\dhori = 0 = \dvert$.

\begin{theorem}\label{prop:double-cplx-tot}
	设 $f: X \to Y$ 为 $\cate{C}^2_f(\mathcal{A})$ 中的态射. 若其诱导态射 $\Hm_{\mathrm{II}} \Hm_{\mathrm{I}} (X) \to \Hm_{\mathrm{II}} \Hm_{\mathrm{I}} (Y)$ 是同构, 则 $\tot(f): \tot(X) \to \tot(Y)$ 是拟同构.
	
	将条件换为诱导态射 $\Hm_{\mathrm{I}} \Hm_{\mathrm{II}} (X) \to \Hm_{\mathrm{I}} \Hm_{\mathrm{II}} (Y)$ 是同构, 结论亦同.
\end{theorem}
\begin{proof}
	条件相当于说 $\cate{C}(\mathcal{A})$ 中的态射 $\Hm_{\mathrm{I}}^n(f): \Hm_{\mathrm{I}}^n(X) \to \Hm_{\mathrm{I}}^n(Y)$ 对每个 $n \in \Z$ 皆是拟同构. 证明第一步是化约到下式成立的情形:
	\begin{equation}\label{eqn:double-cplx-tot-aux}
		q \ll 0 \implies \tilde{\tau}_{\mathrm{I}}^{\leq q}(X) = 0 = \tilde{\tau}_{\mathrm{I}}^{\leq q}(Y).
	\end{equation}
	诚然, 设 $r \in \Z$, 以引理 \ref{prop:truncation-ses} 的函子截断, 可见原条件蕴涵 $\Hm_{\mathrm{I}}^n\left( \tau_{\mathrm{I}}^{\geq r} f \right): \Hm_{\mathrm{I}}^n \left(\tau_{\mathrm{I}}^{\geq r} X \right) \to \Hm_{\mathrm{I}}^n\left( \tau_{\mathrm{I}}^{\geq r} Y \right)$ 对所有 $n$ 也是拟同构. 然而对于给定的 $n$, 从 $\cate{C}^2_f(\mathcal{A})$ 的定义易见 $r \ll 0$ 时有交换图表
	\begin{equation}\label{eqn:double-cplx-tot-aux2}\begin{tikzcd}
		\Hm^n(\tot(X)) \arrow[r, "{\Hm^n(\tot(f))}"] \arrow[d, "\sim"' sloped] & \Hm^n(\tot(Y)) \arrow[d, "\sim" sloped] \\
		\Hm^n\left( \tot(\tau_{\mathrm{I}}^{\geq r} X) \right) \arrow[r, "{\Hm^n\left( \tot (\tau_{\mathrm{I}}^{\geq r} f) \right)}"' inner sep=0.8em] & \Hm^n\left( \tot(\tau_{\mathrm{I}}^{\geq r} Y) \right)
	\end{tikzcd}\end{equation}
	第二行涉及的双复形满足 \eqref{eqn:double-cplx-tot-aux}, 故今后不妨以 $\tau_{\mathrm{I}}^{\geq r} f$ 代替 $f$, 并假定 \eqref{eqn:double-cplx-tot-aux} 成立. 我们的目标是对选定的 $n$ 证 $\Hm^n(\tot(f))$ 为同构.
	
	对任意 $q \in \Z$, 引理 \ref{prop:double-cplx-tot-aux} (ii) 给出行正合的交换图表
	\[\begin{tikzcd}
		0 \arrow[r] & \tot\left( \tilde{\tau}_{\mathrm{I}}^{\leq q-1} X \right) \arrow[r] \arrow[d, "{\alpha_{q-1}}"] & \tot\left( \tau_{\mathrm{I}}^{\leq q} X \right) \arrow[d, "{\beta_q}"] \arrow[r] & \Hm_{\mathrm{I}}^q(X)[-q] \arrow[d, "\simeq"] \arrow[r] & 0 \\
		0 \arrow[r] & \tot\left( \tilde{\tau}_{\mathrm{I}}^{\leq q-1} Y \right) \arrow[r] & \tot\left( \tau_{\mathrm{I}}^{\leq q} Y \right) \arrow[r] & \Hm_{\mathrm{I}}^q(Y)[-q] \arrow[r] & 0
	\end{tikzcd}\]
	其中垂直箭头来自 $f$. 以下证明 $\alpha_q$ 和 $\beta_q$ 皆为拟同构. 若 $\alpha_{q-1}$ 是拟同构, 则相应的长正合列的函子性 (命题 \ref{prop:long-exact-sequence-ses}) 结合命题 \ref{prop:5-lemma} 将蕴涵 $\beta_q$ 也是拟同构. 另一方面, 引理 \ref{prop:double-cplx-tot-aux} (i) 给出交换图表
	\[\begin{tikzcd}
		\tot\left( \tau_{\mathrm{I}}^{\leq q} X \right) \arrow[r, "\text{拟同构}"] \arrow[d, "{\beta_q}"'] & \tot\left( \tilde{\tau}_{\mathrm{I}}^{\leq q} X \right) \arrow[d, "{\alpha_q}"] \\
		\tot\left( \tau_{\mathrm{I}}^{\leq q} Y \right) \arrow[r, "\text{拟同构}"'] & \tot\left( \tilde{\tau}_{\mathrm{I}}^{\leq q} Y \right)
	\end{tikzcd}\]
	所以 $\beta_q$ 是拟同构蕴涵 $\alpha_q$ 亦然. 由于 $q \ll 0$ 时 $\alpha_q$ 无非 $0 \rightiso 0$, 由此便递归地推得 $\alpha_q$, $\beta_q$ 对所有 $q$ 都是拟同构.
	
	回忆到 $n \in \Z$ 已选定. 和 \eqref{eqn:double-cplx-tot-aux2} 的道理类似, 当 $q \gg 0$ 时有交换图表
	\[\begin{tikzcd}
		\Hm^n(\tot(X)) \arrow[r, "{\Hm^n(\tot(f))}"] & \Hm^n(\tot(Y)) \\
		\Hm^n\left( \tot(\tau_{\mathrm{I}}^{\leq q} X) \right) \arrow[r, "{\Hm^n(\beta_q)}"' inner sep=0.8em, "\sim"] \arrow[u, "\sim" sloped] & \Hm^n\left( \tot(\tau_{\mathrm{I}}^{\leq q} Y) \right) \arrow[u, "\sim"' sloped]
	\end{tikzcd}\]
	故 $\Hm^n(\tot(f))$ 确实为同构.
	
	最后, 相同论证仍适用于 $\Hm_{\mathrm{I}} \Hm_{\mathrm{II}} (X) \rightiso \Hm_{\mathrm{I}} \Hm_{\mathrm{II}} (Y)$ 的情形; 另一观点则是以命题 \ref{prop:double-cplx-swap} 对换 $\Hm_{\mathrm{I}}$ 和 $\Hm_{\mathrm{II}}$.
\end{proof}

以上证明颇费周折, 例 \ref{eg:double-cplx-tot-ss} 将介绍基于谱序列的另证.

\begin{corollary}\label{prop:acyclic-assembly}
	设 $X$ 为 $\cate{C}_f^2(\mathcal{A})$ 的对象. 若 $X$ 行正合, 换言之 $\left( X^{\bullet, q}, \dhori \right)$ 对每个 $q \in \Z$ 都正合, 则 $\tot(X)$ 正合. 类似地, 若 $X$ 列正合, 则 $\tot(X)$ 正合.
\end{corollary}
\begin{proof}
	若 $X$ 行正合, 则 $\Hm_{\mathrm{II}} \Hm_{\mathrm{I}}(X) = 0$, 因此定理 \ref{prop:double-cplx-tot} 蕴涵 $\tot(X) \to \tot(0) = 0$ 是拟同构, 亦即 $\tot(X)$ 正合.

	列正合情形的论证完全相同, 或者也可以借助命题 \ref{prop:double-cplx-swap} 来相互过渡.
\end{proof}

\section{解消}\label{sec:resolutions}
设 $\mathcal{A}$ 为 Abel 范畴. 考虑复形 $X \in \Obj(\cate{C}(\mathcal{A}))$. 若 $X^n$ 对每个 $n \in \Z$ 都是 $\mathcal{A}$ 的内射对象 (或投射对象), 则称 $X$ 由内射对象 (或投射对象) 组成. 莫忘一则平凡事实: $0$ 既是内射对象也是投射对象.

\begin{definition}\label{def:resolutions}
	\index{jiexiao@解消 (resolution)}
	\index{jiexiao!内射 (injective)}
	\index{jiexiao!投射 (projective)}
	设 $X \in \Obj(\cate{C}(\mathcal{A}))$.
	\begin{itemize}
		\item 设 $X \to I$ 为 $\cate{C}(\mathcal{A})$ 中的拟同构, 其中 $I \in \Obj(\cate{C}^+(\mathcal{A}))$ 由内射对象组成, 则称 $X \to I$ 为 $X$ 的\emph{内射解消}.
		\item 设 $P \to X$ 为 $\cate{C}(\mathcal{A})$ 中的拟同构, 其中 $P \in \Obj(\cate{C}^-(\mathcal{A}))$ 由投射对象组成, 则称 $P \to X$ 为 $X$ 的\emph{投射解消}.
	\end{itemize}
\end{definition}

以 $\mathcal{A}^{\opp}$ 代 $\mathcal{A}$, 亦即倒转箭头, 可见内射解消和投射解消是相互对偶的概念.

\begin{example}\label{eg:resolution-single}
	作为最初步也最经典的特例, 考虑 $X \in \Obj(\mathcal{A})$ 的内射解消 $X \to I$. 定义只要求 $I \in \Obj(\cate{C}^+(\mathcal{A}))$, 但我们希望取到 $I \in \Obj(\cate{C}^{\geq 0}(\mathcal{A}))$. 这是可行的: 运用截断
	\[ X \to I \to \tau^{\geq 0} I . \]
	为了说明其合成是内射解消, 须验证两条性质. 首先引理 \ref{prop:truncation-ses} 说明这仍是拟同构. 其次, $\tau^{\geq 0} I$ 仍由内射对象组成. 何以故? 考虑短正合列 $0 \to \Image\left(d_I^{n-1} \right) \to I^n \to \Coker\left( d_I^{n-1} \right) \to 0$, 由于 $n \ll 0$ 时 $d_I^{n-1} = 0$ 而 $n < 0$ 时 $\Coker\left( d_I^{n-1} \right) \simeq \Image\left( d_I^n \right)$, 以引理 \ref{prop:inj-proj-split} 和 \ref{prop:prod-injective-projective} 可递归地说明 $n \leq 0$ 时
	\[ I^n \simeq \Image\left(d_I^{n-1} \right) \oplus \Coker\left( d_I^{n-1} \right) , \quad \text{因而}\; \Coker\left(d_I^{n-1} \right) \;\text{是内射对象}; \]
	递归起点是 $n \ll 0$, 此时上式化为 $0 = 0 \oplus 0$. 取 $n=0$ 遂知 $\Coker\left( d_I^{-1}\right)$ 是内射对象, 故 $\tau^{\geq 0} I$ 由内射对象组成.

	由于实践中关心的是足够``深''的解消, 上述观察表明无妨以 $X \to \tau^{\geq 0} I$ 代替 $X \to I$, 使内射解消形如
	\[\begin{tikzcd}[row sep=small]
		\cdots 0 \arrow[r] & 0 \arrow[r] \arrow[d] & X \arrow[d] \arrow[r] & 0 \arrow[r] \arrow[d] & 0 \arrow[r] \arrow[d] & \cdots \\
		\cdots 0 \arrow[r] & 0 \arrow[r] & I^0 \arrow[r] & I^1 \arrow[r] & I^2 \arrow[r] & \cdots
	\end{tikzcd}\]
	这也可以摊平, 视同 $\mathcal{A}$ 的正合列
	\[ 0 \to X \to I^0 \to I^1 \to \cdots, \quad \text{每个 $I^n$ 都是内射对象, $n \geq 0$}. \]
	
	同理, 对于投射解消 $P \to X$, 以 $\tau^{\leq 0} P \to P \to X$ 的合成代替 $P \to X$, 不妨设其来自 $\mathcal{A}$ 的正合列
	\[ \cdots \to P^{-1} \to P^0 \to X \to 0, \quad \text{每个 $P^n$ 都是投射对象, $n \leq 0$.} \]
	上式惯常以链复形的写法记为 $\cdots \to P_1 \to P_0 \to X \to 0$, 其中 $P_n := P^{-n}$.
\end{example}

问题在于内射或投射解消的存在性. 基于对偶性, 无妨先讨论内射解消的构造. 最简单的仍是 $X \in \Obj(\mathcal{A})$ 的情形. 假定 $\mathcal{A}$ 有足够的内射对象, 见定义 \ref{def:enough-injectives}. 首先任取单态射 $X \to I^0$ 使得 $I^0$ 为内射对象. 递归地假设已有正合列
\[ 0 \to X \to I^0 \to \cdots \to I^n, \quad n \geq 0, \]
使得 $I^0, \ldots, I^n$ 皆为内射对象. 存在内射对象 $I^{n+1}$ 和单态射 $\Coker\left[ I^{n-1} \to I^n\right] \to I^{n+1}$, 取合成遂给出 $I^n \to I^{n+1}$, 使得 $0 \to X \to \cdots \to I^{n+1}$ 也正合. 反复操作即是 $X$ 的内射解消.

在 $\mathcal{A}$ 有足够的投射对象的前提下, $X \in \Obj(\mathcal{A})$ 的投射解消 $\cdots \to P^0 \to X \to 0$ 其构造全然是对偶的. 这些构造可以扩及复形, 但需要一些有界条件, 详如下述.

\begin{theorem}\label{prop:resolution-existence}
	设 $\mathcal{A}^\flat$ 是 $\mathcal{A}$ 的加性全子范畴, 而且对 $\mathcal{A}$ 的每个对象 $A$ 都存在单态射 $A \hookrightarrow B$ (或满态射 $B \twoheadrightarrow A$) 使得 $B \in \Obj(\mathcal{A}^\flat)$, 则
	\begin{enumerate}[(i)]
		\item 对 $\cate{C}^+(\mathcal{A})$ (或 $\cate{C}^-(\mathcal{A})$) 的所有对象 $X$, 存在拟同构
		\[ f: X \to I \quad \text{(或 $f: P \to X$)}, \]
		使得 $f$ 单 (或满), 而且 $I \in \Obj(\cate{C}^+(\mathcal{A}^\flat))$ (或 $P \in \Obj(\cate{C}^-(\mathcal{A}^\flat))$).
		\item 更精确地说, 若 $m \in \Z$ 而 $X \in \Obj(\cate{C}^{\geq m}(\mathcal{A}))$ (或 $X \in \Obj(\cate{C}^{\leq m}(\mathcal{A}))$), 则可取 $I \in \Obj(\cate{C}^{\geq m}(\mathcal{A}))$ (或 $P \in \Obj(\cate{C}^{\leq m}(\mathcal{A}))$).
	\end{enumerate}

	作为推论, 若 $\mathcal{A}$ 有足够的内射对象 (或投射对象), 则如上之 $X$ 有内射解消 $X \hookrightarrow I$ (或投射解消 $P \twoheadrightarrow X$).
\end{theorem}
\begin{proof}
	基于对偶性, 以下仅论 $X$ 来自 $\cate{C}^+(\mathcal{A})$ 的版本. 对于 (i), 递归地假设已有交换图表
	\[\begin{tikzcd}
		\cdots \arrow[r] & X^{n-1} \arrow[r, "{d_X^{n-1}}"] \arrow[d, "{f^{n-1}}"'] & X^n \arrow[r, "{d_X^n}"] \arrow[d, "{f^n}"] & X^{n+1} \arrow[r] & \cdots \\
		\cdots \arrow[r] & I^{n-1} \arrow[r, "{d_I^{n-1}}"'] & I^n & &
	\end{tikzcd}\]
	使得第二行是 $\mathcal{A}^\flat$ 的对象构成的复形, $m \ll 0 \implies I^m = 0$, 每个 $f^m$ 皆单 ($m \leq n$), 而且当 $m \leq n-1$ 时 $f^m$ 诱导同构 $\Ker\left( d_X^m \right) / \Image\left( d_X^{m-1} \right) \rightiso \Ker\left( d_I^m \right) / \Image\left( d_I^{m-1} \right)$.

	我们希望将图表第二行右延. 首先假设
	\begin{equation}\label{eqn:resolution-existence-aux0}
		f^n\left( \Image\left(d_X^{n-1} \right) \right) = \Image\left( d^{n-1}_I \right) \cap f^n\left( \Ker\left( d_X^n \right) \right) = \Image\left( d^{n-1}_I \right) \cap f^n(X^n) ;
	\end{equation}
	留意到其中的包含关系 $\cdots \subset \cdots \subset \cdots$ 恒成立. 构造推出图表
	\begin{equation*}\begin{tikzcd}
		X^n / \Ker\left( d_X^n \right) \arrow[hookrightarrow, r, "\delta: \text{由 $d_X^n$ 诱导}"] \arrow[d, "\alpha: \text{由 $f^n$ 诱导}"'] & X^{n+1} \arrow[d, "\beta"] \\
		I^n / \left( f^n \left( \Ker\left( d_X^n \right) \right) + \Image\left( d_I^{n-1}\right) \right) \arrow[r, "\eta"'] & T \arrow[phantom, lu, "\boxplus" description]
	\end{tikzcd}\end{equation*}
	注意到 $f^n$ 单, 而 \eqref{eqn:resolution-existence-aux0} 蕴涵
	\begin{multline*}
		\left( f^n\left( \Ker\left( d_X^n \right) \right) + \Image\left( d_I^{n-1}\right) \right) \cap f^n(X^n) \\
		\xlongequal{\text{定理 \ref{prop:subobject-modularity}}} f^n\left( \Ker\left( d_X^n \right) \right) + \left( \Image\left( d_I^{n-1}\right) \cap f^n(X^n) \right) \\
		= f^n\left( \Ker\left( d_X^n \right) \right) + f^n\left(\Image\left(d_X^{n-1} \right) \right) = f^n\left( \Ker\left( d_X^n \right) \right),
	\end{multline*}
	所以基于 \S\ref{sec:Abel-cat-subobjects} 讨论的基本操作推知 $\alpha$ 也是单态射.
	
	应用命题 \ref{prop:Abel-cat-pull-push} 两次, 可见推出图表中的 $\beta$ 和 $\eta$ 亦单. 取单态射 $T \hookrightarrow I^{n+1}$ 使得 $I^{n+1} \in \Obj(\mathcal{A}^\flat)$, 由之得到合成态射
	\begin{gather*}
		d_I^n: I^n \twoheadrightarrow I^n / \left( f^n \left( \Ker\left( d_X^n \right) \right) + \Image\left( d_I^{n-1}\right) \right) \xrightarrow{\eta} T \hookrightarrow I^{n+1}, \\
		f^{n+1}: X^{n+1} \xrightarrow{\beta} T \hookrightarrow I^{n+1}.
	\end{gather*}
	显见 $d_I^n d_I^{n-1} = 0$ 而 $f^{n+1}$ 单, $d_I^n f^n = f^{n+1} d_X^n$. 单态射 $f^n$ 对上同调诱导
	\begin{align*}
		\frac{\Ker\left( d_X^n \right)}{\Image\left( d_X^{n-1} \right)} \longrightarrow \frac{\Ker\left( d_I^n \right)}{\Image\left( d_I^{n-1} \right)} & =  \frac{ f^n\left(\Ker d_X^n \right) + \Image\left( d_I^{n-1} \right)}{ \Image\left(d_I^{n-1}\right) } \\
		\text{($\because$ 定理 \ref{prop:Abel-cat-isom-thm} (iii))} \quad & \simeq \frac{ f^n\left(\Ker d_X^n \right) }{ f^n\left(\Ker d_X^n \right) \cap \Image\left(d_I^{n-1}\right) };
	\end{align*}
	依据 \eqref{eqn:resolution-existence-aux0} 和 $f^n$ 的单性立见此为同构. 右延成功.

	现在说明在一般状况下如何修改 $I^n$ 以化约到 \eqref{eqn:resolution-existence-aux0} 成立的情形. 兹考虑典范态射
	\[\begin{tikzcd}
		I^n \arrow[hookrightarrow, shift left, r, "\iota_1"] & I^n \oplus \left( X^n / \Image\left( d_X^{n-1} \right) \right) \arrow[twoheadrightarrow, shift left, l, "p_1"] \arrow[twoheadrightarrow, shift right, r, "p_2"'] & X^n / \Image\left( d_X^{n-1} \right) \arrow[hookrightarrow, shift right, l, "\iota_2"'] & X^n \arrow[twoheadrightarrow, l, "q"'] .
	\end{tikzcd}\]
	任选 $J \in \Obj(\mathcal{A}^\flat)$ 和单态射 $j: I^n \oplus \left( X^n / \Image\left( d_X^{n-1} \right) \right) \hookrightarrow J$. 定义合成态射
	\begin{gather*}
		f' : X^n \xrightarrow{(f^n, q)} I^n \oplus \left( X^n / \Image\left( d_X^{n-1} \right) \right) \xrightarrow{j} J, \\
		d': I^{n-1} \xrightarrow{d_I^{n-1}} I^n \xrightarrow{\iota_1} I^n \oplus \left( X^n / \Image\left( d_X^{n-1} \right) \right) \xrightarrow{j} J.
	\end{gather*}
	容易看出 $f'$ 单, $\Ker(d') = \Ker\left( d_I^{n-1} \right)$ 和 $f' d_X^{n-1} = d' f^{n-1}$.
	
	此外, 在 $I^n \oplus \left( X^n / \Image\left( d_X^{n-1} \right) \right)$ 中操作可见
	\[ f'(X^n) \cap \Image(d') = f'\left( \Image\left( d_X^{n-1} \right) \right). \]
	这表明以 $d': I^{n-1} \to J$ (或 $f': X^n \to J$) 代替 $d_I^{n-1}: I^{n-1} \to I^n$ (或 $f^n: X^n \to I^n$), 可以确保 \eqref{eqn:resolution-existence-aux0} 成立.
	
	最后讨论 (ii). 在上述构造中取 $\cdots = I^{m-2} = I^{m-1} = 0$. 显见 \eqref{eqn:resolution-existence-aux0} 对 $n = m-1$ 成立 (各项全为零), 所以图表右延过程中不产生次数 $< m$ 的非零项. 明所欲证.
\end{proof}

\begin{corollary}
	设 $\mathcal{A}$ 有足够的内射对象 (或投射对象), 则 $X \in \Obj(\cate{C}(\mathcal{A}))$ 有内射解消 $X \to I$ (或投射解消 $P \to X$) 当且仅当 $n \ll 0$ (或 $n \gg 0$) 时 $\Hm^n(X) = 0$.
\end{corollary}
\begin{proof}
	就内射解消的情形为例. 解消定义中的拟同构条件导致``仅当''方向. 而在 $m \ll 0$ 时, 截断函子给出拟同构 $X \to \tau^{\geq m} X \in \Obj(\cate{C}^+(\mathcal{A}))$, 依此将``当''的方向化约到命题 \ref{prop:resolution-existence}.
\end{proof}

内射或投射解消的理论价值可以从以下基本结果得到说明, 它还蕴涵内射解消或投射解消在同伦意义下唯一.

\begin{theorem}\label{prop:resolution-homotopy}
	取定 $\cate{C}(\mathcal{A})$ 中的拟同构 $\alpha: X \to Y$.
	\begin{itemize}
		\item 给定态射 $\gamma: X \to I$, 其中 $I$ 是 $\cate{C}^+(\mathcal{A})$ 的对象, 由内射对象组成. 在 $\cate{K}(\mathcal{A})$ 中存在唯一的态射 $\beta$ 使下图交换:
		\[\begin{tikzcd}[row sep=tiny]
			& Y \arrow[dd, "\beta"] \\
			X \arrow[ru, "\alpha"] \arrow[rd, "\gamma"'] & \\
			& I
		\end{tikzcd}\]
		\item 给定态射 $\gamma: P \to Y$, 其中 $P$ 是 $\cate{C}^-(\mathcal{A})$ 的对象, 由投射对象组成. 在 $\cate{K}(\mathcal{A})$ 中存在唯一的态射 $\beta$ 使下图交换:
		\[\begin{tikzcd}[row sep=tiny]
			& X \arrow[ld, "\alpha"'] \\
			Y & \\
			& P \arrow[lu, "\gamma"] \arrow[uu, "\beta"']
		\end{tikzcd}\]
	\end{itemize}
\end{theorem}

两个断言当然相互对偶. 稍后的定理 \ref{prop:cplx-triangulated} 之后将给予简洁的处理. 在 $\alpha$ 单 (或满) 的情形, 还能进一步取到 $\beta$ 使得第一个 (或第二个) 图表在 $\cate{C}(\mathcal{A})$ 中交换, 这是稍后的引理 \ref{prop:ext-alpha-beta} 及其对偶版本的内容. 事实上, 本章的习题部分将对定理 \ref{prop:resolution-homotopy} 给出基于上述引理的直接论证, 附提示, 建议读者一试.

不论取哪种进路, 都离不开以下性质.

\begin{lemma}\label{prop:resolution-homotopy-aux}
	设 $\cate{C}(\mathcal{A})$ 的对象 $X$ 零调.
	\begin{itemize}
		\item 若 $\cate{C}^+(\mathcal{A})$ 的对象 $I$ 由内射对象组成, 则 $\Hom_{\cate{K}(\mathcal{A})}(X, I) = 0$.
		\item 若 $\cate{C}^-(\mathcal{A})$ 的对象 $P$ 由投射对象组成, 则 $\Hom_{\cate{K}(\mathcal{A})}(P, X) = 0$.
	\end{itemize}
\end{lemma}
\begin{proof}
	处理 $I$ 的情形即可. 取定 $\alpha \in \Hom_{\cate{C}(\mathcal{A})}(X,I)$. 我们寻求一族态射 $h^n: X^n \to I^{n-1}$, 使得
	\begin{equation}\label{eqn:alpha-homotopy-aux}
		\alpha^n = d_I^{n-1} h^n + h^{n+1} d_X^n , \quad n \in \Z.
	\end{equation}

	递归地假设已有 $\ldots, h^{n-1}, h^n$ 使 \eqref{eqn:alpha-homotopy-aux} 对 $\ldots, \alpha^{n-2}, \alpha^{n-1}$ 成立. 这是合理的, 因为当 $n \ll 0$ 时可以取 $\ldots, h^{n-1}, h^n$ 全为 $0$. 我们寻求虚线标出的态射
	\[\begin{tikzcd}[row sep=large]
		& I^n & \\
		X \arrow[twoheadrightarrow, r, "{d_X^n}"'] \arrow[ru, "{\alpha^n - d_I^{n-1} h^n}"] & \Image\left( d_X^n \right) \arrow[dashed, u, "\beta"] \arrow[hookrightarrow, r] & X^{n+1} \arrow[dashed, lu, "{h^{n+1}}"']
	\end{tikzcd}\]
	使得全图交换. 首先,
	\[ \left( \alpha^n - d_I^{n-1} h^n \right) d_X^{n-1} = d_I^{n-1} \alpha^{n-1} - d_I^{n-1} \left( \alpha^{n-1} - d_I^{n-2} h^{n-1} \right) = 0. \]
	因为 $X$ 零调, 这诱导图示之态射 $\beta$. 其次, 因为 $I^n$ 是内射对象, $\beta$ 能延拓为 $h^{n+1}: X^{n+1} \to I^n$. 明所欲证.
\end{proof}

按导出范畴的视角, 定理 \ref{prop:resolution-existence} 和 \ref{prop:resolution-homotopy} 提供的信息已足够开展导出函子的研究. 行将介绍的几种解消则适合搭配谱序列来运用, 见 \S\ref{sec:double-cplx-ss}. 以下针对内射版本进行表述. 我们需要一些准备.

\begin{lemma}\label{prop:ext-alpha-beta}
	设 $\alpha: X \to Y$ 是 $\cate{C}(\mathcal{A})$ 中的拟同构, 而且 $\alpha$ 单. 给定 $\cate{C}(\mathcal{A})$ 中的态射 $\gamma: X \to I$, 其中 $I \in \Obj(\cate{C}^+(\mathcal{A}))$ 由内射对象组成. 存在 $\cate{C}(\mathcal{A})$ 中的态射 $\beta: Y \to I$ 使得 $\gamma = \beta\alpha$.
\end{lemma}
\begin{proof}
	借助 $\alpha^n$ 将每个 $X^n$ 视同 $Y^n$ 的子对象, 今后在符号中省略 $\alpha^n$. 我们逐步构造 $\beta^n$; 当 $n \ll 0$ 时命 $\beta^n = 0$. 设已有满足条件的 $\ldots, \beta^{n-2}, \beta^{n-1}$. 所求的 $\beta^n: Y^n \to I^n$ 在子对象 $X^n$ 和 $\Image(d^{n-1}_Y)$ 上能且仅能按下述方式确定.
	\begin{itemize}
		\item 在子对象 $X^n$ 上, $\beta^n$ 等于 $\gamma^n$.
		\item 在子对象 $\Image(d^{n-1}_Y)$ 上, $\beta^n$ 由 $Y^{n-1} \xrightarrow{\beta^{n-1}} I^{n-1} \xrightarrow{d_I^{n-1}} I^n$ 的合成所诱导. 为了使这句话有意义, 兹断言 $d_I^{n-1} \beta^{n-1}$ 限制在 $\Ker(d_Y^{n-1})$ 上为 $0$. 请端详交换图表
		\[\begin{tikzcd}
			\Image(d_X^{n-2}) \arrow[r] \arrow[hookrightarrow, d] & \Ker(d_X^{n-1}) \arrow[r] \arrow[hookrightarrow, d] & X^{n-1} \arrow[rd, "{\gamma^{n-1}}"] \arrow[hookrightarrow, d] & & \\
			\Image(d_Y^{n-2}) \arrow[r] & \Ker(d_Y^{n-1}) \arrow[r] & Y^{n-1} \arrow[r, "{\beta^{n-1}}"'] & I^{n-1} \arrow[r, "{d_I^{n-1}}"'] & I^n . 
		\end{tikzcd}\]
		从对 $\ldots, \beta^{n-2}, \beta^{n-1}$ 的假设和 $d_I^{n-1} d_I^{n-2} = 0$ 可知第二行合成为 $0$; 又因为 $\gamma$ 是复形之间的态射, $\Ker(d_X^{n-1}) \to \cdots \to I^n$ 合成为 $0$. 综之可得交换图表
		\[\begin{tikzcd}
			\Hm^{n-1}(X) \arrow[d] \arrow[r, "0"] & I^n \\
			\Hm^{n-1}(Y) \arrow[ru, "\text{由 $d_I^{n-1} \beta^{n-1}$ 诱导}"'] &
		\end{tikzcd}\]
		然而拟同构的条件说明 $\Hm^{n-1}(X) \rightiso \Hm^{n-1}(Y)$. 断言得证.
	\end{itemize}
	
	如能证明子对象的等式
	\[ X^n \cap \Image\left(d_Y^{n-1} \right) = \Image\left( d_X^{n-1} \right), \]
	则根据命题 \ref{prop:biproduct-sum-intersection} 的推出图表以及推出的泛性质, 态射可以延拓到 $X^n + \Image(d^{n-1}_Y)$, 进而用 $I^n$ 作为内射对象的性质延拓为 $\beta^n: Y^n \to I^n$. 留意到包含关系 $\supset$ 是明白的.
	
	重点在于证 $\mathcal{A}$ 的任意态射 $\phi: T \to Y^n$ 若通过 $X^n \cap \Image\left(d_Y^{n-1} \right)$ 分解, 则自动通过 $\Image\left( d_X^{n-1} \right)$ 分解. 这般态射 $\phi$ 必诱导 $T \to \Ker\left( d_X^n \right) = X^n \cap \Ker\left( d_Y^n \right)$. 现在沿着交换图表
	\[\begin{tikzcd}
		T \arrow[r, "\phi"] \arrow[d, "\phi"'] & \Ker(d_X^n) \arrow[twoheadrightarrow, r] \arrow[hookrightarrow, d] & \Hm^n(X) \arrow[d] \\
		\Image(d_Y^{n-1}) \arrow[hookrightarrow, r] & \Ker(d_Y^n) \arrow[twoheadrightarrow, r] & \Hm^n(Y)
	\end{tikzcd}\]
	的第二行作合成, 其产物为 $0$; 由 $\Hm^n(X) \rightiso \Hm^n(Y)$ 可知第一行也合成为 $0$. 于是 $\phi$ 通过 $\Image\left( d_X^{n-1} \right)$ 分解.
	
	按照构造, $\beta = (\beta^n)_n$ 给出态射 $Y \to I$, 而条件 $\gamma = \beta\alpha$ 同样由构造立得.
\end{proof}

\begin{lemma}\label{prop:split-differential}
	设 $A, C \in \Obj(\cate{C}(\mathcal{A}))$. 对每个 $n \in \Z$ 定义 $B^n := A^n \oplus C^n$, 它带有自明的态射 $A^n \xrightarrow{i^n} B^n \xrightarrow{p^n} C^n$. 我们有双射
	\[ \Hom_{\cate{C}(\mathcal{A})}(C, A[1]) \xrightarrow{1:1} \left\{\begin{array}{r|l}
		\text{态射族}\; (d_B^n)_{n \in \Z} & \text{使}\; (B^n, d_B^n)_n \;\text{成为复形, 而且}  \\
		& 0 \to A \xrightarrow{(i^n)_n} B \xrightarrow{(p^n)_n} C \to 0 \;\text{正合}
	\end{array}\right\}, \]
	具体方式是映 $\delta: C \to A[1]$ 为矩阵表法所确定的
	\[ d_B^n = \begin{pmatrix}
		d_A^n & \delta^n \\
		0 & d_C^n
	\end{pmatrix}: B^n \to B^{n+1}. \]
\end{lemma}
\begin{proof}
	一旦知道 $(B^n, d_B^n)_n$ 成为复形, 而且 $(i^n)_n$ 和 $(p^n)_n$ 是复形的态射, 则 $0 \to A \to B \to C \to 0$ 的正合性可以逐项地检验, 不在话下.
	
	使 $d_B^n i^n = i^{n+1} d_A^n$ 和 $d_C^n p^n = p^{n+1} d_B^n$ 恒成立的充要条件是存在一族 $\delta^n: C^n \to A^{n+1}$, 使得 $d_B^n$ 能表为断言中的上三角矩阵. 问题归结为刻画使 $d_B^{n+1} d_B^n = 0$ 的 $(\delta^n)_n$. 例行的计算给出充要条件 $d_A^{n+1} \delta^n + \delta^{n+1} d_C^n = 0$, 亦即 $d_{A[1]}^n \delta^n = \delta^{n+1} d_C^n$.
\end{proof}

\begin{proposition}[马蹄引理]\label{prop:horseshoe}
	\index{matiyinli@马蹄引理 (Horseshoe Lemma)}
	设 $0 \to A \to B \to C \to 0$ 是 $\cate{C}(\mathcal{A})$ 中的短正合列, $A, B, C \in \Obj(\cate{C}^+(\mathcal{A}))$. 任取性质如定理 \ref{prop:resolution-existence} 所述的内射解消 $\epsilon: A \to I$ 和 $\eta: C \to K$, 则这些资料扩充为 $\cate{C}^+(\mathcal{A})$ 中的行正合交换图表
	\[\begin{tikzcd}
		0 \arrow[r] & I \arrow[r] & J \arrow[r] & K \arrow[r] & 0 \\
		0 \arrow[r] & A \arrow[r] \arrow[u, "\epsilon"] & B \arrow[r] \arrow[u, "\kappa"] & C \arrow[r] \arrow[u, "\eta"'] & 0 
	\end{tikzcd}\]
	使得 $\kappa: B \to J$ 也是内射解消, 而且 $\kappa$ 单.
\end{proposition}

读者不妨尝试对 $A, B, C \in \Obj(\mathcal{A})$ 的特例给出相对简单的证明.

\begin{proof}
	在 $\cate{C}(\mathcal{A})$ 中构造推出图表
	\[\begin{tikzcd}
		0 \arrow[r] & I \arrow[r] & L \arrow[r] \arrow[phantom, ld, "\boxplus" description] & C \arrow[r] & 0 \\
		0 \arrow[r] & A \arrow[r] \arrow[u, "\epsilon"] & B \arrow[r] \arrow[u] & C \arrow[r] \arrow[u, "\identity"'] & 0 
	\end{tikzcd}\]
	此处用到两则事实:
	\begin{compactitem}
		\item 推出保余核 $C$;
		\item 命题 \ref{prop:Abel-cat-pull-push} 和 $A \to B$ 的单性蕴涵 $I \to L$ 也是单态射.
	\end{compactitem}
	特别地, 上图仍是行正合交换图表, 因而 $L \in \Obj(\cate{C}^+(\mathcal{A}))$. 同样应用命题 \ref{prop:Abel-cat-pull-push} 和 $\epsilon$ 单可得 $B \to L$ 亦单.

	由于 $I^n$ 内射, 引理 \ref{prop:inj-proj-split} 蕴涵短正合列 $0 \to I^n \to L^n \to C^n \to 0$ 对每个 $n$ 皆分裂, 给出同构 $\Phi^n: L^n \rightiso I^n \oplus C^n$. 引理 \ref{prop:split-differential} 遂确定 $\delta: C \to I[1]$ 使得
	\begin{equation}\label{eqn:horseshoe-aux}\begin{tikzcd}
		L^n \arrow[r, "{\Phi^n}", "\sim"'] \arrow[d, "{d_L^n}"'] & I^n \oplus C^n \arrow[d, "{e^n}"] \\
		L^{n+1} \arrow[r, "{\Phi^{n+1}}"', "\sim"] & I^{n+1} \oplus C^{n+1}
		\end{tikzcd} \quad \text{恒交换, 其中}\;
		e^n := \begin{pmatrix} d_I^n & \delta^n \\ 0 & d_C^n \end{pmatrix}.
	\end{equation}

	同样可用引理 \ref{prop:split-differential} 从任意 $\theta: K \to I[1]$ 构造复形 $J$, 使得
	\[ J^n := I^n \oplus K^n, \quad d_J^n := \begin{pmatrix}
		d_I^n & \theta^n \\
		0 & d_K^n
	\end{pmatrix}, \quad 0 \to I \to J \to K \to 0 \;\text{正合}. \]
	我们希望取 $\theta$ 使得合成 $L^n \xrightarrow[\sim]{\Phi^n} I^n \oplus C^n \xrightarrow{(\identity, \eta^n)} J^n$ 给出态射 $L \to J$. 倘若此性质成立, 则 $L \to J$ 单; 记 $\kappa$ 为 $B \to L \to J$ 的合成, 它仍然单. 容易验证断言中的图表交换. 由于 $\epsilon$ 和 $\eta$ 都是拟同构, 命题 \ref{prop:long-exact-sequence-ses} 的长正合列, 其函子性连同五项引理 (命题 \ref{prop:5-lemma}) 表明 $\kappa$ 也是拟同构, 因而是所求的内射解消.

	如何取 $\theta$? 为了使上述的 $L^n \to J^n$ 成为复形的态射, 根据 \eqref{eqn:horseshoe-aux}, 充要条件是
	\[\begin{pmatrix}
		\identity_{I^{n+1}} & 0 \\
		0 & \eta^{n+1}
	\end{pmatrix} \begin{pmatrix}
		d_I^n & \delta^n \\
		0 & d_K^n
	\end{pmatrix} = \begin{pmatrix}
		d_I^n & \theta^n \\
		0 & d_K^n \\
	\end{pmatrix} \begin{pmatrix}
		\identity_{I^n} & 0 \\
		0 & \eta^n
	\end{pmatrix}, \quad n \in \Z ; \]
	换言之, $\delta = \theta \eta$. 由于拟同构 $\eta$ 是 $\cate{C}(\mathcal{A})$ 的单态射, 而 $I[1]$ 由内射对象组成, 引理 \ref{prop:ext-alpha-beta} 确保有这般的 $\theta: K \to I[1]$. 证毕.
\end{proof}

为了陈述下一个结果, 我们对给定的复形 $X$ 和每个 $p \in \Z$ 定义
\[ B^p := \Image\left( d_X^{p-1} \right), \quad Z^p := \Ker\left( d_X^p \right), \quad H^p := Z^p/B^p = \Hm^p(X) , \]
依此将 $X$ 拆解为以下短正合列
\[ 0 \to Z^p \to X^p \xrightarrow{d_X^p} B^{p+1} \to 0, \quad 0 \to B^p \to Z^p \to H^p \to 0. \]
即将介绍的 Cartan--Eilenberg 解消可设想为这些短正合列的同步解消.

\begin{theorem}[H.\ Cartan, S.\ Eilenberg]\label{prop:CE-resolution}
	\index{jiexiao!Cartan--Eilenberg}
	设 $\mathcal{A}$ 有足够的内射对象. 对 $\cate{C}^+(\mathcal{A})$ 的每个对象 $X$, 存在满足以下条件的双复形 $I$, 连同 $\cate{C}(\mathcal{A})$ 中的态射 $\epsilon: X \to (I^{\bullet, 0}, \dhori^{\bullet, 0})$, 其中 $\dhori$ 和 $\dvert$ 来自 $I$ 的双复形结构 (定义 \ref{def:double-cplx}):
	\begin{enumerate}[(i)]
		\item 对所有 $(p, q) \in \Z^2$ 都有 $q < 0 \implies I^{p, q} = 0$.
		\item 取 $N \in \Z$ 使得 $n < N \implies X^n = 0$, 则对所有 $(p, q) \in \Z^2$ 都有 $p < N \implies I^{p, q} = 0$.
		\item 对每个 $p \in \Z$, 我们有 $X^p \in \Obj(\mathcal{A})$ 的内射解消
		\[ 0 \to X^p \xrightarrow{\epsilon^p} I^{p, 0} \xrightarrow{\dvert^{p, 0}} I^{p, 1} \xrightarrow{\dvert^{(p, 1)}} \cdots . \]
		\item 上述态射诱导 $Z^p := \Ker\left( d_X^p \right)$ 的内射解消
		\[ 0 \to \Ker\left( d_X^p \right) \to \Ker\left( \dhori^{p, 0} \right) \to \Ker\left( \dhori^{p, 1} \right) \to \cdots . \]
		\item 类似地, $B^{p+1} := \Image\left( d_X^p \right)$ 有内射解消
		\[ 0 \to \Image\left( d_X^p \right) \to \Image\left( \dhori^{p, 0} \right) \to \Image\left( \dhori^{p, 1} \right) \to \cdots . \]
		\item 类似地, $H^p := \Hm^p\left(X\right)$ 有内射解消
		\[ 0 \to \Hm^p\left(X\right) \to \underbracket{\Hm^p\left( I^{\bullet, 0}, \dhori \right) \to \Hm^p\left( I^{\bullet, 1}, \dhori \right) \to \cdots}_{= \Hm^p_{\mathrm{I}}\left( I \right), \;\text{请见 \S\ref{sec:double-cplx-coh}} } . \]
	\end{enumerate}
	具备上述性质的资料 $(I, \epsilon)$ 称为 $X$ 的 \emph{Cartan--Eilenberg 解消}.
\end{theorem}
\begin{proof}
	取 $N \in \Z$ 使得 $n < N \implies X^n = 0$. 之前已经定义了短正合列
	\begin{align*}
		0 \to Z^N \to X^N \to B^{N+1} \to 0, & & 0 \to B^{N+1} \to Z^{N+1} \to H^{N+1} \to 0, \\
		0 \to Z^{N+1} \to X^{N+1} \to B^{N+2} \to 0, & & 0 \to B^{N+2} \to Z^{N+2} \to H^{N+2} \to 0, \\
		\vdots & &
	\end{align*}
	为每个 $H^n$ (或 $B^n$) 循例 \ref{eg:resolution-single} 的方式选定内射解消, 当 $n < N$ (或 $n \leq N$) 时取之为 $0 \to 0$.
	\begin{compactitem}
		\item 注意到 $Z^N = H^N$. 命题 \ref{prop:horseshoe} 将 $Z^N$ 和 $B^{N+1}$ 的内射解消一道扩充为 $X^N$ 的内射解消 $I^{N, \bullet}$, 与第一个短正合列相容.
		\item 其次, 以命题 \ref{prop:horseshoe} 将 $B^{N+1}$ 连同 $H^{N+1}$ 的内射解消一道扩充为 $Z^{N+1}$ 的内射解消, 与第二个短正合列相容.
		\item 现在重复同样操作, 得到 $X^{N+1}$ 的内射解消 $I^{N+1, \bullet}$, 连同 $Z^{N+2}$ 的内射解消, 分别与第三和第四个短正合列相容.
	\end{compactitem}

	依此类推, 对每个 $n$ 得到内射解消 $X^n \to I^{n, 0} \to I^{n,1} \to \cdots$, 当 $n < N$ 时 $I^{n, \bullet} := 0$. 以上构造也指明如何定义 $I^{n,m} \to I^{n,m+1}$ 以得到双复形 $(I, \dhori, \dvert)$; 须验证 $\dhori^2 = 0$, 但这点没有实质困难.
\end{proof}

亦可将 $\epsilon$ 视同双复形的态射 $X \to I$, 前提是将 $X$ 等同于集中在第 $0$ 行 (即横轴 $q=0$) 的双复形.

\begin{remark}\label{rem:CE-split}
	对所有 $p, q \in \Z$, Cartan--Eilenberg 解消涉及的短正合列
	\begin{gather*}
		0 \to \Ker\left( \dhori^{p,q} \right) \to I^{p,q} \xrightarrow{\dhori^{p,q}} \Image\left( \dhori^{p,q} \right) \to 0, \\
		0 \to \Image\left( \dhori^{p-1,q} \right) \to \Ker\left( \dhori^{p,q} \right) \to \underbracket{\Hm^p\left( I^{\bullet, q}, \dhori \right)}_{=: \Hm_{\mathrm{I}}(I)^{p,q}} \to 0, \\
		0 \to \Image\left( \dhori^{p-1,q} \right) \to I^{p, q} \to \Coker\left( \dhori^{p-1,q} \right) \to 0,
	\end{gather*}
	其左端都是内射对象, 故引理 \ref{prop:inj-proj-split} 蕴涵它们分裂. 回忆到分裂短正合列被一切加性函子 $F: \mathcal{A} \to \mathcal{B}$ 保持. 当 $q$ 固定, 上述短正合列对 $F$ 的像遂给出复形 $(F(I^{\bullet, q}), F(\dhori))$ 的相应拆解; 特别地, 由此可见
	\[ F\Hm^p\left( I^{\bullet, q}, \dhori \right) \simeq \Hm^p\left( F(I^{\bullet, q}), F(\dhori) \right). \]
	换言之, $F$ 保持 Cartan--Eilenberg 解消中的横向上同调.
\end{remark}

\begin{remark}\label{rem:double-cplx-resolution}
	Cartan--Eilenberg 解消给出为 $X \in \Obj(\cate{C}^+(\mathcal{A}))$ 构造内射解消的另一种迂回手段. 首先将 $X$ 看成集中在第 $0$ 行的双复形. 易见 $\epsilon: X \to I$ 成为 $\cate{C}^2_f(\mathcal{A})$ 中的态射. 按 \S\ref{sec:double-cplx-coh} 的符号, 对 $X$ 和 $I$ 先取纵向上同调 $\Hm_{\mathrm{II}}$, 再取横向上同调 $\Hm_{\mathrm{I}}$, 则诱导态射 $\Hm_{\mathrm{I}} \Hm_{\mathrm{II}} (X) \to \Hm_{\mathrm{I}} \Hm_{\mathrm{II}} (I)$ 为同构. 定理 \ref{prop:double-cplx-tot} 遂表明
	\[ \tot(\epsilon): X = \tot(X) \to \tot(I) \]
	是 $\cate{C}^+(\mathcal{A})$ 中的拟同构. 注意到 $\tot^n(I)$ 对每个 $n \in \Z$ 都是内射对象的有限直和, 于是我们得到内射解消 $X \to \tot(I)$.
\end{remark}

\section{经典导出函子}\label{sec:derived-primer}
本节目的是在不使用导出范畴语言的前提下, 说明如何对有足够内射对象 (或投射对象) 的 Abel 范畴探讨种种左导出函子 (或右导出函子); 这已经囊括应用中常见的许多上同调理论. 具体定义依赖于 \S\ref{sec:resolutions} 介绍的内射解消 (或投射解消).

\begin{definition}\label{def:derived-primer}
	\index{daochuhanzi@导出函子 (derived functor)}
	\index[sym1]{RnF@$\mathrm{R}^n F$}
	\index[sym1]{LnF@$\mathrm{L}_n F$}
	设 $F: \mathcal{A} \to \mathcal{B}$ 为 Abel 范畴之间的加性函子.
	\begin{itemize}
		\item 设 $\mathcal{A}$ 有足够的内射对象. 对每个 $X \in \cate{C}^{+}(\mathcal{A})$ 选取内射解消 $X \to I$; 对每个 $n \in \Z$, 定义 $F$ 的第 $n$ 次\emph{右导出函子} $\mathrm{R}^n F$ 在 $X$ 处的取值为
		\[ \mathrm{R}^n F(X) := \Hm^n(\cate{C}F(I)). \]
		\item 设 $\mathcal{A}$ 有足够的投射对象. 对每个 $X \in \cate{C}^{-}(\mathcal{A})$ 选取投射解消 $P \to X$; 对每个 $n \in \Z$, 定义 $F$ 的第 $n$ 次\emph{左导出函子} $\mathrm{L}_n F$ 在 $X$ 处的取值为
		\[ \mathrm{L}_n F(X) := \Hm^{-n}(\cate{C}F(P)). \]
	\end{itemize}
	这些函子取值都在 $\mathcal{B}$ 中. 依构造, 它们自带一族典范态射 $\Hm^n(\cate{C}F(X)) \to \mathrm{R}^n F(X)$ 和 $\mathrm{L}_n F(X) \to \Hm^{-n}(\cate{C}F(X))$.
\end{definition}

这一表述留下几个问题. 首先, 它们在何种意义下依赖于解消的选取? 其次, 以上仅是对象层次的定义, 如何将其升级为函子? 左右两种导出函子显然相对偶, 故以下仅论右导出函子. 我们且从第二个问题入手, 将基本工具表述成一则引理.

\begin{lemma}\label{prop:resolution-homotopy-classical}
	考虑 $\cate{C}^+(\mathcal{A})$ 中的图表 (实线部分)
	\[\begin{tikzcd}
		X \arrow[d, "f"'] \arrow[r] & I \arrow[dashed, d, "\beta"]\\
		Y \arrow[r] & J
	\end{tikzcd} \qquad \begin{array}{rl}
		X \to I: & \text{拟同构}, \\
		Y \to J: & \text{内射解消}, \\
		f: & \text{任意态射},
	\end{array}\]
	此时在 $\cate{K}(\mathcal{A})$ 中存在唯一的态射 $\beta$, 使得图表在 $\cate{K}(\mathcal{A})$ 中交换.
\end{lemma}
\begin{proof}
	对 $X \to I$ 和 $X \xrightarrow{f} Y \to J$ 应用定理 \ref{prop:resolution-homotopy}.
\end{proof}

反转箭头可在 $\cate{C}^-(\mathcal{A})$ 中得到投射解消的版本. 事实上, 本章习题将证明当 $X \hookrightarrow I$ 时可取 $\beta$ 使图表在 $\cate{C}(\mathcal{A})$ 中交换. 由于本节只关心 $\Hm^n(\beta)$, 故 $\cate{K}(\mathcal{A})$ 版本已然足够.

转回导出函子的讨论. 在 $\cate{C}^+(\mathcal{A})$ 中考虑 $f: X \to Y$ 和内射解消 $X \to I$, $Y \to J$. 代入引理 \ref{prop:resolution-homotopy-classical}, 可见对于所有 $n \in \Z$, 态射
\[ \mathrm{R}^n F(f) := \Hm^n(\cate{K}F\beta): \; \Hm^n(\cate{K}F(I)) \to \Hm^n(\cate{K}F(J)) \]
仅依赖 $f$ 和 $X \to I$, $Y \to J$, 而且一旦选定内射解消, 则有
\begin{gather*}
	\mathrm{R}^n F(f_1 + f_2) = \mathrm{R}^n F(f_1) + \mathrm{R}^n F(f_2), \\
	\mathrm{R}^n F(gf) = \mathrm{R}^n F(g) \; \mathrm{R}^n F(f), \quad \mathrm{R}^n F(\identity) = \identity.
\end{gather*}

取 $X = Y$, $f = \identity_X$, 则上述讨论还蕴涵不同的内射解消给出相同之 $(\mathrm{R}^n F)(X)$, 精确到唯一的同构. 这一切表明 $\mathrm{R}^n F(X)$ 一如范畴论中种种由泛性质刻画的对象, 在典范同构的意义下不依赖辅助资料 (内射解消) 的选取, 并且 $\mathrm{R}^n F: \cate{C}^{+}(\mathcal{A}) \to \mathcal{B}$ 是加性函子.

按构造, 如果 $F, G: \mathcal{A} \to \mathcal{B}$ 都是加性函子, 则每个态射 $F \to G$ 都典范地对所有 $n \in \Z$ 诱导 $\mathrm{R}^n F \to \mathrm{R}^n G$. 一旦取定内射解消 $X \to I$, 则 $\mathrm{R}^n F (X) \to \mathrm{R}^n G (X)$ 具体由 $\cate{C}F(I) \to \cate{C}G(I)$ 确定. 类似地, 它也诱导 $\mathrm{L}_n F \to \mathrm{L}_n G$.

\begin{convention}
	以下谈论加性函子 $F: \mathcal{A} \to \mathcal{B}$ 的右导出函子 (或左导出函子) 时, 总默认 $\mathcal{A}$ 有足够的内射对象 (或投射对象).
\end{convention}

\begin{theorem}[导出函子的长正合列]\label{prop:long-exact-sequence-primer}
	\index{changzhenghelie}
	设 $F: \mathcal{A} \to \mathcal{B}$ 为 Abel 范畴之间的加性函子. 考虑 $\cate{C}(\mathcal{A})$ 中的短正合列 $0 \to X \to Y \to Z \to 0$.
	\begin{itemize}
		\item 设 $X, Y, Z \in \Obj\left(\cate{C}^+(\mathcal{A})\right)$, 此时存在一族典范态射 $\delta^n = \delta^n_{X,Y,Z}: \mathrm{R}^n F(Z) \to \mathrm{R}^{n+1} F(X)$ (其中 $n \in \Z$), 使得我们有正合列
		\[ \cdots \to \mathrm{R}^{n-1} F(Z) \xrightarrow{\delta^{n-1}} \mathrm{R}^n F(X) \to \mathrm{R}^n F(Y) \to \mathrm{R}^n F(Z) \xrightarrow{\delta^n} \mathrm{R}^{n+1} F(X) \to \cdots . \]
		\item 设 $X, Y, Z \in \Obj\left(\cate{C}^-(\mathcal{A})\right)$, 此时存在一族典范态射 $\partial_n = \partial_n^{X,Y,Z}: \mathrm{L}_n F(Z) \to \mathrm{L}_{n-1} F(X)$ (其中 $n \in \Z$), 使得我们有正合列
		\[ \cdots \to \mathrm{L}_{n+1} F(Z) \xrightarrow{\partial_{n+1}} \mathrm{L}_n F(X) \to \mathrm{L}_n F(Y) \to \mathrm{L}_n F(Z) \xrightarrow{\partial_n} \mathrm{L}_{n-1} F(X) \to \cdots . \]
	\end{itemize}
	而且连接态射 $\delta^n$ 和 $\partial_n$ 具以下函子性: 设
	\[\begin{tikzcd}
		0 \arrow[r] & X \arrow[d, "\alpha"] \arrow[r] & Y \arrow[d, "\beta"] \arrow[r] & Z \arrow[d, "\gamma"] \arrow[r] & 0 \\
		0 \arrow[r] & X' \arrow[r] & Y' \arrow[r] & Z' \arrow[r] & 0
	\end{tikzcd}\]
	是行正合交换图表, 则图表
	\[\begin{tikzcd}
		\mathrm{R}^n F(Z) \arrow[r, "{\delta^n}"] \arrow[d, "{\mathrm{R}^n F(\gamma)}"'] & \mathrm{R}^{n+1} F(X) \arrow[d, "{\mathrm{R}^{n+1} F(\alpha)}"] \\
		\mathrm{R}^n F(Z') \arrow[r, "{\delta^n}"'] & \mathrm{R}^{n+1} F(X')
	\end{tikzcd} \quad \text{或} \quad \begin{tikzcd}
		\mathrm{L}_n F(Z) \arrow[r, "{\partial_n}"] \arrow[d, "{\mathrm{L}_n F(\gamma)}"'] & \mathrm{L}_{n-1} F(X) \arrow[d, "{\mathrm{L}_{n-1} F(\alpha)}"] \\
		\mathrm{L}_n F(Z') \arrow[r, "{\partial_n}"'] & \mathrm{L}_{n-1} F(X')
	\end{tikzcd}\]
	对所有 $n \in \Z$ 皆交换.
\end{theorem}
\begin{proof}
	仅论 $\mathrm{R}^n F$ 情形. 对短正合列 $0 \to X \to Y \to Z \to 0$ 运用命题 \ref{prop:horseshoe}, 得到行正合交换图表
	\[\begin{tikzcd}
		0 \arrow[r] & I \arrow[r] & J \arrow[r] & K \arrow[r] & 0 \\
		0 \arrow[r] & X \arrow[r] \arrow[u] & Y \arrow[r] \arrow[u] & Z \arrow[r] \arrow[u] & 0
	\end{tikzcd}\]
	其中每一列都是内射解消. 引理 \ref{prop:inj-proj-split} 确保 $0 \to I^n \to J^n \to K^n \to 0$ 分裂, 于是它们在 $F$ 下的像也是分裂短正合列; 作为推论, $0 \to \cate{C}F(I) \to \cate{C}F(J) \to \cate{C}F(K) \to 0$ 仍然正合. 因此命题 \ref{prop:long-exact-sequence-ses} 给出态射族 $\delta^n$ 以及长正合列
	\[ \cdots \to \mathrm{R}^{n-1} F(Z) \xrightarrow{\delta^{n-1}} \mathrm{R}^n F(X) \to \mathrm{R}^n F(Y) \to \mathrm{R}^n F(Z) \xrightarrow{\delta^n} \cdots \]
	
	连接态射 $\delta^n$ 的函子性比较复杂, 这里运用例 \ref{eg:A2-injective-projective} 的 Abel 范畴 $\mathcal{A}^{\mathbf{2}}$ 来处理. 首先将给定的行正合交换图表写作 $\mathcal{A}^{\mathbf{2}}$ 的短正合列
	\[ 0 \to [X \xrightarrow{\alpha} X'] \to [Y \xrightarrow{\beta} Y'] \to [Z \xrightarrow{\gamma} Z'] \to 0. \]
	由例 \ref{eg:A2-injective-projective} 已知 $\mathcal{A}^{\mathbf{2}}$ 有足够的内射对象, 于是可取内射解消 $[X \to X'] \hookrightarrow \mathcal{I}$ 和 $[Z \to Z'] \hookrightarrow \mathcal{K}$, 再以命题 \ref{prop:horseshoe} 将之延拓为 $\cate{C}(\mathcal{A}^{\mathbf{2}})$ 中的行正合交换图表
	\begin{equation}\label{eqn:long-exact-sequence-primer-aux}\begin{tikzcd}
		0 \arrow[r] & \mathcal{I} \arrow[r] & \mathcal{J} \arrow[r] & \mathcal{K} \arrow[r] & 0 \\
		0 \arrow[r] & {[X \to X']} \arrow[r] \arrow[hookrightarrow, u] & {[Y \to Y']} \arrow[r] \arrow[hookrightarrow, u] & {[Z \to Z']} \arrow[r] \arrow[hookrightarrow, u] & 0
	\end{tikzcd}\end{equation}
	使 $[Y \to Y'] \to \mathcal{J}$ 也是内射解消. 接着将 $\mathcal{J}^n$ 展开为 $[J^n \to (J')^n]$ 等等. 运用例 \ref{eg:A2-injective-projective} 的函子 $\mathrm{ev}_0$ 和 $\mathrm{ev}_1$ 可知 $J^n$, $(J')^n$ 等等都是 $\mathcal{A}$ 的内射对象, 而 $\cate{C}(\mathcal{A})$ 中的态射 $Y \to J$, $Y' \to J'$ 等等是内射解消. 综之, 这在原给定的交换图表之上竖起了相容的内射解消.
	
	现将 \eqref{eqn:long-exact-sequence-primer-aux} 的第一行展开为 $\cate{C}(\mathcal{A})$ 中的行正合交换图表
	\[\begin{tikzcd}
		0 \arrow[r] & I \arrow[d] \arrow[r] & J \arrow[d] \arrow[r] & K \arrow[d] \arrow[r] & 0 \\
		0 \arrow[r] & I' \arrow[r] & J' \arrow[r] & K' \arrow[r] & 0
	\end{tikzcd}\]
	因为每个 $I^n$ 和 $(I')^n$ 皆是内射对象, 上图取 $\cate{C}F$ 之后依然行正合. 故 $\delta^n$ 的函子性化约为命题 \ref{prop:long-exact-sequence-ses} 的相应陈述.
\end{proof}

在经典场景下, 主要考虑的是函子 $\mathrm{R}^n F$ 或 $\mathrm{L}_n F$ 在 $\mathcal{A}$ 上的限制, 仍写作 $\mathrm{R}^n F(X)$ 或 $\mathrm{L}_n F(X)$ 的形式, $X \in \Obj(\mathcal{A})$. 这是通过``摊平''为正合列的内射解消 $0 \to X \to I^0 \to I^1 \to \cdots$ (或投射解消 $\cdots \to P^1 \to P^0 \to X \to 0$) 来计算的; 见例 \ref{eg:resolution-single} 的说明. 对复形定义的导出函子在早期文献中称为\emph{超导出函子}, 更适合以导出范畴或谱序列来处理. 是故以下聚焦于 $X \in \Obj(\mathcal{A})$ 的经典情形.
\index{chaodaochuhanzi@超导出函子 (hyper-derived functor)}

第一步是将定理 \ref{prop:long-exact-sequence-primer} 的长正合列提炼为 $\delta$-函子的概念.

\begin{definition}\label{def:delta-functor}
	\index{delta hanzi@$\delta$-函子 ($\delta$-functor)}
	设 $\mathcal{A}$, $\mathcal{B}$ 为 Abel 范畴. 从 $\mathcal{A}$ 到 $\mathcal{B}$ 的一族\emph{上同调 $\delta$-函子}意谓以下资料:
	\begin{compactitem}
		\item 一族加性函子 $F^n: \mathcal{A} \to \mathcal{B}$, 其中 $n \in \Z_{\geq 0}$;
		\item 对 $\mathcal{A}$ 中的每个短正合列 $0 \to X \to Y \to Z \to 0$ 指定一族态射 $\delta^n: F^n Z \to F^{n+1} X$ (称为连接态射), 其中 $n \in \Z_{\geq 0}$, 它们对短正合列之间的态射具有函子性, 并且有正合列
		\begin{equation*}\begin{tikzcd}[row sep=large]
			0 \arrow[r] & F^0(X) \arrow[r] & F^0(Y) \arrow[r] \arrow[phantom, d, ""{coordinate, name=A}] & F^0(Z) \arrow[dll, rounded corners, "{\delta^0}" description, to path={
				-- ([xshift=2ex]\tikztostart.east)
				|- (A) [near end]\tikztonodes
				-| ([xshift=-2ex]\tikztotarget.west)
				-- (\tikztotarget)}] \\
			& F^1(X) \arrow[r] & F^1(Y) \arrow[r] \arrow[phantom, d, ""{coordinate, name=B}] & F^1(Z) \arrow[dll, rounded corners, "{\delta^1}" description, to path={
				-- ([xshift=2ex]\tikztostart.east)
				|- (B) [near end]\tikztonodes
				-| ([xshift=-2ex]\tikztotarget.west)
				-- (\tikztotarget)}] \\
			& F^2(X) \arrow[r] & F^2(Y) \arrow[r] & \cdots .
		\end{tikzcd}\end{equation*}
	\end{compactitem}
	从上同调 $\delta$-函子 $(F^n, \delta^n)_{n \geq 0}$ 到 $(G^n, \eta^n)_{n \geq 0}$ 之间的态射是一族态射 $\varphi^n: F^n \to G^n$, 要求和短正合列给出的连接态射相容; 换言之, 要求图表
	\[\begin{tikzcd}
		G^{n-1} (X) \arrow[r, "{\eta^{n-1}}"] & G^n (X) \\
		F^{n-1} (X) \arrow[r, "{\delta^{n-1}}"'] \arrow[u, "{\varphi^{n-1}}"] & F^n (X) \arrow[u, "\varphi^n"']
	\end{tikzcd}\]
	对所有 $n \geq 1$ 和短正合列 $0 \to X \to Y \to Z \to 0$ 皆交换.
	
	对偶地, 从 $\mathcal{A}$ 到 $\mathcal{B}$ 的一族\emph{同调 $\delta$-函子}意谓以下资料:
	\begin{compactitem}
		\item 一族加性函子 $F_n: \mathcal{A} \to \mathcal{B}$, 其中 $n \in \Z_{\geq 0}$;
		\item 对 $\mathcal{A}$ 中的每个短正合列 $0 \to X \to Y \to Z \to 0$ 指定一族连接态射 $\partial_n: F_n Z \to F_{n-1} X$, 对短正合列之间的态射具有函子性, 并且有正合列
		\begin{equation*}\begin{tikzcd}[row sep=large]
			\cdots \arrow[r] & F_2(Y) \arrow[r] \arrow[phantom, d, ""{coordinate, name=A}] & F_2(Z) \arrow[dll, rounded corners, "{\partial_2}" description, to path={
				-- ([xshift=2ex]\tikztostart.east)
				|- (A) [near end]\tikztonodes
				-| ([xshift=-2ex]\tikztotarget.west)
				-- (\tikztotarget)}] & \\
			F_1(X) \arrow[r] & F_1(Y) \arrow[r] \arrow[phantom, d, ""{coordinate, name=B}] & F_1(Z) \arrow[dll, rounded corners, "{\partial_1}" description, to path={
				-- ([xshift=2ex]\tikztostart.east)
				|- (B) [near end]\tikztonodes
				-| ([xshift=-2ex]\tikztotarget.west)
				-- (\tikztotarget)}] & \\
			F_0 (X) \arrow[r] & F_0 (Y) \arrow[r] & F_0 (Z) \arrow[r] & 0.
		\end{tikzcd}\end{equation*}
	\end{compactitem}
	同调 $\delta$-函子之间的态射按和先前类似的方法定义, 要求与连接态射相容.
\end{definition}

留意到定义中的长正合列自动蕴涵 $F^0$ 左正合, $F_0$ 右正合. 现在回到导出函子.

\begin{proposition}\label{prop:derived-primer-long}
	对 Abel 范畴之间的加性函子 $F: \mathcal{A} \to \mathcal{B}$, 当 $\mathcal{A}$ 有足够的内射对象时, 以下性质成立:
	\begin{compactitem}
		\item $n < 0 \implies \mathrm{R}^n F = 0$;
		\item 若 $I$ 是 $\mathcal{A}$ 的内射对象, 则 $n > 0 \implies \mathrm{R}^n F(I) = 0$;
		\item $(\mathrm{R}^n F, \delta^n)_{n \geq 0}$ 成为上同调 $\delta$-函子;
		\item 若 $F$ 左正合, 则有典范同构 $F \rightiso \mathrm{R}^0 F$.
	\end{compactitem}

	对偶地, 当 $\mathcal{A}$ 有足够的投射对象时, 以下性质成立:
	\begin{compactitem}
		\item $n < 0 \implies \mathrm{L}_n F = 0$;
		\item 若 $P$ 是 $\mathcal{A}$ 的投射对象, 则 $n > 0 \implies \mathrm{L}_n F(P) = 0$;
		\item $(\mathrm{L}_n F, \partial_n)_{n \geq 0}$ 成为同调 $\delta$-函子;
		\item 若 $F$ 右正合, 则有典范同构 $\mathrm{L}_0 F \rightiso F$.
	\end{compactitem}
\end{proposition}
\begin{proof}
	基于对偶性, 讨论 $\mathrm{R}^n F$ 的情形即可. 取 $X \in \Obj(\mathcal{A})$ 的内射解消 $0 \to X \to I^0 \to I^1 \to \cdots$. 于是
	\[ \mathrm{R}^n F(X) = \Hm^n ( \cdots \to 0 \to 0 \to \underbracket{FI^0}_{\text{$0$ 次项}} \to \underbracket{FI^1}_{\text{$1$ 次项}} \to \cdots ). \]
	由此立见 $n < 0$ 时 $\mathrm{R}^n F(X)$ 为 $0$. 若 $X = I$ 是 $\mathcal{A}$ 的内射对象, 则可取内射解消为 $0 \to I \xrightarrow{\identity} I \to 0 \cdots$ (后续全为 $0$), 从而 $n > 0$ 时 $\mathrm{R}^n F(I) = 0$.
	
	有鉴于此, 关于上同调 $\delta$-函子的断言无非是复述定理 \ref{prop:long-exact-sequence-primer}.

	最后设 $F$ 左正合. 注意到 $X \rightiso \Ker\left[ I^0 \to I^1 \right]$; 左正合函子保核, 故 $\mathrm{R}^0 F(X) = \Ker\left[ FI^0 \to FI^1 \right]$ 典范地同构于 $FX$.
\end{proof}

一般来说, 我们仅对左正合函子考虑右导出函子, 对右正合函子考虑左导出函子. 何以故? 初等的解释基于正合列的补项问题: 以 $F$ 左正合的情形为例, 给定 $\mathcal{A}$ 中的短正合列 $0 \to X \to Y \to Z \to 0$, 我们希望将正合列 $0 \to FX \to FY \to FZ$ 尽量右延. 右导出函子起到的正是这一作用.

习惯将 $n \geq 1$ 时的 $\mathrm{R}^n F$ 或 $\mathrm{L}_n F$ 称为 $F$ 的高次导出函子, 默认定义在 $\mathcal{A}$ 的对象上, 而非复形上. 作为补项的一则应用, 下述推论表明正合性的阻碍恰是高次导出函子.

\begin{corollary}\label{prop:obstruction-exactness}
	对于左正合 (或右正合) 加性函子 $F: \mathcal{A} \to \mathcal{B}$, 以下性质等价.
	\begin{enumerate}[(i)]
		\item $F: \mathcal{A} \to \mathcal{B}$ 正合.
		\item 对于所有 $X \in \Obj(\mathcal{A})$ 和 $n > 0$ 皆有 $\mathrm{R}^n F(X) = 0$ (或 $\mathrm{L}_n F(X) = 0$).
		\item 对于所有 $X \in \Obj(\mathcal{A})$ 皆有 $\mathrm{R}^1 F(X) = 0$ (或 $\mathrm{L}_1 F(X) = 0$).
	\end{enumerate}
\end{corollary}
\begin{proof}
	考虑左正合情形. (i) $\implies$ (ii): 计算导出函子所用的复形 $I^0 \to I^1 \to \cdots$ 在第零项之外都是正合的, 故它对 $F$ 的像亦然. (ii) $\implies$ (iii) 平凡. 至于 (iii) $\implies$ (i), 利用 $\mathrm{R}^0 F \simeq F$, 对 $\mathcal{A}$ 中的短正合列 $0 \to X \to Y \to Z \to 0$ 考察相应的正合列
	\[ 0 \to FX \to FY \to FZ \to \mathrm{R}^1 F(X) \]
	即是.
\end{proof}

\begin{convention}\label{con:F-acyclic}\index{lingdiaoduixiang@$\cdots$-零调对象 ($\cdots$-acyclic object)}
	对于左正合 (或右正合) 加性函子 $F: \mathcal{A} \to \mathcal{B}$, 若 $A \in \Obj(\mathcal{A})$ 对 $F$ 的高次右导出函子 (或左导出函子) 都取 $0$, 则称 $A$ 是 \emph{$F$-零调}的; 例如当 $F$ 左正合 (或右正合) 时, 命题 \ref{prop:derived-primer-long} 蕴涵所有内射对象 (或投射对象) 都是 $F$-零调的.
\end{convention}

以下的经典技巧称为移维, 常用于对导出函子进行一些递归论证.

\begin{proposition}[移维]\label{prop:dimension-shifting}
	\index{yiwei@移维 (dimension shifting)}
	设加性函子 $F: \mathcal{A} \to \mathcal{B}$ 左正合 (或右正合), $\mathcal{A}$ 有足够的内射对象 (或投射对象). 设有 $\mathcal{A}$ 中的短正合列
	\[ 0 \to X \to A \to B \to 0, \quad (\text{或}\; 0 \to B \to A \to X \to 0 ), \]
	而且 $A$ 是 $F$-零调的, 则当 $n \geq 1$ 时有自然同构
	\[\mathrm{R}^n F(X) \simeq \begin{cases}
		\mathrm{R}^{n-1} F(B), & n \geq 2, \\
		\Coker\left[FA \to FB \right], & n = 1
	\end{cases}\]
	或
	\[ \mathrm{L}_n F(X) \simeq \begin{cases}
		\mathrm{L}_{n-1} F(B), & n \geq 2, \\
		\Ker\left[ FB \to FA \right], & n = 1.
	\end{cases} \]
\end{proposition}
\begin{proof}
	基于对偶性, 仅证 $\mathrm{R}^n F(X)$ 的情形. 仔细打量相应的长正合列
	\[ \mathrm{R}^{n-1} F(A) \to \mathrm{R}^{n-1} F(B) \to \mathrm{R}^n F(X) \to \underbracket{\mathrm{R}^n F(A)}_{= 0}, \quad n \in \Z_{\geq 1}, \]
	并应用 $\mathrm{R}^0 F(A) \simeq FA$ 和 $\mathrm{R}^0 F(B) = FB$ 便是.
\end{proof}

次一关键结果表明 $F$-零调对象也可用来求导出函子.

\begin{corollary}\label{prop:acyclic-resolution}
	设 $F: \mathcal{A} \to \mathcal{B}$ 左正合 (或右正合), 而且有 $\mathcal{A}$ 中的正合列
	\[ 0 \to X  \to A^0 \to A^1 \to \cdots \quad (\text{或}\; \cdots \to A_1 \to A_0 \to X \to 0 ), \]
	其中 $A^n$ (或 $A_n$) 对所有 $n \geq 0$ 都是 $F$-零调对象; 另外命 $A^{-n} = 0$ (或 $A_{-n} = 0$). 此时有自然同构
	\[ \mathrm{R}^n F(X) \simeq \Hm^n\left( FA^\bullet \right) \quad \text{或}\quad \mathrm{L}_n F(X) \simeq \Hm_n\left(FA_\bullet\right). \]
\end{corollary}
\begin{proof}
	仅证 $\mathrm{R}^n F(X)$ 情形. 对 $m \geq 1$ 命
	\[ B^m := \Image\left[ A^{m-1} \to A^m \right] = \Ker\left[ A^m \to A^{m+1} \right]; \]
	另记 $B^0 := X$. 原正合列遂拆解为
	\[ 0 \to B^m \to A^m \to B^{m+1} \to 0, \quad m \in \Z_{\geq 0}. \]
	当 $n=0$, 断言归结为 $F$ 保核: $FX \simeq \Ker[FA^0 \to FA^1]$. 当 $n \geq 1$, 以命题 \ref{prop:dimension-shifting} 反复移维, 得到一系列自然同构
	\[ \mathrm{R}^n F(X) = \mathrm{R}^n F(B^0) \simeq \cdots \simeq \mathrm{R}^1 F(B^{n-1}) \simeq \Coker\left[ FA^{n-1} \to FB^n \right]; \]
	又因为 $F$ 保核, $B^n$ 的定义蕴涵 $FB^n$ 等同于 $\Ker\left[ FA^n \to FA^{n+1} \right]$, 由此知 $n \geq 1$ 时
	\[ \mathrm{R}^n F(X) \simeq \frac{\Ker\left[ FA^n \to FA^{n+1} \right]}{\Image\left[ FA^{n-1} \to FA^n \right]}. \]
	明所欲证.
\end{proof}

留意到以上论证主要依赖长正合列, 鲜少动用导出函子的定义, 这就提示我们在一般的上同调 (或同调) $\delta$-函子之中刻画 $F$ 的导出函子. 关键在于要求高次 $\delta$-函子在适当的对象上零化.

\begin{definition}[A.\ Grothendieck]\label{def:erasable}
	\index{hanzi!可拭, 余可拭 (effaceable, co-effaceable)}
	设 $E: \mathcal{A} \to \mathcal{B}$ 为 Abel 范畴之间的加性函子. 若对所有 $X \in \Obj(\mathcal{A})$ 皆存在单态射 $a: X \hookrightarrow Y$ (或满态射 $a: Y \twoheadrightarrow X$) 使得 $Ea = 0$, 则称 $E$ \emph{可拭} (或\emph{余可拭}).
\end{definition}

我们即将从可拭性质推导以下泛性质.

\begin{definition}\label{def:universal-delta-functor}
	\index{delta hanzi!泛 (universal)}
	设 $(R^n, \delta^n)_{n \geq 0}$ 为从 $\mathcal{A}$ 到 $\mathcal{B}$ 的上同调 $\delta$-函子. 若对于所有从 $\mathcal{A}$ 到 $\mathcal{B}$ 的上同调 $\delta$-函子 $(F^n, \delta_F^n)_{n \geq 0}$, 所有态射 $\varphi^0: R^0 \to F^0$ 皆能唯一地延拓为 $\delta$-函子之间的态射 $(\varphi^n)_{n \geq 0}$, 则称 $(R^n, \delta^n)_{n \geq 0}$ 为\emph{泛上同调 $\delta$-函子}.
	
	对偶地, 设同调 $\delta$-函子 $(L_n, \partial_n)_{n \geq 0}$ 满足以下性质: 对任何同调 $\delta$-函子 $(F_n, \partial^F_n)_{n \geq 0}$, 任何态射 $\varphi_0: F_0 \to L_0$ 皆可唯一地延拓为 $(\varphi_n)_{n \geq 0}$, 则称 $(L_n, \partial_n)_{n \geq 0}$ 为\emph{泛同调 $\delta$-函子}.
\end{definition}

今后目标是说明给定左正合 (或右正合) 函子 $F: \mathcal{A} \to \mathcal{B}$, 满足 $R^0 = F$ 的泛上同调 $\delta$-函子 (或满足 $L_0 = F$ 的泛同调 $\delta$-函子) 若存在则唯一, 精确到唯一的同构.

\begin{lemma}\label{prop:erasable-aux}
	设 $E: \mathcal{A} \to \mathcal{B}$ 可拭, $0 \to X \xrightarrow{u} I \xrightarrow{v} B \to 0$ 为 $\mathcal{A}$ 的短正合列, $f: X \to Y$ 是任意态射, 则存在行正合的交换图表
	\begin{equation*}\begin{tikzcd}
			0 \arrow[r] & X \arrow[d, "f"'] \arrow[r, "u"] & I \arrow[r, "v"] \arrow[d] & B \arrow[d] \arrow[r] & 0 \\
			0 \arrow[r] & Y \arrow[r, "\iota"'] & J \arrow[r] & \arrow[r] C \arrow[r] & 0 ,
	\end{tikzcd}\end{equation*}
	使得 $E\iota = 0$.
\end{lemma}
\begin{proof}
	取单态射 $h: Y \hookrightarrow J_0$ 使得 $Eh = 0$. 构造推出图表
	\[\begin{tikzcd}
		X \arrow[hookrightarrow, r, "u"] \arrow[d, "hf"'] & I \arrow[d] \\
		J_0 \arrow[r, "k"'] & J \arrow[phantom, lu, "\boxplus" description] ,
	\end{tikzcd}\]
	其中的 $k$ 由命题 \ref{prop:Abel-cat-pull-push} 可知为单. 取 $\iota = kh: Y \to J$, 取 $C := \Coker(\iota)$, 这给出所求图表的左方块; 从余核的函子性得到右方块.
\end{proof}

\begin{proposition}\label{prop:erasable-univ-delta}
	设 $(R^n, \delta^n)_{n \geq 0}$ 是从 $\mathcal{A}$ 到 $\mathcal{B}$ 的上同调 $\delta$-函子. 若 $n > 0$ 蕴涵 $R^n$ 可拭, 则 $(R^n, \delta^n)_{n \geq 0}$ 是泛上同调 $\delta$-函子.
	
	对偶地, $n > 0$ 部分余可拭的同调 $\delta$-函子也必然是泛的.
\end{proposition}
\begin{proof}
	只论上同调版本. 令 $(F^n, \delta_F^n)_{n \geq 0}$ 为从 $\mathcal{A}$ 到 $\mathcal{B}$ 的上同调 $\delta$-函子, $\varphi^0: R^0 \to F^0$ 给定. 以下递归地对 $n \geq 1$ 构造 $\varphi^n$. 给定 $X \in \Obj(\mathcal{A})$, 根据假设, 存在短正合列
	\[ 0 \to X \to I \to B \to 0 \]
	使得 $R^n(X \to I) = 0$. 相应的长正合列给出实线部分的行正合交换图表
	\[\begin{tikzcd}
		F^{n-1} I \arrow[r] & F^{n-1} B \arrow[r, "{\delta_F^{n-1}}"] & F^n X \arrow[r] & F^n I \\
		R^{n-1} I \arrow[u, "{\varphi^{n-1}_I}"] \arrow[r] & R^{n-1} B \arrow[u, "{\varphi^{n-1}_B}"] \arrow[r, "{\delta^{n-1}}"'] & R^n X \arrow[dashed, u, "{\varphi^n_X}"'] \arrow[r] & 0 .
	\end{tikzcd}\]
	根据行的正合性和余核的泛性质, 可知存在唯一的虚线箭头 $\varphi^n_X$ 使全图交换; 这刻画了所求之 $\varphi^n_X$, 但它目前还依赖 $X \hookrightarrow I$ 的选取.
	
	给定任意态射 $f: X \to Y$, 兹断言存在行正合交换图表
	\begin{equation}\label{eqn:erasable-univ-delta}\begin{tikzcd}
		0 \arrow[r] & X \arrow[d, "f"'] \arrow[r] & I \arrow[r] \arrow[d] & B \arrow[d] \arrow[r] & 0 \\
		0 \arrow[r] & Y \arrow[r] & J \arrow[r] & \arrow[r] C \arrow[r] & 0 ,
	\end{tikzcd}\end{equation}
	使得 $R^n(X \to I)$ 和 $R^n(Y \to J)$ 皆为 $0$: 诚然, 先取 $X \hookrightarrow I$, 再代入引理 \ref{prop:erasable-aux} 便是. 将此代入 $\varphi^n_X$ 和 $\varphi^n_Y$ 的构造, 得到
	\[\begin{tikzcd}[row sep=small, column sep=small]
		& F^{n-1} B \arrow[rr] \arrow[dd] & & F^n X \arrow[dd] \\
		R^{n-1} B \arrow[ru] \arrow[rr, crossing over] \arrow[dd] & & R^n X \arrow[ru, "{\varphi_X^n}"'] & \\
		& F^{n-1} C \arrow[rr] & & F^n Y \\
		R^{n-1} C \arrow[ru] \arrow[rr] & & R^n Y \arrow[ru, "{\varphi_Y^n}"'] \arrow[leftarrow, uu, crossing over] &
	\end{tikzcd}\]
	其中除右面以外, 每一面按构造, 定义或递归假设都交换. 又因为 $R^{n-1} B \to R^n X$ 已知满, 根据熟悉的技巧, 右面也随之交换.
	
	以上论证同时说明了 $\varphi^n_X$ 不依赖 $X \hookrightarrow I$ 的选择 (取 $f = \identity_X$ 并注意到根据引理 \ref{prop:erasable-aux} 证明, 任两个 $X \hookrightarrow I_i$ 都能通过推出映入同一个 $X \hookrightarrow J$, 其中 $i=1,2$), 而且对 $X$ 具有函子性 (取任意 $f$). 这就递归地完成了 $\varphi^n$ 的构造.
	
	最后说明 $(\varphi^n)_{n \geq 0}$ 和连接态射相容. 给定短正合列 $0 \to X \to Y \to Z \to 0$, 以引理 \ref{prop:erasable-aux} 取行正合交换图表
	\[\begin{tikzcd}
		0 \arrow[r] & X \arrow[d, "\identity_X"'] \arrow[r] & Y \arrow[r] \arrow[d] & Z \arrow[d, "\alpha"] \arrow[r] & 0 \\
		0 \arrow[r] & X \arrow[r] & I \arrow[r] & \arrow[r] B \arrow[r] & 0
	\end{tikzcd}\]
	使得 $R^n(X \to I) = 0$. 由此对每个 $n \geq 1$ 构造图表
	\[\begin{tikzcd}[column sep=large]
		F^{n-1} Z \arrow[r, "{F^{n-1}\alpha}"] & F^{n-1} B \arrow[r, "{\delta_F^n}"] & F^n X \\
		R^{n-1} Z \arrow[u, "{\varphi_Z^{n-1}}"] \arrow[r, "{R^{n-1}\alpha}"'] & R^{n-1} B \arrow[u, "{\varphi_B^{n-1}}"] \arrow[r, "{\delta^n}"'] & R^n X \arrow[u, "{\varphi_X^n}"'] 
	\end{tikzcd}\]
	根据前一个交换图表和上同调 $\delta$-函子的性质, 两行分别合成为欲考察的连接态射 $F^{n-1} Z \to F^n X$ 和 $R^{n-1} Z \to R^n X$; 问题化为证全图交换. 左块依 $\varphi^{n-1}$ 的函子性交换, 右块依 $\varphi^n_X$ 的构造交换. 证毕.
\end{proof}

\begin{corollary}[导出函子的刻画]\label{prop:derived-erasable}
	设 $F: \mathcal{A} \to \mathcal{B}$ 左正合, $\mathcal{A}$ 有足够的内射对象, 则 $(\mathrm{R}^n F, \delta^n)_{n \geq 0}$ 是满足 $\mathrm{R}^0 F \simeq F$ 的泛上同调 $\delta$-函子.
	
	对偶地, 设 $F$ 右正合而 $\mathcal{A}$ 有足够的投射对象, 则 $(\mathrm{L}_n F, \partial_n)_{n \geq 0}$ 是满足 $\mathrm{L}_0 F \simeq F$ 的泛同调 $\delta$-函子.
\end{corollary}
\begin{proof}
	讨论左正合情形即可. 当 $n > 0$ 时 $\mathrm{R}^n F$ 可拭: 这是因为任何 $X \in \Obj(\mathcal{A})$ 皆可嵌入某个内射对象 $I$, 而命题 \ref{prop:derived-primer-long} 蕴涵 $\mathrm{R}^n F(I) = 0$; 此外它也蕴涵 $\mathrm{R}^0 F \simeq F$. 将此代入命题 \ref{prop:erasable-univ-delta} 便是.
\end{proof}

\section{实例: \texorpdfstring{$\lim\nolimits^1$}{lim1}}\label{sec:lim1}
对于导出函子, 我们选择的首例来自一类常见的 $\varprojlim$. 将 $\Z_{\geq 0}$ 按照标准的全序结构作成范畴, 于是范畴 $\Z_{\geq 0}^{\opp}$ 可以表作 $\cdots \to 2 \to 1 \to 0$. 对任意范畴 $\mathcal{A}$, 考虑函子范畴
\begin{equation}\label{eqn:InvSys}
	\index[sym1]{InvSys@$\cate{InvSys}$}
	\cate{InvSys}(\mathcal{A}) := \mathcal{A}^{\Z_{\geq 0}^{\opp}},
\end{equation}
其对象可以视同资料 $(A_n, f_n)_{n \geq 0}$, 其中 $A_n \in \Obj(\mathcal{A})$ 而 $f_n \in \Hom(A_{n+1}, A_n)$; 这般资料也称为 $\mathcal{A}$ 中的\emph{逆向系}. 本节将以导出函子为工具, 在 Abel 范畴的情形研究它们的 $\varprojlim$.

\begin{convention}\label{con:exact-product}
	\index{Abel fanchou!有正合的可数积或余积 (with exact countable products or coproducts)}
	如果 Abel 范畴 $\mathcal{A}$ 有可数积 (或可数余积, 亦即直和), 而且取积函子 $\prod: \mathcal{A}^{\Z_{\geq 0}} \to \mathcal{A}$ (或直和函子 $\oplus: \mathcal{A}^{\Z_{\geq 0}} \to \mathcal{A}$) 是正合函子, 则称 $\mathcal{A}$ 有正合的可数积 (或可数余积).
\end{convention}

取积函子总是左正合的, 因此在可数积存在的前提下, 其正合性等价于 $\prod$ 保持满态射: 对于任一族满态射 $(g_n: A_n \to B_n)_{n \geq 0}$, 相应的 $\prod_{n \geq 0} g_n: \prod_{n \geq 0} A_n \to \prod_{n \geq 0} B_n$ 亦满. 可数余积的情形则是对偶的.

\begin{hypothesis}\label{hyp:lim1}
	本节考虑的 Abel 范畴 $\mathcal{A}$ 均假定有正合的可数积.
\end{hypothesis}

任意环 $R$ 上的模范畴 $R\dcate{Mod}$ 即有此性质.

\begin{definition}\label{def:Delta-diff}
	\index[sym1]{DeltaA@$\Delta_A$}
	给定 $\cate{InvSys}(\mathcal{A})$ 的对象 $A = (A_n, f_n)_n$, 定义平移态射 $T_A: \prod_{n \geq 0} A_n \to \prod_{n \geq 0} A_n$: 当 $\mathcal{A} = \cate{Ab}$ 时, $T_A((a_n)_{n \geq 0}) := (f_n(a_{n+1}))_{n \geq 0}$, 一般情形准此可知. 再定义典范态射
	\[ \Delta_A := T_A - \identity: \prod_{n \geq 0} A_n \to \prod_{n \geq 0} A_n, \]
	当 $A$ 变动, 这给出取积函子 $\prod_{n \geq 0}$ 的自同态 $T$ 和 $\Delta = T - \identity$.
\end{definition}

留意到以上定义只需要 $\mathcal{A}$ 是有可数积的 $\cate{Ab}$-范畴.

回忆到 $\varprojlim$ 一般通过等化子和积来构造. 对于眼下的情形, $\cate{InvSys}(\mathcal{A})$ 自然地成为 Abel 范畴 (命题 \ref{prop:functor-cat-Abel}), 故构造化为 $\varprojlim A = \Ker\left[ \Delta_A: \prod_n A_n \to \prod_n A_n \right]$.

此外 $\cate{InvSys}(\mathcal{A})$ 也有正合的可数积, 因为函子范畴的极限可以逐项地构造, 见 \S\ref{sec:limit-functor}.

\begin{lemma}
	以上构造给出左正合加性函子 $\varprojlim: \cate{InvSys}(\mathcal{A}) \to \mathcal{A}$. 另一方面, 取积函子 $\prod_{n \geq 0}: \cate{InvSys}(\mathcal{A}) \to \mathcal{A}$ 则是正合函子.
\end{lemma}
\begin{proof}
	两个函子的加性皆显然, 左正合性来自例 \ref{eg:limit-exactness}. 根据假设, 函子 $\prod_{n \geq 0}: (A_n, f_n)_n \mapsto \prod_n A_n$ 也保满态射, 因为它逐项如此. 故注记 \ref{rem:exactness-mono-epi} 说明它正合.
\end{proof}

\begin{definition}
	\index[sym1]{lim1@$\lim\nolimits^1$}
	按以下方式定义一族加性函子 $\lim^n: \cate{InvSys}(\mathcal{A}) \to \mathcal{A}$:
	\begin{align*}
		\lim\nolimits^0 & := \varprojlim = \Ker \Delta: \; A \mapsto \Ker \Delta_A, \\
		\lim\nolimits^1 & := \Coker \Delta: \; A \mapsto \Coker \Delta_A , \\
		\lim\nolimits^n & := 0, \quad n \in \Z_{\geq 2}.
	\end{align*}
\end{definition}

今设 $0 \to X \to Y \to Z \to 0$ 为 $\cate{InvSys}(\mathcal{A})$ 中的短正合列; 换言之, 逐项正合. 应用正合函子 $\prod_{n \geq 0}$ 及其自同态 $\Delta$, 得到行正合交换图表
\[\begin{tikzcd}
	0 \arrow[r] & \prod_n X_n \arrow[r] \arrow[d, "\Delta_X"'] & \prod_n Y_n \arrow[r] \arrow[d, "\Delta_Y"] & \prod_n Z_n \arrow[d, "\Delta_Z"] \arrow[r] & 0 \\
	0 \arrow[r] & \prod_n X_n \arrow[r] & \prod_n Y_n \arrow[r] & \prod_n Z_n \arrow[r] & 0 .
\end{tikzcd}\]
应用定理 \ref{prop:snake-lemma} 得到长正合列
\[ 0 \to \lim\nolimits^0 X \to \lim\nolimits^0 Y \to \lim\nolimits^0 Z \xrightarrow[\text{连接态射}]{\delta^0} \lim\nolimits^1 X \to \lim\nolimits^1 Y \to \lim\nolimits^1 Z \to 0 ; \]
它对短正合列具有函子性. 另对 $n > 0$ 取 $\delta^n := 0$. 总结如下.

\begin{lemma}
	资料 $\left( \lim^n, \delta^n \right)_{n \geq 0}$ 给出从 $\cate{InvSys}(\mathcal{A})$ 到 $\mathcal{A}$ 的上同调 $\delta$-函子 (定义 \ref{def:delta-functor}), 满足 $\lim^0 = \varprojlim$.
\end{lemma}

目光转向 $\varprojlim$ 的右导出函子. 设 $\mathcal{A}$ 有足够的内射对象, 则 $\cate{InvSys}(\mathcal{A})$ 亦然. 事实上, \CHref{sec:Abel-cat}的习题已经讨论过一般的函子范畴 $\mathcal{A}^{\mathcal{C}}$ 的情形, 而这里涉及的是特例 $\mathcal{C} = \Z_{\geq 0}^{\opp}$; 对此, 我们直接验证以下稍加精确的结果.

\begin{lemma}\label{prop:lim1-prep}
	在假设 \ref{hyp:lim1} 的条件下, 定义 $\cate{InvSys}(\mathcal{A})$ 的加性全子范畴 $\mathcal{R}$ 使得
	\[ \Obj(\mathcal{R}) = \left\{\begin{array}{l|l}
		R = (R_k, r_k)_{k \geq 0} & R \;\text{是内射对象}, \; \Delta_R \; \text{满}, \\
		\; \in \Obj\left( \cate{InvSys}(\mathcal{A}) \right) & \forall k, \; R_k \;\text{是 $\mathcal{A}$ 的内射对象}
	\end{array}\right\}, \]
	则对每个 $A \in \Obj\left(\cate{InvSys}(\mathcal{A})\right)$ 皆存在 $R \in \Obj(\mathcal{R})$ 和单态射 $A \hookrightarrow R$.
\end{lemma}
\begin{proof}
	对每个 $m \in \Z_{\geq 0}$ 定义函子 $\mathcal{R}_m: \mathcal{A} \to \cate{InvSys}(\mathcal{A})$, 映对象 $X$ 为
	\begin{equation}\label{eqn:Rm-construction}\begin{tikzcd}[row sep=tiny]
		\cdots \arrow[r, "{\identity}"] & X \arrow[r, "{\identity}"] & X \arrow[r] & 0 \arrow[r] & \cdots \arrow[r] & 0. \\
		\cdots & \text{第 $m+1$ 项} & \text{第 $m$ 项} & & &
	\end{tikzcd}\end{equation}
	易见它是 $\mathrm{ev}_m: (A_n, f_n)_{n \geq 0} \mapsto A_m$ 的右伴随, 从而映内射对象为内射对象 (命题 \ref{prop:adjoint-injective-projective}). 此外不难验证 $\Delta_{\mathcal{R}_m X}$ 是满态射, 对之可以明确写下一个右逆.
	
	给定 $A = (A_n, f_n)_{n \geq 0}$, 对每个 $m$ 取 $\mathcal{A}$ 的内射对象 $I_m$ 和单态射 $A_m \hookrightarrow I_m$, 则 $A \hookrightarrow \prod_{m \geq 0} \mathcal{R}_m (I_m) =: R$, 右式根据引理 \ref{prop:prod-injective-projective} 仍是内射对象; 事实上, $R_k$ 对所有 $k$ 按构造都是 $\mathcal{A}$ 的内射对象. 既然 $\prod_m$ 正合, $\Delta_R = \prod_m \Delta_{\mathcal{R}_m(I_m)}$ 仍然满. 综上, $R \in \Obj(\mathcal{R})$.
\end{proof}

依此便能谈论 $\mathrm{R}^n \varprojlim$, 其中 $n \geq 0$. 回忆到 $\mathrm{R}^n \varprojlim$ 可以刻画为泛上同调 $\delta$-函子 (推论 \ref{prop:derived-erasable}).

\begin{theorem}[S.\ Eilenberg]\label{prop:lim1}
	在假设 \ref{hyp:lim1} 的条件下, 当 $n > 0$ 时 $\lim\nolimits^n$ 是定义 \ref{def:erasable} 所谓的可拭函子. 作为推论, 存在典范同构 $\mathrm{R}^1 \varprojlim \simeq \lim^1$, 而 $n \notin \{0, 1\}$ 时 $\mathrm{R}^n \varprojlim = 0$.
\end{theorem}
\begin{proof}
	引理 \ref{prop:lim1-prep} 表明 $\lim^1$ 可拭, 故 $(\lim^n)_{n \geq 0}$ 也是泛 $\delta$-函子 (命题 \ref{prop:erasable-univ-delta}).
\end{proof}

既然明确了 $\varprojlim$ 的正合性以 $\lim^1$ 为障碍, 现在来研究何时能确保 $\lim^1 = 0$.

\begin{example}\label{eg:quotient-lim1-0}
	若 $\cate{InvSys}(\mathcal{A})$ 的对象 $A$ 满足 $\lim^1 A = 0$, 则对任何商对象 $Q$ 也有 $\lim^1 Q = 0$, 这是因为短正合列 $0 \to K \to A \to Q \to 0$ 诱导的长正合列以 $\lim^1 A \to \lim^1 Q \to 0$ 收尾.
\end{example}

\begin{example}\label{eg:lim1-section}
	如果每个 $f_n: A_{n+1} \to A_n$ 都有截面 (即右逆) $s_n$, 则 $\Delta_A$ 也有截面 $\Sigma_A: \prod_n A_n \to \prod_n A_n$, 因而 $\Delta_A$ 满. 如果使用元素的写法, 则可递归地取 $\Sigma_A((b_n)_n) = (a_n)_n$, 其中 $a_0 = 0$ 而 $a_{n+1} := s_n\left( a_n + b_n \right)$. 一般情形依此可知.
\end{example}

\begin{definition}[Mittag-Leffler 条件]\label{def:ML}
	\index{Mittag-Leffler tiaojian@Mittag-Leffler 条件 (Mittag-Leffler condition)}
	设范畴 $\mathcal{C}$ 的所有态射都有像 (定义 \ref{def:Im-Coim}). 考虑 $\cate{InvSys}(\mathcal{C})$ 的对象 $X = (X_k, f_k)_{k \geq 0}$. 对于 $b \geq a$, 定义 $f^b_a: X_b \to X_a$ 为合成 $f_a \cdots f_{b-1}$ (约定 $f^a_a = \identity$). 若以下条件成立, 则称 $X$ 是 Mittag-Leffler 的:
	\[ \forall k \geq 0, \; \exists N \geq k, \quad n \geq N \implies \Image f^n_k = \Image f^N_k. \]
\end{definition}

举例明之, 若每个 $f_k$ 皆满, 则 $(X_k, f_k)_{k \geq 0}$ 自动是 Mittag-Leffler 的.

对于 Mittag-Leffler 的 $(X_k, f_k)_{k \geq 0}$, 每个 $X_k$ 都有良定义的子对象 $X_k^\flat := \Image f^N_k$, 其中 $N \gg k$. 不难看出这使相应的态射列
\[ \cdots \to X^\flat_2 \xrightarrow{f^\flat_1} X^\flat_1 \xrightarrow{f^\flat_0} X^\flat_0 \]
中的每个 $f^\flat_k := f_k|_{X^\flat_{k+1}}$ 都成为满的.

\begin{lemma}\label{prop:ML-nonempty}
	取 $\mathcal{C}$ 为 $\cate{Set}$ 或 $R\dcate{Mod}$, 其中 $R$ 是任意环. 假设 $\cate{InvSys}(\mathcal{C})$ 的对象 $(X_n, f_n)_{n \geq 0}$ 是 Mittag-Leffler 的.
	\begin{enumerate}[(i)]
		\item 包含态射族 $X_n^\flat \hookrightarrow X_n$ 诱导典范同构 $\varprojlim_n X^\flat_n \rightiso \varprojlim_n X_n$
		\item 对于 $\mathcal{C} = \cate{Set}$ 的情形, 若每个 $X_n$ 皆非空, 则 $\varprojlim_n X_n \neq \emptyset$.
	\end{enumerate}
\end{lemma}
\begin{proof}
	对于 (i), 写下 $\varprojlim$ 在这些范畴中的具体构造
	\[ \varprojlim_n X_n = \left\{ (x_n)_{n \geq 0} \in \prod_{n \geq 0} X_n : \forall n \geq 0,\; f_n(x_{n+1}) = x_n \right\}; \]
	上式的每个 $x_n$ 自动属于 $\bigcap_{N \geq n} \Image(f^N_n)$, 由此立见 $\varprojlim_n X_n = \varprojlim_n X^\flat_n$.
	
	对于 (ii), 按定义可见 $X_n^\flat$ 必然非空. 于是问题按 (i) 化约到每个 $f_n$ 皆满的情形, 因而是显然的.
\end{proof}

\begin{proposition}\label{prop:ML-exactness}
	取 $R$ 为环, $\mathcal{A} := R\dcate{Mod}$. 若 $\cate{InvSys}(\mathcal{A})$ 的对象 $A = (A_n, f_n)_{n \geq 0}$ 是 Mittag-Leffler 的, 则 $\lim^1 A = 0$; 等价地说, 此时对于 $\cate{InvSys}(\mathcal{A})$ 中的所有短正合列 $0 \to A \to B \to C \to 0$, 相应的 $0 \to \varprojlim A \to \varprojlim B \to \varprojlim C \to 0$ 也正合.
\end{proposition}
\begin{proof}
	先论证两个断言的等价性. 设 $A \in \cate{InvSys}(\mathcal{A})$. 若 $\lim^1 A = 0$, 则 $0 \to \varprojlim A \to \varprojlim B \to \varprojlim C \to 0$ 正合是长正合列的直接推论. 反之, 设 $0 \to \varprojlim A \to \varprojlim B \to \varprojlim C \to 0$ 恒正合. 取单态射 $A \hookrightarrow B$ 使得 $B$ 是 $\cate{InvSys}(\mathcal{A})$ 的内射对象, 则移维 (命题 \ref{prop:dimension-shifting}) 给出 $\mathrm{R}^1 \varprojlim A = 0$, 继而由定理 \ref{prop:lim1} 导出 $\lim^1 A = 0$.
	
	进入正题. 设短正合列 $0 \to \underbracket{(A_n, f_n)_n}_{= A} \xrightarrow{\varphi} \underbracket{(B_n, g_n)_n}_{= B} \xrightarrow{\psi} \underbracket{(C_n, h_n)}_{= C} \to 0$ 中的 $A$ 是 Mittag-Leffler 的. 目标是证 $\varprojlim \psi$ 满. 鉴于引理 \ref{prop:ML-nonempty} (ii), 问题化为对每个 $c = (c_n)_{n \geq 0} \in \varprojlim C$ 证非空集的映射列
	\[ \cdots \to \psi_2^{-1}(c_2) \xrightarrow{g_1} \psi_1^{-1}(c_1) \xrightarrow{g_0} \psi_0^{-1}(c_0) \]
	是 Mittag-Leffler 的, 因为由此可说明 $\psi^{-1}(c)$ 非空.
	
	给定 $k \geq 0$, 取 $N \geq k$ 充分大, 使得 $n \geq N$ 时 $\Image f^n_k = \Image f^N_k$. 设 $b_N \in \psi_N^{-1}(c_N)$; 为了验证 Mittag-Leffler 条件, 以下说明对每个 $n \geq N$ 皆存在 $b_n \in \psi_n^{-1}(c_n)$ 使得 $g^n_k(b_n) = g^N_k(b_N)$.
	
	任取 $b'_n \in \psi_n^{-1}(c_n)$, 则 $\psi_N\left( g^n_N(b'_n)\right) = c_N = \psi_N\left(b_N \right)$, 故存在 $a_N$ 使 $g^n_N(b'_n) - b_N = \varphi_N(a_N)$. 再取 $a_n$ 使得 $f^n_k(a_n) = f^N_k(a_N)$, 则 $b_n := b'_n - \varphi_n(a_n)$ 即所求.
\end{proof}

对于满足假设 \ref{hyp:lim1} 的一般 Abel 范畴 $\mathcal{A}$, 存在反例说明 Mittag-Leffler 条件未必蕴涵 $\lim^1 = 0$, 详细讨论见 \cite{Roo06}.

以下结果是一则简单的操练, 将用于 \S\ref{sec:K-injectives}.

\begin{corollary}\label{prop:ML-4-terms}
	设 $\mathcal{A}$ 满足假设 \ref{hyp:lim1}, 而 $A \to B \to C \to D$ 是 $\cate{InvSys}(\mathcal{A})$ 中的正合列. 若 $\lim^1 A = 0$, 则
	\[ \varprojlim B \to \varprojlim C \to \varprojlim D \]
	是 $\mathcal{A}$ 中的正合列.
\end{corollary}
\begin{proof}
	命 $H := \Ker[C \to D]$, $X := \Image[A \to B]$. 因为 $\varprojlim$ 左正合, 故有典范同构
	\[ \Ker\left[ \varprojlim C \to \varprojlim D\right] \simeq \varprojlim H . \]
	态射 $B \to C$ 分解为 $B \twoheadrightarrow H \hookrightarrow C$. 问题化为证 $\varprojlim B \to \varprojlim H$ 满. 例 \ref{eg:quotient-lim1-0} 说明 $\lim^1 X = 0$, 故对短正合列 $0 \to X \to B \to H \to 0$ 取 $\varprojlim$ 可见 $\varprojlim B \to \varprojlim H$ 满.
\end{proof}

\section{实例: \texorpdfstring{$\Ext$}{Ext} 和 \texorpdfstring{$\Tor$}{Tor}}\label{sec:Ext-Tor}
我们即将探讨 $\Hom$ 和 $\otimes$ 的导出函子. 宜先引入一些辅助概念.

\begin{definition}\label{def:balanced-functor}
	\index{shuanghanzi!平衡 (balanced)}
	设 $\mathcal{A}_1$, $\mathcal{A}_2$, $\mathcal{B}$ 为 Abel 范畴, 而双函子 $F: \mathcal{A}_1 \times \mathcal{A}_2 \to \mathcal{B}$ (约定 \ref{con:bifunctor}) 对两个变元都是左正合函子. 若对于所有内射对象 $I_i \in \Obj(\mathcal{A}_i)$ (其中 $i = 1, 2$),
	\[ F(I_1, \cdot): \mathcal{A}_2 \to \mathcal{B} \quad \text{和}\quad F(\cdot, I_2): \mathcal{A}_1 \to \mathcal{B} \]
	都是正合函子, 则称 $F$ 是\emph{平衡}的.
	
	对偶地, 如果 $F$ 对两个变元都右正合, 而且对于所有投射对象 $P_i \in \Obj(\mathcal{A}_i)$, 函子 $F(P_1, \cdot)$ 和 $F(\cdot, P_2)$ 都正合, 则称 $F$ 是平衡的.
\end{definition}

基于对偶性, 以下定理仅针对两个变元皆左正合的双函子来陈述.

\begin{theorem}\label{prop:balanced-primer}
	\index[sym1]{RInF@$\mathrm{R}_{\mathrm{I}}^n F$, $\mathrm{R}_{\mathrm{II}}^n F$}
	设 $\mathcal{A}_1$, $\mathcal{A}_2$, $\mathcal{B}$ 为 Abel 范畴, $\mathcal{A}_1$ 和 $\mathcal{A}_2$ 皆有足够的内射对象. 设双函子 $F: \mathcal{A}_1 \times \mathcal{A}_2 \to \mathcal{B}$ 对两个变元都左正合. 由此可对每个变元取右导出函子, 记为
	\[ \mathrm{R}_{\mathrm{I}}^n F(\cdot, X_2): \mathcal{A}_1 \to \mathcal{B}, \quad \mathrm{R}_{\mathrm{II}}^n F(X_1, \cdot): \mathcal{A}_2 \to \mathcal{B} , \]
	其中 $X_i \in \Obj(\mathcal{A}_i)$ 固定, $i=1,2$. 若 $F$ 是平衡的, 则有双函子 $\mathcal{A}_1 \times \mathcal{A}_2 \to \mathcal{B}$ 之间的典范同构
	\[ \mathrm{R}_{\mathrm{I}}^n F(X_1, X_2) \simeq \mathrm{R}_{\mathrm{II}}^n F(X_1, X_2). \]
\end{theorem}
\begin{proof}
	下述论证依赖双复形的理论, 见 \S\ref{sec:double-cplx} 和 \S\ref{sec:double-cplx-coh}. 对 $i = 1, 2$ 取内射解消 $0 \to X_i \to I_i^0 \to I_i^1 \to \cdots$; 另对 $n < 0$ 定义 $I_i^n := 0$. 由此定义双复形 $F\left(I_1^\bullet, I_2^\bullet\right)$.
	
	另一方面, 定义复形 $F(X_1, I_2^\bullet)$ 和 $F(I_1^\bullet, X_2)$. 将它们视为双复形, 分别集中在纵轴 $(0, \bullet)$ 和横轴 $(\bullet, 0)$ 上. 因此 $X_i \rightiso \Ker[I_i^0 \to I_i^1] \hookrightarrow I_i^0$ 诱导 $\cate{C}^2_f(\mathcal{B})$ 中的态射
	\[ F(X_1, I_2^\bullet) \xrightarrow{\lambda} F\left(I_1^\bullet, I_2^\bullet\right) \xleftarrow{\rho} F(I_1^\bullet, X_2). \]
	兹断言 $\tot(\lambda)$ 和 $\tot(\rho)$ 都是拟同构.
	
	先处理 $\lambda$. 对其两边同取横向上同调 $\Hm_{\mathrm{I}}$, 这不改变落在纵轴上的 $F(X_1, I_2^\bullet)$; 而因为 $F(\cdot, I_2^q)$ 对每个 $q$ 都是正合函子, 我们有
	\[ \left( \Hm_{\mathrm{I}} F\left(I_1^\bullet, I_2^\bullet\right)\right)^{p,q} = \begin{cases}
		0, & p \neq 0, \\
		\Ker\left[F(I_1^0, I_2^q) \to F(I_1^1, I_2^q) \right] = F(X_1, I_2^q), & p = 0.
	\end{cases}\]
	这就说明 $\Hm_{\mathrm{I}}(\lambda): \Hm_{\mathrm{I}} F(X_1, I_2^\bullet) \to \Hm_{\mathrm{I}} F\left(I_1^\bullet, I_2^\bullet\right)$ 是同构. 因此 $\Hm_{\mathrm{II}} \Hm_{\mathrm{I}}(\lambda)$ 仍是同构. 代入定理 \ref{prop:double-cplx-tot} 可知 $\tot(\lambda)$ 是拟同构.
	
	对于 $\rho$, 相同论证给出 $\Hm_{\mathrm{I}} \Hm_{\mathrm{II}}(\rho)$ 是同构, 故定理 \ref{prop:double-cplx-tot} 蕴涵 $\tot(\rho)$ 是拟同构.
	
	观察到 $F(X_1, I_2^\bullet)$ 的全复形仍是它自身, 取 $\Hm^n$ 给出 $\mathrm{R}_{\mathrm{II}}^n F(X_1, X_2)$. 类似地, 对 $F(I_1^\bullet, X_2)$ 或其全复形取 $\Hm^n$ 给出 $\mathrm{R}_{\mathrm{I}}^n F(X_1, X_2)$. 通过 $\tot\left(F(I_1^\bullet, I_2^\bullet)\right)$ 的中介, 遂有
	\[ \mathrm{R}_{\mathrm{I}}^n F(X_1, X_2) \simeq \mathrm{R}_{\mathrm{II}}^n F(X_1, X_2). \]
	由于任何态射 $X_i \to Y_i$ 都可以提升为内射解消之间的态射, 提升在同伦意义下唯一, 故实际得到的是双函子的同构 $\mathrm{R}_{\mathrm{I}}^n F \simeq \mathrm{R}_{\mathrm{II}}^n F$.
\end{proof}

\begin{remark}\label{rem:balanced-primer}
	上述论证实际给出了一套同时解消两个变元, 以双复形 $F(I_1^\bullet, I_2^\bullet)$ 的全复形``平衡地''对 $F$ 求导的技术.
\end{remark}

现在取 $\mathcal{A}$ 为 Abel 范畴, 考虑双函子 $\Hom = \Hom_{\mathcal{A}}: \mathcal{A}^{\opp} \times \mathcal{A} \to \cate{Ab}$, 命题 \ref{prop:Hom-left-exact} 表明 $\Hom$ 对两个变元皆左正合. 对每个 $n \in \Z$ 作如下定义.
\begin{itemize}
	\item 设 $\mathcal{A}$ 有足够的投射对象, 命 $\Ext^n_{\mathcal{A}, \mathrm{I}}(X, Y) := \mathrm{R}^n_{\mathrm{I}} \Hom(X, Y)$, 它由 $X$ 在 $\mathcal{A}$ 中的投射解消确定.
	\item 设 $\mathcal{A}$ 有足够的内射对象, 命 $\Ext^n_{\mathcal{A}, \mathrm{II}}(X, Y) := \mathrm{R}^n_{\mathrm{II}} \Hom(X, Y)$, 它由 $Y$ 在 $\mathcal{A}$ 中的内射解消确定.
\end{itemize}

\begin{definition-proposition}[$\Ext$ 函子]\label{def:Ext-classical}
	\index{Ext hanzi@$\Ext$ 函子 ($\Ext$ functor)}
	\index[sym1]{Extn@$\Ext^n$}
	双函子 $\Hom_{\mathcal{A}}$ 是平衡的. 因此, 只要 Abel 范畴 $\mathcal{A}$ 有足够的内射对象和投射对象, 即可对每个 $n \in \Z$ 定义双函子
	\[ \Ext^n = \Ext^n_{\mathcal{A}}: \mathcal{A}^{\opp} \times \mathcal{A} \to \cate{Ab} \]
	为 $\Ext^n_{\mathcal{A}}(X, Y) := \Ext^n_{\mathrm{I}}(X, Y) \simeq \Ext^n_{\mathrm{II}}(X, Y)$ (典范同构).
\end{definition-proposition}
\begin{proof}
	平衡性正是内射对象和投射对象的定义, 因此适用定理 \ref{prop:balanced-primer}.
\end{proof}

若 $R$ 是环而 $\mathcal{A} = R\dcate{Mod}$ 或 $\cated{Mod}R$, 则记 $\Ext^n_{\mathcal{A}}$ 为 $\Ext^n_R$.

注意到 $\Ext^0 = \Hom$, 而 $(\Ext^n)_{n \geq 0}$ 对变元 $X$ 和 $Y$ 分别有形式如下的长正合列
\begin{gather*}
	\cdots \to \Ext^{n-1}(X', Y) \xrightarrow{\delta^{n-1}} \Ext^n(X'', Y) \to \Ext^n(X, Y) \to \Ext^n(X', Y) \to \cdots, \\
	\cdots \to \Ext^{n-1}(X, Y'') \xrightarrow{\delta^{n-1}} \Ext^n(X, Y') \to \Ext^n(X, Y) \to \Ext^n(X, Y'') \to \cdots,
\end{gather*}
其中 $0 \to X' \to X \to X'' \to 0$ 和 $0 \to Y' \to Y \to Y'' \to 0$ 都是给定的短正合列.

当 $\mathcal{A}$ 是 $\Bbbk$-线性 Abel 范畴时, $\Ext^n$ 的取值当然能升级到范畴 $\Bbbk\dcate{Mod}$ 里.

\begin{remark}
	符号 $\Ext$ 意指``扩张'', 这是由于 $\Ext^1(X, Y)$ 的元素和扩张 $0 \to Y \to E \to X \to 0$ (亦即短正合列) 的同构类一一对应; 更高次的 $\Ext^n$ 也有类似诠释. 这一事实宜从导出范畴解释, 详见 \S\ref{sec:Hom-Ext}.
\end{remark}

基于推论 \ref{prop:obstruction-exactness} 对函子正合性的刻画, 立见
\begin{itemize}
	\item 对象 $I$ 内射 $\iff \Ext^1(\cdot, I) = 0 \iff \Ext^{\geq 1}(\cdot, I) = 0$;
	\item 对象 $P$ 投射 $\iff \Ext^1(P, \cdot) = 0 \iff \Ext^{\geq 1}(P, \cdot) = 0$.
\end{itemize}

\begin{example}[$\HHm^n$ 作为 $\Ext^n$]\label{eg:HH-Ext}
	\index[sym1]{HH}
	回到 \S\ref{sec:HH} 关于 Hochschild 同调与上同调的基本框架: $\Bbbk$ 为交换环, $R$ 为 $\Bbbk$-代数, 现在进一步要求 $R$ 作为 $\Bbbk$-模是投射的. 记 $\otimes := \otimes_{\Bbbk}$. 运用投射模可以刻画为自由模的直和项这一事实\footnote{或运用张量积的伴随关系 \cite[定理 6.6.5]{Li1}.}, 可见 $R^{\otimes n}$ 仍是投射 $\Bbbk$-模. 现将 $R \otimes R^{\otimes n} \otimes R$ 作成左 $R^e := R \otimes R^{\opp}$-模. 因为 $R \otimes (\cdot) \otimes R: \Bbbk\dcate{Mod} \to R^e\dcate{Mod}$ 是忘却函子的左伴随 \cite[推论 6.6.8]{Li1}, 它保投射对象 (命题 \ref{prop:adjoint-injective-projective}), 故引理 \ref{prop:HH-bar-exactness} 说明杠复形 $\mathsf{B}R$ 给出 $R$ 作为左 $R^e$-模的投射解消
	\begin{equation}\label{eqn:bar-proj-resolution}
		\cdots \to \mathsf{B}_1 R \to \mathsf{B}_0 R \to R \to 0.
	\end{equation}
	现在设 $M$ 是 $(R,R)$-双模, 代入 Hochschild 上同调 $\HHm^n(M)$ 的原始定义 \ref{def:HH}, 遂得
	\[ \HHm^n(M) = \Hm^n\left( \Hom_{R^e}(\mathsf{B} R, M) \right) = \Ext^n_{R^e}(R, M). \]
\end{example}

次一主题是 $\Tor$ 函子. 选定环 $R$. 模范畴 $R\dcate{Mod}$ (或 $\cated{Mod}R$) 有足够的投射对象\footnote{事实上, 自由模必然是投射模, 见 \cite[例 6.9.6]{Li1}}. 张量积给出双函子
\[ \otimes_R: \cated{Mod}R \times R\dcate{Mod} \to \cate{Ab}, \]
它对每个变元都是右正合的, 见 \cite[命题 6.9.2]{Li1}; 对之可套用先前理论的右正合版本, 对两个变元定义左导出函子 $\mathrm{L}_{\mathrm{I}, n} \otimes_R$ 和 $\mathrm{L}_{\mathrm{II}, n} \otimes_R$, 它们仅在 $n \geq 0$ 时非零. 也请回忆何谓平坦模 \cite[定义 6.9.4]{Li1}.

\begin{definition-proposition}[$\Tor$ 函子]\label{def:Tor-classical}
	\index{Tor hanzi@$\Tor$ 函子 ($\Tor$ functor)}
	\index[sym1]{Torn@$\Tor^R_n$}
	对于任何环 $R$, 双函子 $\otimes_R$ 是平衡的. 因此可对每个 $n \in \Z$ 定义双函子
	\[ \Tor_n = \Tor_n^R: \cated{Mod}R \times R\dcate{Mod} \to \cate{Ab} \]
	为 $\Tor_n^R(X, Y) := (\mathrm{L}_{\mathrm{I}, n} \otimes_R) (X, Y) \simeq (\mathrm{L}_{\mathrm{II}, n} \otimes_R) (X, Y)$ (典范同构).
\end{definition-proposition}
\begin{proof}
	平衡性缘于投射模必平坦 \cite[推论 6.9.9]{Li1}. 应用定理 \ref{prop:balanced-primer} 的左导出版本.
\end{proof}

我们有 $\Tor^R_0(X, Y) = X \dotimes{R} Y$, 而 $(\Tor_n)_{n \geq 0}$ 有如下形式的长正合列
\begin{gather*}
	\cdots \to \Tor^R_{n+1}(X'', Y) \xrightarrow{\partial_{n+1}} \Tor^R_n(X', Y) \to \Tor^R_n(X, Y) \to \Tor^R_n(X'', Y) \to \cdots, \\
	\cdots \to \Tor^R_{n+1}(X, Y'') \xrightarrow{\partial_{n+1}} \Tor^R_n(X, Y') \to \Tor^R_n(X, Y) \to \Tor^R_n(X, Y'') \to \cdots,
\end{gather*}
其中 $0 \to X' \to X \to X'' \to 0$ 和 $0 \to Y' \to Y \to Y'' \to 0$ 都是给定的短正合列.

继续重复和 $\Ext$ 情形相似的套话.
\begin{itemize}
	\item 右 $R$-模 $M$ 平坦 $\iff \Tor^R_1(M, \cdot) = 0 \iff \Tor^R_{\geq 1}(M, \cdot) = 0$;
	\item 左 $R$-模 $N$ 平坦 $\iff \Tor^R_1(\cdot, N) = 0 \iff \Tor^R_{\geq 1}(\cdot, N) = 0$;
\end{itemize}

设 $M$ 为 $R$-模. 仿照投射解消的定义, $M$ 的\emph{平坦解消}定义为如下形式的正合列 \index{jiexiao!平坦 (flat)}
\[ \cdots \to F^2 \to F^1 \to F^0 \to M \to 0, \quad \text{每个 $F^n$ 都是平坦 $R$-模}. \]

鉴于推论 \ref{prop:acyclic-resolution} 和上述观察, $\Tor^R_n(X,Y)$ 亦可用 $X$ 或 $Y$ 的平坦解消来计算. 这是计算 $\Tor$ 的实际技术, 而投射解消主要是在理论层面起作用.

这类计算自然也可以``平衡''. 任取平坦解消 $\cdots \to P_1 \to P_0 \to X \to 0$ 和 $\cdots \to Q_1 \to Q_0 \to X \to 0$. 构造链复形意义下的双复形 $P_\bullet \dotimes{R} Q_\bullet$, 并且简记其全复形为 $P \otimes Q$. 注记 \ref{rem:balanced-primer} 的同调版本表明 $\Tor_n^R(X, Y) \simeq \Hm_n(P \otimes Q)$; 该处论证所需的性质只有 $P_p \dotimes{R} (\cdot)$ 和 $(\cdot) \dotimes{R} Q_q$ 的正合性, 这正是平坦性所保证的.

\begin{remark}[双模结构]\label{rem:ExtTor-bimodule}
	选定环 $A$, $B$; 设 $X$ 是 $(A, R)$-双模, $Y$ 是 $(R, B)$-双模. 由于 $\Tor^R_n$ 是双函子, $\Tor^R_n(X, Y)$ 具有典范的 $(A, B)$-双模结构. 以 $A$ 的左乘为例, 每个 $a \in A$ 都给出 $X$ 作为右 $R$-模的自同态, 因而诱导 $\Tor^R_n(X, Y)$ 的自同态 $L_a$; 它们满足 $L_a L_{a'} = L_{aa'}$ 和 $L_1 = \identity$. 对 $B$ 的右乘也在第二个变元作类似处理. 取 $n=0$ 便回到 $X \dotimes{R} Y$ 上的自然双模结构.
	
	若选定 $Y$ 作为左 $R$-模的平坦解消 $\cdots \to P_0 \to Y \to 0$, 则 $A$ 的左乘作用可以直接在复形 $X \dotimes{R} P_\bullet$ 上读出: 它来自 $A$ 对 $X$ 的左乘. 右乘亦作如是观.

	作为特例, 选定环同态 $R \to S$, 则 $\Tor^R_n(\cdot, S)$ (或 $\Tor^R_n(S, \cdot)$) 升级为函子 $\cated{Mod}R \to \cated{Mod}S$ (或 $R\dcate{Mod} \to S\dcate{Mod}$).

	类似的论证表明: 对于 $(R, A)$-双模 $X$ 和 $(R, B)$-双模 $Y$, 每个 $\Ext^n_R(X, Y)$ 都自然地成为 $(A, B)$-双模.
\end{remark}

交换环 $R$ 上的模不分左右, 对应的范畴统一写作 $R\dcate{Mod}$; 此时任何 $R$-模自然也是 $(R, R)$-双模, 故 $\Tor^R_n$ 升级为双函子 $(R\dcate{Mod})^2 \to R\dcate{Mod}$.

\begin{proposition}\label{prop:Tor-symmetry}
	当 $R$ 是交换环时, 存在双函子的典范同构 $\Tor_n^R(X, Y) \simeq \Tor_n^R(Y, X)$, 其中 $n \in \Z$.
\end{proposition}
\begin{proof}
	分别取 $X$ 和 $Y$ 的平坦解消 $P_\bullet$ 和 $Q_\bullet$. 命 $P \otimes Q$ 为双复形 $P_\bullet \dotimes{R} Q_\bullet$ 的全复形; 类似地定义 $Q \otimes P$. 张量积的交换约束 \cite[命题 6.5.14]{Li1} 给出典范同构
	\[ P_p \dotimes{R} Q_q \simeq Q_q \dotimes{R} P_p, \quad p, q \in \Z. \]
	换言之 $P_\bullet \dotimes{R} Q_\bullet \simeq \mathrm{swap}(Q_\bullet \dotimes{R} P_\bullet)$. 代入命题 \ref{prop:double-cplx-swap} 即可导出全复形的典范同构 $P \otimes Q \simeq Q \otimes P$. 证毕.
\end{proof}

\begin{remark}
	符号 $\Tor$ 意指``挠元'', 源自以下特例. 取环 $R$ 的左理想 $\mathfrak{a}$, 再取任意右 $R$-模 $X$. 因为自由左模 $R$ 平坦, 对 $0 \to \mathfrak{a} \to R \to R/\mathfrak{a} \to 0$ 应用命题 \ref{prop:dimension-shifting} 来移维, 可得典范同构
	\begin{align*}
		\Tor^R_1(X, R/\mathfrak{a}) & \simeq \Ker\left[ X \dotimes{R} \mathfrak{a} \to X \dotimes{R} R \simeq X \right] \\
		& \simeq \left\{ \sum_i x_i \otimes a_i \in X \dotimes{R} \mathfrak{a} : \sum_i x_i a_i = 0 \right\}, \\
		\Tor^R_n(X, R/\mathfrak{a}) & \simeq \Tor^R_{n-1}(X, \mathfrak{a}), \quad (n > 1).
	\end{align*}
	如进一步取 $t \in R$ 非右零因子, 再取 $\mathfrak{a} := Rt \leftiso R$ (映 $r$ 为 $rt$), 则 $\Tor^R_1(X, R/Rt) \simeq \left\{ x \in X: xt = 0 \right\}$, 即 $X$ 的 $t$-挠元子模, 而 $n > 1$ 时 $\Tor^R_n(X, R/Rt) = 0$.
\end{remark}

\begin{example}[$\HHm_n$ 作为 $\Tor_n$]\label{eg:HH-Tor}
	\index[sym1]{HH}
	接续例 \ref{eg:HH-Ext} 的起手式, 但将关于 $R$ 的前提弱化为 $R$ 是平坦 $\Bbbk$-模. 此时 $R^{\otimes n}$ 也平坦, 而结合约束 $(\cdot) \dotimes{R^e} (R \otimes R^{\otimes n} \otimes R) \simeq (\cdot) \otimes R^{\otimes n}$ 表明 $R \otimes R^{\otimes n} \otimes R$ 是平坦左 $R^e$-模. 故 $\mathsf{B}R$ 是 $R$ 作为左 $R^e$-模的平坦解消, 如 \eqref{eqn:bar-proj-resolution}.
	
	对任意 $(R, R)$-双模 $M$, 代入 Hochschild 同调的原始定义 \ref{def:HH} 可得
	\[ \HHm_n(M) = \Hm_n\left( M \dotimes{R^e} \mathsf{B}R \right) = \Tor^{R^e}_n\left(M, R \right). \]
\end{example}

现在可以结合例 \ref{eg:HH-Ext} 和 \ref{eg:HH-Tor} 来计算更多 Hochschild 同调与上同调的实例. 首先取多项式代数 $R = \Bbbk[t]$, 将 $R^e = R \otimes R^{\opp}$ 等同于 $\Bbbk[x, y]$, 则 $R$ 作为 $R^e$-模的纯量乘法表作 $h(x, y) \cdot f(t) = h(t, t) f(t)$, 而且 $R$ 有自由解消
\[ 0 \to R^e \xrightarrow{\text{乘以}\; x-y} R^e \xrightarrow{f \mapsto f(t,t)} R \to 0. \]
由此计算 $\Ext^n_{R^e}(R, \cdot)$ 与 $\Tor_n^{R^e}(\cdot, R)$, 立得 $n \in \{0,1\}$ 时 $\HHm_n(R) \simeq R \simeq \HHm^n(R)$, 而 $n > 1$ 时 $\HHm_n(R) = 0 = \HHm^n(R)$.

其次取 $R = \Bbbk[t]/(t^2)$, 则 $R^e = \Bbbk[x,y]/(x^2, y^2)$. 对 $R^e$-模 $R$ 有周期 $2$ 的自由解消
\[ \cdots \xrightarrow{x+y} R^e \xrightarrow{x-y} R^e \xrightarrow{x+y} R^e \xrightarrow{\text{乘以}\; x-y} R^e \xrightarrow{f \mapsto f(t,t)} R \to 0 , \]
正合性敬请读者直接验证. 对任意 $R$-模 $M$, 取 $M \dotimes{R^e} (\cdot)$ 和 $\Hom_{R^e}(\cdot, M)$ 分别给出
\begin{gather*}
	\cdots \xrightarrow{2t} M \xrightarrow{0} M \xrightarrow{2t} M \xrightarrow{0} M \to M \dotimes{R^e} R \to 0 , \\
	0 \to \Hom_{R^e}(R, M) \to M \xrightarrow{0} M \xrightarrow{2t} M \xrightarrow{0} M \xrightarrow{2t} \cdots .
\end{gather*}
因此 $\HHm_k(M)$ 和 $\HHm^k(M)$ 仅依赖 $k \bmod 2$; 倘若仅按定义计算, 此周期性恐怕不易察觉.

在 $\Bbbk$ 为域的情形, 由此可读出 $\HHm_n(M)$ 和 $\HHm^n(M)$, 但答案取决于 $\mathrm{char}(\Bbbk)$. 习题将讨论 $R = \Bbbk[t]/(t^n)$ 的一般情形.

以下结论常用于代数拓扑学, 同时也是先前内容的一次小验收, 它涉及链复形及其同调 (注记 \ref{rem:cochain-vs-chain}). 设 $C = (C_\bullet, d^C_\bullet)$ (或 $D = (D_\bullet, d^D_\bullet)$) 是右 $R$-模 (或左 $R$-模) 构成的链复形. 由之构造链双复形 $C_\bullet \otimes D_\bullet$, 并将其全复形记为
\[ C \otimes D := \tot_{\oplus}\left(C_\bullet \otimes D_\bullet\right); \]
张量积的下标 $R$ 省略. 具体地说, 全复形 $C \otimes D$ 的 $n$ 次项是 $\bigoplus_{p+q=n} C_p \otimes D_q$, 而其 $d_n$ 拉回到 $C_p \otimes D_q$ 上等于 $d^C_p \otimes \identity + (-1)^p \identity \otimes d^D_q$. 这给出自明的典范同态
\[ \kappa: \bigoplus_{p+q=n} \Hm_p(C) \otimes \Hm_q(D) \to \Hm_n\left( C \otimes D \right). \]

\begin{theorem}[同调 Künneth 定理]\label{prop:Kunneth-homology}
	\index{Kunneth dingli@Künneth 定理 (Künneth Theorem)}
	取链复形 $C$, $D$ 如上. 若每个 $C_p$ 和 $\Image\left( d^C_p\right)$ 都是平坦右 $R$-模, 则有典范短正合列
	\[\begin{tikzcd}[column sep=small, every cell/.append style={font = \small}]
		0 \arrow[r] & \displaystyle\bigoplus_{p+q=n} \Hm_p(C) \otimes \Hm_q(D) \arrow[r, "\kappa"] & \Hm_n\left( C \otimes D \right) \arrow[r] & \displaystyle\bigoplus_{p+q=n-1} \Tor_1^R\left( \Hm_p(C), \Hm_q(D) \right) \arrow[r] & 0.
	\end{tikzcd}\]
\end{theorem}	
\begin{proof}
	对所有 $p \in \Z$ 定义 $C_p$ 的子模 $B_p := \Image\left( d^C_{p+1} \right)$ 和 $Z_p := \Ker\left( d^C_p \right)$. 按 $d^Z = d^B := 0$ 将它们作成链复形 $Z = Z_\bullet$ 和 $B = B_\bullet$. 由此易得链复形的短正合列
	\[\begin{tikzcd}[column sep=large]
		0 \arrow[r] & Z \arrow[r] & C \arrow[r, "{d^C = (d^C_p)_p}" inner sep=0.6em] & B[-1] \arrow[r] & 0.
	\end{tikzcd}\]
	
	短正合列 $0 \to Z_p \to C_p \xrightarrow{d^C_p} B_{p-1} \to 0$ 和 $\Tor$ 函子的长正合列蕴涵 $Z_p$ 亦平坦. 又因为 $B_{p-1}$ 平坦, 故长正合列也说明
	\[ 0 \to Z_p \otimes D_q \to C_p \otimes D_q \to B_{p-1} \otimes D_q \to 0 \]
	正合. 既然模的直和 (容许无穷) 保持正合性, 上式遂给出链复形的短正合列
	\[ 0 \to Z \otimes D \to C \otimes D \xrightarrow{d^C \otimes \identity_D} B[-1] \otimes D \to 0. \]
	又因为 $d^Z = d^{B[-1]} = 0$, 对应的同调长正合列根据平坦性化为
	\begin{multline*}
		\overbracket{\displaystyle\bigoplus_{p+q=n+1} B_{p-1} \otimes \Hm_q(D)}^{= \bigoplus_{p+q = n} B_p \otimes \Hm_q(D)} \xrightarrow{\partial_{n+1}} \displaystyle\bigoplus_{p+q=n} Z_p \otimes \Hm_q(D) \to \Hm_n\left( C \otimes D \right) \\
		\to \underbracket{\displaystyle\bigoplus_{p+q=n} B_{p-1} \otimes \Hm_q(D)}_{= \bigoplus_{p+q=n-1} B_p \otimes \Hm_q(D)} \xrightarrow{\partial_n} \displaystyle\bigoplus_{p+q=n-1} Z_p \otimes \Hm_q(D).
	\end{multline*}
	待明确的仅有连接同态 $\partial_{n+1}$ 和 $\partial_n$. 兹断言它们皆来自包含同态 $B_p \hookrightarrow Z_p$, 至多差一些简单的正负号. 为此须返归长正合列的明确构造, 即命题 \ref{prop:long-exact-sequence-ses} 的同调版本, 但此处可运用图追踪, 见 \cite[命题 6.8.6]{Li1} 以上的说明; 细节留给读者验证.

	作为推论, $\Coker(\partial_{n+1})$ 等同于 $\bigoplus_{p+q=n} \Hm_p(C) \otimes \Hm_q(D)$, 从它到 $\Hm_n(C \otimes D)$ 的单态射等同于 $\kappa$.

	最后, $0 \to B_p \to Z_p \to \Hm_p(C) \to 0$ 是 $\Hm_p(C)$ 的平坦解消, 故 $\Ker(\partial_n) \simeq \displaystyle\bigoplus_{p+q=n-1} \Tor_1^R\left( \Hm_p(C) , \Hm_q(D) \right)$. 至此得到所断言的短正合列.
\end{proof}

留意: 当 $R$ 是除环, $\kappa$ 必然是同构.

因为 $D$ 在拓扑场景下充当了同调的系数, 定理 \ref{prop:Kunneth-homology} 在 $D$ 集中于零次项的特例也称为同调\emph{泛系数定理}. 这类结果还有种种变体, 同时环结构也赋予 $\Tor$ 不共于其他导出双函子的运算, 这些结构似乎更适合以导出范畴或谱序列的语言来梳理.
\index{fanxishudingli@泛系数定理 (Universal Coefficient Theorem)}

\section{K-内射和 K-投射复形}\label{sec:K-injectives}
复形的内射和投射解消在 \S\ref{sec:derived-primer} 给出导出函子的初步定义. 这些构造需要复形上有界或下有界, 但涉及无界复形及其导出范畴的应用则需要更广的一类解消. K-内射或 K-投射解消的理论肇自 \cite{Spa88}. 本节旨在提供一些相关的准备, 论证取法于 \cite{stacks}, 读者也可以参照 \cite[Chapter 14]{KS06} 或 \cite{BN93}.

\begin{definition}[N.\ Spaltenstein]\label{def:K-resolutions}
	\index{fuxing!K-内射, K-投射 (K-injective, K-projective)}
	\index{jiexiao!K-内射, K-投射 (K-injective, K-projective)}
	\index{zugoudeKneishefuxing@足够的 K-内射复形 (enough K-injectives)}
	\index{zugoudeKtoushefuxing@足够的 K-投射复形 (enough K-projectives)}
	设 $\mathcal{A}$ 为 Abel 范畴.
	\begin{itemize}
		\item 满足以下条件的 $I \in \Obj(\cate{C}(\mathcal{A}))$ 称为 \emph{K-内射复形}\footnote{更合理的术语兴许是同伦内射复形, 以及同伦投射复形.}: 当 $X \in \Obj(\cate{C}(\mathcal{A}))$ 零调时, $\Hom^\bullet(X, I)$ 也零调.
		
		若 $f: A \to I$ 是 $\cate{C}(\mathcal{A})$ 中的拟同构, $I$ 是 K-内射的, 则称 $f$ 是 $A$ 的 \emph{K-内射解消}.
		\item 满足以下条件的 $P \in \Obj(\cate{C}(\mathcal{A}))$ 称为 \emph{K-投射复形}: 当 $X \in \Obj(\cate{C}(\mathcal{A}))$ 零调时, $\Hom^\bullet(P, X)$ 也零调.
		
		若 $f: P \to A$ 是 $\cate{C}(\mathcal{A})$ 中的拟同构, $P$ 是 K-投射的, 则称 $f$ 是 $A$ 的 \emph{K-投射解消}.
	\end{itemize}

	若 $\cate{C}(\mathcal{A})$ 的所有对象都有 K-内射解消 (或 K-投射解消), 则称 $\mathcal{A}$ 有足够的 K-内射 (或 K-投射) 复形.
\end{definition}

两套定义当然对偶. 实际操作中, 以下判准往往更方便.
\begin{lemma}\label{prop:K-injective-criterion}
	复形 $I$ (或 $P$) 是 K-内射 (或 K-投射) 的当且仅当对于所有零调之 $X$, 皆有 $\Hom_{\cate{K}(\mathcal{A})}(X, I) = 0$ (或 $\Hom_{\cate{K}(\mathcal{A})}(P, X) = 0$).
\end{lemma}
\begin{proof}
	设 $n \in \Z$. 引理 \ref{prop:Hom-cplx-d-translate} 说明
	\[ \Hm^n \Hom^\bullet(X, I) \simeq \Hom_{\cate{K}(\mathcal{A})}(X[-n], I), \quad \Hm^n \Hom^\bullet(P, X) \simeq \Hom_{\cate{K}(\mathcal{A})}(P, X[n]); \]
	然而 $X$ 零调等价于 $X[-n]$ 零调, 也等价于 $X[n]$ 零调.
\end{proof}

\begin{example}\label{eg:bdd-K-resolution}
	引理 \ref{prop:resolution-homotopy-aux} 立刻转译为以下陈述.
	\begin{compactitem}
		\item 若 $I \in \Obj(\cate{C}^+(\mathcal{A}))$ 由内射对象组成, 则 $I$ 是 K-内射的.
		\item 若 $P \in \Obj(\cate{C}^-(\mathcal{A}))$ 由投射对象组成, 则 $P$ 是 K-投射的.
	\end{compactitem}
	这将内射 (或投射) 解消化为 K-内射 (或 K-投射) 解消的特例.
\end{example}

\begin{example}[A.\ Dold]
	例 \ref{eg:bdd-K-resolution} 的单边有界条件不可或缺. 例如取 $\mathcal{A} = \Z/4\Z\dcate{Mod}$, 并考虑其中的零调复形 $Q$ 如下
	\[ \cdots \xrightarrow{2} \Z/4\Z \xrightarrow{2} \Z/4\Z \xrightarrow{2} \Z/4\Z \xrightarrow{2} \cdots, \]
	每项都是自由模. 另一方面, 取逐项的张量积 $Q \dotimes{\Z/4\Z} \Z/2\Z$, 得到复形
	\[ \cdots \xrightarrow{0} \Z/2\Z \xrightarrow{0} \Z/2\Z \xrightarrow{0} \Z/2\Z \xrightarrow{0} \cdots. \]
	因此 $Q$ 并非 K-投射的, 否则 $\identity_Q$ 将同伦于 $0$, 而 $\identity_{Q \otimes \Z/2\Z}$ 亦然, 与 $\Hm^n(Q \dotimes{\Z/4\Z} \Z/2\Z) = \Z/2\Z$ 矛盾.
	
	换个角度, 此例也说明投射对象组成的无界复形并非合理的解消: $0 \to Q$ 和 $0 \to 0$ 都是拟同构, 而上一段说明 $\identity_Q$ 不同伦于 $0$, 从而 $Q$ 和 $0$ 在 $\cate{K}(\mathcal{A})$ 中并不同构, 由此可见定理 \ref{prop:resolution-homotopy} 无法简单地扩及无界情形.
\end{example}

次一结果是定理 \ref{prop:resolution-homotopy} 的自然推广, 其证明也同样付与定理 \ref{prop:cplx-triangulated} 之后的讨论.

\begin{theorem}\label{prop:K-resolution-homotopy}
	取定 $\cate{C}(\mathcal{A})$ 中的拟同构 $\alpha: X \to Y$.
	\begin{itemize}
		\item 给定态射 $\gamma: X \to I$, 其中 $I$ 是 K-内射复形. 存在 $\cate{K}(\mathcal{A})$ 中的交换图表
		\[\begin{tikzcd}[row sep=tiny]
			& Y \arrow[dd, "\beta"] \\
			X \arrow[ru, "\alpha"] \arrow[rd, "\gamma"'] & \\
			& I
		\end{tikzcd}\]
		\item 给定态射 $\gamma: P \to Y$, 其中 $P$ 是 K-投射复形. 存在 $\cate{K}(\mathcal{A})$ 中的交换图表
		\[\begin{tikzcd}[row sep=tiny]
			& X \arrow[ld, "\alpha"'] \\
			Y & \\
			& P \arrow[lu, "\gamma"] \arrow[uu, "\beta"']
		\end{tikzcd}\]
	\end{itemize}
	无论在哪种情形, $\beta$ 作为 $\cate{K}(\mathcal{A})$ 的态射都是唯一的.
\end{theorem}

\begin{proposition}\label{prop:K-sigma-duality}
	定义--命题 \ref{def:sigma} 的等价 $\sigma$ 满足: $X \in \Obj(\cate{C}(\mathcal{A}^{\opp}))$ 是 K-内射 (或 K-投射) 的当且仅当 $\sigma X \in \Obj(\cate{C}(\mathcal{A})^{\opp}) = \Obj(\cate{C}(\mathcal{A}))$ 是 K-投射 (或 K-内射) 的.
\end{proposition}
\begin{proof}
	函子 $\sigma$ 保持零调复形, 诱导 $\cate{K}(\mathcal{A}^{\opp}) \rightiso \cate{K}(\mathcal{A})^{\opp}$ (命题 \ref{prop:sigma-homotopy}), 故保持引理 \ref{prop:K-injective-criterion} 的判准.
\end{proof}

继而考虑伴随函子. 命题 \ref{prop:adjoint-injective-projective} 在此有相应的版本。

\begin{proposition}\label{prop:adjoint-injective-projectives-K}
	考虑 Abel 范畴之间的一对函子
	$\begin{tikzcd}
		\mathcal{A} \arrow[r, shift left, "F"] & \mathcal{B} \arrow[l, shift left, "G"]
	\end{tikzcd}$,
	并且假设 $F$ 是正合函子.
	\begin{compactitem}
		\item 若 $G$ 是 $F$ 的左伴随, 则 $\cate{K}G: \cate{K}(\mathcal{B}) \to \cate{K}(\mathcal{A})$ 映 K-投射复形为 K-投射复形;
		\item 若 $G$ 是 $F$ 的右伴随, 则 $\cate{K}G: \cate{K}(\mathcal{B}) \to \cate{K}(\mathcal{A})$ 映 K-内射复形为 K-内射复形.
	\end{compactitem}
\end{proposition}
\begin{proof}
	基于对偶性, 考虑 $G$ 是左伴随的情形即可. 此时 $G$ 自动是加性函子 (推论 \ref{prop:automatic-additivity}). 设 $P \in \Obj(\cate{C}(\mathcal{B}))$ 为 K-投射复形. 对于任意零调复形 $X \in \Obj(\cate{C}(\mathcal{A}))$, 命题 \ref{prop:KF-adjoint} 给出同构
	\[ \Hom_{\cate{K}(\mathcal{A})}(\cate{K}G(P), X) \simeq \Hom_{\cate{K}(\mathcal{B})}(P, \cate{K}F(X)). \]
	由 $F$ 正合得 $\cate{K}F(X)$ 零调, 右式为 $0$. 应用引理 \ref{prop:K-injective-criterion} 完成证明.
\end{proof}

本节后续的任务是探讨 K-内射或 K-投射解消的存在性, 这部分需要比较曲折的论证. 且先从 K-内射情形入手.

\begin{lemma}\label{prop:K-injective-projlimit-aux}
	考虑 $\mathcal{A}$ 上的一列复形 $\cdots \to I_2 \to I_1 \to I_0$. 假设
	\begin{itemize}
		\item 每个 $I_k$ 都是 K-内射的;
		\item 对每个 $k$ 和 $n$, 态射 $I_{k+1}^n \to I_k^n$ 是有截面的满态射 (换言之它有右逆, 见命题 \ref{prop:split-ses} 后的解说);
		\item 对每个 $n$, 在 $\mathcal{A}$ 中存在 $I^n := \varprojlim_k I_k^n$.
	\end{itemize}
	此时 诱导的 $d^n_I := \varprojlim_k d_{I_k}^n : I^n \to I^{n+1}$ 使 $I := (I^n, d_I^n)_n$ 成为 K-内射复形.
\end{lemma}
\begin{proof}
	设 $X \in \Obj(\cate{C}(\mathcal{A}))$ 零调. 对每个 $k \in \Z_{\geq 0}$, 从复形 $\Hom^\bullet\left( X, I_k \right)$ 截取
	\begin{equation}\label{eqn:K-injective-projlimit-aux}\begin{tikzcd}
		[row sep=small, column sep=tiny, every cell/.append style = {font = \small}]
		\displaystyle\prod_n \Hom\left( X^n, I_k^{n-2} \right) \arrow[r] & \displaystyle\prod_n \Hom\left( X^n, I_k^{n-1} \right) \arrow[r] & \displaystyle\prod_n \Hom\left( X^n, I_k^n \right) \arrow[r] & \displaystyle\prod_n \Hom\left( X^n, I_k^{n+1} \right) \\
		\Hom^{-2}\left( X, I_k \right) \arrow[equal, u] & \Hom^{-1}\left( X, I_k \right) \arrow[equal, u] & \Hom^0\left( X, I_k \right) \arrow[equal, u] & \Hom^1\left( X, I_k \right) \arrow[equal, u]
	\end{tikzcd}\end{equation}
	根据 $I_k$ 为 K-内射复形的假设, 此列正合. 现在考虑由 $I_{k+1} \to I_k$ 诱导的同态
	\[ \Hom\left( X^n, I_{k+1}^{n-2} \right) \to \Hom\left( X^n, I_k^{n-2} \right); \]
	因为存在截面 $I_k^{n-2} \to I_{k+1}^{n-2}$, 上式为满. 取 $\prod_n$ 后亦满, 因之 \eqref{eqn:K-injective-projlimit-aux} 的左端项当 $k$ 变动时是 Mittag-Leffler 的 (定义 \ref{def:ML}).
	
	现在让 $k \in \Z_{\geq 0}$ 在 \eqref{eqn:K-injective-projlimit-aux} 中变动, 由推论 \ref{prop:ML-4-terms} 导出
	\[\begin{tikzcd}
		[column sep=small, every cell/.append style = {font = \small}]
		\displaystyle\varprojlim_k \displaystyle\prod_n \Hom\left( X^n, I_k^{n-1} \right) \arrow[r] & \displaystyle\varprojlim_k \displaystyle\prod_n \Hom\left( X^n, I_k^n \right) \arrow[r] & \displaystyle\varprojlim_k \displaystyle\prod_n \Hom\left( X^n, I_k^{n+1} \right)
	\end{tikzcd}\]
	正合. 其次, 交换 $\varprojlim_k$ 和 $\prod_n$ 给出
	\[\begin{tikzcd}
		[every cell/.append style = {font = \small}]
		\displaystyle\prod_n \varprojlim_k \Hom\left( X^n, I_k^{n-1} \right) \arrow[r] & \displaystyle\prod_n \varprojlim_k \Hom\left( X^n, I_k^n \right) \arrow[r] & \displaystyle\prod_n \varprojlim \Hom\left( X^n, I_k^{n+1} \right) \\
		\displaystyle\prod_n \Hom\left( X^n, I^{n-1} \right) \arrow[u, "\sim" sloped] & \displaystyle\prod_n \Hom\left( X^n, I^n \right) \arrow[u, "\sim" sloped] & \displaystyle\prod_n \Hom\left( X^n, I^{n+1} \right) \arrow[u, "\sim" sloped]
	\end{tikzcd}\]
	也正合. 然而这也是 $\Hom^\bullet(X, I)$ 的一段, 中项的上同调正是 $\Hom_{\cate{K}(\mathcal{A})}(X, I)$. 综上, $\Hom_{\cate{K}(\mathcal{A})}(X, I) = 0$. 最后应用引理 \ref{prop:K-injective-criterion} 即可完成证明.
\end{proof}

\begin{lemma}\label{prop:K-injective-truncated}
	设 $\mathcal{A}$ 有足够的内射对象, 则对所有 $A \in \Obj(\cate{C}(\mathcal{A}))$, 都存在 $\cate{C}(\mathcal{A})$ 中的交换图表
	\[\begin{tikzcd}
		\cdots \arrow[r] & I_2 \arrow[r] & I_1 \arrow[r] & I_0 \\
		\cdots \arrow[r] & \tau^{\geq -2} A\arrow[u, "f_2"'] \arrow[r] & \tau^{\geq -1} A \arrow[r] \arrow[u, "f_1"'] & \tau^{\geq 0} A \arrow[u, "f_0"']
	\end{tikzcd}\]
	其中 $\tau^{\geq k} A$ 及其间的箭头如定义 \ref{def:truncation-cplx} 所述, 使得
	\begin{itemize}
		\item 每个 $I_k$ 都是 $\cate{C}^+(\mathcal{A})$ 的对象, 由内射对象组成;
		\item 每个垂直箭头 $f_k$ 既是单态射, 也是拟同构;
		\item 对每个 $k$ 和 $n$, 态射 $I_{k+1}^n \to I_k^n$ 是有截面的满态射.
	\end{itemize}
\end{lemma}
\begin{proof}
	从右向左构造. 以下论及的内射解消都默认为单态射.

	首先对 $\tau^{\geq 0} A \in \Obj(\cate{C}^+(\mathcal{A}))$ 应用定理 \ref{prop:resolution-existence} 得到内射解消 $f_0: \tau^{\geq 0} A \to I_0$.

	其次, 观定义可见 $\tau^{\geq -k-1} A \to \tau^{\geq -k} A$ 逐项满, 因而是 $\cate{C}^+(\mathcal{A})$ 中的满态射. 现在假设已构造所需之拟同构
	\[ \tau^{\geq 0} A \xrightarrow{f_0} I_0, \quad \ldots, \quad \tau^{\geq -k} A \xrightarrow{f_k} I_k \]
	连同相应的交换图表. 以定理 \ref{prop:resolution-existence} 取 $H := \Ker\left[ \tau^{\geq -k-1} A \to \tau^{\geq -k} A \right]$ 的内射解消 $H \xrightarrow{g} J$, 则应用命题 \ref{prop:horseshoe} 可得 $\cate{C}(\mathcal{A})$ 中的行正合交换图表
	\[\begin{tikzcd}
		0 \arrow[r] & J \arrow[r] & I_{k+1} \arrow[r] & I_k \arrow[r] & 0 \\
		0 \arrow[r] & H \arrow[r] \arrow[u, "g"] & \tau^{\geq -k-1} A \arrow[u, "{f_{k+1}}"] \arrow[r] & \tau^{\geq -k} A \arrow[r] \arrow[u, "f_k"'] & 0
	\end{tikzcd}\]
	使得 $f_{k+1}$ 是内射解消. 第一行正合相当于 $0 \to J^n \to I_{k+1}^n \to I_k^n \to 0$ 对所有 $n$ 皆正合, 而每项都是 $\mathcal{A}$ 的内射对象, 故引理 \ref{prop:inj-proj-split} 说明 $I_{k+1}^n \to I_k^n$ 有截面. 明所欲证.
\end{proof}

后继的讨论将用到 \S\ref{sec:lim1} 的一些符号.

取 $A$ 如上. 当 $k \geq 0$ 变动, 诸 $\tau^{\geq -k} A$ 给出 $\cate{InvSys}(\cate{C}(\mathcal{A}))$ 的对象, 记为 $\tau A$. 截断函子的定义给出一族典范态射 $A \to \tau^{\geq -k} A$. 逐次地考虑复形 $\tau^{\geq -k} A$, 可见它们的 $\varprojlim_k$ 存在, 而且上述典范态射诱导同构
\begin{equation}\label{eqn:A-lim-truncation}
	A \rightiso \varprojlim_{k \geq 0} \left( \tau^{\geq -k} A \right).
\end{equation}

其次考察 $I_k$. 假定 $\mathcal{A}$ 有可数积, 因此 $\cate{C}(\mathcal{A})$ 也有逐次地构造的可数积. 对 $A \in \Obj(\cate{C}(\mathcal{A}))$ 考虑引理 \ref{prop:K-injective-truncated} 给出的交换图表. 我们有
\[ I := \varprojlim_k I_k \quad \text{存在}. \]
更具体地说, 根据定义 \ref{def:Delta-diff} 及其后的讨论, $I$ 和 $A$ 皆可置入正合列
\begin{gather*}
	0 \to I \to \prod_k I_k \xrightarrow{\Delta_I} \prod_k I_k, \\
	0 \to A \to \prod_k \tau^{\geq -k} A \xrightarrow{\Delta_{\tau A}} \prod_k \tau^{\geq -k} A,
\end{gather*}
态射 $\Delta_I$ 和 $\Delta_{\tau A}$ 见诸上引定义. 由此可得交换图表
\begin{equation}\label{eqn:K-injective-diagonal}\begin{tikzcd}
	I \arrow[hookrightarrow, r] \arrow[d] & \displaystyle\prod_k I_k \\
	\Cone(\Delta_I)[-1] \arrow[ru, "{\beta(\Delta_I)[-1]}"'] &
\end{tikzcd} \quad \begin{tikzcd}
	A \arrow[hookrightarrow, r] \arrow[d] & \displaystyle\prod_k \tau^{\geq -k} A \\
	\Cone(\Delta_{\tau A})[-1] \arrow[ru, "{\beta(\Delta_{\tau A})[-1]}"']
\end{tikzcd}\end{equation}
垂直箭头的定义都是嵌入映射锥的第一个坐标, 请读者简单验证; 另一种小题大做的解释则是用同伦核的泛性质 (命题 \ref{prop:homotopy-kernel-cokernel}).

另一方面, 从 $(f_k)_{k \geq 0}$ 和 $\varprojlim$ 的泛性质可得态射
\[ f: A \simeq \varprojlim_k (\tau^{\geq -k} A) \to \varprojlim_k I_k = I, \]
它使下图交换
\begin{equation}\label{eqn:K-injective-A-I}\begin{tikzcd}
	I \arrow[r, "\text{如 \eqref{eqn:K-injective-diagonal}}" inner sep=0.4em] & \Cone\left(\Delta_I\right)[-1] \\
	A \arrow[r] \arrow[u, "f"] & \Cone\left( \Delta_{\tau A} \right)[-1] \arrow[u, "\text{由诸 $f_k$ 诱导}"'].
\end{tikzcd}\end{equation}
引理 \ref{prop:K-injective-projlimit-aux} 连同引理 \ref{prop:K-injective-truncated} 说明 $I$ 是 K-内射复形. 若能说明 $f$ 是拟同构, 则 $f$ 给出 $A$ 的 K-内射解消. 最后这一击需要另加条件.

\begin{lemma}\label{prop:K-injective-equiv}
	在上述场景中, 进一步设 $\mathcal{A}$ 有正合的可数积 (约定 \ref{con:exact-product}), 则 $f$ 是拟同构当且仅当 $A \to \Cone(\Delta_{\tau A})[-1]$ 是拟同构.
\end{lemma}
\begin{proof}
	对自然的交换图表
	\[\begin{tikzcd}
		\prod_k I_k \arrow[r, "\Delta_I"] & \prod_k I_k \arrow[r, "{\alpha(\Delta_I)}"] & \Cone\left(\Delta_I\right) \\
		\prod_k \tau^{\geq -k} A \arrow[r, "\Delta_A"'] \arrow[u, "{\prod_k f_k}"] & \prod_k \tau^{\geq -k} A \arrow[u, "{\prod_k f_k}"] \arrow[r, "{\alpha(\Delta_{\tau A})}"'] & \Cone\left( \Delta_{\tau A} \right) \arrow[u]
	\end{tikzcd}\]
	取映射锥的长正合列 \eqref{eqn:cone-long-exact-sequence}, 导出行正合交换图表
	\[\begin{tikzcd}
		[column sep=tiny, every cell/.append style = {font = \small}]
		\cdots \Hm^n\left( \displaystyle\prod_k I_k \right) \arrow[r] & \Hm^n\left( \Cone(\Delta_I) \right) \arrow[r] & \Hm^{n+1}\left( \displaystyle\prod_k I_k \right) \arrow[r, "{\Hm^{n+1}(\Delta_I)}" inner sep=1em] & \Hm^{n+1}\left( \displaystyle\prod_k I_k \right) \\
		\cdots \Hm^n\left( \displaystyle\prod_k \tau^{\geq -k} A \right) \arrow[r] \arrow[u, "{\Hm^n(\prod_k f_k)}"] & \Hm^n\left( \Cone(\Delta_{\tau A}) \right) \arrow[r] \arrow[u] & \Hm^{n+1}\left(\displaystyle\prod_k \tau^{\geq -k} A \right) \arrow[u, "{\Hm^{n+1}(\prod_k f_k)}"] \arrow[r, "{\Hm^{n+1}(\Delta_{\tau A})}"' inner sep=1.5em] & \Hm^{n+1}\left( \displaystyle\prod_k \tau^{\geq -k} A \right) \arrow[u, "{\Hm^{n+1}(\prod_k f_k)}"'] .
	\end{tikzcd}\]
	回忆到每个 $f_k$ 都是拟同构, 可数积正合故 $\prod_k f_k$ 仍是拟同构. 代入命题 \ref{prop:5-lemma} 遂得 $\Cone(\Delta_{\tau_A}) \to \Cone(\Delta_I)$ 也是拟同构.
	
	其次证明 $I \to \Cone(\Delta_I)[-1]$ 是拟同构. 首先断言 $\Delta_I$ 是满的, 逐项检查如下. $\Delta_{I^n}: \prod_k I_k^n \to \prod_k I_k^n$ 之所以满, 是因为 $I_{k+1}^n \to I_k^n$ 有截面, 见例 \ref{eg:lim1-section}. 将短正合列 $0 \to I \to \prod_k I_k \xrightarrow{\Delta_I} \prod_k I_k \to 0$ 代入引理 \ref{prop:cone-qis} 作比较, 可见先前定义的态射 $I \to \Cone(\Delta_I)[-1]$ 确实是拟同构.
	
	将这些结果代入交换图表 \eqref{eqn:K-injective-A-I} 并取上同调, 即所求充要条件.
\end{proof}

\begin{theorem}\label{prop:K-injective-resolution}
	满足以下两则条件的 Abel 范畴 $\mathcal{A}$ 有足够的 K-内射复形:
	\begin{compactitem}
		\item $\mathcal{A}$ 有正合的可数积;
		\item $\mathcal{A}$ 有足够的内射对象.
	\end{compactitem}
\end{theorem}
\begin{proof}
	引理 \ref{prop:K-injective-equiv} 将问题化为对 $A \in \Obj(\cate{C}(\mathcal{A}))$ 证 $A \to \Cone(\Delta_{\tau A})[-1]$ 为拟同构. 选定 $n \in \Z$. 既然可数积正合, 引理 \ref{prop:truncation-ses} 给出典范同构
	\[ \Hm^n\left( \prod_k \tau^{\geq -k} A \right) \simeq \prod_k \Hm^n\left( \tau^{\geq -k} A \right) \simeq \prod_{k \geq -n} \Hm^n(A). \]
	假如对 $\mathcal{A}$ 的对象可以谈论元素, 则态射 $\Delta_{\tau A}$ 在取 $\Hm^n$ 以后表作
	\[\begin{tikzcd}[row sep=tiny]
		\prod_{k \geq -n} \Hm^n(A) \arrow[r, "\nabla^n"] & \prod_{k \geq -n} \Hm^n(A) \\
		(a_k)_{k \geq -n} \arrow[mapsto, r] & (a_{k+1} - a_k)_{k \geq -n}.
	\end{tikzcd}\]
	易见 $\nabla^n$ 满 (可写下截面), 而 $\Ker(\nabla^n) = \left\{ (a)_{k \geq -n} : a \in \Hm^n(A) \right\}$, 即其``对角''部分. 这些观察不难移植到一般的 $\mathcal{A}$ 上. 现在考虑映射锥的长正合列
	\[ \cdots \to \Hm^n \Cone(\Delta_{\tau A})[-1] \to \prod_{k \geq -n} \Hm^n(A) \xrightarrow[\text{满}]{\nabla^n} \prod_{k \geq -n} \Hm^n(A) \to \Hm^n \Cone(\Delta_{\tau A}) \cdots \]
	这将 $\Hm^n \Cone(\Delta_{\tau A})[-1] \to \prod_{k \geq -n} \Hm^n(A)$ 等同于 $\Ker(\nabla^n)$ 
	的嵌入. 此外, \eqref{eqn:K-injective-diagonal} 蕴涵
	\[\begin{tikzcd}
		\Hm^n(A) \arrow[d] \arrow[hookrightarrow, r, "\text{对角态射}"] & \prod_{k \geq -n} \Hm^n(A) \\
		\Hm^n {\Cone(\Delta_{\tau A})[-1]} \arrow[hookrightarrow, ru] &
	\end{tikzcd}\]
	交换. 综上可知 $\Hm^n(A) \to \Hm^n\left( \Cone(\Delta_{\tau A})[-1] \right)$ 对每个 $n$ 皆是同构.
\end{proof}

以命题 \ref{prop:K-sigma-duality} 倒转箭头, 遂给出定理 \ref{prop:K-injective-resolution} 的对偶版本.

\begin{theorem}\label{prop:K-projective-resolution}
	满足以下两则条件的 Abel 范畴 $\mathcal{A}$ 有足够的 K-投射复形:
	\begin{compactitem}
		\item $\mathcal{A}$ 有正合的可数余积;
		\item $\mathcal{A}$ 有足够的投射对象.
	\end{compactitem}
\end{theorem}

\begin{corollary}\label{prop:Grothendieck-K-projective}
	若 $\mathcal{A}$ 是有足够投射对象的 Grothendieck 范畴 (定义 \ref{def:Grothendieck-cat}), 则它也有足够的 K-投射复形.
\end{corollary}
\begin{proof}
	Grothendieck 范畴 $\mathcal{A}$ 按定义有可数余积. 命题 \ref{prop:Grothendieck-cat-coprod-exact} 蕴涵可数余积正合. 将此代入定理 \ref{prop:K-projective-resolution}.
\end{proof}

\begin{example}\label{eg:Mod-K-injectives}
	设 $R$ 为环. 定理 \ref{prop:K-injective-resolution} 和推论 \ref{prop:Grothendieck-K-projective} 表明 $R\dcate{Mod}$ 有足够的 K-内射和 K-投射复形.
\end{example}

事实上, Grothendieck 范畴上的所有复形皆有 K-内射解消, 而且 K-内射解消可以如定理 \ref{prop:Grothendieck-injective} 一般作成函子, 这是 \cite{Serp03} 的主定理, 亦见 \cite[Tag 079P]{stacks}; 它足以对付定理 \ref{prop:K-injective-resolution} 所不及, 但几何学中常见的一些场景.

\begin{Exercises}
	\item 设 $\Bbbk$ 为交换环, 因此 $\cate{C}(\Bbbk\dcate{Mod})$ 和 $\cate{K}(\Bbbk\dcate{Mod})$ 自然地成为定义 \ref{def:cat-k-linear} 所谓的 $\Bbbk\dcate{Mod}$-范畴. 说明对所有 $n \in \Z$ 和 $\cate{C}(\Bbbk\dcate{Mod})$ 的对象 $X$, 皆有典范 $\Bbbk$-模同构
	\[ \Hm^n(X) \simeq \Hom_{\cate{K}(\Bbbk\dcate{Mod})}(\Bbbk, X[n]) \simeq \Hom_{\cate{K}(\Bbbk\dcate{Mod})}(\Bbbk[-n], X); \]
	右式的 $\Bbbk$ 视同集中在 $0$ 次项的复形.

	\item 本题旨在联系态射的同伦以及映射锥的同构. 设 $\mathcal{A}$ 为加性范畴, 考虑 $\cate{C}(\mathcal{A})$ 的任一对态射 $f, g \in \Hom(X,Y)$. 建立双射
	\[ \left\{ s \in \Hom^{-1}(X, Y) : d^{-1}_{\Hom^\bullet(X, Y)} (s) = f - g \right\} \xrightarrow{1:1}
	\left\{ \begin{array}{ll}
		h: \Cone(f) \to \Cone(g), \\
		\text{复形态射, 使下图交换}
	\end{array}\right\}, \]
	\begin{equation*}\begin{tikzcd}[row sep=small]
		& \Cone(f) \arrow[dr, "\beta(f)"] \arrow[dd, "h"] & \\
		Y \arrow[ru, "\alpha(f)"] \arrow[rd, "\alpha(g)"'] & & X[1] \\
		& \Cone(g) \arrow[ru, "\beta(g)"'] &
	\end{tikzcd}, \end{equation*}
	而且如是态射 $h$ 必为复形的同构; 特别地, $f$ 和 $g$ 同伦当且仅当存在如是之 $h$.

	\begin{hint}
		图表在 $n$ 次项的交换性等价于矩阵等式
		\[ h^n = \begin{pmatrix} \identity_{X^{n+1}} & 0 \\ s^{n+1} & \identity_{Y^n} \end{pmatrix}, \quad \text{其中}\; s^{n+1}: X^{n+1} \to Y^n. \]
	\end{hint}

	\item (分裂正合性) 选定 Abel 范畴 $\mathcal{A}$ 以及其上的复形 $X$. 试证以下陈述等价.
	\begin{enumerate}[(i)]
		\item $X$ 在 $\cate{K}(\mathcal{A})$ 中等于 $0$;
		\item $\identity_X$ 零伦;
		\item $X$ 零调, 而且存在 $(s^n)_n \in \Hom^{-1}(X, X)$ 使得 $d^n s^{n+1} d^n = d^n$ 对所有 $n$ 成立;
		\item 存在 $\mathcal{A}$ 的一列对象 $(Y^n)_n$ 和复形的同构 $\Phi: X \rightiso (Y^n \oplus Y^{n+1}, \overline{d}^n)_n$, 其中 $\overline{d}$ 用矩阵表为
		$\bigl(\begin{smallmatrix} 0 & 1 \\ 0 & 0 \end{smallmatrix}\bigr)$.
	\end{enumerate}

	满足以上任一条件的复形也称为\emph{分裂正合}的
	\footnote{推而广之, 设 $X$ 是加性范畴 $\mathcal{A}$ 上的复形. 若存在两列对象 $(Y^n)_n$ 和 $(H^n)_n$ 使得 $X^n \simeq Y^n \oplus H^n \oplus Y^{n+1}$ 而 $d$ 等同于 $\bigl(\begin{smallmatrix} 0 & 0 & 1 \\ 0 & 0 & 0 \\ 0 & 0 & 0 \end{smallmatrix}\bigr)$, 则称 $X$ 为\emph{分裂}的.}.
	对于 $X = \left[ \cdots 0 \to A \to B \to C \to 0 \cdots\right]$ 的情形, 说明这等价于 $X$ 是分裂短正合列.
	\index{fuxing!分裂正合 (split exact)}
	
	\begin{hint}
		仅论 (iii) $\implies$ (iv). 条件蕴涵 $d^{n-1} s^n \in \End_{\mathcal{A}}(X^n)$ 是幂等元; 由此得到分解 $X^n = Y^n \oplus Z^n$, 使得 $Y^n = \Image(d^{n-1} s^n)$. 论证 $Y^n = \Image(d^{n-1})$ 而 $d^n$ 限制为同构 $Z^n \rightiso \Image(d^n)$, 然后让 $\Phi$ 对应到
		$\bigl( \begin{smallmatrix} ds \\ d \end{smallmatrix} \bigr)$.
	\end{hint}

	\item 选定 Abel 范畴 $\mathcal{A}$. 证明 $X$ 是 $\cate{C}(\mathcal{A})$ 的内射对象当且仅当 $X$ 由 $\mathcal{A}$ 的内射对象组成, 而且在 $\cate{K}(\mathcal{A})$ 中等于 $0$. 陈述投射对象的版本.
	
	\begin{hint}
		对于``仅当''情形, 注意到 $\cate{C}(\mathcal{A})$ 的短正合列 $0 \to X \to \Cone(\identity_X) \to X[1] \to 0$ 分裂; 由命题 \ref{prop:cone-homotopy} 遂知 $X$ 零调. 取截面 $X[1] \to \Cone(\identity_X)$, 表之为
		$\bigl(\begin{smallmatrix} 1 \\ -\theta \end{smallmatrix}\bigr)$, 其中 $\theta^n: X^{n+1} \to X^n$. 验证 $d_X^{n-1} \theta^{n-1} + \theta^n d_X^n = \identity_{X^n}$ 以说明 $\identity_X$ 零伦.
		
		选定 $n$, 按上一道习题将 $X^n$ 分解为 $Y^n \oplus Y^{n+1}$, 后续问题化为证 $Y^n$ 内射. 任取 $\mathcal{A}$ 的态射 $f: A \to Y^n$ 和单态射 $g: A \hookrightarrow B$, 考虑 $\cate{C}(\mathcal{A})$ 的行正合交换图表
		\[\begin{tikzcd}[column sep=large]
			0 \arrow[r] & {A[-n]} \arrow[r, "{g[-n]}"] \arrow[d, "\Psi"'] & {B[-n]}, \arrow[dashed, ld, "\exists"] \\
			& X &
		\end{tikzcd}\]
		其中 $\Psi^n = \big(\begin{smallmatrix} f \\ 0 \end{smallmatrix} \bigr)$. 以此验证 $Y^n$ 的内射条件.
		
		对于``当''的情形, 仍将 $X^n$ 表示成 $Y^n \oplus Y^{n+1}$, 具体描述所有可能的态射 $f: Z \to X$.
	\end{hint}

	\item 设 $\Bbbk$ 是交换环, $R$ 是 $\Bbbk$-代数, 按本书惯例要求含幺元. 记 $M_n(R)$ 为 $R$ 上的 $n \times n$ 矩阵 $\Bbbk$-代数. 试给出典范同构 $\HHm_0(M_n(R)) \simeq \HHm_0(R)$ 和 $\HHm^0(M_n(R)) \simeq \HHm^0(R)$. 事实上, 对任意 $\HHm_m$ 和 $\HHm^m$ 都有这般同构, 见 \cite[Theorem 1.2.4]{Lo98}.

	\item 在 \S\ref{sec:HH} 的情境中, 对每个 $n \in \Z_{\geq 0}$ 定义 $\mathsf{B}_n R$ 的子双模
	\begin{align*}
		\mathrm{Triv}_n R & := \sum_{h=1}^n R \otimes ( \cdots \otimes \underbracket{\;\Bbbk\;}_{\text{第 $h$ 个}} \otimes \cdots ) \otimes R, \quad \mathrm{Triv}_0 R = 0.
	\end{align*}
	\begin{enumerate}[(i)]
		\item 验证这给出 $\mathsf{B} R$ 的子复形 $\mathrm{Triv} R$, 而且 $\mathrm{Triv} R$ 正合.
		\begin{hint}
			将引理 \ref{prop:HH-bar-exactness} 证明中的 $(h^n)_n$ 限制到 $\mathrm{Triv} R$ 上.
		\end{hint}
		\item 记 $\overline{R} := R/\Bbbk$ (作为 $\Bbbk$-模), 定义 $R$ 的既约杠复形为商
			\[ \overline{\mathsf{B}} R := \mathsf{B} R / \mathrm{Triv} R = \left( R \otimes \overline{R}^{\otimes n} \otimes R, b_n \right)_{n \geq 0}. \]
			参照 \eqref{eqn:HH-cplx} 的模式来简化 $M \dotimes{R^e} \overline{\mathsf{B}} R$ 和 $\Hom_{R^e}\left(\overline{\mathsf{B}} R, M \right)$.
		\item 给出典范同构 $\Hm_n\left( M \dotimes{R^e} \overline{\mathsf{B}} R \right) \simeq \HHm_n(M)$ 和 $\Hm^n\left( \Hom_{R^e}(\overline{\mathsf{B}} R, M) \right) \simeq \HHm^n(M)$.
		\begin{hint}
			一般而言, 此事实归结为 Dold--Kan 对应理论中关于正规化复形的定理 \ref{prop:Dold-Kan-N-C}, 前提是得从例 \ref{eg:HH-comonad} 的高度来理解 $\mathsf{B}R$.
		\end{hint}
	\end{enumerate}

	\item 设 $\Bbbk$ 为交换环, $\tilde{R}$ 和 $R$ 为 $\Bbbk$-代数, 按本书惯例要求含幺元, 并且作为 $\Bbbk$-模有分裂短正合列
	\[\begin{tikzcd}
		0 \arrow[r] & M \arrow[r] & \tilde{R} \arrow[shift left, r, "p"] & R \arrow[r] \arrow[shift left, l, "s"] & 0,
	\end{tikzcd}\]
	其中 $M$ 是 $\tilde{R}$ 的双边理想, $ps = \identity_R$, 而且 $M^2 = \{0\}$. 通过 $s$ 将 $\tilde{R}$ 等同于 $R \oplus M$.
	\begin{enumerate}[(i)]
		\item 对 $m \in M$ 和 $r \in R$, 任取 $\tilde{r} \in p^{-1}(r)$, 说明 $\tilde{r}m$ 和 $m\tilde{r}$ 只和 $m, r$ 有关, 此运算使 $M$ 成为 $(R,R)$-双模.
		\item 说明存在双线性映射 $f: R^2 \to M$ 使得 $\tilde{R}$ 的乘法由下式确定:
		\[ (r_1, m_1)(r_2, m_2) = (r_1 r_2, r_1 m_2 + m_1 r_2 + f(r_1, r_2)), \quad r_1, r_2 \in R, \; m_1, m_2 \in M. \]
		\item 这样的资料 $(R, \tilde{R}, M)$ 称为 $R$ 对 $M$ 的零平方扩张. 证明在合适的同构意义下, 零平方扩张由 $\HHm^2(M)$ 分类, 零元对应到平凡扩张 $R \times M$.
		\begin{hint}
			使用前一道习题的既约杠复形计算 $\HHm^2(M)$.
		\end{hint}
	\end{enumerate}

	\item 补全引理 \ref{prop:CC-cplx} 的证明.

	% Rosenberg, p.311
	\item 在定义 \ref{def:HC} 的框架下, 对 $R = \Bbbk$ 的特例验证
	\begin{enumerate}[(i)]
		\item 当 $n$ 是奇数时, $\mathrm{HP}_n(\Bbbk) = \mathrm{HC}_n(\Bbbk) = 0$;
		\item 当 $n$ 是偶数 (或者非负偶数) 时, $\mathrm{HP}_n(\Bbbk)$ (或 $\mathrm{HC}_n(\Bbbk)$) 同构于 $\Bbbk$.
	\end{enumerate}

	% Rosenberg, p.313
	\item 在定义 \ref{def:HC} 和定理 \ref{prop:Connes-exact} 的框架下, 取多项式环 $R = \Bbbk[t]$.
	\begin{enumerate}[(i)]
		\item 说明例 \ref{eg:HH1} 的 $\Omega_{R|\Bbbk}$ 是秩 $1$ 自由 $R$-模, 以 $\dd t$ 为基.
		\item 描述连接同态 $B: \mathrm{HC}_0(R) \to \mathrm{HH}_1(R)$; 注意到两边都同构于 $R$.
		\item 尽量明确地描述所有的 $\mathrm{HC}_n(R)$. 在 $\Bbbk = \Z$ 的情形说明 $\mathrm{HC}_1(\Z[t]) = \bigoplus_{n \geq 2} \Z/n\Z$.
	\end{enumerate}
	\begin{hint}
		运用定理 \ref{prop:Connes-exact} 的长正合列, 以及例 \ref{eg:HH-Tor} 对 $\HHm_n(\Bbbk[t])$ 在高次情形的计算.
	\end{hint}

	\item 设 $\mathcal{A}$ 为 Abel 范畴, $0 \to X \to I^0 \to I^1 \to \cdots$ 为 $\cate{C}(\mathcal{A})$ 中的正合列. 视 $I := [I^0 \to I^1 \to \cdots]$ 为双复形, $I^{p, q} := (I^p)^q$. 在 $I \in \Obj(\cate{C}^2_f(\mathcal{A}))$ 的前提下, 证明自明的态射 $X \to \tot(I)$ 是拟同构.

	\item 对 $X \in \Obj(\cate{C}^2_f(\mathcal{A}))$, 其中 $\mathcal{A}$ 是 Abel 范畴, 证明若第 $i$ 列 $\left( X^{i, \bullet}, \dvert \right)$ 和第 $j$ 行 $\left( X^{\bullet, j}, \dhori \right)$ 对所有 $i, j \in \Z \smallsetminus \{0\}$ 皆正合, 则对所有 $n \in \Z$ 皆有典范同构 $\Hm^n\left( X^{0, \bullet}, \dvert \right) \simeq \Hm^n\left( X^{\bullet, 0}, \dhori \right)$.
	
	\begin{hint}	
		定义从 $\cate{C}^2(\mathcal{A})$ 到自身的暴力截断函子 $\sigma_{\mathrm{I}}^{\leq n}$ (或 $\sigma_{\mathrm{I}}^{\geq n}$), 它将双复形 $(X^{i,j})_{(i,j) \in \Z^2}$ 中 $i > n$ (或 $i < n$) 的项换为 $0$, 其余不变. 以定理 \ref{prop:double-cplx-tot} 论证
		\[ \text{第 $0$ 列} = \sigma_{\mathrm{I}}^{\leq 0} \sigma_{\mathrm{I}}^{\geq 0}(X) \twoheadleftarrow \sigma_{\mathrm{I}}^{\geq 0} X \hookrightarrow X \]
		的两段态射都诱导全复形的拟同构. 行列角色对调亦同.
	\end{hint}

	\item 运用引理 \ref{prop:ext-alpha-beta} 以证明定理 \ref{prop:resolution-homotopy} 中关于 $\beta$ 的存在性部分.
	
	\begin{hint}
		讨论第一种情形即可. 关于映射柱的引理 \ref{prop:cone-beta} 将 $X \xrightarrow{\alpha} Y$ 分解为 $X \xrightarrow{\alpha'} \Cyl(\alpha) \xrightarrow{\psi} Y$, 其中 $\alpha'$ 单, $\psi$ 在 $\cate{K}(\mathcal{A})$ 中可逆.
	\end{hint}

	\item 承上题, 证明定理 \ref{prop:resolution-homotopy} 中关于 $\beta$ 的唯一性.

	\begin{hint}
		讨论第一种情形. 仍以映射柱化约到 $\alpha^n: X^n \hookrightarrow Y^n$ 对每个 $n$ 都是直和项的嵌入的情形. 设 $\cate{C}(\mathcal{A})$ 的态射 $\beta_i: Y \to I$ ($i=1,2$) 满足 $\gamma - \beta_i \alpha = d^{-1}_{\Hom^\bullet(X, I)} h_i$, 其中 $h_i \in \Hom^{-1}(X, I)$. 取零调复形 $C := \Coker(\alpha)$, 定义 $h^n: Y^n \simeq X^n \oplus C^n \to I^{n-1}$ 使得它拉回 $X^n$ (或 $C^n$) 等于 $h_1^n - h_2^n$ (或 $0$). 验证
		\[ \beta_1 - \beta_2 - d^{-1}_{\Hom^\bullet(Y, I)} h \; : Y \to I \]
		合成 $\alpha$ 为 $0$, 因此 $\beta_1 - \beta_2$ 精确到同伦分解为 $Y \twoheadrightarrow C \to I$; 再对其后段应用引理 \ref{prop:resolution-homotopy-aux}.
	\end{hint}

	\item 设 $\mathcal{A}$ 为 Abel 范畴, 证明 $(\Hm^n)_{n \geq 0}$ 给出从 Abel 范畴 $\cate{C}^{\geq 0}(\mathcal{A})$ (见定义 \ref{def:cplx-cat-variant}) 到 $\mathcal{A}$ 的泛上同调 $\delta$-函子.
	\begin{hint}
		为了证明 $n > 0$ 时 $\Hm^n$ 可拭, 取 $\alpha(\identity_X): X \hookrightarrow \Cone(\identity_X)$, 它通过 $\Cone(\identity_X)$ 的暴力截断 $\sigma^{\geq 0} \Cone(\identity_X)$ 分解; 运用命题 \ref{prop:cone-homotopy} (i).
	\end{hint}

	% https://ncatlab.org/nlab/show/lim%5E1+and+Milnor+sequences
	% Weibel, Theorem 3.5.8 for the chain version
	\item (Milnor 正合列的复形版本) 考虑任意环 $R$ 和 $\cate{InvSys}(\cate{C}(R\dcate{Mod}))$ 的对象 $(X_k, f_k)_{k \geq 0}$, 要求 $(X_k^n, f_k^n)_{k \geq 0}$ 对每个 $n \in \Z$ 都是 Mittag-Leffler 的; 今后省略符号 $f_k$. 按下述步骤建立短正合列
	\[ 0 \to \lim\nolimits^1_k \Hm^{n-1}(X_k) \to \Hm^n( \varprojlim_k X_k ) \to \varprojlim_k \Hm^n(X_k) \to 0. \]
	\begin{enumerate}[(i)]
		\item 按照定理 \ref{prop:CE-resolution} 的方式定义 $B_k^n \subset Z_k^n \subset X_k^n$ 和 $H_k^n$, 作成复形 $B_k \subset Z_k \subset X_k$ 和 $H_k$, 使得 $d_{B_k}$, $d_{Z_k}$ 和 $d_{H_k}$ 全为 $0$. 证明 $(B_k^n)_{k \geq 0}$ 对每个 $n$ 都是 Mittag-Leffler 的.
		
		\item 说明 $0 \to \varprojlim_k Z_k \to \varprojlim_k X_k \xrightarrow{d} \left( \varprojlim_k X_k \right)[1]$ 正合.
		
		\item 定义 $B^n := \Image\left( d^{n-1}: \varprojlim_k X_k^{n-1} \to \varprojlim X_k^n \right)$, 作成复形使得 $d_B = 0$. 证明有短正合列 $0 \to B[1] \to \varprojlim_k B_k[1] \to \lim\nolimits^1_k Z_k \to 0$.
		
		\begin{hint}
			应用短正合列 $0 \to Z_k \to X_k \xrightarrow{d} B_k[1] \to 0$ 和 $(X_k^n)_{k \geq 0}$ 的 Mittag-Leffler 条件.
		\end{hint}
		
		\item 推导同构 $\lim\nolimits^1_k Z_k \rightiso \lim\nolimits^1_k H_k$ 和短正合列 $0 \to \varprojlim_k B_k \to \varprojlim_k Z_k \to \varprojlim_k H_k \to 0$.
		
		\begin{hint}
			应用短正合列 $0 \to B_k \to Z_k \to H_k \to 0$ 和 $(B_k^n)_{k \geq 0}$ 的 Mittag-Leffler 条件.
		\end{hint}
		
		\item 取适当的嵌入 $B \subset \varprojlim_k B_k \subset \varprojlim_k Z_k$, 讨论对应的商来推导所求的短正合列.
	\end{enumerate}

	\item 设 $\Bbbk$ 为域. 定义 $\cate{InvSys}(\Bbbk\dcate{Mod})$ 的对象 $A$, $B$ 使得 $A_k := X^k \Bbbk\llbracket X\rrbracket$ 而 $B_k := X^k \Bbbk[X]$, 对应的态射为包含态射. 它们不满足 Mittag--Leffler 条件. 证明 $\lim\nolimits^1 A = 0$ 而 $\lim\nolimits^1 B \simeq \Bbbk\llbracket X \rrbracket/\Bbbk[X]$.
	\begin{hint}
		将这些理想列嵌入环, 用命题 \ref{prop:dimension-shifting} 的移维技巧来计算 $\lim\nolimits^1$.
	\end{hint}

	\item 设 $D$ 是除环. 说明 $n > 0$ 时 $\Ext^n_D$ 和 $\Tor^D_n$ 全为 $0$.

	\item 对所有环 $R$, 严谨地表述并证明典范同构 $\Tor^R_n(X, Y) \simeq \Tor^{R^{\opp}}_n(Y, X)$.

	\item 设整环 $R$ 为主理想环. 试对所有有限生成 $R$-模 $M$, $N$ 和 $n \in \Z_{\geq 0}$ 精确描述 $\Ext^n_R(M, N)$ 和 $\Tor_n^R(M, N)$.

	\item 设 $M$ 为 $\Z$-模, 说明 $\Tor_1^{\Z}(\Q/\Z, M)$ 是 $M$ 的所有挠元构成的子群.

	\item (上同调泛系数定理) 考虑环 $R$ 和左 $R$-模构成的链复形 $C = (C_\bullet, d^C_\bullet)$. 对于任意左 $R$-模 $M$, 逐项取 $\Hom_R(\cdot, M)$ 给出上链复形 $C^\bullet := \Hom(C_\bullet, M)$, 其第 $n$ 次上同调记为 $\Hm^n\left(C, M \right)$. 假设 $B_n := \Image(d_{n+1})$ 和 $Z_n := \Ker(d_n)$ 对所有 $n \in \Z$ 都是投射模, 并且命 $H_n := Z_n/B_n = \Hm_n(C)$. 证明对每个 $n \in \Z$ 都有典范的短正合列
	\[ 0 \to \Ext^1_R\left( \Hm_{n-1}(C), M \right) \to \Hm^n(C, M) \to \Hom_R\left( \Hm_n(C), M \right) \to 0. \]
	进一步证明若 $R$ 是主理想环, 而且每个 $C_n$ 都自由, 则条件必成立.
	\index{fanxishudingli}
	
	\begin{hint}
		定义从 $(R\dcate{Mod})^{\opp}$ 到 $\cate{Ab}$ 的函子 $D(\cdot) := \Hom_R(\cdot, M)$. 运用短正合列 $0 \to H_n \to C_n/B_n \to B_{n-1} \to 0$ 和条件得到行正合交换图表
		\[\begin{tikzcd}
			0 \arrow[r] & D(Z_{n-1}) \arrow[r, "\identity"] \arrow[d] & D(Z_{n-1}) \arrow[r] \arrow[d, "{D(d^C_n)}"] & 0 \arrow[r] \arrow[d] & 0 \\
			0 \arrow[r] & D(B_{n-1}) \arrow[r, "{D(d^C_n)}"' inner sep=0.5em] & D(C_n/B_n) \arrow[r] & D(H_n) \arrow[r] & 0 .
		\end{tikzcd}\]
		说明 $d_{D(C)}^{n-1}$ 分解为 $D(C_{n-1}) \twoheadrightarrow D(Z_{n-1}) \xrightarrow{D(d^C_n)} D(C_n/B_n) \hookrightarrow D(C_n)$, 而 $D(C_n/B_n) \simeq \Ker(d_C^n)$, 故图表中路的余核是 $\Hm^n(C, M)$. 另一方面, $0 \to B_{n-1} \to Z_{n-1} \to H_{n-1} \to 0$ 说明左路余核是 $\Ext^1_R\left(\Hm_{n-1}(C), M \right)$. 应用定理 \ref{prop:snake-lemma}.
	\end{hint}

	% Weibel, Exercise 9.1.4
	\item 在 \S\ref{sec:HH} 的框架下, 取 $R = \Bbbk[t]/(t^n)$, 其中 $t$ 是多项式的变元, $n \in \Z_{\geq 1}$, 而 $R^e = R[x,y]/(x^n, y^n)$.
	\begin{enumerate}
		\item 说明以下链复形正合
		\[ \cdots \xrightarrow{v} R^e \xrightarrow{u} R^e \xrightarrow{v} R^e \xrightarrow{u} R^e \to R \to 0 , \]
		其中 $R^e \to R$ 映 $f(x,y)$ 为 $f(t,t)$, 而
		\[ u := x-y, \quad v := \sum_{\substack{a+b=n-1 \\ a, b \geq 0}} x^a y^b . \]
		以此说明对任何 $R$-模 $M$ 都有 $\HHm_{k+2}(M) \simeq \HHm_k(M)$ 和 $\HHm^{k+2}(M) \simeq \HHm^k(M)$.

		\item 设 $\Bbbk$ 为域, 给出 $\HHm_k(M)$ 和 $\HHm^k(M)$ 的描述, 按照 $k \geq 1$ 和 $\mathrm{char}(\Bbbk)$ 互素与否分开讨论.
	\end{enumerate}

	\item 对于任意环 $R$ 和 $n \in \Z_{\geq 0}$, 证明 $\Tor^R_n$ 和小 $\varinjlim$ 交换, 亦即
	\[ \Tor^R_n(\varinjlim_i X_i, Y) \simeq \varinjlim_i \Tor^R_n(X_i, Y), \quad \Tor^R_n(X, \varinjlim_i Y_i) \simeq \varinjlim_i \Tor^R_n(X, Y_i). \]
	类似地, 在 Abel 范畴 $\mathcal{A}$ 有足够内射对象和投射对象的前提下, 证明
	\[ \Ext_{\mathcal{A}}^n(\varinjlim_i X_i, Y) \simeq \varprojlim_i \Ext_{\mathcal{A}}^n(X_i, Y), \quad \Ext_{\mathcal{A}}^n(X, \varprojlim_i Y_i) \simeq \varprojlim_i \Ext_{\mathcal{A}}^n(X, Y_j) . \]

	\item 证明 Abel 范畴 $\mathcal{A}$ 的对象 $I$ 是内射对象当且仅当它视为复形是 K-内射的. 陈述其对偶版本.
	\begin{hint}
		一个方向来自例 \ref{eg:bdd-K-resolution}. 对于反方向, 将 $\mathcal{A}$ 中的短正合列 $0 \to A \to B \to C \to 0$ 视同零调复形, 从给定的态射 $A \to I$ 构造从此短正合列到 $I$ 的态射.
	\end{hint}

	\item (N.\ Nitsure) 考虑 Abel 范畴之间的左正合函子 $F: \mathcal{A} \to \mathcal{B}$. 设 $\mathcal{A}$ 有足够的内射对象, $X \in \Obj(\mathcal{A})$. 取 $X$ 的内射解消 $0 \to X \to I^0 \to I^1 \to \cdots$ 和 $F$-零调解消 $0 \to X \to A^0 \to A^1 \to \cdots$. 记
	\[ I := [\cdots \to 0 \to I^0 \to I^1 \to \cdots], \quad A := [\cdots \to 0 \to A^0 \to A^1 \to \cdots ]. \]
	\begin{itemize}
		\item 引理 \ref{prop:resolution-homotopy-classical} 在 $\cate{K}(\mathcal{A})$ 中确定唯一的态射 $\beta: A \to I$, 和 $A \leftarrow X \rightarrow I$ 兼容. 它给出 $c^n := \Hm^n(\beta): \Hm^n(A) \to \Hm^n(I) = \mathrm{R}^n F(X)$.
		\item 另一方面, 推论 \ref{prop:acyclic-resolution} 又通过移维给出同构 $d^n: \Hm^n(A) \rightiso \mathrm{R}^n F(X)$.
	\end{itemize}
	证明对所有 $n \geq 0$ 都有 $d^n = (-1)^{n(n+1)/2} c^n$.
	\begin{hint}
		这是 \cite{Ni09} 的内容.
	\end{hint}
\end{Exercises}
